\documentclass{oblivoir}
\usepackage{amsmath,amssymb,kotex,multicol}
\usepackage[margin=1cm]{geometry}
\addtolength{\topmargin}{1cm}

\newcounter{num}
\newcommand{\exam}[1]
{\noindent\refstepcounter{num}\textbf{예시 \arabic{num}) #1}\par\noindent}
\renewcommand{\arraystretch}{1.5}

\title{06 명제의 증명}
\date{\today}
\author{}

\begin{document}

\begin{multicols}{2}

%
\exam{}
실수 \(a\), \(b\), \(c\)에 대하여 (\(a\neq0\)),
\[ax^2+bx+c=0\]
형태의 식을 \emph{이차방정식}이라고 한다.
그리고, 이 식을 만족시키는 \(x\)의 값을 이 이차방정식의 \emph{근} 이라고 부른다.
예를 들어,
\[x^2-4x+3=0\]
는 \(a=1\), \(b=-4\), \(c=3\)인 형태의 식이므로 이차방정식이다.
또한,
\begin{align*}
x=1:&\quad1^2-3\times1+2=0\\
x=2:&\quad2^2-3\times2+2\neq0\\
x=3:&\quad3^2-3\times3+2=0\\
x=4:&\quad4^2-3\times4+2\neq0\\
x=5:&\quad5^2-3\times5+2\neq0
\end{align*}
이므로 \(x=1\), \(x=3\)은 이 이차방정식의 근이지만, \(x=2\), \(x=4\), \(x=5\)는 근이 아니다.

이차방정식의 두 근을 \(\alpha\), \(\beta\)라고 할 때, 다음 두 식 (1), (2)가 성립한다.
\begin{align}
\alpha+\beta&=-\frac ba\\
\alpha\beta&=\frac ca.
\end{align}
위의 예에서 \(\alpha=1\), \(\beta=3\)이므로 \(\alpha+\beta=4\), \(\alpha\beta=3\)이다.
그런데 \(-\frac ba=-\frac{-4}1=4\), \(\frac ca=\frac31=3\)이므로 (1), (2) 식을 확인할 수 있다.
이것이 성립하는 이유는 다음과 같이 설명할 수 있다.

\begin{align*}
ax^2+bx+c=0
&\iff x=\alpha\text{ 또는 }x=\beta\\
&\iff (x-\alpha)(x-\beta)=0\\
&\iff x^2-(\alpha+\beta)x+\alpha\beta=0
\end{align*}
즉, 두 이차방정식 \(ax^2+bx+c=0\)과 \(x^2-(\alpha+\beta)x+\alpha\beta=0\)이 같은 이차방정식이어야 한다.
%다시 말해, 첫번째 \(ax^2+bx+c=0\)의 양변을 \(a\)로 나눈 이차방정식인 \(x^2+\frac bax+\frac ca=0\)의 좌변과 두번째 
다시 말해, 다음 식이 성립하여야 한다.
\[x^2+\frac bax+\frac ca=0=x^2-(\alpha+\beta)x+\alpha\beta\]
즉, \(\frac ba=-(\alpha+\beta)\)와 \(\frac ca=\alpha\beta\)가 성립해야 한다.
이것을 정리하면 각각 (1), (2)가 된다.

\columnbreak
%
\exam{}
두 실수 \(a\), \(b\)에 대하여 \(z=a+bi\) 형태의 수를 \emph{복소수}라고 부른다.
이때, \(i\)는
\[i^2=-1\]
을 만족시키는 새로운 수이다.

예를 들어
\[z_1=2+3i\]
는 \(a=2\), \(b=3\)인 형태의 복소수이다.
또한,
\[z_2=5i\]
는 \(a=0\), \(b=5\)인 형태의 복소수이다.
이처럼 \(a=0\)인 형태의 복소수를 \emph{순허수}라고 부른다.
\[z_3=4\]
는 \(a=4\), \(b=0\)인 형태의 복소수인데, 이때 \(z_3\)은 실수이다.
즉, \(b=0\)이면, 복소수 \(a+bi\)는 실수이다.

복소수 \(z=a+bi\)에 대하여,
\[\bar z=a-bi\]
를 \(z\)의 \emph{켤레복소수}라고 부른다.
켤레복소수와 관련하여, 다음과 같이 몇가지 사실들이 성립한다.
\begin{enumerate}[(a)]\tightlist
\item
\(z=\bar z\)이면 \(z\)는 실수이다.
\item
\(z=-\bar z\)이면 \(z\)는 순허수거나 0이다.
\item
\(z+\bar z\)는 실수이다.
\item
\(z\bar z\)는 0보다 크거나 같은 실수이다.
\end{enumerate}

\(z=a+bi\)라고 두자. (단, \(a\), \(b\)는 실수)

(a) \(z=\bar z\)이면 \(a+bi=a-bi\)이다.
따라서 \(b=-b\)이고 \(b=0\)이다.
그러므로 \(z=a\)는 실수이다.

(b) \(z=-\bar z\)이면 \(a+bi=-(a-bi)=-a+bi\)이다.
따라서 \(a=-a\)이고 \(a=0\)이다.
그러면 \(z\)는 \(z=bi\)의 형태가 되는데, \(b\neq0\)이면 \(z\)는 순허수이고, \(b=0\)이면, \(z=0\)이다.

(c) \(z+\bar z=(a+bi)+(a-bi)=2a\)이다.
따라서 \(z+\bar z\)는 실수이다.

(d) \(z\bar z=(a+bi)(a-bi)=a^2+b^2\)이다.
다라서 \(z\bar z\)는 0보다 크거나 같은 실수이다.
\end{multicols}

\end{document}