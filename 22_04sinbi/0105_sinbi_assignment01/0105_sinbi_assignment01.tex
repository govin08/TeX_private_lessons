\documentclass[a4paper]{oblivoir}
\usepackage{amsmath,amssymb,kotex,mdframed,paralist,tabu}
\usepackage[T1]{fontenc}
%%%Default packages
\usepackage{amsmath,amssymb,amsthm,kotex,tabu,graphicx,pifont}
\usepackage{../kswrapfig}

\usepackage{gensymb} %\degree

%%%More packages
%\usepackage{caption,subcaption}
%\usepackage[perpage]{footmisc}
%
\usepackage[skipabove=10pt,innertopmargin=10pt,nobreak=true]{mdframed}

\usepackage[inline]{enumitem}
\setlist[enumerate,1]{label=(\arabic*)}
\setlist[enumerate,2]{label=(\alph*)}

\usepackage{multicol}
\setlength{\columnsep}{30pt}
\setlength{\columnseprule}{1pt}
%
%\usepackage{forest}
%\usetikzlibrary{shapes.geometric,arrows.meta,calc}
%
%%%defi theo exam prob rema proo
%이 환경들 아래에 문단을 쓸 경우 살짝 들여쓰기가 되므로 \hspace{-.7em}가 필요할 수 있다.

\newcounter{num}
\newcommand{\defi}[1]
{\noindent\refstepcounter{num}\textbf{정의 \arabic{num})} #1\par\noindent}
\newcommand{\theo}[1]
{\noindent\refstepcounter{num}\textbf{정리 \arabic{num})} #1\par\noindent}
\newcommand{\revi}[1]
{\noindent\refstepcounter{num}\textbf{복습 \arabic{num})} #1\par\noindent}
\newcommand{\exam}[1]
{\bigskip\bigskip\noindent\refstepcounter{num}\textbf{예시 \arabic{num})} #1\par\noindent}
\newcommand{\prob}[1]
{\bigskip\bigskip\noindent\refstepcounter{num}\textbf{문제 \arabic{num})} #1\par\noindent}
\newcommand{\rema}[1]
{\bigskip\bigskip\noindent\refstepcounter{num}\textbf{참고 \arabic{num})} #1\par\noindent}
\newcommand{\proo}
{\bigskip\noindent\textsf{증명)}}

\newenvironment{talign}
 {\let\displaystyle\textstyle\align}
 {\endalign}
\newenvironment{talign*}
 {\let\displaystyle\textstyle\csname align*\endcsname}
 {\endalign}
%
%%%Commands

\newcommand{\procedure}[1]{\begin{mdframed}\vspace{#1\textheight}\end{mdframed}}

\newcommand\an[1]{\par\bigskip\noindent\textbf{문제 \ref{#1})}\par\noindent}

\newcommand\ann[2]{\par\bigskip\noindent\textbf{문제 \ref{#1})}\:\:#2\par\medskip\noindent}

\newcommand\ans[1]{\begin{flushright}\textbf{답 : }#1\end{flushright}}

\newcommand\anssec[1]{\bigskip\bigskip\noindent{\large\bfseries#1}}

\newcommand{\pb}[1]%\Phantom + fBox
{\fbox{\phantom{\ensuremath{#1}}}}

\newcommand\ba{\,|\,}

\newcommand\ovv[1]{\ensuremath{\overline{#1}}}
\newcommand\ov[2]{\ensuremath{\overline{#1#2}}}
%
%%%% Settings
%\let\oldsection\section
%
%\renewcommand\section{\clearpage\oldsection}
%
%\let\emph\textsf
%
%\renewcommand{\arraystretch}{1.5}
%
%%%% Footnotes
%\makeatletter
%\def\@fnsymbol#1{\ensuremath{\ifcase#1\or
%*\or **\or ***\or
%\star\or\star\star\or\star\star\star\or
%\dagger\or\dagger\dagger\or\dagger\dagger\dagger
%\else\@ctrerr\fi}}
%
%\renewcommand{\thefootnote}{\fnsymbol{footnote}}
%\makeatother
%
%\makeatletter
%\AtBeginEnvironment{mdframed}{%
%\def\@fnsymbol#1{\ensuremath{\ifcase#1\or
%*\or **\or ***\or
%\star\or\star\star\or\star\star\star\or
%\dagger\or\dagger\dagger\or\dagger\dagger\dagger
%\else\@ctrerr\fi}}%
%}   
%\renewcommand\thempfootnote{\fnsymbol{mpfootnote}}
%\makeatother
%
%%% 객관식 선지
\newcommand\one{\ding{172}}
\newcommand\two{\ding{173}}
\newcommand\three{\ding{174}}
\newcommand\four{\ding{175}}
\newcommand\five{\ding{176}}
\usepackage{tabto,pifont}
%\TabPositions{0.2\textwidth,0.4\textwidth,0.6\textwidth,0.8\textwidth}

\newcommand\taba[5]{\par\noindent
\one\:{#1}
\tabto{0.2\textwidth}\two\:\:{#2}
\tabto{0.4\textwidth}\three\:\:{#3}
\tabto{0.6\textwidth}\four\:\:{#4}
\tabto{0.8\textwidth}\five\:\:{#5}}

\newcommand\tabb[5]{\par\noindent
\one\:{#1}
\tabto{0.33\textwidth}\two\:\:{#2}
\tabto{0.67\textwidth}\three\:\:{#3}\medskip\par\noindent
\four\:\:{#4}
\tabto{0.33\textwidth}\five\:\:{#5}}

\newcommand\tabc[5]{\par\noindent
\one\:{#1}
\tabto{0.5\textwidth}\two\:\:{#2}\medskip\par\noindent
\three\:\:{#3}
\tabto{0.5\textwidth}\four\:\:{#4}\medskip\par\noindent
\five\:\:{#5}}

\newcommand\tabd[5]{\par\noindent
\one\:{#1}\medskip\par\noindent
\two\:\:{#2}\medskip\par\noindent
\three\:\:{#3}\medskip\par\noindent
\four\:\:{#4}\medskip\par\noindent
\five\:\:{#5}}
%
%%%% fonts
%
%\usepackage{fontspec, xunicode, xltxtra}
%\setmainfont[]{은 바탕}
%\setsansfont[]{은 돋움}
%\setmonofont[]{은 바탕}
%\XeTeXlinebreaklocale "ko"
\begin{document}
\begin{center}
\LARGE신비, 미니테스트 1
\end{center}
\begin{center}
날짜 : \today
,\qquad
점수 : \pb{20} / \pb{20}
\end{center}

%
\prob
다음 \(\square\)에 알맞은 수를 써넣으시오.
\\[-10pt]
\begin{enumerate}[(1)]
\item
\(5\times5\times5=5^\square\)
\item
\(5=5^\square\)
\item
\(\frac15=5^\square\)
\item
\(\frac1{25}=5^\square\)
\item
\(\sqrt5 = 5^\square\)
\item
\(5\sqrt5 = 5^\square\)
\item
\(\frac1{\sqrt5}=5^\square\)
\end{enumerate}

%
\prob
다음 \(\square\)에 알맞은 수를 써넣으시오.
\\[-10pt]
\begin{enumerate}[(1)]
\item
\(5^3\times5^4=5^\square\)
\item
\(8^2\times4^2=2^\square\)
\item
\(6^5\div6^2=6^\square\)
\end{enumerate}

%
\prob
다음 식을 간단히 하시오.(단, \(a\neq0\), \(b\neq0\))
\begin{enumerate}[(1)]
\item
\(2^4\times3^{-2}\div6^{-3}\)
\item
\((3^3\times9^{-2})^{-1}\)
\item
\(7^{\frac12}\times7^{-\frac13}\)
\item
\((5^{\sqrt2})^{2\sqrt2}\)
\end{enumerate}

%
\prob
다음 등식을 만족시키는 \(x\)의 값을 각각 구하여라.
\begin{enumerate}[(1)]
\item
\(2^x=8\)
\item
\(2^x=\frac12\)
\item
\(2^x=1\)
\item
\(2^x=2\sqrt2\)
\end{enumerate}

%
\prob
<보기>와 같은 과정을 통해 로그의 값을 계산하여라.
\begin{enumerate}[(1)]
\item
\(\log_381\)
\begin{mdframed}[frametitle=<보기>]
\[x=\log_381\qquad\longrightarrow\qquad 3^x=81\qquad\longrightarrow\qquad x=4\]
\end{mdframed}
\item
\(\log_232\)
\item
\(\log_{10}100\)
\item
\(\log_44\)
\item
\(\log_2\sqrt2\)
\item
\(\log_3\frac13\)
\item
\(\log_3\frac19\)
\item
\(\log_51\)
\end{enumerate}

%
\prob
다음 \(\square\)에 알맞은 수를 써넣으시오
\begin{enumerate}[(1)]
\item
\(\log_35+\log_32=\log_3\square\)
\item
\(\log_230+\log_25=\log_2\square\)
\item
\(2\log_35=\log_3\square\)
\item
\(3\log_{10}5=\log_{10}\square\)
\item
\(\frac12\log_57=\log_5\square\)
\end{enumerate}

%
\prob{다음 식을 간단히 하시오.}
\begin{enumerate}[(1)]
\item
\(\log_69+\log_64\)
\item
\(\log_798-\log_72\)
\item
\(\log_{\frac23}27-\log_{\frac23}8\)
\item
\(\log_3{\frac{\sqrt3}5}+\log_345\)
\item
\(\log_212+\log_26-2\log_23\)
\item
\(\frac12\log_3\frac95+\log_3\sqrt5\)
\end{enumerate}


\newpage
\setcounter{num}{0}

%
\ans
\begin{enumerate}[(1)]
\item
3
\item
1
\item
\(-1\)
\item
\(-2\)
\item
\(\frac12\)
\item
\(\frac32\)
\item
\(-\frac12\)
\end{enumerate}

%
\ans
\begin{enumerate}[(1)]
\item
7
\item
10
\item
3
\end{enumerate}

%
\ans
\begin{enumerate}[(1)]
\item
\(2^7\times3(=384)\)
\item
3
\item
\(7^{\frac16}(=\sqrt[6]7)\)
\item
\(5^4(=625)\)
\end{enumerate}

%
\ans
\begin{enumerate}[(1)]
\item
3
\item
\(-1\)
\item
0
\item
\(\frac32\)
\end{enumerate}

%
\ans
\begin{enumerate}[(1)]
\setcounter{enumi}{1}
\item
5
\item
2
\item
1
\item
\(\frac12\)
\item
\(-1\)
\item
\(-2\)
\item
\(0\)
\end{enumerate}

%
\ans
\begin{enumerate}[(1)]
\item
10
\item
6
\item
25
\item
\(125(=5^3)\)
\item
\(\sqrt7(=7^{\frac12})\)
\end{enumerate}

%
\ans
\begin{enumerate}[(1)]
\item
2
\item
2
\item
\(-3\)
\item
\(\frac52\)
\item
3
\item
1
\end{enumerate}

\end{document}