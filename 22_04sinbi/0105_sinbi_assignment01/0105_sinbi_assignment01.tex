\documentclass[a4paper]{oblivoir}
\usepackage{amsmath,amssymb,kotex,mdframed,paralist,tabu}
\usepackage[T1]{fontenc}
\usepackage{amsmath,amssymb,kotex,paralist,graphicx}
\usepackage{mdframed}
\usepackage{../kswrapfig}
\usepackage{fapapersize}
\usefapapersize{210mm,297mm,20mm,*,20mm,*}
%\pagestyle{empty}
\usepackage{multicol}
\setlength{\columnsep}{30pt}
\setlength{\columnseprule}{1pt}
%\def\columnseprulecolor{\color{blue}}

%%% 객관식 선지

\usepackage{tabto,pifont}
\TabPositions{0.2\textwidth,0.4\textwidth,0.6\textwidth,0.8\textwidth}

\newcommand\one{\ding{172}}
\newcommand\two{\ding{173}}
\newcommand\three{\ding{174}}
\newcommand\four{\ding{175}}
\newcommand\five{\ding{176}}

\newcommand\taba[5]{\par\bigskip\noindent
\one\:{\ensuremath{#1}}
\tab\two\:\:{\ensuremath{#2}}
\tab\three\:\:{\ensuremath{#3}}
\tab\four\:\:{\ensuremath{#4}}
\tab\five\:\:{\ensuremath{#5}}}

\newcommand\tabb[5]{\par\bigskip\noindent
\one\:{\ensuremath{#1}}
\tabto{0.16\textwidth}\two\:\:{\ensuremath{#2}}
\tabto{0.33\textwidth}\three\:\:{\ensuremath{#3}}\medskip\par\noindent
\four\:\:{\ensuremath{#4}}.
\tabto{0.16\textwidth}\five\:\:{\ensuremath{#5}}}

\newcommand\tabc[5]{\par\bigskip\noindent
\one\:{\ensuremath{#1}}
\tabto{0.25\textwidth}\two\:\:{\ensuremath{#2}}\medskip\par\noindent
\three\:\:{\ensuremath{#3}}
\tabto{0.25\textwidth}\four\:\:{\ensuremath{#4}}\medskip\par\noindent
\five\:\:{\ensuremath{#5}}}

\newcommand\tabd[5]{\par\bigskip\noindent
\one\:{#1}\medskip\par\noindent
\two\:\:{#2}\medskip\par\noindent
\three\:\:{#3}\medskip\par\noindent
\four\:\:{#4}\medskip\par\noindent
\five\:\:{#5}}

%%% Counters
\newcounter{num}

%%% Commands
\newcommand{\prob}[1]
{\vs\par\noindent\refstepcounter{num}\textbf{문제 \arabic{num})}\label{#1}\par\noindent}

\newcommand\vs[1]{\vspace{70pt}}

\newcommand\inc[1]{\begin{center}\includegraphics[width=0.95\columnwidth]{#1}\end{center}}

\newcommand\pb[1]{\ensuremath{\fbox{\phantom{#1}}}}

\newcommand\ba{\ensuremath{\:|\:}}

\newcommand\an[2]{\par\bigskip\noindent\textbf{문제 \ref{#1})} #2\\}

\newcommand\ans[1]{\begin{flushright}\textbf{답 : }#1\end{flushright}}

\renewcommand{\arraystretch}{1.5}

%%% Meta Commands
\let\oldsection\section
\renewcommand\section{\clearpage\oldsection}
\let\emph\textsf

\begin{document}
\begin{center}
\LARGE신비, 미니테스트 1
\end{center}
\begin{center}
날짜 : \today
,\qquad
점수 : \pb{20} / \pb{20}
\end{center}

%
\prob
다음 \(\square\)에 알맞은 수를 써넣으시오.
\\[-10pt]
\begin{enumerate}[(1)]
\item
\(5\times5\times5=5^\square\)
\item
\(5=5^\square\)
\item
\(\frac15=5^\square\)
\item
\(\frac1{25}=5^\square\)
\item
\(\sqrt5 = 5^\square\)
\item
\(5\sqrt5 = 5^\square\)
\item
\(\frac1{\sqrt5}=5^\square\)
\end{enumerate}

%
\prob
다음 \(\square\)에 알맞은 수를 써넣으시오.
\\[-10pt]
\begin{enumerate}[(1)]
\item
\(5^3\times5^4=5^\square\)
\item
\(8^2\times4^2=2^\square\)
\item
\(6^5\div6^2=6^\square\)
\end{enumerate}

%
\prob
다음 식을 간단히 하시오.(단, \(a\neq0\), \(b\neq0\))
\begin{enumerate}[(1)]
\item
\(2^4\times3^{-2}\div6^{-3}\)
\item
\((3^3\times9^{-2})^{-1}\)
\item
\(7^{\frac12}\times7^{-\frac13}\)
\item
\((5^{\sqrt2})^{2\sqrt2}\)
\end{enumerate}

%
\prob
다음 등식을 만족시키는 \(x\)의 값을 각각 구하여라.
\begin{enumerate}[(1)]
\item
\(2^x=8\)
\item
\(2^x=\frac12\)
\item
\(2^x=1\)
\item
\(2^x=2\sqrt2\)
\end{enumerate}

%
\prob
<보기>와 같은 과정을 통해 로그의 값을 계산하여라.
\begin{enumerate}[(1)]
\item
\(\log_381\)
\begin{mdframed}[frametitle=<보기>]
\[x=\log_381\qquad\longrightarrow\qquad 3^x=81\qquad\longrightarrow\qquad x=4\]
\end{mdframed}
\item
\(\log_232\)
\item
\(\log_{10}100\)
\item
\(\log_44\)
\item
\(\log_2\sqrt2\)
\item
\(\log_3\frac13\)
\item
\(\log_3\frac19\)
\item
\(\log_51\)
\end{enumerate}

%
\prob
다음 \(\square\)에 알맞은 수를 써넣으시오
\begin{enumerate}[(1)]
\item
\(\log_35+\log_32=\log_3\square\)
\item
\(\log_230+\log_25=\log_2\square\)
\item
\(2\log_35=\log_3\square\)
\item
\(3\log_{10}5=\log_{10}\square\)
\item
\(\frac12\log_57=\log_5\square\)
\end{enumerate}

%
\prob{다음 식을 간단히 하시오.}
\begin{enumerate}[(1)]
\item
\(\log_69+\log_64\)
\item
\(\log_798-\log_72\)
\item
\(\log_{\frac23}27-\log_{\frac23}8\)
\item
\(\log_3{\frac{\sqrt3}5}+\log_345\)
\item
\(\log_212+\log_26-2\log_23\)
\item
\(\frac12\log_3\frac95+\log_3\sqrt5\)
\end{enumerate}


\newpage
\setcounter{num}{0}

%
\ans
\begin{enumerate}[(1)]
\item
3
\item
1
\item
\(-1\)
\item
\(-2\)
\item
\(\frac12\)
\item
\(\frac32\)
\item
\(-\frac12\)
\end{enumerate}

%
\ans
\begin{enumerate}[(1)]
\item
7
\item
10
\item
3
\end{enumerate}

%
\ans
\begin{enumerate}[(1)]
\item
\(2^7\times3(=384)\)
\item
3
\item
\(7^{\frac16}(=\sqrt[6]7)\)
\item
\(5^4(=625)\)
\end{enumerate}

%
\ans
\begin{enumerate}[(1)]
\item
3
\item
\(-1\)
\item
0
\item
\(\frac32\)
\end{enumerate}

%
\ans
\begin{enumerate}[(1)]
\setcounter{enumi}{1}
\item
5
\item
2
\item
1
\item
\(\frac12\)
\item
\(-1\)
\item
\(-2\)
\item
\(0\)
\end{enumerate}

%
\ans
\begin{enumerate}[(1)]
\item
10
\item
6
\item
25
\item
\(125(=5^3)\)
\item
\(\sqrt7(=7^{\frac12})\)
\end{enumerate}

%
\ans
\begin{enumerate}[(1)]
\item
2
\item
2
\item
\(-3\)
\item
\(\frac52\)
\item
3
\item
1
\end{enumerate}

\end{document}