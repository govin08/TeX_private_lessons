\documentclass[a4paper,twocolumn]{article}
\usepackage{amsmath,amssymb,kotex,mdframed,paralist,tabu}
\usepackage[T1]{fontenc}
\usepackage[a4paper, total={190mm, 270mm}]{geometry}
%%%%Default packages
\usepackage{amsmath,amssymb,amsthm,kotex,tabu,graphicx,pifont}
\usepackage{../kswrapfig}

\usepackage{gensymb} %\degree

%%%More packages
%\usepackage{caption,subcaption}
%\usepackage[perpage]{footmisc}
%
\usepackage[skipabove=10pt,innertopmargin=10pt,nobreak=true]{mdframed}

\usepackage[inline]{enumitem}
\setlist[enumerate,1]{label=(\arabic*)}
\setlist[enumerate,2]{label=(\alph*)}

\usepackage{multicol}
\setlength{\columnsep}{30pt}
\setlength{\columnseprule}{1pt}
%
%\usepackage{forest}
%\usetikzlibrary{shapes.geometric,arrows.meta,calc}
%
%%%defi theo exam prob rema proo
%이 환경들 아래에 문단을 쓸 경우 살짝 들여쓰기가 되므로 \hspace{-.7em}가 필요할 수 있다.

\newcounter{num}
\newcommand{\defi}[1]
{\noindent\refstepcounter{num}\textbf{정의 \arabic{num})} #1\par\noindent}
\newcommand{\theo}[1]
{\noindent\refstepcounter{num}\textbf{정리 \arabic{num})} #1\par\noindent}
\newcommand{\revi}[1]
{\noindent\refstepcounter{num}\textbf{복습 \arabic{num})} #1\par\noindent}
\newcommand{\exam}[1]
{\bigskip\bigskip\noindent\refstepcounter{num}\textbf{예시 \arabic{num})} #1\par\noindent}
\newcommand{\prob}[1]
{\bigskip\bigskip\noindent\refstepcounter{num}\textbf{문제 \arabic{num})} #1\par\noindent}
\newcommand{\rema}[1]
{\bigskip\bigskip\noindent\refstepcounter{num}\textbf{참고 \arabic{num})} #1\par\noindent}
\newcommand{\proo}
{\bigskip\noindent\textsf{증명)}}

\newenvironment{talign}
 {\let\displaystyle\textstyle\align}
 {\endalign}
\newenvironment{talign*}
 {\let\displaystyle\textstyle\csname align*\endcsname}
 {\endalign}
%
%%%Commands

\newcommand{\procedure}[1]{\begin{mdframed}\vspace{#1\textheight}\end{mdframed}}

\newcommand\an[1]{\par\bigskip\noindent\textbf{문제 \ref{#1})}\par\noindent}

\newcommand\ann[2]{\par\bigskip\noindent\textbf{문제 \ref{#1})}\:\:#2\par\medskip\noindent}

\newcommand\ans[1]{\begin{flushright}\textbf{답 : }#1\end{flushright}}

\newcommand\anssec[1]{\bigskip\bigskip\noindent{\large\bfseries#1}}

\newcommand{\pb}[1]%\Phantom + fBox
{\fbox{\phantom{\ensuremath{#1}}}}

\newcommand\ba{\,|\,}

\newcommand\ovv[1]{\ensuremath{\overline{#1}}}
\newcommand\ov[2]{\ensuremath{\overline{#1#2}}}
%
%%%% Settings
%\let\oldsection\section
%
%\renewcommand\section{\clearpage\oldsection}
%
%\let\emph\textsf
%
%\renewcommand{\arraystretch}{1.5}
%
%%%% Footnotes
%\makeatletter
%\def\@fnsymbol#1{\ensuremath{\ifcase#1\or
%*\or **\or ***\or
%\star\or\star\star\or\star\star\star\or
%\dagger\or\dagger\dagger\or\dagger\dagger\dagger
%\else\@ctrerr\fi}}
%
%\renewcommand{\thefootnote}{\fnsymbol{footnote}}
%\makeatother
%
%\makeatletter
%\AtBeginEnvironment{mdframed}{%
%\def\@fnsymbol#1{\ensuremath{\ifcase#1\or
%*\or **\or ***\or
%\star\or\star\star\or\star\star\star\or
%\dagger\or\dagger\dagger\or\dagger\dagger\dagger
%\else\@ctrerr\fi}}%
%}   
%\renewcommand\thempfootnote{\fnsymbol{mpfootnote}}
%\makeatother
%
%%% 객관식 선지
\newcommand\one{\ding{172}}
\newcommand\two{\ding{173}}
\newcommand\three{\ding{174}}
\newcommand\four{\ding{175}}
\newcommand\five{\ding{176}}
\usepackage{tabto,pifont}
%\TabPositions{0.2\textwidth,0.4\textwidth,0.6\textwidth,0.8\textwidth}

\newcommand\taba[5]{\par\noindent
\one\:{#1}
\tabto{0.2\textwidth}\two\:\:{#2}
\tabto{0.4\textwidth}\three\:\:{#3}
\tabto{0.6\textwidth}\four\:\:{#4}
\tabto{0.8\textwidth}\five\:\:{#5}}

\newcommand\tabb[5]{\par\noindent
\one\:{#1}
\tabto{0.33\textwidth}\two\:\:{#2}
\tabto{0.67\textwidth}\three\:\:{#3}\medskip\par\noindent
\four\:\:{#4}
\tabto{0.33\textwidth}\five\:\:{#5}}

\newcommand\tabc[5]{\par\noindent
\one\:{#1}
\tabto{0.5\textwidth}\two\:\:{#2}\medskip\par\noindent
\three\:\:{#3}
\tabto{0.5\textwidth}\four\:\:{#4}\medskip\par\noindent
\five\:\:{#5}}

\newcommand\tabd[5]{\par\noindent
\one\:{#1}\medskip\par\noindent
\two\:\:{#2}\medskip\par\noindent
\three\:\:{#3}\medskip\par\noindent
\four\:\:{#4}\medskip\par\noindent
\five\:\:{#5}}
%
%%%% fonts
%
%\usepackage{fontspec, xunicode, xltxtra}
%\setmainfont[]{은 바탕}
%\setsansfont[]{은 돋움}
%\setmonofont[]{은 바탕}
%\XeTeXlinebreaklocale "ko"
\begin{document}
\begin{center}
\LARGE신비, 지수와 로그 (간단히)
\end{center}

%%
\section{지수}
\(a^x\)
와 같이 생긴 것을 \fbox{거듭제곱}이라고 부른다.
이때 \(a\)를 \fbox{밑}, \(x\)를 \fbox{지수}라고 부른다.

%
\subsection{자연수 지수}
\(2\times 2\times 2\)를 \(2^3\)으로 표현한다.
\begin{itemize}
\item
\(3^3=\square\)
\item
\(2^4=\square\)
\item
\(7^1=\square\)
\end{itemize}

%
\subsection{정수 지수}
\(\frac15\)를 \(5^{-1}\)로 표시한다.
따라서 \(\frac1{25}\)는
\[\frac1{25}=\left(\frac15\right)^2=(5^{-1})^2=5^{-2}\]
이다.
\begin{itemize}
\item
\(3^3=\square\)
\item
\(3^2=\square\)
\item
\(3^1=\square\)
\item
\(3^0=\square\)
\item
\(3^{-1}=\square\)
\item
\(3^{-2}=\square\)
\item
\(3^{-3}=\square\)
\end{itemize}

%
\subsection{유리수 지수}
\(\sqrt5\)을 \(5^{\frac12}\)로 표시한다.
마찬가지로, \(\sqrt[3]5\)는 \(5^{\frac13}\)로 표시한다.
\begin{itemize}
\item
\(2^{\frac12}=\square\)
\item
\(6^{\frac12}=\square\)
\item
\(7^{\frac13}=\square\)
\item
\(10^{\frac15}=\square\)
\item
\(3^{\frac32}=\square\)
\end{itemize}

%
\subsection{지수 법칙}
\(a>0\), \(b>0\)이고 \(x\), \(y\)가 실수일 때,
\begin{enumerate}[(1)]
\item
\(a^x\times a^y=a^{x+y}\)
\item
\(a^x\div a^y=a^{x-y}\)
\item
\((a^x)^y=a^{xy}\)
\item
\((ab)^x=a^xb^x\)
\end{enumerate}

\begin{itemize}
\item
\(5^3\times5^2=(5\times5\times5)\times(5\times5)=5^5\)
\item
\(5^3\div5^2=\frac{5\times5\times5}{5\times5}=5=5^1\)
\item
\((5^3)^2=(5\times5\times5)^2=(5\times5\times5)\times(5\times5\times5)=5^6\)
\item
\((2\times3)^2=(2\times3)\times(2\times3)=2^2\times3^2\)
\end{itemize}

%%
\section{로그}

%
\subsection{로그의 정의}
지수에서의 문제가
\[2^3=\square\]
인 \(\square\)를 구하는 문제였다면, 로그에서의 문제는
\[2^\square=8\]
인 \(\square\)를 구하는 문제이다.
(즉, 지수와 로그는 서로 반대이다)
이러한 \(\square\)의 값을 \(\log_28\)이라고 쓴다.
따라서 \(\log_28=3\)이다.
\begin{itemize}
\item
\(\log_28\qquad\longrightarrow\qquad2^\square=8\qquad\longrightarrow\qquad\square=3\)\\[10pt]
\(\therefore\log_28=3\)
\item
\(\log_39\qquad\longrightarrow\qquad3^\square=9\qquad\longrightarrow\qquad\square=2\)\\[10pt]
\(\therefore\log_39=2\)
\item
\(\log_2\sqrt2\qquad\longrightarrow\qquad2^\square=\sqrt2\qquad\longrightarrow\qquad\square=\frac12\)\\[10pt]
\(\therefore\log_2\sqrt2=\frac12\)
\item
\(\log_216\qquad\longrightarrow\qquad\phantom{2^\square=8}\qquad\longrightarrow\qquad\square=\)\\[10pt]
\(\therefore\log_216=\)
\item
\(\log_3\frac13\qquad\longrightarrow\qquad\phantom{2^\square=8}\qquad\longrightarrow\qquad\square=\)\\[10pt]
\(\therefore\log_3\frac13=\)
\item
\(\log_51\qquad\longrightarrow\qquad\phantom{2^\square=8}\qquad\longrightarrow\qquad\square=\)\\[10pt]
\(\therefore\log_51=\)
\end{itemize}

\bigskip
\begin{itemize}
\item
\(\log_24=\square\)
\item
\(\log_61=\square\)
\item
\(\log_5\frac1{25}=\square\)
\item
\(\log_416=\square\)
\end{itemize}

방금까지는 계산이 잘 되는 로그의 값들만을 나열해보았다.
하지만, 대부분의 경우에 로그의 값은 계산해내기 어렵다.
\begin{itemize}
\item
\(\log_27\qquad\longrightarrow\qquad2^\square=7\qquad\longrightarrow\qquad\square=?\)
\end{itemize}
이때의 \(\square\)값은 \(2<\square<3\)일 것이다.
왜냐하면 \(2^2<7<2^3\)이기 때문이다.
하지만 정확한 \(\log_27\)의 값은 계산기를 사용하지 않는 이상 구할 수 없다.
다시 말해, \(\log_27\)이란 \(2^\square=7\)를 만족시키는 값을 나타내는 표현법이다.

%
\subsection{로그의 (정확한) 정의}
정리하면
\[a^\square=N\]
를 만족시키는 값 \(\square\)를 \(\log_aN\)이라고 쓴다.
이때 \(a\)를 \fbox{밑}, \(N\)을 \fbox{진수}라고 부른다.

한편, 실수 지수가 정의되려면 \(a>0\)이어야 한다(따라서 \(N>0\)이다).
또한 \(a=1\)이면
\[1^\square=N\]
이 되어 이 문제가 의미 없어진다.
따라서 로그값 \(\log_aN\)을 정의할 때, 다음 두 가지 조건을 만족해야 한다.
\begin{itemize}
\item
밑조건 : \(a>0\), \(a\neq1\)
\item
진수조건 : \(N>0\)
\end{itemize}

%
\subsection{로그의 기본적인 성질}
\(a>0\), \(a\neq1\), \(M>0\), \(N>0\)일 때,
\begin{enumerate}[(1)]
\item
\(\log_a1=0,\quad\log_aa=1\)
\item
\(\log_aM+\log_aN=\log_aMN\)
\item
\(\log_aM-\log_aN=\log_a{\frac MN}\)
\item
\(\log_a{M^k}=k\log_aM\)\quad(단, \(k\)는 실수)
\end{enumerate}

\begin{itemize}
\item
\(3^0=1\)이고 \((\frac15)^0=1\)이다.
따라서
\begin{center}
\fbox{\(\log_31=0\)}
\qquad
\fbox{\(\log_{\frac15}1=0\)}
\end{center}
한편, \(3^1=3\)이고 \((\frac15)^1=\frac15\)이다.
따라서
\begin{center}
\fbox{\(\log_33=1\)}
\qquad
\fbox{\(\log_{\frac15}{\frac15}=1\)}
\end{center}
\item
\(\log_28=3\), \(\log_24=2\), \(\log_232=5\)이다.
\(3+2=5\)에서
\begin{mdframed}[rightmargin=.1\textwidth,leftmargin=.1\textwidth]
\centering
\(\log_28+\log_24=\log_232\)
\end{mdframed}
\item
또, \(\log_22=1\)이므로 \(3-2=1\)에서
\begin{mdframed}[rightmargin=.1\textwidth,leftmargin=.1\textwidth]
\centering
\(\log_28-\log_24=\log_22\)
\end{mdframed}
\item
\(\log_232=5\), \(\log_22=1\)이므로
\begin{mdframed}[rightmargin=.1\textwidth,leftmargin=.1\textwidth]
\centering
\(\log_2{2^5}=5\log_22\)
\end{mdframed}
\end{itemize}

\par\bigskip\noindent
\textbf{예제)} 다음 식을 간단히 하시오.
\begin{itemize}
\item
\(\log_612+\log_63\)
\item
\(\log_3\sqrt{27}\)
\item
\(\log_23-2\log_2\sqrt6\)
\item
\(\log_{10}\sqrt5+\frac12\log_{10}2\)
\end{itemize}

\noindent
\begin{mdframed}[frametitle=풀이]
\begin{itemize}
\item
\(\log_612+\log_63
\stackrel{(2)}=
\log_636=2\)
\item
\(\log_3\sqrt{27}=\log_3{3^{\frac32}}
\stackrel{(4)}=
\frac32\log_33=\frac32\)
\item
\(\log_23-2\log_2\sqrt6
\stackrel{(4)}=
\log_23-\log_26
\stackrel{(2)}=
\log_2\frac12=-1\)
\item
\(\log_{10}\sqrt5+\frac12\log_{10}2
\stackrel{(4)}=
\log_{10}\sqrt5+\log_{10}\sqrt2
\stackrel{(2)}=
\log_{10}\sqrt{10}=\frac12\)
\end{itemize}
\end{mdframed}

\textbf{문제)} 다음 식을 간단히 하시오.
\begin{itemize}\label{prop5}
\item
\(\log_3\sqrt[3]{81}\)
\item
\(\log_798-\log_72\)
\item
\(\log_{\frac23}27-\log_{\frac23}8\)
\item
\(\log_3{\frac{\sqrt3}5}+\log_345\)
\item
\(\log_212+\log_26-2\log_23\)
\item
\(\frac12\log_3\frac95+\log_3\sqrt5\)
\end{itemize}

%
\subsection{로그의 추가적인 성질}
\begin{enumerate}[(1)]
\setcounter{enumi}{4}
\item
\(a>0\), \(a\neq1\), \(b>0\), \(c>0\), \(c\neq1\)일 때,
\[\phantom{\qquad(\text{밑의 변환 공식})}\log_ab=\frac{\log_cb}{\log_ca}\qquad(\text{밑의 변환 공식})\]
\item
\(a>0\), \(a\neq1\), \(b>0\), \(b\neq1\)일 때,
\[\log_ab=\frac1{\log_ba}\]
\item
\(a>0\), \(a\neq1\), \(b>0\)이고 \(m\), \(n\)이 실수일 때,
\[\log_{a^n}{b^m}=\frac mn \log_ab\]
\item
\(\displaystyle a^{\log_ab}=b\)
\item
\(\displaystyle a^{\log_bc}=c^{\log_ba}\)

\end{enumerate}


\end{document}