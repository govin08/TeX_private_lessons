\documentclass[a4paper]{oblivoir}
\usepackage{amsmath,amssymb,kotex,paralist,graphicx}
\usepackage{mdframed}
\usepackage{../kswrapfig}
\usepackage{fapapersize}
\usefapapersize{210mm,297mm,20mm,*,20mm,*}
%\pagestyle{empty}
\usepackage{multicol}
\setlength{\columnsep}{30pt}
\setlength{\columnseprule}{1pt}
%\def\columnseprulecolor{\color{blue}}

%%% 객관식 선지

\usepackage{tabto,pifont}
\TabPositions{0.2\textwidth,0.4\textwidth,0.6\textwidth,0.8\textwidth}

\newcommand\one{\ding{172}}
\newcommand\two{\ding{173}}
\newcommand\three{\ding{174}}
\newcommand\four{\ding{175}}
\newcommand\five{\ding{176}}

\newcommand\taba[5]{\par\bigskip\noindent
\one\:{\ensuremath{#1}}
\tab\two\:\:{\ensuremath{#2}}
\tab\three\:\:{\ensuremath{#3}}
\tab\four\:\:{\ensuremath{#4}}
\tab\five\:\:{\ensuremath{#5}}}

\newcommand\tabb[5]{\par\bigskip\noindent
\one\:{\ensuremath{#1}}
\tabto{0.16\textwidth}\two\:\:{\ensuremath{#2}}
\tabto{0.33\textwidth}\three\:\:{\ensuremath{#3}}\medskip\par\noindent
\four\:\:{\ensuremath{#4}}.
\tabto{0.16\textwidth}\five\:\:{\ensuremath{#5}}}

\newcommand\tabc[5]{\par\bigskip\noindent
\one\:{\ensuremath{#1}}
\tabto{0.25\textwidth}\two\:\:{\ensuremath{#2}}\medskip\par\noindent
\three\:\:{\ensuremath{#3}}
\tabto{0.25\textwidth}\four\:\:{\ensuremath{#4}}\medskip\par\noindent
\five\:\:{\ensuremath{#5}}}

\newcommand\tabd[5]{\par\bigskip\noindent
\one\:{#1}\medskip\par\noindent
\two\:\:{#2}\medskip\par\noindent
\three\:\:{#3}\medskip\par\noindent
\four\:\:{#4}\medskip\par\noindent
\five\:\:{#5}}

%%% Counters
\newcounter{num}

%%% Commands
\newcommand{\prob}[1]
{\vs\par\noindent\refstepcounter{num}\textbf{문제 \arabic{num})}\label{#1}\par\noindent}

\newcommand\vs[1]{\vspace{70pt}}

\newcommand\inc[1]{\begin{center}\includegraphics[width=0.95\columnwidth]{#1}\end{center}}

\newcommand\pb[1]{\ensuremath{\fbox{\phantom{#1}}}}

\newcommand\ba{\ensuremath{\:|\:}}

\newcommand\an[2]{\par\bigskip\noindent\textbf{문제 \ref{#1})} #2\\}

\newcommand\ans[1]{\begin{flushright}\textbf{답 : }#1\end{flushright}}

\renewcommand{\arraystretch}{1.5}

%%% Meta Commands
\let\oldsection\section
\renewcommand\section{\clearpage\oldsection}
\let\emph\textsf

\begin{document}
\begin{center}
\LARGE 신비, 미니테스트 4
\end{center}
\begin{center}
날짜 : \today
,\qquad
점수 : \pb{20} / \pb{20}
\end{center}

%
\begin{Exercise}
1, 2, 3, 5, 9의 평균, 분산, 표준편차를 구하여라.
\end{Exercise}

\begin{Answer}
평균 : 4, 분산 : 8, 표준편차 : \(2\sqrt2\)
\end{Answer}

%
\begin{Exercise}
3, 5, 9, 15의 평균, 분산, 표준편차를 구하여라.
\end{Exercise}

\begin{Answer}
평균 : 8, 분산 : 21, 표준편차 : \(\sqrt{21}\)
\end{Answer}

%
\begin{Exercise}
\begin{enumerate}[(1)]
\item
1, 2, 3, 4, 5의 평균, 분산, 표준편차를 구하여라.
\item
2, 3, 4, 5, 6의 평균, 분산, 표준편차를 구하여라.
\item
23, 24, 25 ,26 27의 평균, 분산, 표준편차를 구하여라.
\item
$-2$, $-1$, 0, 1, 2의 평균, 분산, 표준편차를 구하여라.
\end{enumerate}
\end{Exercise}

\begin{Answer}
\begin{enumerate}[(1)]
\item
평균 : 3, 분산 : 2, 표준편차 : \(\sqrt2\)
\item
평균 : 4, 분산 : 2, 표준편차 : \(\sqrt2\)
\item
평균 : 25, 분산 : 2, 표준편차 : \(\sqrt2\)
\item
평균 : 0, 분산 : 2, 표준편차 : \(\sqrt2\)
\end{enumerate}
\end{Answer}

%
\begin{Exercise}
\begin{enumerate}[(1)]
\item
1, 2, 3, 4, 5의 평균, 분산, 표준편차를 구하여라.
\item
2, 4, 6, 8, 10의 평균, 분산, 표준편차를 구하여라.
\item
\(\frac12\), 1, \(\frac32\), 2, \(\frac52\)의 평균, 분산, 표준편차를 구하여라.
\end{enumerate}
\end{Exercise}

\begin{Answer}
\begin{enumerate}[(1)]
\item
평균 : 3, 분산 : 2, 표준편차 : \(\sqrt2\)
\item
평균 : 6, 분산 : 8, 표준편차 : \(2\sqrt2\)
\item
평균 : \(\frac32\), 분산 : \(\frac12\), 표준편차 : \(\frac{\sqrt2}2\)
\end{enumerate}
\end{Answer}

\newpage
\shipoutAnswer
\end{document}