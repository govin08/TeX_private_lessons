\documentclass[a4paper]{oblivoir}
\usepackage{amsmath,amssymb,kotex,mdframed,paralist,tabu}
\usepackage[T1]{fontenc}
\usepackage{amsmath,amssymb,kotex,paralist,graphicx}
\usepackage{mdframed}
\usepackage{../kswrapfig}
\usepackage{fapapersize}
\usefapapersize{210mm,297mm,20mm,*,20mm,*}
%\pagestyle{empty}
\usepackage{multicol}
\setlength{\columnsep}{30pt}
\setlength{\columnseprule}{1pt}
%\def\columnseprulecolor{\color{blue}}

%%% 객관식 선지

\usepackage{tabto,pifont}
\TabPositions{0.2\textwidth,0.4\textwidth,0.6\textwidth,0.8\textwidth}

\newcommand\one{\ding{172}}
\newcommand\two{\ding{173}}
\newcommand\three{\ding{174}}
\newcommand\four{\ding{175}}
\newcommand\five{\ding{176}}

\newcommand\taba[5]{\par\bigskip\noindent
\one\:{\ensuremath{#1}}
\tab\two\:\:{\ensuremath{#2}}
\tab\three\:\:{\ensuremath{#3}}
\tab\four\:\:{\ensuremath{#4}}
\tab\five\:\:{\ensuremath{#5}}}

\newcommand\tabb[5]{\par\bigskip\noindent
\one\:{\ensuremath{#1}}
\tabto{0.16\textwidth}\two\:\:{\ensuremath{#2}}
\tabto{0.33\textwidth}\three\:\:{\ensuremath{#3}}\medskip\par\noindent
\four\:\:{\ensuremath{#4}}.
\tabto{0.16\textwidth}\five\:\:{\ensuremath{#5}}}

\newcommand\tabc[5]{\par\bigskip\noindent
\one\:{\ensuremath{#1}}
\tabto{0.25\textwidth}\two\:\:{\ensuremath{#2}}\medskip\par\noindent
\three\:\:{\ensuremath{#3}}
\tabto{0.25\textwidth}\four\:\:{\ensuremath{#4}}\medskip\par\noindent
\five\:\:{\ensuremath{#5}}}

\newcommand\tabd[5]{\par\bigskip\noindent
\one\:{#1}\medskip\par\noindent
\two\:\:{#2}\medskip\par\noindent
\three\:\:{#3}\medskip\par\noindent
\four\:\:{#4}\medskip\par\noindent
\five\:\:{#5}}

%%% Counters
\newcounter{num}

%%% Commands
\newcommand{\prob}[1]
{\vs\par\noindent\refstepcounter{num}\textbf{문제 \arabic{num})}\label{#1}\par\noindent}

\newcommand\vs[1]{\vspace{70pt}}

\newcommand\inc[1]{\begin{center}\includegraphics[width=0.95\columnwidth]{#1}\end{center}}

\newcommand\pb[1]{\ensuremath{\fbox{\phantom{#1}}}}

\newcommand\ba{\ensuremath{\:|\:}}

\newcommand\an[2]{\par\bigskip\noindent\textbf{문제 \ref{#1})} #2\\}

\newcommand\ans[1]{\begin{flushright}\textbf{답 : }#1\end{flushright}}

\renewcommand{\arraystretch}{1.5}

%%% Meta Commands
\let\oldsection\section
\renewcommand\section{\clearpage\oldsection}
\let\emph\textsf

\begin{document}
\begin{center}
\LARGE신비, 미니테스트 2
\end{center}
\begin{center}
날짜 : \today
,\qquad
점수 : \pb{20} / \pb{20}
\end{center}

%%
\section{로그}

%
\prob
<보기>를 참고하여 다음 빈 칸을 채워라.
\begin{mdframed}[frametitle=<보기>]
\(\sqrt3\)은 "제곱해서 3이 되는 양수"이다.
즉, \(x^2=3\)을 만족시키는 양수이다.
따라서 \((\sqrt3)^2=3\)이다.
또한
\[(\sqrt3)^3=\sqrt3\times\sqrt3\times\sqrt3=(\sqrt3)^2\times\sqrt3=3\times\sqrt3=3\sqrt3\]
이다.
\end{mdframed}
\begin{enumerate}[(1)]
\item
\(\sqrt5\)은 "\pb{제곱해서 5이 되는 양수}"이다.
즉, \pb{\(x^2=5\)}을 만족시키는 양수이다.
따라서 \((\sqrt5)^2=\pb{5}\)이다.
또한
\((\sqrt5)^3=\pb{$5\sqrt5$}\)
이다.
\item
\(\sqrt2\)은 "\pb{제곱해서 2이 되는 양수}"이다.
즉, \pb{\(x^2=2\)}을 만족시키는 양수이다.
따라서 \((\sqrt2)^2=\pb{2}\)이다.
또한
\((\sqrt2)^5=\pb{$4\sqrt2$}\)
이다.
\end{enumerate}

%
\prob
다음 \(\square\)에 알맞은 수를 써넣으시오.
\\[-10pt]
\begin{enumerate}[(1)]
\item
\(5^3=\square\)
\item
\(5^1=\square\)
\item
\(5^0=\square\)
\item
\(5^{-1}=\square\)
\item
\(5^{-3}=\square\)
\item
\(5^{\frac12}=\square\)
\item
\(5^{\frac13}=\square\)
\item
\(5^{\frac32}=\square\)
\item
\(5^{\frac52}=\square\)
\end{enumerate}

%
\prob
<보기>와 같은 과정을 통해 로그의 값을 계산하여라.
\begin{enumerate}[(1)]
\item
\(\log_381=4\)
\begin{mdframed}[frametitle=<보기>]
\[x=\log_381\qquad\longrightarrow\qquad 3^x=81\qquad\longrightarrow\qquad x=4\]
\end{mdframed}
\item
\(\log_327\)
\item
\(\log_264\)
\item
\(\log_{10}\frac1{10}\)
\item
\(\log_55\)
\item
\(\log_5\frac15\)
\item
\(\log_39\sqrt3\)
\item
\(\log_22\sqrt2\)
\item
\(\log_7\sqrt7\)
\item
\(\log_42\)
\item
\(\log_{16}2\)
\item
\(\log_51\)
\item
\(\log_{10}1\)
\end{enumerate}

%
\prob
다음 등식을 만족하는 \(N\)의 값을 구하시오.
\begin{enumerate}[(1)]
\item
\(\log_2N=4\)
\item
\(\log_2N=3\)
\item
\(\log_5N=-1\)
\item
\(\log_5N=-2\)
\item
\(\log_{\frac12}N=3\)
\item
\(\log_{\frac13}N=2\)
\item
\(\log_5N=1\)
\item
\(\log_4N=1\)
\item
\(\log_3N=0\)
\item
\(\log_{10}N=0\)
\end{enumerate}

%
\prob
다음 \(\square\)에 알맞은 수를 써넣으시오
\begin{enumerate}[(1)]
\item
\(\log_510+\log_53=\log_3\square\)
\item
\(\log_245-\log_23=\log_2\square\)
\item
\(4\log_62=\log_6\square\)
\item
\(2\log_{10}3=\log_{10}\square\)
\item
\(\log_58=\square\log_52\)
\item
\(\frac12\log_57=\log_5\square\)
\end{enumerate}

%
\prob
다음 값을 계산하시오.
\begin{enumerate}[(1)]
\item
\(\log_{10}2+\log_{10}5\)
\item
\(\log_510+\log_5\frac{25}2\)
\item
\(\log_260-\log_215\)
\item
\(\log_575-\log_53\)
\item
\(\log_{\frac12}12-\log_{\frac12}3\)
\item
\(\log_{\frac13}21-\log_{\frac13}7\)
\item
\(\log_{\frac32}9-\log_{\frac32}4\)
\item
\(\log_2{\frac{\sqrt2}7}+\log_214\)
\item
\(\log_32+\log_3\frac{9\sqrt3}2\)
\item
\(\log_2{20}+\log_2\frac65-\log_23\)
\item
\(\log_330-\log_35+\log_3\frac{27}2\)
\item
\(\log_510+\log_52-2\log_2\frac{\sqrt5}2\)
\item
\(\frac12\log_2\frac83+\log_2\sqrt3\)
\item
\(\frac12\log_560-\log_52\sqrt3\)
\item
\(\log_36-\frac12\log_312\)
\item
\(\frac12\log_3\frac95+\log_3\sqrt5\)
\end{enumerate}

%
\prob
다음 빈칸에 알맞은 말을 써넣으시오.
\begin{mdframed}
\(\log_aN\)은
\(a^x=N\)을 만족시키는 실수 \(x\)값을 의미한다.
예를 들어, \(\log_381\)은
\pb{\(a^x=N\)}을 만족시키는 실수 \(x\)이므로 \(x=\square\)이다.
따라서 \(\log_381=\square\)이다.

이때, \(a\)는 \pb{밑}이라고 부르고 \(N\)은 \pb{진수}라고 부른다.
또한 \(a\)는 \pb{\(a>0\), \(a\neq1\)}이어야 하고, \(N\)은 \pb{\(N>0\)}이어야 한다.
\end{mdframed}

%%
\section{삼각함수}
%
\prob
\(\pi=180\degree\)임을 활용하여 다음 표를 완성하여라.\\[10pt]
예를 들어, \(45\degree=180\degree\times\frac14=\pi\times\frac14=\frac\pi4\)이므로 아래 표의 세번째 부분을 채울 수 있다(굵은 글씨)
\par\bigskip\noindent
\begin{tabu}{|X[c$]|X[c$]|X[c$]|X[c$]|X[c$]|X[c$]|X[c$]|X[c$]|X[c$]|X[c$]|X[c$]|}
\hline
 0\degree
&30\degree
&45\degree
&
&
&120\degree
&
&150\degree
&180\degree
&
&
\\\hline
0
&
&\pmb{\frac\pi4}
&\frac\pi3
&\frac\pi2
&\frac23\pi
&\frac34\pi
&
&\pi
&\frac32\pi
&2\pi
\\\hline
\end{tabu}

%
\prob
다음 삼각비의 값을 계산하시오.
\begin{enumerate}[(1)]
\item
\(\sin 30\degree\)
\item
\(\sin 45\degree\)
\item
\(\sin 60\degree\)
\item
\(\cos 30\degree\)
\item
\(\cos 45\degree\)
\item
\(\cos 60\degree\)
\end{enumerate}

\newpage
\setcounter{num}{0}

%
\ans

\begin{enumerate}[(1)]
\item
\(\sqrt5\)은 "\fbox{제곱해서 5이 되는 양수}"이다.
즉, \fbox{\(x^2=5\)}을 만족시키는 양수이다.
따라서 \((\sqrt5)^2=\fbox{5}\)이다.
또한
\((\sqrt5)^3=\fbox{$5\sqrt5$}\)
이다.
\item
\(\sqrt2\)은 "\fbox{제곱해서 2이 되는 양수}"이다.
즉, \fbox{\(x^2=2\)}을 만족시키는 양수이다.
따라서 \((\sqrt2)^2=\fbox{2}\)이다.
또한
\((\sqrt2)^5=\fbox{$4\sqrt2$}\)
이다.
\end{enumerate}

%
\ans
\begin{enumerate}[(1)]
\item
125
\item
5
\item
1
\item
\(\frac15\)
\item
\(\frac1{125}\)
\item
\(\sqrt5\)
\item
\(\sqrt[3]5\)
\item
\(5\sqrt5(=\sqrt{5^3}=\sqrt{125})\)
\item
\(25\sqrt5(=\sqrt{5^5}=\sqrt{3125})\)
\end{enumerate}

%
\ans
\begin{enumerate}[(1)]
\setcounter{enumi}{1}
\item
3
\item
6
\item
\(-1\)
\item
1
\item
\(-1\)
\item
\(\frac52\)
\item
\(\frac32\)
\item
\(\frac12\)
\item
\(\frac12\)
\item
\(\frac14\)
\item
0
\item
0
\end{enumerate}

%
\ans
\begin{enumerate}[(1)]
\item
16
\item
8
\item
\(\frac15\)
\item
\(\frac1{25}\)
\item
\(\frac18\)
\item
\(\frac19\)
\item
5
\item
4
\item
1
\item
1
\end{enumerate}

%
\ans
\begin{enumerate}[(1)]
\item
30
\item
15
\item
16
\item
9
\item
3
\item
\(\sqrt7(=7^{\frac12})\)
\end{enumerate}

%
\ans
\begin{enumerate}[(1)]
\setcounter{enumi}{1}
\item
1
\item
3
\item
2
\item
2
\item
\(-2\)
\item
\(-1\)
\item
2
\item
\(\frac32\)
\item
\(\frac52\)
\item
3
\item
4
\item
\(\frac32\)
\item
\(\frac32\)
\item
\(\frac12\)
\item
\(\frac12\)
\item
1
\end{enumerate}

%
\ans
\(\log_aN\)은
\(a^x=N\)을 만족시키는 실수 \(x\)값을 의미한다.
예를 들어, \(\log_381\)은
\fbox{\(a^x=N\)}을 만족시키는 실수 \(x\)이므로 \(x=\square\)이다.
따라서 \(\log_381=\square\)이다.

이때, \(a\)는 \fbox{밑}이라고 부르고 \(N\)은 \fbox{진수}라고 부른다.
또한 \(a\)는 \fbox{\(a>0\), \(a\neq1\)}이어야 하고, \(N\)은 \fbox{\(N>0\)}이어야 한다.

%
\ans
\par\bigskip\noindent
\begin{tabu}{|X[c$]|X[c$]|X[c$]|X[c$]|X[c$]|X[c$]|X[c$]|X[c$]|X[c$]|X[c$]|X[c$]|}
\hline
 0\degree
&30\degree
&45\degree
&60\degree
&90\degree
&120\degree
&135\degree
&150\degree
&180\degree
&270\degree
&360\degree
\\\hline
0
&\frac\pi6
&\pmb{\frac\pi4}
&\frac\pi3
&\frac\pi2
&\frac23\pi
&\frac34\pi
&\frac56\pi
&\pi
&\frac32\pi
&2\pi
\\\hline
\end{tabu}


%
\ans
\begin{enumerate}[(1)]
\item
\(\sin 30\degree=\frac12\)
\item
\(\sin 45\degree=\frac{\sqrt2}2\)
\item
\(\sin 60\degree=\frac{\sqrt3}2\)
\item
\(\cos 30\degree=\frac{\sqrt3}2\)
\item
\(\cos 45\degree=\frac{\sqrt2}2\)
\item
\(\cos 60\degree=\frac12\)
\end{enumerate}

\end{document}