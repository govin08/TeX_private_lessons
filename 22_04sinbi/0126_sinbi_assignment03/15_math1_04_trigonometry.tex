\documentclass{oblivoir}
\usepackage{amsmath,amssymb,kotex,paralist,graphicx}
\usepackage{mdframed}
\usepackage{../kswrapfig}
\usepackage{fapapersize}
\usefapapersize{210mm,297mm,20mm,*,20mm,*}
%\pagestyle{empty}
\usepackage{multicol}
\setlength{\columnsep}{30pt}
\setlength{\columnseprule}{1pt}
%\def\columnseprulecolor{\color{blue}}

%%% 객관식 선지

\usepackage{tabto,pifont}
\TabPositions{0.2\textwidth,0.4\textwidth,0.6\textwidth,0.8\textwidth}

\newcommand\one{\ding{172}}
\newcommand\two{\ding{173}}
\newcommand\three{\ding{174}}
\newcommand\four{\ding{175}}
\newcommand\five{\ding{176}}

\newcommand\taba[5]{\par\bigskip\noindent
\one\:{\ensuremath{#1}}
\tab\two\:\:{\ensuremath{#2}}
\tab\three\:\:{\ensuremath{#3}}
\tab\four\:\:{\ensuremath{#4}}
\tab\five\:\:{\ensuremath{#5}}}

\newcommand\tabb[5]{\par\bigskip\noindent
\one\:{\ensuremath{#1}}
\tabto{0.16\textwidth}\two\:\:{\ensuremath{#2}}
\tabto{0.33\textwidth}\three\:\:{\ensuremath{#3}}\medskip\par\noindent
\four\:\:{\ensuremath{#4}}.
\tabto{0.16\textwidth}\five\:\:{\ensuremath{#5}}}

\newcommand\tabc[5]{\par\bigskip\noindent
\one\:{\ensuremath{#1}}
\tabto{0.25\textwidth}\two\:\:{\ensuremath{#2}}\medskip\par\noindent
\three\:\:{\ensuremath{#3}}
\tabto{0.25\textwidth}\four\:\:{\ensuremath{#4}}\medskip\par\noindent
\five\:\:{\ensuremath{#5}}}

\newcommand\tabd[5]{\par\bigskip\noindent
\one\:{#1}\medskip\par\noindent
\two\:\:{#2}\medskip\par\noindent
\three\:\:{#3}\medskip\par\noindent
\four\:\:{#4}\medskip\par\noindent
\five\:\:{#5}}

%%% Counters
\newcounter{num}

%%% Commands
\newcommand{\prob}[1]
{\vs\par\noindent\refstepcounter{num}\textbf{문제 \arabic{num})}\label{#1}\par\noindent}

\newcommand\vs[1]{\vspace{70pt}}

\newcommand\inc[1]{\begin{center}\includegraphics[width=0.95\columnwidth]{#1}\end{center}}

\newcommand\pb[1]{\ensuremath{\fbox{\phantom{#1}}}}

\newcommand\ba{\ensuremath{\:|\:}}

\newcommand\an[2]{\par\bigskip\noindent\textbf{문제 \ref{#1})} #2\\}

\newcommand\ans[1]{\begin{flushright}\textbf{답 : }#1\end{flushright}}

\renewcommand{\arraystretch}{1.5}

%%% Meta Commands
\let\oldsection\section
\renewcommand\section{\clearpage\oldsection}
\let\emph\textsf

\usepackage{fapapersize}
\usefapapersize{210mm,297mm,40mm,*,20mm,*}
\DeclareSymbolFont{yhlargesymbols}{OMX}{yhex}{m}{n}
\DeclareMathAccent{\wideparen}{\mathord}{yhlargesymbols}{"F3}
\usepackage{multirow,multicol}
%%%%
\begin{document}

\begin{center}
{\Large 신비, 미니테스트 3}
\end{center}

%
\prob{각도 \(\theta\)의 동경을 \(OP\)라고 할 때, \(P=(2,1)\)이다.}\label{property4}
\par\noindent
\begin{minipage}{.5\textwidth}
\begin{talign*}
\sin\theta&=\frac1{\sqrt5}\\
\cos\theta&=\frac2{\sqrt5}\\
\tan\theta&=\frac12
\end{talign*}
\end{minipage}
\begin{minipage}{.5\textwidth}
\vspace{10pt}
\includegraphics[width=.5\textwidth]{property_4}
\vspace{10pt}
\end{minipage}
\noindent
이때 다음 각도들에 대한 삼각비의 값을 차례로 구하여라.\\[10pt]
\begin{enumerate*}[itemjoin=\qquad\qquad]
\item
\(\theta-2\pi\)
\item
\(-\theta\)
\item
\(\theta-\pi\)
\item
\(\frac\pi2-\theta\)
\end{enumerate*}

\begin{enumerate}
\item
\begin{minipage}{.5\textwidth}
\begin{talign*}
\sin(\theta-2\pi)=&\\
\cos(\theta-2\pi)=&\\
\tan(\theta-2\pi)=&\\
\end{talign*}
\end{minipage}
\begin{minipage}{.5\textwidth}
\vspace{10pt}
\includegraphics[width=.5\textwidth]{property_4-0}
\vspace{10pt}
\end{minipage}
\item
\begin{minipage}{.5\textwidth}
\begin{talign*}
\sin(-\theta)=&\\
\cos(-\theta)=&\\
\tan(-\theta)=&\\
\end{talign*}
\end{minipage}
\begin{minipage}{.5\textwidth}
\vspace{10pt}
\includegraphics[width=.5\textwidth]{property_4-0}
\vspace{10pt}
\end{minipage}
\item
\begin{minipage}{.5\textwidth}
\begin{talign*}
\sin(\theta-\pi)=&\\
\cos(\theta-\pi)=&\\
\tan(\theta-\pi)=&\\
\end{talign*}
\end{minipage}
\begin{minipage}{.5\textwidth}
\vspace{10pt}
\includegraphics[width=.5\textwidth]{property_4-0}
\vspace{10pt}
\end{minipage}
\item
\begin{minipage}{.5\textwidth}
\begin{talign*}
\sin(\frac\pi2-\theta)=&\\
\cos(\frac\pi2-\theta)=&\\
\tan(\frac\pi2-\theta)=&\\
\end{talign*}
\end{minipage}
\begin{minipage}{.5\textwidth}
\vspace{10pt}
\includegraphics[width=.5\textwidth]{property_4-0}
\vspace{10pt}
\end{minipage}
\end{enumerate}

\newpage
%
\textbf{답 1)}
\begin{enumerate}
\item
\begin{minipage}{.5\textwidth}
\begin{talign*}
\sin(\theta-2\pi)=&\frac1{\sqrt5}\\
\cos(\theta-2\pi)=&\frac2{\sqrt5}\\
\tan(\theta-2\pi)=&\frac12\\
\end{talign*}
\end{minipage}
\begin{minipage}{.5\textwidth}
\vspace{10pt}
\includegraphics[width=.5\textwidth]{property_4-1}
\vspace{10pt}
\end{minipage}
\item
\begin{minipage}{.5\textwidth}
\begin{talign*}
\sin(-\theta)=&-\frac1{\sqrt5}\\
\cos(-\theta)=&\frac2{\sqrt5}\\
\tan(-\theta)=&-\frac12\\
\end{talign*}
\end{minipage}
\begin{minipage}{.5\textwidth}
\vspace{10pt}
\includegraphics[width=.5\textwidth]{property_4-2}
\vspace{10pt}
\end{minipage}
\item
\begin{minipage}{.5\textwidth}
\begin{talign*}
\sin(\theta-\pi)=&-\frac1{\sqrt5}\\
\cos(\theta-\pi)=&-\frac2{\sqrt5}\\
\tan(\theta-\pi)=&\frac12\\
\end{talign*}
\end{minipage}
\begin{minipage}{.5\textwidth}
\vspace{10pt}
\includegraphics[width=.5\textwidth]{property_4-3}
\vspace{10pt}
\end{minipage}
\item
\begin{minipage}{.5\textwidth}
\begin{talign*}
\sin(\frac\pi2-\theta)=&\frac2{\sqrt5}\\
\cos(\frac\pi2-\theta)=&\frac1{\sqrt5}\\
\tan(\frac\pi2-\theta)=&2\\
\end{talign*}
\end{minipage}
\begin{minipage}{.5\textwidth}
\vspace{10pt}
\includegraphics[width=.5\textwidth]{property_4-4}
\vspace{10pt}
\end{minipage}
\end{enumerate}

\end{document}