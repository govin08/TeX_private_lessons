\documentclass{oblivoir}
\usepackage{amsmath,amssymb,amsthm,kotex,paralist,kswrapfig}

\usepackage[skipabove=10pt,innertopmargin=10pt]{mdframed}

\usepackage{tabto,pifont}
\TabPositions{0.2\textwidth,0.4\textwidth,0.6\textwidth,0.8\textwidth}
\newcommand\tabb[5]{\par\bigskip\noindent
\ding{172}\:{\ensuremath{#1}}
\tab\ding{173}\:\:{\ensuremath{#2}}
\tab\ding{174}\:\:{\ensuremath{#3}}
\tab\ding{175}\:\:{\ensuremath{#4}}
\tab\ding{176}\:\:{\ensuremath{#5}}}

\usepackage{enumitem}
\setlist[enumerate]{label=(\arabic*)}

\newcounter{num}
\newcommand{\defi}[1]
{\noindent\refstepcounter{num}\textbf{정의 \arabic{num}) #1}\par\noindent}
\newcommand{\theo}[1]
{\noindent\refstepcounter{num}\textbf{정리 \arabic{num}) #1}\par\noindent}
\newcommand{\exam}[1]
{\bigskip\bigskip\noindent\refstepcounter{num}\textbf{예시 \arabic{num}) #1}\par\noindent}
\newcommand{\prob}[1]
{\bigskip\bigskip\noindent\refstepcounter{num}\textbf{문제 \arabic{num}) #1}\par\noindent}
\newcommand{\proo}
{\bigskip\textsf{증명)}\par}

\newcommand{\ans}{
{\par\raggedleft\textbf{답 : (\qquad\qquad\qquad\qquad\qquad\qquad)}\par}\bigskip\bigskip}
\newcommand\an[1]{\par\bigskip\noindent\textbf{문제 #1)}\\}

\newcommand{\pb}[1]%\Phantom + fBox
{\fbox{\phantom{\ensuremath{#1}}}}

\newcommand\ba{\,|\,}

\let\oldsection\section
\renewcommand\section{\clearpage\oldsection}

\newenvironment{talign}
 {\let\displaystyle\textstyle\align}
 {\endalign}
\newenvironment{talign*}
 {\let\displaystyle\textstyle\csname align*\endcsname}
 {\endalign}

%%%%
\begin{document}

\title{규현 : 06 수열(4)}
\author{}
\date{\today}
\maketitle
\tableofcontents
\newpage

%%
\section{수열의 귀납적 정의(1)}
%
\exam{}
첫째항이 \(1\)이고 공차가 \(2\)인 등차수열
\[1,\quad3,\quad5,\quad7,\quad\cdots\]
는 일반항 \(a_n=1+(n-1)\cdot 2=2n-1\)로 나타내기도 하지만 첫째항 \(1\)부터 차례로 일정한 수 \(2\)를 더해서 얻어지는 수열이므로
\[a_1=1,\quad a_{n+1}=a_n+2\:\:(n=1,2,3,\cdots)\]
와 같이 첫째항과 이웃하는 항들 사이의 관계식을 이용하여 수열을 정의하기도 한다.

\begin{mdframed}
%
\defi{수열의 귀납적 정의}
일반적으로 \(a_1\)의 값과 \(a_n\)에서 \(a_{n+1}\)을 구할 수 있는 관계식이 주어지면 수열 \(\{a_n\}\)의 모든 항 \(a_1\), \(a_2\), \(a_3\), \(\cdots\)이 정해진다.
이와 같이 첫째항과 이웃하는 항들 사이의 관계식으로 수열을 정의하는 것을 수열의 \textbf{귀납적 정의}라고 한다.
이때, 이웃하는 항들 사이의 관계식을 \textbf{점화식}이라고 한다.
\end{mdframed}

\exam{}
\(a_1=1\),\:\:\(a_{n+1}=2a_n+1\:\:(n=1,2,3,\cdots)\)로 정의된 수열 \(\{a_n\}\)에서 제 4항을 구하면
\[
\begin{aligned}
n=1\text{을 대입 : }	&a_2=2\cdot1+1=3\\
n=2\text{을 대입 : }	&a_3=2\cdot3+1=7\\
n=3\text{을 대입 : }	&a_4=2\cdot7+1=15
\end{aligned}
\]
따라서 수열 \(\{a_n\}\)의 제4항은 \(15\)이다.

%
\prob{}
다음과 같이 귀납적으로 정의된 수열 \(\{a_n\}\)의 제5항을 구하여라.
\begin{enumerate}
\item
\(a_1=2\),~~ \(a_{n+1}=a_n+3\)
\item
\(a_1=2\),~~ \(a_{n+1}=\frac12a_n\)
\end{enumerate}

%
\prob{}\label{iterating}
다음과 같이 귀납적으로 정의된 수열 \(\{a_n\}\)의 제5항을 구하여라.
\begin{enumerate}
\item\label{iterating_sum}
\(a_1=3\),~~ \(a_{n+1}=a_n+n\)~\((n=1,2,3,\cdots)\)
\begin{mdframed}
\vspace{0.1\textheight}
\end{mdframed}
\ans
\item\label{iterating_multiplication}
\(a_1=1\),~~ \(a_{n+1}=\frac{n+2}{n}a_n\)~\((n=1,2,3,\cdots)\)
\begin{mdframed}
\vspace{0.1\textheight}
\end{mdframed}
\ans
\item\label{iterating_difference}
\(a_1=1\),~~ \(a_{n+1}=3a_n+2\)~\((n=1,2,3,\cdots)\)
\begin{mdframed}
\vspace{0.1\textheight}
\end{mdframed}
\ans
\end{enumerate}

%
\prob{}
다음과 같이 귀납적으로 정의된 수열 \(\{a_n\}\)에서 \(a_5+a_{10}\)의 값을 구하여라.
\begin{enumerate}
\item\label{Fibonacci}
\(a_1=1\),~~\(a_2=1\),~~\(a_{n+2}=a_n+a_{n+1}\)~\((n=1,2,3,\cdots)\)
\begin{mdframed}
\vspace{0.2\textheight}
\end{mdframed}
\ans
\item
\(a_1=1\),~~\(a_2=2\),~~\(a_{n+2}=a_n+1\)~\((n=1,2,3,\cdots)\)
\begin{mdframed}
\vspace{0.2\textheight}
\end{mdframed}
\ans
\end{enumerate}

위 문제에서 \ref{Fibonacci}의 수열은 \textbf{피보나치 수열}이라고 불린다.

%\begin{mdframed}
%%
%\theo{등차수열과 등비수열의 귀납적 정의}
%수열 \(\{a_n\}\)에 대하여 이웃하는 항 사이의 관계식이
%\begin{enumerate}
%\item
%\(a_{n+1}-a_n=d\)이면 공차가 \(d\)인 등차수열이다.
%\item
%\(2a_{n+1}=a_n+a_{n+2}\)이면 등차수열이다.
%\item
%\(\frac{a_{n+1}}{a_n}=r\)이면 공비가 \(r\)인 등비수열이다.
%\item
%\({a_{n+1}}^2=a_na_{n+2}\)이면 등비수열이다.
%\end{enumerate}
%\end{mdframed}

\clearpage
%
\exam{}
다음과 같이 귀납적으로 정의된 수열의 일반항을 구하여라.
\begin{enumerate}
\item
\(a_1=5\),~~\(a_{n+1}=a_n+8\)~\((n=1,2,3,\cdots)\)
\begin{mdframed}[skipabove=-1pt]
\(a_{n+1}-a_n=8\)에서 공차가 \(8\)인 등차수열임을 알 수 있다.
첫항은 \(a=5\)이므로
\[a_n=5+(n-1)8=8n-3\]
\end{mdframed}
{\par\raggedleft\textbf{답 : \(a_n=8n-3\)\quad}\par}\bigskip
\item
\(a_1=2\),~~\(a_2=6\),~~\(2a_{n+1}=a_{n+2}+a_n\)~\((n=1,2,3,\cdots)\)
\begin{mdframed}[skipabove=-1pt]
\(2a_{n+1}=a_{n+2}+a_n\)에서 \(a_{n+1}\)은 \(a_n\)과 \(a_{n+2}\)의 등차중항이다.
따라서 인접한 세 항은 등차수열을 이루고, 전체 수열도 등차수열임을 알 수 있다.
첫항은 \(a=2\)이고 공차는 \(d=a_2-a_1=4\)이므로
\[a_n=2+(n-1)4=4n-2\]
\end{mdframed}
{\par\raggedleft\textbf{답 : \(a_n=4n-2\)\quad}\par}\bigskip
\item
\(a_1=1\),~~\(a_{n+1}=2a_n\)~\((n=1,2,3,\cdots)\)
\begin{mdframed}[skipabove=-1pt]
\(\frac{a_{n+1}}{a_n}=2\)에서 공비가 \(2\)인 등비수열임을 알 수 있다.
첫항은 \(a=1\)이므로
\[a_n=1\cdot2^{n-1}=2^{n-1}\]
\end{mdframed}
{\par\raggedleft\textbf{답 : \(a_n=2^{n-1}\)\quad}\par}\bigskip
\item
\(a_1=4\),~~\(a_2=6\),~~\({a_{n+1}}^2=a_na_{n+2}\)~\((n=1,2,3,\cdots)\)
\begin{mdframed}[skipabove=-1pt]
\({a_{n+1}}^2=a_na_{n+2}\)에서 \(a_{n+1}\)은 \(a_n\)과 \(a_{n+2}\)의 등비중항이다.
따라서 인접한 세 항은 등차수열을 이루고, 전체 수열도 등차수열임을 알 수 있다.
첫항은 \(a=4\)이고 공차는 \(d=a_2\div a_1=\frac32\)이므로
\[a_n=4\times\left(\frac32\right)^{n-1}\]
\end{mdframed}
{\par\raggedleft\textbf{답 : \(a_n=4\times\left(\frac32\right)^{n-1}\)\quad}\par}\bigskip
\end{enumerate}

%
\prob{}
다음과 같이 귀납적으로 정의된 수열의 일반항을 구하여라.
\begin{enumerate}
\item
\(a_1=1\),~~ \(a_{n+1}-a_n=5\)~\((n=1,2,3,\cdots)\)
\begin{mdframed}[skipabove=-2pt]
\vspace{0.11\textheight}
\end{mdframed}
{\par\raggedleft\textbf{답 : \(a_n=\)\qquad\qquad\qquad\qquad}\par}\bigskip
\item
\(a_1=4\),~~ \(a_2=2\),~~ \(\displaystyle a_{n+1}=\frac{a_{n+2}+a_n}2\)~~\((n=1,2,3,\cdots)\)
\begin{mdframed}[skipabove=-2pt]
\vspace{0.11\textheight}
\end{mdframed}
{\par\raggedleft\textbf{답 : \(a_n=\)\qquad\qquad\qquad\qquad}\par}\bigskip
\item
\(a_1=4\),~~ \(\displaystyle\frac{a_{n+1}}{a_n}=-2\)~~\((n=1,2,3,\cdots)\)
\begin{mdframed}[skipabove=-2pt]
\vspace{0.11\textheight}
\end{mdframed}
{\par\raggedleft\textbf{답 : \(a_n=\)\qquad\qquad\qquad\qquad}\par}\bigskip
\item
\(a_1=\frac13\),~~ \(a_2=1\),~~ \(a_{n+1}=\sqrt{a_na_{n+2}}\)~~\((n=1,2,3,\cdots)\)
\begin{mdframed}[skipabove=-2pt]
\vspace{0.11\textheight}
\end{mdframed}
{\par\raggedleft\textbf{답 : \(a_n=\)\qquad\qquad\qquad\qquad}\par}\bigskip
\end{enumerate}

%%
\section{수열의 귀납적 정의(2)}

%
\exam{}
문제 \ref{iterating}의 \ref{iterating_sum}, \ref{iterating_multiplication}, \ref{iterating_difference}의 일반항을 구해보자.
\begin{enumerate}
%
\item
\(a_1=3\),~~ \(a_{n+1}=a_n+n\)
\begin{mdframed}
점화식의 \(n\)에 \(1\), \(2\), \(3\), \(\cdots\), \(n\)을 차례로 대입하여 나열하면
\begin{align*}
a_2=&\:a_1+1\\
a_3=&\:a_2+2\\
a_4=&\:a_3+3\\
&\vdots\\
a_n=&\:a_{n-1}+(n-1)
\end{align*}
이다.
이 식들을 모두 더하면
%\[a_2+a_3+a_4+\cdots+a_n=a_1+a_2+a_3+\cdots+a_{n-1}+(1+2+3+\cdots+n)\]
%이고 중복된 항들을 지우면
\begin{align*}
a_n
&=a_1+(1+2+3+\cdots+(n-1))\\
&=a_1+\sum_{k=1}^{n-1}k=3+\frac{n(n-1)}2\\
&=\frac{n^2-n+6}2
\end{align*}
이다.
\end{mdframed}
{\par\raggedleft\textbf{답 : \(a_n=\frac{n^2-n+6}2\)\quad}\par}\bigskip

%
\clearpage
\item
\(a_1=1\),~~ \(a_{n+1}=\frac{n+2}{n}a_n\)
\begin{mdframed}
점화식의 \(n\)에 \(1\), \(2\), \(3\), \(\cdots\), \(n\)을 차례로 대입하여 나열하면
\begin{talign*}
\textstyle
a_2=&\:\frac31a_1\\
a_3=&\:\frac42a_2\\
a_4=&\:\frac53a_3\\
a_5=&\:\frac64a_3\\
&\vdots\\
a_{n-1}=&\:\frac{n}{n-2}a_{n-2}\\
a_n=&\:\frac{n+1}{n-1}a_{n-1}
\end{talign*}
이다.
이 식들을 모두 곱하면
%\[a_2a_3a_4\cdots a_n=a_1a_2a_3\cdots a_{n-1}
%\times\frac31\times\frac42\times\frac53\times\frac64\times\cdots\times\frac{n+1}{n-1}\times\frac{n+2}n\]
%이고 중복된 항들을 지우고 약분하면
\begin{talign*}
\textstyle
a_n&=a_1\times\frac31\times\frac42\times\frac53\times\frac64\times\cdots\times\frac{n}{n-2}\times\frac{n+1}{n-1}\\
&=a_1\times\frac{n(n+1)}{1\times2}\\
&=\frac{n(n+1)}2
\end{talign*}
이다.
\end{mdframed}
{\par\raggedleft\textbf{답 : \(a_n=\frac{n(n+1)}2\)\quad}\par}\bigskip

%
\clearpage
\item
\(a_1=1\),~~ \(a_{n+1}=3a_n+2\)
\begin{mdframed}
주어진 점화식
\[a_{n+1}=3a_n+2\]
은
\[a_{n+1}+k=3(a_n+k)\]
꼴로 만들 수 있다.
두 식을 비교해보면 \(2k=2\)이므로 \(k=1\)이고
\[a_{n+1}+1=3(a_n+1)\]
이다.
\(b_n=a_n+1\)을 만족시키는 수열 \(\{b_n\}\)을 생각하면
\[b_{n+1}=3b_n\]
이므로 \(\{b_n\}\)은 공비가 3인 등비수열이다.
\(b_1=a_1+1=1+1=2\)이므로
\[b_n=2\cdot3^{n-1}\]
이고
\[a_n+1=2\cdot3^{n-1}\]
이다.
따라서
\[a_n=2\cdot3^{n-1}-1\]
이다.
\end{mdframed}
{\par\raggedleft\textbf{답 : \(a_n=2\cdot3^{n-1}-1\)\quad}\par}\bigskip
\end{enumerate}
\clearpage

%
\prob{}
다음과 같이 귀납적으로 정의된 수열의 일반항을 구하여라.
\begin{enumerate}
\item
\(a_1=1\), \(a_{n+1}=a_n+4n\)
\begin{mdframed}
\vspace{0.8\textheight}
\end{mdframed}
{\par\raggedleft\textbf{답 : \(a_n=\)\qquad\qquad\qquad\qquad}\par}\bigskip
\clearpage
\item
\(a_1=1\), \(a_{n+1}=2^na_n\)
\begin{mdframed}
\vspace{0.8\textheight}
\end{mdframed}
{\par\raggedleft\textbf{답 : \(a_n=\)\qquad\qquad\qquad\qquad}\par}\bigskip
\clearpage
\item
\(a_1=2\), \(a_{n+1}=2a_n+1\)
\begin{mdframed}
\vspace{0.8\textheight}
\end{mdframed}
{\par\raggedleft\textbf{답 : \(a_n=\)\qquad\qquad\qquad\qquad}\par}\bigskip
\end{enumerate}

%%
\section{수학적 귀납법}
\vspace{-20pt}
%
\exam{}
\(1+3=2^2\), \(1+3+5=3^2\), \(1+3+5+7=4^2\)이므로 다음을 추측할 수 있다.
\vspace{-5pt}
\[
1+3+5+7+\cdots+(2n-1)=n^2
\tag{㉠}
\]
하지만 이 사실만으로 모든 자연수 \(n\)에 대하여 식 ㉠이 성립한다고 할 수는 없다.
또한, 무한히 많은 자연수 \(n\)에 대해 일일이 성립하는지 확인해볼 수도 없을 것이다.

\bigskip
이제 모든 자연수 \(n\)에 대하여 ㉠이 성립함을 증명하는 방법을 알아보자.
\begin{enumerate}[label=\emph{[\arabic*]}]
\item
\(n=1\)일때 식 ㉠이 성립함을 보인다.
즉
\vspace{-5pt}
\[(좌변)=1,\quad (우변)=1^2\]
이므로 식 ㉠은 \(n=1\)일 때 성립한다.
\item
\(n=k\)일때 식 ㉠이 성립한다고 가정하고, \(n=k+1\)일 때도 식 ㉠이 성립함을  보인다.
즉
\vspace{-5pt}
\[
1+3+5+\cdots+(2k-1)=k^2
\]
을 가정하고 이 식의 양변에 \(2k+1\)을 더하면.
그러면
%\begin{align*}
\[1+3+5+\cdots+(2k-1)+(2k+1)=k^2+(2k+1)=(k+1)^2\]
이다.
따라서 식 ㉠은 \(n=k+1\)일 때에도 성립한다.
\end{enumerate}

\emph{[1]}에서 \(n=1\)일 때, 식 ㉠이 성립함을 보였으므로 위에서 증명한 사실 \emph{[2]}로부터
\begin{center}
\(n=1\)일 때 식 ㉠이 성립하므로 \(n=2\)일 때 식 ㉠이 성립한다.\\
\(n=2\)일 때 식 ㉠이 성립하므로 \(n=3\)일 때 식 ㉠이 성립한다.\\
\(n=3\)일 때 식 ㉠이 성립하므로 \(n=4\)일 때 식 ㉠이 성립한다.\\
\(\vdots\)
\end{center}

따라서 \emph{[1]}, \emph{[2]}가 성립하면 모든 자연수 \(n\)에 대하여 식 ㉠이 성립함을 알 수 있다.

\bigskip
자연수에 대한 어떤 명제가 성립함을 보일 때, 위와 같은 방법으로 증명하는 것을 \textbf{수학적 귀납법}이라고 한다.

\begin{mdframed}[skipabove=-5pt]
%
\theo{수학적 귀납법}
자연수 \(n\)에 대한 명제 \(p(n)\)이 모든 자연수 \(n\)에 대하여 성립함을 증명하려면 다음 두 가지를 보이면 된다.
\smallskip
\begin{enumerate}[label=\emph{[\arabic*]}]\tightlist
\item
\(n=1\)일 때 명제 \(p(n)\)이 성립한다.
\item
\(n=k\)일 때 명제 \(p(n)\)이 성립한다고 가정하면 \(n=k+1\)일 때도 \(p(n)\)이 성립한다.
\end{enumerate}
\end{mdframed}

%
\exam{}
모든 자연수 \(n\)에 대하여 다음 등식이 성립함을 수학적 귀납법을 사용하여 증명하여라.
\vspace{-5pt}
\[1^2+2^2+3^2+\cdots+n^2=\frac{n(n+1)(2n+1)}6\tag{㉠}\]

\begin{mdframed}[skipabove=-10pt]
\begin{enumerate}[label=\emph{[\arabic*]}]
\item
\(n=1\)이면
\vspace{-10pt}
\[(좌변)=1^2=1,\quad (우변)=\frac{1\cdot2\cdot3}6=1\]
이므로 식 ㉠은 \(n=1\)일 때 성립한다.
\item
\(n=k\)일때 식 ㉠이 성립한다고 가정하자.
즉
\[
1^2+2^2+3^2+\cdots+k^2=\frac{k(k+1)(2k+1)}6
\]
을 가정하자.
\(n=k+1\)일때에 식 ㉠이 성립한다는 것을 보이기 위해 양변에 \((k+1)^2\)을 더하면
\vspace{-5pt}
\begin{align*}
1^2+2^2+3^2+\cdots+k^2+(k+1)^2
&=\frac{k(k+1)(2k+1)}6+(k+1)^2\\
&=\frac{k+1}6\left\{k(2k+1)+6(k+1)\right\}\\
&=\frac{k+1}6(2k^2+7k+6)\\
&=\frac{(k+1)(k+2)(2k+3)}6\\
\end{align*}

\vspace{-30pt}
이다.
따라서 \(n=k+1\)일 때에도 식 ㉠이 성립한다.
\end{enumerate}

\vspace{15pt}
\emph{[1]}, \emph{[2]}에서 모든 자연수에 대해 식 ㉠이 성립한다.
\end{mdframed}

\clearpage
%
\prob{}
모든 자연수 \(n\)에 대하여 다음 등식이 성립함을 수학적 귀납법을 사용하여 증명하여라.
\[1+2+3+\cdots+n=\frac{n(n+1)}2\tag{㉠}\]
\begin{mdframed}
\vspace{0.8\textheight}
\end{mdframed}

%%
\section*{답}
\begin{minipage}{0.49\textwidth}
%
\an{4}
(1) 14\\
(2) \(\frac18\)

%
\an{5}
(1) 13\\
(2) 15\\
(3) 161

%
\an{6}
(1) 60\\
(2) 9
\end{minipage}
\begin{minipage}{0.49\textwidth}
%
\an{8}
(1) \(a_n=5n-4\)\\
(2) \(a_n=-2n+6\)\\
(3) \(a_n=4\cdot(-2)^{n-1}\)\\
(4) \(a_n=3^{n-2}\)

%
\an{10}
(1) \(a_n=2n^2-2n+1\)\\
(2) \(a_n=2^{\frac{n(n-1)}2}\)\\
(3) \(a_n=3\times2^{n-1}-1\)
\end{minipage}

%
\an{14}
\begin{enumerate}[label=\emph{[\arabic*]}]
\item
\(n=1\)이면
\[(좌변)=1,\quad (우변)=\frac{1\cdot2}2=1\]
이므로 식 ㉠은 \(n=1\)일 때 성립한다.
\item
\(n=k\)일때 식 ㉠이 성립한다고 가정하자.
즉
\[
1+2+3+\cdots+k=\frac{k(k+1)}2
\]
을 가정하자.
\(n=k+1\)일때에 식 ㉠이 성립한다는 것을 보이기 위해 양변에 \(k+1\)을 더하면
\begin{align*}
1+2+3+\cdots+k+(k+1)
&=\frac{k(k+1)}2+(k+1)\\
&=\frac{(k+1)(k+2)}2
\end{align*}
이다.
따라서 \(n=k+1\)일 때에도 식 ㉠이 성립한다.
\end{enumerate}
\emph{[1]}, \emph{[2]}에서 모든 자연수에 대해 식 ㉠이 성립한다.

\end{document}