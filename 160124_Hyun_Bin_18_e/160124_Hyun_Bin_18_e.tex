\documentclass{article}
\usepackage{amsmath,amssymb,amsthm,kotex,mdframed,paralist}

\begin{document}

\title{현빈 18 : 숫자 \(e\)}
\author{}
\date{\today}
\maketitle

\newpage

\section{\(e\)에 대한 두 가지 정의}
숫자 \(e\)의 정의는 다음의 두 가지이다.\footnote{홍성대, ``수학의 정석, 미적분2'', 성지출판, 2014, p 139.}
\begin{equation}e=\lim_{x\to0}\left(1+x\right)^{\frac1x}\end{equation}
\begin{equation}e=\lim_{t\to\infty}\left(1+\frac1t\right)^t\end{equation}

%(만약 \(e\)를 \[(\log_ax)'\bigg|_{x=1}=1\]인 \(a\)의 값으로 정의한다면 이것은 (1)의 정의와 같다.)
두 정의는 동등한 정의일까?

(1)을 다시 살펴보면, (1)은
\[e=\lim_{x\to0+}\left(1+x\right)^{\frac1x}\tag{1-a}\]
\[e=\lim_{x\to0-}\left(1+x\right)^{\frac1x}\tag{1-b}\]
로 이루어져 있다.
다시 말해
\[(1)\iff (1-a)\wedge(1-b)\]
이다.\footnote{\(\wedge\)는 \(and\)를 의미한다.}
한편 \(x=\frac1t\)로 치환하면
\[(1-a)\iff (2)\]
임을 알 수 있다.
다시 말해 두 정의가 동등하려면, 즉 \[(1)\iff(2)\]이려면,\[(1-a)\wedge(1-b)\iff(1-a)\]여야 하고,
그러려면 \[(1-a)\Rightarrow(1-b)\]여야 한다.

따라서 다음을 증명해보자.
\begin{mdframed}[skipabove=3pt]
\center\(\displaystyle e=\lim_{x\to0+}\left(1+x\right)^{\frac1x}\)이면 \(\displaystyle e=\lim_{x\to0-}\left(1+x\right)^{\frac1x}\)이다.
\end{mdframed}

\newpage

\section{은행 예금}

\paragraph{(1)}
연이율이 6\%인 은행에 100만원을 예금한다고 하자.
1년 후 돌려받는 금액은
\[100\text{만원}\times\left(1+\frac6{100}\right)=106\text{만원}\]
이다.

\paragraph{(2)}
이번에도 연이율 6\%과 원금 100만원을 예금하되, 이자를 6개월마다 한 번씩 받는 상황을 생각해보자.
연이율이 6\%이므로, 6개월 동안의 이율은 연이율의 절반인 3\%로 생각하자.
처음에 100만원이었던 원금은 6개월 후에
\[100\text{만원}\times\left(1+\frac3{100}\right)=103\text{만원}\]
이 되고 다시 6개월이 흐른 1년 후에는 
\[103\text{만원}\times\left(1+\frac3{100}\right)=106\text{만 }900\text{원}\]
이 된다.

\paragraph{(3)}
이번에는 이자를 4개월마다 한 번씩 받는 상황을 생각해보자.
연이율이 6\%이므로, 4개월 동안의 이율은 연이율의 \(1/3\)인 \(2\%\)로 생각하자.
같은 방식으로 계산하면 1년 후에
\[100\text{만원}\times\left(1+\frac2{100}\right)^3=106\text{만 }1208\text{원}\]
을 받을 수 있게 된다.

\paragraph{(4)}
100만원을 한 번에 걸쳐 이자를 붙이면 106만원이 되고, 두 번에 걸쳐 이자를 붙이면 106만 900원이 되며, 세 번에 걸쳐 이자를 붙이면 106만 1208원이 된다.
따라서 여러 번에 걸쳐 이자를 붙일 수록 이익인 것 같다.

그렇다면 한없이 여러 번에 걸쳐 이자를 붙인다면, 얼마까지 이익을 볼 수 있을까?
예를 들어 한 만 번 정도에 걸쳐 이자를 붙이면 원금의 두 배인 200만원 정도를 받을 수 있을까?

\end{document}