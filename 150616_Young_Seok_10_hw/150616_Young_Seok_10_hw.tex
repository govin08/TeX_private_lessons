\documentclass{article}
\usepackage{amsmath,amssymb,amsthm,kotex,mdframed,paralist,chngcntr}

\newcounter{num}
%\newcommand{\defi}[1]
%{\bigskip\noindent\refstepcounter{num}\textbf{정의 \arabic{num}) #1}\par}
%\newcommand{\theo}[1]
%{\bigskip\noindent\refstepcounter{num}\textbf{정리 \arabic{num}) #1}\par}
%\newcommand{\exam}[1]
%{\bigskip\noindent\refstepcounter{num}\textbf{예시 \arabic{num}) #1}\par}
%\newcommand{\prob}[1]
%{\bigskip\noindent\refstepcounter{num}\textbf{문제 \arabic{num}) #1}\par}
\newcommand{\howo}[1]
{\bigskip\noindent\refstepcounter{num}\textbf{숙제 \arabic{num}) #1}\par\bigskip}


\renewcommand{\proofname}{증명)}
\counterwithout{subsection}{section}


%%%
\begin{document}

\title{영석 : 10 숙제(\(\sim\)2015. 6. 20. 토요일)}
\author{}
\date{\today}
\maketitle
%\tableofcontents
\newpage

%
\howo{}
직선 \(l:x+2y=4\)에 대해 다음을 구하여라.\\
(1) \(l\)을 \(x\)축 방향으로 \(2\)만큼, \(y\)축 방향으로 \(-3\)만큼 평행이동한 도형의 방정식\\
(2) \(l\)을 \(x\)축에 대하여 대칭이동한 도형의 방정식\\
(3) \(l\)을 \(y\)축에 대하여 대칭이동한 도형의 방정식\\
(4) \(l\)을 원점에 대하여 대칭이동한 도형의 방정식\\
(5) \(l\)을 \(y=x\)에 대하여 대칭이동한 도형의 방정식

%
\howo{}
다음 부등식의 영역을 좌표 평면 위에 나타내어라.\\
(1) \(y>2x+1\)\\
(3) \(3x-y+9<0\)

%
\howo{}
다음 부등식의 영역을 좌표 평면 위에 나타내어라.\\
(1) \(x^2+y^2<1\)\\
(3) \((x-2)^2+(y-1)^2<1\)

%
\howo{}
다음 연립부등식의 영역을 좌표 평면 위에 나타내어라.\\
\[
\begin{cases}
y<x+2\\
x^2+y^2<9
\end{cases}
\]

%
\howo{}
다음 부등식의 영역을 좌표 평면 위에 나타내어라.\\
\[
(x^2+y^2-4)(x+y)<0\]

%
\howo{}
연립부등식 \(x\ge0\), \(y\ge0\), \(x+y\le 3\)에서 \(y-x\)의 최댓값을 구하여라.

\end{document}