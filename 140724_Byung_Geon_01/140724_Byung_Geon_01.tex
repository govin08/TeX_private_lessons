\documentclass{oblivoir}
\usepackage{amsmath,amssymb,amsthm,mdframed,kotex,paralist}
\newcommand{\bd}{\begin{description}}
\newcommand{\ed}{\end{description}}
\newcommand{\intt}{\ensuremath{\int_a^b}}
\newcommand{\inttt}{\ensuremath{\int_\alpha^\beta}}
\newcommand{\dx}{\ensuremath{\,dx}}
\newcommand{\dt}{\ensuremath{\,dt}}
\newcommand{\delx}{\ensuremath{\Delta x}}
%\usepackage{tabto}
%\TabPositions{0.33\textwidth,0.66\textwidth}

\begin{document}
\title{병건, 적분단원 개념정리}
\author{}
\date{\today}
\maketitle
\tableofcontents

\newpage


%새 節
\section{부정적분}

\subsection{부정적분}
\bd
\item[정의)]
\(F'(x)=f(x)\)이면 \(F(x)\)는 \(f(x)\)의 부정적분이다.
\item[구하는 방법)]
역미분.
\item[기호)]
\(\int f(x)\,dx=F(x)+C\).
\item[용어)]
피적분함수, 적분변수, 적분상수.
\item[성질)]
부정적분은 선형적(linear)이다; 임의의 두 함수 \(f\), \(g\)와 임의의 두 실수 \(p\), \(q\)에 대해, \[\int(pf(x)+qg(x))\,dx=p\int f(x)\,dx+q\int g(x)\,dx.\]
특히,
\begin{enumerate}[(1)]\tightlist
\item
\(\int(f(x)+g(x))\,dx=\int f(x)\,dx+\int g(x)\,dx\).
\item
\(\int(f(x)-g(x))\,dx=\int f(x)\,dx-\int g(x)\,dx\).
\item
\(\int(kf(x))\,dx=k\int f(x)\,dx\).
\end{enumerate}
\item[증명)]
(1)과 (3)을 증명한 후 조합하면 된다.
\ed

\subsection{치환적분법}
\bd
\item[정리)]
\(x\)를 미분가능한 함수 \(g(t)\)로 치환하면 \(\int f(x)\,dx=\int f(g(t))g'(t)\,dt\)이다.
\item[증명)]
\(F'(x)=f(x)\)라고 하면 합성함수의 미분법(연쇄법칙)에 의해%\footnote{RHS : Right hand side, 우변, LHS=좌변}
\[
\int f(g(t))g'(t)\,dt=\int\frac d{dt}F(g(t))\,dt=F(g(t))+C=F(x)+C=\int f(x)\,dx.
\]
혹은,
\[
\int f(g(t))g'(t)\,dt=\int f(x)\frac{\dx}{\dt}\,dt=\int f(x)\,dx
\]
와 같이 생각할 수도 있다.
\item[예시)]
\(\int\frac{f'(x)}{f(x)}\,dx=\ln{|f(x)|}+C\).
\ed

\subsection{부분적분법}
\bd
\item[정리)]
\(f(x)\)와 \(g(x)\)가 모두 미분가능하면, \(\int f(x)g'(x)\,dx=f(x)g(x)-\int f'(x)g(x)\,dx\).
\item[증명)]
곱의 미분법과 부분적분의 성질 (2)에 의해
\(\int f(x)g'(x)\,dx+\int f'(x)g(x)\,dx=f(x)g(x)\).
\ed


%새 節
\section{정적분}
\subsection{구분구적법}
\bd
\item[정의)]
평면 혹은 공간의 일정한 영역을 직사각형, 직육면체 등으로 일정하게 잘라 해당 영역의 넓이, 부피 따위를 어림하는 방법. 극한을 사용.
\ed
\subsection{정적분}
\bd
\item[정의)]
\(f(x)\)는 \([a,b]\)에서 정의된 연속 함수라고 하자.\\
\(f(x)\ge0\)일 때, \(\intt f(x)\,dx\)는 ``\(x=a\)'',~``\(x=b\)'',~``\(y=0\)'',~``\(y=f(x)\)''로 둘러싸인 영역의 넓이이다.
구분구적법을 이용하면,
\[
\intt f(x)\,dx
=
\lim_{n\to\infty}\sum_{k=1}^nf(x_k)\delx
=
\lim_{n\to\infty}\sum_{k=1}^nf\left(a+\frac{b-a}nk\right)\frac{b-a}n.
\]
\(f(x)<0\)일 때,
\[\intt f(x)\,dx=-\intt(-f)(x)\,dx.\]
\(f(x)\)가 \([a,b]\) 내에서 양의 값과 음의 값을 모두 가지면, 구간을 나눠서 정적분을 정의하면 된다.
예를 들어
\([a,m]\)에서 \(f(x)\ge0\)이고 \((m,c]\)에서 \(f(x)<0\)이면,
\[\intt f(x)\,dx=\int_a^mf(x)\,dx+\int_m^bf(x)\,dx.\]
\item[용어)]
위끝, 아래끝.
\item[참고)]
정적분은 고정된 실수이므로 적분변수는 의미를 가지지 않는다(dummy variable);
\[\intt f(x)\,dx=\intt f(t)\,dt=\intt f(s)\,ds=\cdots.\]
\item[정리) 미적분의 기본정리 (1)]
\([a,b]\)에서 \(f(x)\)가 연속일 때 \(a\le x\le b\)에 대해,
\[\frac d{dx}\int_a^xf(t)\,dt=f(x).\]
\item[증명)]
\(S(x)=\int_a^xf(t)\dt\)라고 하자.
\(S(x)\)의 미분을 생각하기 위해 \(x\)의 증분인 \(\delx\)와 \(S\)의 증분인 \(\Delta S=S(x+\delx)-S(x)\)를 고려하자.
\begin{enumerate}[(a)]
\item
\(f(x)\ge0\)일 때\\ 
\(\delx>0\)이면 연속함수 \(f(x)\)는 구간 \([x, x+\delx]\)에서 최대값 \(M\)과 최소값 \(m\)을 갖는다.
따라서
\[m\delx\le S(x+\delx)-S(x)=\int_{x}^{x+\delx}f(t)\dt\le M\delx.\]
각변을 \(\delx\)로 나누고 \delx를 \(0\)으로 보내는 극한을 취하면 \(m\)과 \(M\)은 모두 \(f(x)\)로 수렴하므로
\[f(x)=m\le\lim_{\delx\to0+}\frac{S(x+\delx)-S(x)}\delx\le M=f(x).\]
따라서
\[\lim_{\delx\to0+}\frac{S(x+\delx)-S(x)}\delx=f(x).\]
\par
\delx<0이면 마찬가지로 \([x+\delx,x]\)에서
\[m(-\delx)\le -S(x+\delx)+S(x)\le M(-\delx).\]
같은 논리에 의해
\[\lim_{\delx\to0-}\frac{S(x+\delx)-S(x)}\delx=f(x).\]
따라서 \(\frac d{dx}\int_a^xf(t)\,dt=S'(x)=f(x).\)
\item
\(f(x)<0\)일 때\\
\((-f)(x)=-f(x)\)로 정의된 함수 \(-f\)는 \((-f)(x)\ge0\)를 만족하므로 (a)에 의해,
\[\frac d{dx}\int_a^x(-f)(t)\,dt=(-f)(x).\]
따라서
\[\frac d{dx}\int_a^xf(t)\,dt=-\frac d{dx}\int_a^x(-f)(t)\,dt=-(-f)(x)=f(x).\]
\item
\(f(x)\)가 어떤 구간에서는 양수를, 어떤 구간에서는 음수를 취할 때, 구간을 나누어 생각하면 같은 결과를 얻을 수 있다.
\end{enumerate}
\item[미적분의 기본정리 (2)]
\([a,b]\)에서 \(f(x)\)가 연속이고 \(F'(x)=f(x)\)라고 하면,
\[\intt f(x)\dx=\left[F(x)\right]_a^b=F(b)-F(a).\]
\item[증명)]
미적분의 기본정리 (1)에 의해,
\[\int_a^xf(t)\dt=F(x)+C.\]
\(x=a\)를 대입하면 \(C=-F(a)\)를 얻고, 다시 \(x=b\)를 대입하면
\[
\intt f(x)\dx=F(b)-F(a).
\]
\item[기호)]
정적분을 나타낼 때, 경우에 따라 다양한 기호들 중 하나를 선택해서 사용하면 된다;
\begin{align*}
&\left[F(x)\right]_a^b
=\left.F(x)\right]_a^b
=\left.F(x)\right|_a^b\\
=&\left[F(x)\right]_{x=a}^{x=b}
=\left.F(x)\right]_{x=a}^{x=b}
=\left.F(x)\right|_{x=a}^{x=b}.
\end{align*}
\ed

\subsection{정적분의 계산}
\bd
\item[성질)]
부정적분과 마찬가지로 정적분도 선형적(linear)이다;
임의의 두 함수 \(f\), \(g\)와 임의의 두 실수 \(p\), \(q\)에 대해, \[\intt(pf(x)+qg(x))\,dx=p\intt f(x)\,dx+q\intt g(x)\,dx.\]
특히,
\begin{enumerate}[(1)]\tightlist
\item
\(\intt(f(x)+g(x))\,dx=\intt f(x)\,dx+\intt g(x)\,dx\).
\item
\(\intt(f(x)-g(x))\,dx=\intt f(x)\,dx-\intt g(x)\,dx\).
\item
\(\intt(kf(x))\,dx=k\intt f(x)\,dx\).
\end{enumerate}
\item[증명)]
(1)과 (3)을 증명한 후 조합하면 된다.
\item[성질)]
임의의 세 실수 \(a\), \(b\), \(c\)에 대해,
\[
\int_a^bf(x)\dx+\int_b^cf(x)\dx=\int_a^cf(x)\dx.
\]
\item[증명)]
\(f(x)\)의 한 부정적분 \(F(x)\)에 대해
\[좌변=[F(b)-F(a)]+[F(c)-F(b)]=F(c)-F(a)=우변.\]
\ed

\subsection{치환적분법과 부분적분법}
\bd
\item[치환적분법)]
\([a,b]\)에서 연속인 함수 \(f(x)\)에 대해 미분가능한 함수 \(g(t)\)의 도함수 \(g'(t)\)가 구간 \([\alpha,\beta]\)에서 연속이고 \(a=g(\alpha)\), \(b=g(\beta)\)일 때,
\[\intt f(x)\,dx=\inttt f(g(t))g'(t)\,dt\]
\item[증명)]
\(F'(x)=f(x)\)이면,
\begin{align*}
\inttt f(g(t))g'(t)\,dt
&=\inttt\frac d{dt}F(g(t))\dt\\
&=\left[F(g(t))\right]_{t=\alpha}^{t=\beta}=\left[F(x)\right]_{x=a}^{x=b}
=\intt f(x)\,dx.
\end{align*}
\item[부분적분법)]
두 함수 \(f(x)\), \(g(x)\)가 미분가능하고, \(f'(x)\), \(g'(x)\)가 연속이면,
\[\intt f(x)g'(x)\dx=[f(x)g(x)]_a^b-\intt f'(x)g(x)\dx.\]
\item[증명)]
부정적분의 부분적분법 식을 적용해 곱의 미분법 식을 정적분하면 된다.
\ed

\newpage
%새 節
\section*{미분이 포함된 간단한 방정식}
다음에서 \(f(x)\)는 미분가능한 함수, \(a\), \(b\)는 실수, \(g(x)\), \(h(x)\)는 연속함수이다.
\begin{enumerate}[(1)]
\item
\(f'(x)=af(x)\)\\
\(f(x)\neq0\)인 \(x\)에 대해 \(\frac{f'(x)}{f(x)}=a\)이고, 양변을 적분하면
\(\ln\left|f(x)\right|=ax+C\).
따라서 \(f(x)=\pm e^{ax+C}=Ae^{ax}(A\neq0)\).
\(f(x)\)의 연속성에 의해, 모든 \(x\)에 대해 \(f(x)=0\)이거나 모든 \(x\)에 대해 \(f(x)=Ae^{ax}(A\neq0)\)이다.
정리하면,
\[f(x)=Ae^{ax},\]
\(A\)는 임의의 실수.
\item
\(f'(x)=af(x)+b(a\neq0)\)\\
\(f(x)\neq-\frac ba\)인 \(x\)에 대해 \(\frac{f'(x)}{f(x)+\frac ba}=a\)이고, 양변을 적분하면 \(\ln|f(x)+\frac ba|=ax+C\).
따라서 \(f(x)+\frac ba=\pm e^{ax+C}=Ae^{ax}(A\neq0)\).
\(f(x)+\frac ba\)의 연속성에 의해, 모든 \(x\)에 대해 \(f(x)+\frac ba=0\)이거나 모든 \(x\)에 대해 \(f(x)+\frac ba=Ae^{ax}(A\neq0)\)이다.
정리하면,
\[f(x)=Ae^{ax}-\frac ba,\]
\(A\)는 임의의 실수.
\item
\(f'(x)+g(x)f(x)=h(x)\).(적분인자integrating factor를 곱하는 방법)\\
양변에 \(k(x)\)를 곱했을 때 좌변이 \((f(x)k(x))'\)와 같게 되도록 \(k(x)\)를 찾으면, 즉,
\[f'(x)k(x)+g(x)f(x)k(x)=f'(x)k(x)+f(x)k'(x)\]
에서
\[g(x)k(x)=k'(x).\]
가 성립해야 하므로 양변을 \(k(x)\)로 나누고 적분하면
\[\ln|k(x)|=\int g(x)\dx.\]
\(g(x)\)의 한 부정적분을 \(G(x)\)라고 하면
\[\ln|k(x)|=G(x)+C.\]
\(C=0\), \(k(x)>0\)이라고 가정하면
\[k(x)=e^{G(x)}.\tag{\(\ast\)}\]
로 택할 수 있다.
한편, 원래 식은
\[(f(x)k(x))'=h(x)k(x)\]
이었으므로 양변을 적분하면
\[f(x)k(x)=\int h(x)k(x)\dx+C',\]
\(C'\)는 임의의 실수.
\((\ast)\)를 넣어 정리하면 \(k(x)\neq0\)이므로,
\[f(x)=\frac1{k(x)}\left[\int h(x)k(x)\dx+C'\right]=e^{-G(x)}\left[\int h(x)k(x)\dx+C'\right].\]

\subsection*{예제1 :~\(f'(x)+xf(x)=x\)}
\(f'(x)k(x)+xf(x)k(x)=(xk(x))'\)가 되는 \(k(x)\)를 찾으면
\[xk(x)=k'(x)\]
에서
\[\ln|k(x)|=\frac12x^2+C.\]
\(k(x)>0\), \(C=0\)을 가정하면
\[k(x)=e^{\frac12x^2}.\]
이제 원래 식인 \((f(x)k(x))'=xk(x)=xe^{\frac12x^2}\)에서
\[f(x)k(x)=\int xe^{\frac12x^2}\dx=e^{\frac12x^2}+C',\]
이므로 (\(C'\)는 임의의 실수)
\[f(x)=\frac1{k(x)}\left[e^{\frac12x^2}+C'\right]=e^{-\frac12x^2}\left[e^{\frac12x^2}+C'\right]=1+C'e^{-\frac12x^2}.\]
\end{enumerate}
\end{document}