\documentclass{article}
\usepackage[a4paper,margin=3cm,footskip=.5cm]{geometry}
\usepackage{amsmath,amssymb,amsthm,amsfonts,mdframed,kotex}
\newcounter{num}
\newcommand{\bp}
{\stepcounter{num}
\begin{mdframed}
[frametitle={최상위\thenum},skipabove=10pt,skipbelow=10pt]}
\newcommand{\ep}
{\vspace{0.1\textheight}
\par
\end{mdframed}}
\newcommand{\parall}{\mathbin{\!/\mkern-5mu/\!}}
\mdfsetup{nobreak=true}

\title{미리 : 06 중간고사 대비(3학년 1학기)(1)}
\date{\today}
\author{}

\begin{document}
\maketitle
\newpage

\bp
음이 아닌 두 정수 \(a\), \(b\)에 대하여 \(\sqrt a+\sqrt b\le2\)이고, \(\sqrt a+\sqrt b\)가 정수가 되는 순서쌍 \((a,b)\)를 모두 구하여라.
\ep

\bp
음이 아닌 두 정수 \(a\), \(b\)에 대하여 \(\sqrt a+\sqrt b\le3\)이고, \(\sqrt a+\sqrt b\)가 정수가 되는 순서쌍 \((a,b)\)를 모두 구하여라.
\ep

\bp
\(\sqrt{891-81a}\)가 자연수일 때, 자연수 \(a\)의 값의 합을 구하여라.
\ep

\bp
\(\sqrt{384-12x}\)가 자연수일 때, 자연수 \(x\)의 값의 합을 구하여라.
\ep

\bp
\(\sqrt{\frac{252}x}\)가 자연수가 되는 가장 작은 자연수 \(x\)의 값을 구하여라.
\ep

\bp
\(\sqrt{\frac{21600}x}\)이 정수가 되는 정수 \(x\)의 개수를 구하여라.
\ep

\bp
\(A=3\sqrt5-2\sqrt2\), \(B=2\sqrt{10}-3\)의 대소관계를 정하여라.
\ep

\bp
\(a=\sqrt{11}+\sqrt{19}\), \(b=\sqrt{10}+\sqrt{20}\), \(c=\sqrt{18}+\sqrt{12}\)의 대소관계를 정하여라.
\ep

\bp
\(100\) 이하의 자연수 \(n\)에 대하여 \(\sqrt n\)이 무리수가 되는 \(n\)의 개수를 구하여라.
\ep

\bp
\(100\) 이하의 자연수 \(n\)에 대하여 \(\sqrt{2n}\)이 무리수가 되는 \(n\)의 개수를 구하여라.
\ep

\bp
\(100\) 이하의 자연수 \(n\)에 대하여 \(\sqrt{3n}\)이 무리수가 되는 \(n\)의 개수를 구하여라.
\ep

\bp
\(100\) 이하의 자연수 \(n\)에 대하여 \(\sqrt{5n}\)이 무리수가 되는 \(n\)의 개수를 구하여라.
\ep

\bp
\(100\) 이하의 자연수 \(n\)에 대하여 \(\sqrt n\), \(\sqrt{2n}\)이 모두 무리수가 되는 \(n\)의 개수를 구하여라.
\ep

\bp
\(100\) 이하의 자연수 \(n\)에 대하여 \(\sqrt n\), \(\sqrt{3n}\)이 모두 무리수가 되는 \(n\)의 개수를 구하여라.
\ep

\bp
\(100\) 이하의 자연수 \(n\)에 대하여 \(\sqrt n\), \(\sqrt{2n}\), \(\sqrt{3n}\)이 모두 무리수가 되는 \(n\)의 개수를 구하여라.
\ep

\bp
\(100\) 이하의 자연수 \(n\)에 대하여 \(\sqrt n\), \(\sqrt{2n}\), \(\sqrt{3n}\), \(\sqrt{5n}\)이 모두 무리수가 되는 \(n\)의 개수를 구하여라.
\ep

\bp
\(1000\) 이하의 자연수 \(n\)에 대하여 \(\sqrt n\), \(\sqrt{2n}\), \(\sqrt{3n}\), \(\sqrt{5n}\)이 모두 무리수가 되는 \(n\)의 개수를 구하여라.
\ep

\bp
연립방정식
\begin{gather*}
\sqrt2x+\sqrt3y=1\\
\sqrt3x-\sqrt2y=1
\end{gather*}
을 풀어라.
\ep

\bp
\(a>0\), \(b>0\)일 때, 다음의 대소 관계를 정하여라.
\begin{quote}
\[\sqrt a+\sqrt b,\quad\sqrt{a+b}\]
\end{quote}
\ep

\bp
\(a>0\), \(b>0\), \(x>0\), \(y>0\)일 때, 다음의 대소 관계를 정하여라.
\begin{quote}
\[\sqrt{a^2+b^2}\sqrt{x^2+y^2},\quad ax+by\]
\end{quote}
\ep

\bp
\(\sqrt{3a}+\sqrt{b}=5\)가 성립하도록 하는 자연수 \(a\), \(b\)에 대하여 \(\sqrt{2a^2+b}\)의 정수 부분을 \(x\), 소수 부분을 \(y\)라고 할 때, \(x-y\)의 값을 구하여라.
\ep

\bp
\(\sqrt{\frac{54}{n^2}}\)가 유리수가 되도록 하는 자연수 \(n\)의 최솟값을 구하여라.
\ep

\bp
\(5\sqrt{1\times2\times3\times4\times5}\)의 정수 부분의 자릿수를 구하여라.
\ep

\bp
\(5\sqrt{\frac{1\times2\times3\times\cdots\times10}{(1\times2\times3\times4\times5)^2}}\)
\ep

\bp
\(125\sqrt{1\times2\times3\times\cdots\times10}\)의 정수 부분의 자릿수를 구하여라.
\ep

\bp
자연수 \(n\)에 대하여 다음 두 조건을 만족하는 \(x\)의 값의 합이 \(6\)일 때, \(n\)의 값을 구하여라.
\begin{quote}
(가) \(nx\)는 자연수이다.\\
(나) \(\sqrt{nx}\)의 정수 부분은 \(2\)이다.
\end{quote}
\ep

\bp
자연수 \(n\)에 대하여 다음 두 조건을 만족하는 \(x\)의 값의 합이 \(28\)일 때, \(n\)의 값을 구하여라.
\begin{quote}
(가) \(nx\)는 자연수이다.\\
(나) \(\sqrt{nx}\)의 정수 부분은 \(3\)이다.
\end{quote}
\ep

\bp
자연수 \(n\)에 대하여 다음 두 조건을 만족하는 \(x\)의 값의 합이 \(36\)일 때, \(n\)의 값을 구하여라.
\begin{quote}
(가) \(nx\)는 자연수이다.\\
(나) \(\sqrt{nx}\)의 정수 부분은 \(4\)이다.
\end{quote}
\ep

\bp
\(\sqrt{n^2+3}=m\)이 되도록 하는 자연수 \(m\), \(n\)의 순서쌍 \((m,n)\)을 구하여라.
\ep

\bp
\(\sqrt{n^2+15}=m\)이 되도록 하는 자연수 \(m\), \(n\)의 순서쌍 \((m,n)\)을 구하여라.
\ep

\bp
\(\sqrt{n^2+77}=m\)이 되도록 하는 자연수 \(m\), \(n\)의 순서쌍 \((m,n)\)을 구하여라.
\ep

\bp
\(\sqrt{n^2+85}=m\)이 되도록 하는 자연수 \(m\), \(n\)의 순서쌍 \((m,n)\)을 구하여라.
\ep

\end{document}