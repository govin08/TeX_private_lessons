\documentclass{oblivoir}
\usepackage{amsmath,amssymb,amsthm,kotex,multicol}

\usepackage[skipabove=10pt,innertopmargin=10pt,nobreak=true]{mdframed}

\newcounter{num}
\newcommand{\theo}[1]
{\noindent\refstepcounter{num}\textbf{정리 \arabic{num}) #1}\par\noindent}
\newcommand{\prob}[1]
{\bigskip\bigskip\noindent\refstepcounter{num}\textbf{문제 \arabic{num})} #1\par\noindent}
\newcommand{\proo}
{\bigskip\noindent\textsf{증명)}}

\newcommand{\pb}[1]%\Phantom + fBox
{\fbox{\phantom{\ensuremath{#1}}}}

\renewcommand{\arraystretch}{1.5}

\newcommand{\procedure}[1]{\begin{mdframed}\vspace{#1\textheight}\end{mdframed}}

\title{이차방정식의 근과 계수와의 관계}
\date{\today}
\author{}

\begin{document}
\maketitle

%%
\section{이차방정식의 근과 계수와의 관계}
%
\prob{이차방정식 \(x^2-6x+5=0\)에 대하여 다음 물음에 답하여라.}
\begin{enumerate}[(1)]
\item
이차방정식의 두 근을 구하여라.
\item
두 근의 합을 구하여라.
\item
두 근의 곱을 구하여라.
\item
\(a=1\), \(b=-6\), \(c=5\)라고 두었을 때, \(-\frac ba\)와 \(\frac ca\)의 값을 구하여라.
\end{enumerate}

%
\prob{이차방정식 \(2x^2+5x+2=0\)에 대하여 다음 물음에 답하여라.}
\begin{enumerate}[(1)]
\item
이차방정식의 두 근을 구하여라.
\item
두 근의 합을 구하여라.
\item
두 근의 곱을 구하여라.
\item
\(a=2\), \(b=5\), \(c=2\)라고 두었을 때, \(-\frac ba\)와 \(\frac ca\)의 값을 구하여라.
\end{enumerate}

\begin{mdframed}
%
\theo{이차방정식의 근과 계수와의 관계}
이차방정식 \(ax^2+bx+c=0\) (\(a\neq0\))의 두 근을 \(\alpha\), \(\beta\)라고 할 때 다음 식이 성립한다.
\[\alpha+\beta= -\frac ba,\qquad\alpha\beta=\frac ca.\]
\end{mdframed}

%%
\section{증명}
이차방정식 \(ax^2+bx+c=0\) (\(a\neq0\))의 근의 공식은
\[x=\frac{-b\pm\sqrt{b^2-4ac}}{2a}\]
이다.
두 근 \(\alpha\), \(\beta\)를
\[
\alpha=\frac{-b+\sqrt{b^2-4ac}}{2a}
\beta=\frac{-b-\sqrt{b^2-4ac}}{2a}
\]
로 둘 수 있다.
그러면
\begin{align*}
\alpha+\beta
&=\frac{-b+\sqrt{b^2-4ac}}{2a}+\frac{-b-\sqrt{b^2-4ac}}{2a}\\
&=\frac{\pb{-b+\sqrt{b^2-4ac}-b-\sqrt{b^2-4ac}}}{2a}\\
&=\frac{-2b}{2a}\\
&=-\frac ba
\end{align*}
이다.
또한, 합차공식 [\((X+Y)(X-Y)=X^2-Y^2\)]을 활용하면
\begin{align*}
\alpha\beta
&=\frac{-b+\sqrt{b^2-4ac}}{2a}\times\frac{-b-\sqrt{b^2-4ac}}{2a}\\
&=\frac{(-b+\sqrt{b^2-4ac})(-b-\sqrt{b^2-4ac})}{4a^2}\\
&=\frac{\pb{(-b)}^2-\pb{(\sqrt{b^2-4ac})}^2}{4a^2}\\
&=\frac{b^2-(b^2-4ac)}{4a^2}\\
&=\frac{4ac}{4a^2}\\
&=\frac ca
\end{align*}
을 얻는다.\footnote{이 증명을 세 번 정도 공책에 써보세요!}

\newpage
\begin{multicols}{2}
%%
\section*{답}
\setcounter{num}0

%
\prob{}
\begin{enumerate}[(1)]
\item
\(x=1\), \(x=5\)
\item
\(6\)
\item
\(5\)
\item
\(-\frac ba=6\), \(\frac ca=5\)
\end{enumerate}

%
\prob{}
\begin{enumerate}[(1)]
\item
\(x=-\frac12\), \(x=-2\)
\item
\(-\frac52\)
\item
\(1\)
\item
\(-\frac ba=-\frac52\), \(\frac ca=1\)
\end{enumerate}
\end{multicols}
\begin{align*}
\alpha+\beta
&=\frac{-b+\sqrt{b^2-4ac}}{2a}+\frac{-b-\sqrt{b^2-4ac}}{2a}\\
&=\frac{\fbox{$-b+\sqrt{b^2-4ac}-b-\sqrt{b^2-4ac}$}}{2a}\\
&=\frac{-2b}{2a}\\
&=-\frac ba
\end{align*}
이다.
또한, 합차공식 [\((X+Y)(X-Y)=X^2-Y^2\)]을 활용하면
\begin{align*}
\alpha\beta
&=\frac{-b+\sqrt{b^2-4ac}}{2a}\times\frac{-b-\sqrt{b^2-4ac}}{2a}\\
&=\frac{(-b+\sqrt{b^2-4ac})(-b-\sqrt{b^2-4ac})}{4a^2}\\
&=\frac{\fbox{$(-b)^2$}-\fbox{$\sqrt{b^2-4ac}^2$}}{4a^2}\\
&=\frac{b^2-(b^2-4ac)}{4a^2}\\
&=\frac{4ac}{4a^2}\\
&=\frac ca
\end{align*}

\end{document}