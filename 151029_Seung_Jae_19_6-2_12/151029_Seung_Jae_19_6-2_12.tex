\documentclass{oblivoir}
\usepackage{amsmath,amssymb,amsthm,kotex,mdframed,paralist,kswrapfig}

\newcounter{num}
\newcommand{\prob}
{\bigskip\noindent\refstepcounter{num}\textbf{문제 \arabic{num})}\par}

\newcommand{\ans}{{\raggedleft\textbf{답 : (\qquad\qquad\qquad\qquad\qquad\qquad)}
\par}\bigskip\bigskip}


%%%
\begin{document}
\Large

\title{승재 19 - 6학년 2학기 - 12}
\author{}
\date{\today}
\maketitle
%\tableofcontents

\prob
다음 \(\square\)에 들어갈 알맞은 숫자를 쓰세요.

\begin{enumerate}[(1)]
\item
\(5:3=\square:\frac67\)
\item
\(9:\frac15=3:\square\)
\item
\(\frac12:\frac13=6:\square\)
\item
\(\frac23:1=4:\square\)
\item
\(2:\frac32=\frac34:\square\)
\item
\(\frac76:\frac52=\square:10\)
\item
\(\frac{13}2:52=\square:6\)
\item
\(\frac{48}5:7=\frac{12}7:\square\)
\item
\(4.9:2.8=14:\square\)
\item
\(400:150=0.16:\square\)
\item
\(270:126=\frac94:\square\)
\end{enumerate}

\newpage

\prob
다음을 가장 간단한 자연수의 비로 나타내어 보시오.

\begin{enumerate}[(1)]
\item
\(16:42\)
\item
\(32:56\)
\item
\(32:72\)
\item
\(21:27\)
\item
\(\frac12:\frac27\)
\item
\(\frac15:\frac56\)
\item
\(1\frac12:2\frac13\)
\item
\(1.5:2\frac34\)
\item
\(0.48:\frac9{10}\)
\item
\(0.56:\frac{16}{25}\)
\item
\(2.1:3\frac35\)
\item
\(0.27:0.63\)
\item
\(1.68:0.96\)
\item
\(\frac1{25}:\frac1{42}\)
\item
\(\frac1{60}:\frac2{75}\)
\end{enumerate}

\end{document}