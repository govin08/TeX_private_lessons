\documentclass[a4paper,twocolumn]{oblivoir}
\usepackage{amsmath,amssymb,amsthm,kotex,graphicx,mdframed}
\usepackage{fapapersize}
\usefapapersize{210mm,297mm,10mm,*,10mm,*}
\newcounter{num}
\newcommand{\defi}[1]
{\bigskip\noindent\refstepcounter{num}\textbf{정의 \arabic{num}) #1}\par\noindent}
\newcommand{\theo}[1]
{\bigskip\noindent\refstepcounter{num}\textbf{정리 \arabic{num}) #1}\par\noindent}
\newcommand{\exam}[1]
{\bigskip\noindent\refstepcounter{num}\textbf{예시 \arabic{num}) #1}\par\noindent}
\newcommand{\prob}[1]
{\bigskip\noindent\refstepcounter{num}\textbf{문제 \arabic{num}) #1}\par\noindent}
\newcommand{\proo}
{\bigskip\textsf{증명)}\par}

\newcommand{\pb}[1]%\Phantom + fBox
{\fbox{\phantom{\ensuremath{#1}}}}

\title{수지 : 함수의 연속 복습}
\date{\today}
\author{}
%%%%
\begin{document}
\maketitle

%
\begin{mdframed}[frametitle=함수의 극한]
극한값 \(\displaystyle\lim_{x\to a}f(x)\)가 존재한다.\\[2em]
\(\displaystyle\iff\lim_{x\to a-}f(x)=\lim_{x\to a+}f(x)\)\\
\begin{enumerate}[(1)]
\item
좌극한값 \(\displaystyle\lim_{x\to a-}f(x)\)가 존재한다.
\item
우극한값 \(\displaystyle\lim_{x\to a+}f(x)\)가 존재한다.
\item
좌극한값과 우극한값이 같다.
\end{enumerate}
\end{mdframed}

%
\begin{mdframed}[frametitle=함수의 연속]
\(f(x)\)가 \(x=a\)에서 연속이다.\\[2em]
\(\displaystyle\lim_{x\to a}f(x)=f(a)\)\\
\begin{enumerate}[(1)]
\item
함숫값 \(f(a)\)가 존재한다.
\item
극한값 \(\displaystyle\iff\lim_{x\to a}f(x)\)가 존재한다.
\item
함숫값과 극한값이 같다.
\end{enumerate}
\end{mdframed}
\newpage

%
\exam{\(f(x)=x^2-1\)}
{\centering\includegraphics[width=0.4\textwidth]{55}}
\begin{enumerate}[(1)]
\item
좌극한값 \(\displaystyle\lim_{x\to2-}f(x)\)가 존재한다 / 존재하지 않는다.\\
존재한다면 그 값은 \pb{3}이다.
\item
우극한값 \(\displaystyle\lim_{x\to2+}f(x)\)가 존재한다 / 존재하지 않는다.\\
존재한다면 그 값은 \pb{3}이다.
\item
\(\displaystyle\lim_{x\to2}f(x)\)가 존재한다 / 존재하지 않는다.\\
존재한다면 그 값은 \pb{3}이다.
\item
\(f(2)\)의 값은 존재한다 / 존재하지 않는다.\\
존재한다면 그 값은 \(\pb{3}\)이다.
\item
\(f(x)\)는 \(x=2\)에서 연속 / 연속이 아니다.
\end{enumerate}
\newpage

%
\exam{\(f(x)=[x]\)}
{\centering\includegraphics[width=0.4\textwidth]{55}}
\begin{enumerate}[(1)]
\item
좌극한값 \(\displaystyle\lim_{x\to2-}f(x)\)가 존재한다 / 존재하지 않는다.\\
존재한다면 그 값은 \pb{3}이다.
\item
우극한값 \(\displaystyle\lim_{x\to2+}f(x)\)가 존재한다 / 존재하지 않는다.\\
존재한다면 그 값은 \pb{3}이다.
\item
\(\displaystyle\lim_{x\to2}f(x)\)가 존재한다 / 존재하지 않는다.\\
존재한다면 그 값은 \pb{3}이다.
\item
\(f(2)\)의 값은 존재한다 / 존재하지 않는다.\\
존재한다면 그 값은 \(\pb{3}\)이다.
\item
\(f(x)\)는 \(x=2\)에서 연속 / 연속이 아니다.
\end{enumerate}
\newpage

%
\exam{\(f(x)=|x-2|\)}
{\centering\includegraphics[width=0.4\textwidth]{55}}
\begin{enumerate}[(1)]
\item
좌극한값 \(\displaystyle\lim_{x\to2-}f(x)\)가 존재한다 / 존재하지 않는다.\\
존재한다면 그 값은 \pb{3}이다.
\item
우극한값 \(\displaystyle\lim_{x\to2+}f(x)\)가 존재한다 / 존재하지 않는다.\\
존재한다면 그 값은 \pb{3}이다.
\item
\(\displaystyle\lim_{x\to2}f(x)\)가 존재한다 / 존재하지 않는다.\\
존재한다면 그 값은 \pb{3}이다.
\item
\(f(2)\)의 값은 존재한다 / 존재하지 않는다.\\
존재한다면 그 값은 \(\pb{3}\)이다.
\item
\(f(x)\)는 \(x=2\)에서 연속 / 연속이 아니다.
\end{enumerate}
\newpage

%
\exam{\(f(x)=\frac{x-2}{|x-2|}\)}
{\centering\includegraphics[width=0.4\textwidth]{55}}
\begin{enumerate}[(1)]
\item
좌극한값 \(\displaystyle\lim_{x\to2-}f(x)\)가 존재한다 / 존재하지 않는다.\\
존재한다면 그 값은 \pb{3}이다.
\item
우극한값 \(\displaystyle\lim_{x\to2+}f(x)\)가 존재한다 / 존재하지 않는다.\\
존재한다면 그 값은 \pb{3}이다.
\item
\(\displaystyle\lim_{x\to2}f(x)\)가 존재한다 / 존재하지 않는다.\\
존재한다면 그 값은 \pb{3}이다.
\item
\(f(2)\)의 값은 존재한다 / 존재하지 않는다.\\
존재한다면 그 값은 \(\pb{3}\)이다.
\item
\(f(x)\)는 \(x=2\)에서 연속 / 연속이 아니다.
\end{enumerate}
\newpage

%
\prob{\(f(x)=\frac{x^2-3x+2}{x-2}\)}
{\centering\includegraphics[width=0.4\textwidth]{55}}
\begin{enumerate}[(1)]
\item
좌극한값 \(\displaystyle\lim_{x\to2-}f(x)\)가 존재한다 / 존재하지 않는다.\\
존재한다면 그 값은 \pb{3}이다.
\item
우극한값 \(\displaystyle\lim_{x\to2+}f(x)\)가 존재한다 / 존재하지 않는다.\\
존재한다면 그 값은 \pb{3}이다.
\item
\(\displaystyle\lim_{x\to2}f(x)\)가 존재한다 / 존재하지 않는다.\\
존재한다면 그 값은 \pb{3}이다.
\item
\(f(2)\)의 값은 존재한다 / 존재하지 않는다.\\
존재한다면 그 값은 \(\pb{3}\)이다.
\item
\(f(x)\)는 \(x=2\)에서 연속 / 연속이 아니다.
\end{enumerate}
\newpage

\end{document}