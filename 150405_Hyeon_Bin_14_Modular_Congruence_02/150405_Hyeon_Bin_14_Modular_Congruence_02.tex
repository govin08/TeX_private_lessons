\documentclass{article}
\usepackage{amsmath,amssymb,amsthm,kotex,mdframed,paralist}
\newcounter{num}
\newcommand{\defi}[1]
{\bigskip\noindent\refstepcounter{num}\textbf{정의 \arabic{num}) #1}\par}
\newcommand{\theo}[1]
{\bigskip\noindent\refstepcounter{num}\textbf{정리 \arabic{num}) #1}\par}
\newcommand{\exam}[1]
{\bigskip\noindent\refstepcounter{num}\textbf{예시 \arabic{num}) #1}\par}
\newcommand{\prob}[1]
{\bigskip\noindent\refstepcounter{num}\textbf{연습문제 \arabic{num}) #1}\par}

\newcommand{\pr}
{\begin{mdframed}[skipabove=10pt,skipbelow=10pt]
증명)
\vspace{0.5\textheight}
\par
\end{mdframed}
\par
}
\newcommand{\sol}
{\begin{mdframed}[skipabove=10pt]
풀이)
\vspace{0.5\textheight}
\par
\end{mdframed}
\par
}

\renewcommand{\proofname}{증명)}
\newcommand{\mo}[1]{\ensuremath{\:(\text{mod}\:\:#1)}}

%%%
\begin{document}

\title{현빈 : 14 일차합동식 추가문제들}
\author{}
\date{\today}
\maketitle
%\tableofcontents
%
%\newpage

%%
\setcounter{section}{4}
\setcounter{num}{29}
\section{일차합동식 추가문제들}

%
\prob{}
다음 일차합동식들의 해를 구하시오.
\begin{enumerate}[(1)]
\item
\(3x\equiv2\mo{7}\)
\item
\(6x\equiv3\mo{9}\)
\item
\(17x\equiv14\mo{21}\)
\item
\(15x\equiv9\mo{25}\)
\item
\(128x\equiv833\mo{1001}\)
\item
\(987x\equiv610\mo{1597}\)
\end{enumerate}

%\sol

%%
%\section{중국인의 나머지 정리(The Chinese Remainder Theorem)}
%
%%
%\exam{}
%다음과 같은 문제를 풀어보자.
%
%\begin{quote}
%5으로 나누었을 때의 나머지가 1이고, 6으로 나누었을 때의 나머지가 2이고, 7로 나누었을 때의 나머지가 3인 숫자를 구하여라.
%\end{quote}
%
%이 문제는 다음 연립 일차 합동식을 푸는 것과 같다.
%\begin{gather}
%x\equiv1\mo{5}\\
%x\equiv2\mo{6}\\
%x\equiv3\mo{7}
%\end{gather}
%
%(1)에 의해 \(x=5t+1\)이다.
%이를 (2)에 넣으면
%\[5t+1\equiv2\mo{6}\]
%이고 1을 이항하면
%\[5t\equiv1\mo{6}\]
%이다.
%따라서 \(t\)는 \(5\)의 역원인 \(5\)이다(mod 6).
%즉
%\[t\equiv5\mo{6}\]
%혹은
%\[t=6u+5\]
%이다.
%따라서
%\[x=5(6u+5)+1=30u+26\]
%이다.
%이것을 다시 (3)에 넣으면
%\[30u+26\equiv3\mo{7}\]
%이고 \(30\equiv2\mo{7}\)이므로
%\[2u\equiv30u\equiv3-26\equiv-23\equiv5\mo{7}\]
%이다.
%양변에 \(2^{-1}(\equiv4)\)를 곱하면
%\[u\equiv20\equiv6\mo{7}\]
%따라서
%\[u=7v+6\]
%이고
%\[x=30(7v+6)+26=210v+206\]
%이다.
%즉
%\[x\equiv206\mo{210}\]
%인 모든 정수이다.
%
%
%%
%\theo{중국인의 나머지 정리(The Chinese Remainder Theorem)}
%\pr
\end{document}