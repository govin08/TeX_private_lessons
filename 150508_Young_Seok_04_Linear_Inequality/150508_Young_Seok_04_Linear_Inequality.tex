\documentclass{article}
\usepackage{amsmath,amssymb,amsthm,kotex,mdframed,paralist,chngcntr}

\newcounter{num}
\newcommand{\defi}[1]
{\bigskip\noindent\refstepcounter{num}\textbf{정의 \arabic{num}) #1}\par}
\newcommand{\theo}[1]
{\bigskip\noindent\refstepcounter{num}\textbf{정리 \arabic{num}) #1}\par}
\newcommand{\exam}[1]
{\bigskip\noindent\refstepcounter{num}\textbf{예시 \arabic{num}) #1}\par}
\newcommand{\prob}[1]
{\bigskip\noindent\refstepcounter{num}\textbf{문제 \arabic{num}) #1}\par}


\renewcommand{\proofname}{증명)}
\newcommand{\mo}[1]{\ensuremath{\:(\text{mod}\:\:#1)}}
\counterwithout{subsection}{section}


%%%
\begin{document}

\title{영석 : 04 일차부등식}
\author{}
\date{\today}
\maketitle
\tableofcontents
\newpage

%%
\subsection{부등호와 부등식}
%
\defi{부등호와 부등식}
\[<,>,\le,\ge\]
등의 기호를 \textbf{부등호}라고 한다.
또 부등호의 양 옆에 수나 식이 나타나있는 경우 이것을 \textbf{부등식}이라고 한다.

%
\exam{}
(1) 
\begin{gather*}
1<2\\
2x\ge3\\
3(x+4)+5>2x-2\\
x+y>2\\
x^2>1\\
x^2-x-2\le0
\end{gather*}
등은 모두 부등식이다.

(2)
\[1<2\]
는 `\(1\)이 \(2\)보다 작다'라고 읽는다.
혹은 `\(1\)이 \(2\) 미만이다'라고 읽는다.
\[2>1\]
은 `\(2\)가 \(1\)보다 크다'라고 읽는다.
혹은 `\(2\)가 \(1\) 초과이다'라고 읽는다.
\[1\le2\]
은 `\(1\)이 \(2\)보다 작거나 같다'라고 읽는다.
혹은 \(1\)이 \(2\) 이하이다.'라고 읽는다.
\[2\ge1\]
는 `\(2\)가 \(1\)보다 크거나 같다'라고 읽는다.
혹은 \(2\)가 \(1\) 이상이다.'라고 읽는다.

(3)
허수에 관해서는 수의 대소관계를 나타내지 않는다.
예를 들어 
\begin{gather*}
i<1\\
i^3+i^2-2i\ge0
\end{gather*}
등의 식은 쓰지 않는다.
\bigskip

부등식은 다음과 같은 성질을 만족한다.

%
\defi{부등식의 성질 1}\label{prop1}
\(a\), \(b\), \(c\)가 실수이고 \(a<b\)이면\\
(1) \(a+c<b+c\)\\
(2) \(a-c<b-c\)

%
\defi{부등식의 성질 2}\label{prop2}
\(a\), \(b\), \(c\)가 실수이고 \(a<b\), \(c>0\)이면\\
(1) \(ac<bc\)\\
(2) \(\frac ac<\frac bc\)

%
\defi{부등식의 성질 3}\label{prop3}
\(a\), \(b\), \(c\)가 실수이고 \(a<b\), \(c<0\)이면\\
(1) \(ac>bc\)\\
(2) \(\frac ac>\frac bc\)

정리하면, 부등식의 양 변에 같은 값을 더하거나 빼도 부등호의 방향은 그대로 유지된다.
부등식의 양 변에 같은 값을 곱하거나 나눌 경우, 곱하거나 나누는 수가 양수이면 부등호의 방향이 그대로 유지되고, 곱하거나 나누는 수가 음수이면 부등호의 방향이 바뀐다.

이 결과는 `\(>\)'이 `\(\le\)'와 \(\ge\)'로 바뀐 경우에도 여전히 비슷하게 성립한다.

%
\exam{}
(1)
\[1<3\]
에서 양변에 \(2\)를 더한 
\[3<5\]
와, 양변에 \(2\)를 뺀
\[-1<1\]
은 모두 성립한다.
(부등호의 방향은 바뀌지 않았다.)

(2)
\[1<3\]
에서 양변에 \(2\)를 곱한
\[2<6\]
과, 양변에 \(2\)를 나눈
\[\frac12<\frac32\]
는 모두 성립한다.
(부등호의 방향은 바뀌지 않았다.)

(3)
\[1<3\]
에서 양변에 \(-2\)를 곱한
\[-2>-6\]
과, 양변에 \(-2\)를 나눈
\[-\frac12>-\frac32\]
는 모두 성립한다.
(부등호의 방향이 바뀌었다.)

%%
\subsection{일차부등식}
%
\defi{일차부등식}
정의 \ref{prop1}, \ref{prop2}, \ref{prop3}의 결과로 부등식에서도 등식에서와 마찬가지로 이항을 마음대로 할 수 있음을 알 수 있다.
예를 들어
\[4x+3<x+1\]
과 같은 부등식의 경우 양변에 1을 빼어
\[4x+2<x\]
를 얻고, 다시 양변에 \(x\)를 빼어
\[3x+2<0\]
를 얻을 수 있다.
이처럼, 이항하여
\begin{gather*}
ax+b>0\\
ax+b<0\\
ax+b\ge0\\
ax+b\le0
\end{gather*}
꼴(\(a\), \(b\)는 실수, \(a\neq0\))로 나타낼 수 있는 부등식을 \textbf{일차부등식}이라고 한다.
또 이러한 일차부등식을 만족하는 \(x\)의 범위를 찾는 과정을 `일차부등식을 푼다'라고 한다.

%
\prob{}
다음 중 일차부등식을 찾아보자.
\begin{gather}
x>1\\
2x\ge x\\
x+1<x+2\\
0\cdot x+3>1\\
x^2+2x+1\ge0\\
3(x+4)<2(x+1)
\end{gather}

%
\exam{일차방정식의 풀이}
(1)
일차방정식
\[2x+3>0\]
를 풀어보자.

양변에서 \(3\)을 빼면
\[2x>-3\]
이다.
양변을 \(2\)로 나누면
\[x>-\frac32\]
이다.

(2)
일차방정식
\[-3x-5\ge0\]
을 풀어보자.

양변에서 \(5\)를 더하면
\[-3x\ge5\]
이다.
양변을 \(-3\)으로 나누면
\[x\le-\frac53\]
이다.

(3) 일차방정식
\[2x+4<3x+2\]
를 풀어보자.

양변에서 \(3x\)를 빼면
\[-x+4<2\]
이다.
양변에서 \(4\)를 빼면
\[-x<-2\]
이다.
양변에 \(-1\)을 곱하면
\[x>2\]
이다.

%
\prob{}
다음 일차부등식을 풀어라.
\setcounter{equation}{0}
\begin{gather}
x+4\le0\\
x-\frac12<0\\
2x+1>0\\
3x+4\ge0\\
\frac32x+\frac43<0\\
-x+7>0\\
-2x+4\ge0\\
-\frac12x+3<0
\end{gather}

%
\prob{}
다음 일차부등식을 풀어라.
\setcounter{equation}{0}
\begin{gather}
x+2\ge0\\
x-\frac34>0\\
3x+4<0\\
7x+24\ge0\\
\frac12x+\frac78>0\\
-x>0\\
-5x+3\ge0\\
-\frac32x+6>0
\end{gather}

%
\prob{}
다음 일차부등식을 풀어라.
\setcounter{equation}{0}
\begin{gather}
2x+3\ge x+2\\
2x-3<3x-3\\
-3x+4\le x+3\\
-x+1>2x+5\\
\frac12x+4<\frac32+1\\
\frac13x+\frac13>x+1\\
\frac25x+1<\frac13x+\frac23\\
2(x+3)+4>3(x-1)+1\\
\frac12(x+4)+1\ge2(x-2)+2
\end{gather}

%
\prob{}
다음 일차부등식을 풀어라.
\setcounter{equation}{0}
\begin{gather}
3x+1\ge x+1\\
2x-5<3x-6\\
-2x+1\le x-3\\
-2x+2\ge x+3\\
\frac32x+1>\frac12+1\\
\frac32x+\frac23>2x+2\\
\frac14x+2>\frac15x+\frac45\\
x+4<2(x+3)+1\\
\frac13(2x+4)+2\le3(x-4)+2
\end{gather}

%%
\subsection{부정과 불능}

%
\exam{}
(1)
부등식
\[x+3>x+2\]
의 경우 우변의 모든 항을 좌변으로 이항하면
\[0\cdot x+1>0\]
혹은
\[1>0\]
이 되므로 엄밀히 말해 일차부등식은 아니다.
그래도 부등식의 해를 구해보면, 즉 \(x\)가 어떤 값일 때 이 부등식이 성립하는 지 살펴보면, \(x\)가 어떤 값을 가지더라도 위의 부등식이 성립함을 알 수 있다.
따라서 구하고자 하는 부등식의 해는 `\(x\)는 모든 실수'이다.

이처럼 부등식의 해가 실수 전체인 경우, \textbf{부정(不定)}이라고 말한다.

(2)
반면 부등식
\[2x+4\ge 2x+5\]
의 경우 정리하면
\[0\cdot x-1\ge0\]
혹은
\[-1\ge0\]
이 되어 역시 일차부등식은 아니다.
이 부등식의 경우 어떤 실수 \(x\)에 대해서도 성립하지 않기 때문에 `해는 없다.'

이처럼 부등식을 만족하는 실수 \(x\)의 값이 존재하지 않을 경우 \textbf{불능(不能)}이라고 말한다.


%
\exam{}
다음 부등식의 해를 구하시오.

\setcounter{equation}{0}
\begin{gather}
2x+3>2x+2\\
3x+1\le x+2x\\
2(x+1)\ge 3(x+3)-(x+7)\\
\frac12(x+4)+\frac12(x-2)<x+4\\
\frac{2x+3}6<\frac13 x+\frac12
\end{gather}

%
\exam{}
다음 부등식이 부정(해가 실수 전체)이기 위한 상수 \(a\)의 범위를 구하시오.
\setcounter{equation}{0}
\begin{gather}
x+a>x\\
2x-a<2x+a\\
\frac12(x+a)\ge\frac12x+1
\end{gather}

%
\exam{}
다음 부등식이 불능(해가 없음)이기 위한 상수 \(a\)의 값을 구하시오.
\setcounter{equation}{0}
\begin{gather}
ax+1>2\\
(a+2)x+4<0\\
\frac{2x+1}4>\frac{ax+3}2
\end{gather}
\end{document}