\documentclass{oblivoir}
\usepackage{amsmath,amssymb,amsthm,kotex,paralist,kswrapfig}

\usepackage[skipabove=10pt,innertopmargin=10pt]{mdframed}

\usepackage{tabto,pifont}
\TabPositions{0.2\textwidth,0.4\textwidth,0.6\textwidth,0.8\textwidth}
\newcommand\tabb[5]{\par\bigskip\noindent
\ding{172}\:{\ensuremath{#1}}
\tab\ding{173}\:\:{\ensuremath{#2}}
\tab\ding{174}\:\:{\ensuremath{#3}}
\tab\ding{175}\:\:{\ensuremath{#4}}
\tab\ding{176}\:\:{\ensuremath{#5}}}

\usepackage{enumitem}
\setlist[enumerate]{label=(\arabic*)}

\newcounter{num}
\newcommand{\defi}[1]
{\noindent\refstepcounter{num}\textbf{정의 \arabic{num}) #1}\par\noindent}
\newcommand{\theo}[1]
{\noindent\refstepcounter{num}\textbf{정리 \arabic{num}) #1}\par\noindent}
\newcommand{\exam}[1]
{\bigskip\bigskip\noindent\refstepcounter{num}\textbf{예시 \arabic{num}) #1}\par\noindent}
\newcommand{\prob}[1]
{\bigskip\bigskip\noindent\refstepcounter{num}\textbf{문제 \arabic{num}) #1}\par\noindent}
\newcommand{\proo}
{\bigskip\textsf{증명)}\par}

\newcommand{\ans}{
{\par
\raggedleft\textbf{답 : (\qquad\qquad\qquad\qquad\qquad\qquad)}
\par}\bigskip\bigskip}

\newcommand{\pb}[1]%\Phantom + fBox
{\fbox{\phantom{\ensuremath{#1}}}}

\newcommand\ba{\,|\,}

\let\oldsection\section
\renewcommand\section{\clearpage\oldsection}

%%%%
\begin{document}

\title{민형 : 01 도함수의 활용(1)}
\author{}
\date{\today}
\maketitle
\tableofcontents
\newpage

%%

%%
\section{복습}
\begin{mdframed}
%
\defi{미분계수}
함수 \(f\)와 실수 \(a\)에 대해
\[
\lim_{h\to0}\frac{f(a+h)-f(a)}h
\]
가 존재하면 `\(f\)가 \(x=a\)에서 \textbf{미분가능}하다'라고 말하고 이 값을 함수 \(f\)의 \(x=a\)에서의 \textbf{미분계수}라고 말한다.
구간 \(I\)에 대해, 모든 \(a\in I\)에 대해 \(f\)가 \(x=a\)에서 미분가능하면 `\(f\)가 구간 \(I\)에서 \textbf{미분가능}하다'라고 말한다.
\end{mdframed}

%
\exam{}\label{differential_coefficient}
함수 \(f(x)=x^2\)에 대해 \(x=2\)에서의 미분계수는
\[
\lim_{h\to0}\frac{f(2+h)-f(2)}h=\lim_{h\to0}\frac{(2+h)^2-2^2}h
=\lim_{h\to0}\frac{h^2+4h}h=\lim_{h\to0}(h+4)=4
\]
이다.
따라서 \(f\)는 \(x=2\)에서 미분가능하다.

\begin{mdframed}
%
\defi{도함수}
함수 \(f\)가 구간 \(I\)에서 미분가능할 때
\[
f'(x)=\lim_{h\to0}\frac{f(x+h)-f(x)}h
\]
로 함수 \(f'\)를 정의할 수 있다.
이 함수를 함수 \(f\)의 \textbf{도함수}라고 말한다.
\end{mdframed}

%
\exam{}\label{derivative}
예제 \ref{differential_coefficient}에서의 함수 \(f(x)=x^2\)는 실수 전체에서 미분가능하다.
이때의 도함수 도함수 \(f'\)는
\[
f'(x)=\lim_{h\to0}\frac{f(x+h)-f(x)}h=\lim_{h\to0}\frac{(x+h)^2-x^2}h
=\lim_{h\to0}\frac{h^2+2hx}h=\lim_{h\to0}(h+2x)=2x
\]
로 정의된다.
죽
\[f'(x)=2x\]
이다.

\begin{mdframed}
%
\defi{Leibniz Notation}
함수 \(f\)에 대해 \(y=f(x)\)라고 하면 \(f\)의 도함수 \(f'(x)\)를
\[
\frac{{d}y}{{d}x},\quad\frac{{d}f(x)}{{d}x},\quad\frac{{d}}{{d}x}f(x),\quad\left(f(x)\right)'
\]
등으로 쓰기도 한다.
함수 \(f\)의 \(x=a\)에서의 미분계수 \(f'(a)\)는
\[
\left.\frac{{d}y}{{d}x}\right|_{x=a},\quad\left.\frac{{d}f(x)}{{d}x}\right|_{x=a},\quad\left.\frac{{d}}{{d}x}f(x)\right|_{x=a},\quad\left.\left(f(x)\right)'\right|_{x=a}
\]
등으로 쓰기도 한다.
\end{mdframed}

%
\exam{}
예시 \ref{differential_coefficient}, 예시 \ref{derivative}의 결과를
\begin{gather*}
\frac{d(x^2)}{dx}=2x,\quad\left.\frac{d(x^2)}{dx}\right|_{x=2}=4\\
(x^2)'=2x,\quad \left.(x^2)'\right|_{x=2}=4
\end{gather*}
등으로 쓸 수도 있다.

\begin{mdframed}
%
\theo{도함수의 성질}
실수 \(c\), 미분가능한 함수 \(f\), \(g\)에 대해
\begin{enumerate}[label=(\emph{\alph*})]
\item
\((c)'=0\)
\item
\(\left(cf(x)\right)'=cf'(x)\)
\item
\(\left(f(x)+g(x)\right)'=f'(x)+g'(x)\)
\item
\(\left(f(x)-g(x)\right)'=f'(x)-g'(x)\)
\item
\(\left\{f(x)g(x)\right\}'=f(x)g'(x)+f'(x)g(x)\)
\item
\(\left(\frac{f(x)}{g(x)}\right)'=\frac{f'(x)g(x)-f(x)g'(x)}{\left\{g(x)\right\}^2}\)
\end{enumerate}
\end{mdframed}

%\begin{mdframed}
%%
%\theo{곱의 미분법과 몫의 미분법}
%미분가능한 함수 \(f\), \(g\)에 대해 (\(g(x)\neq0\))
%\begin{align*}
%\left\{f(x)g(x)\right\}'		&=f(x)g'(x)+f'(x)g(x)\\
%\left(\frac{f(x)}{g(x)}\right)'	&=\frac{f'(x)g(x)-f(x)g'(x)}{\left\{g(x)\right\}^2}
%\end{align*}
%\end{mdframed}

\begin{mdframed}
%
\theo{여러 가지 함수의 도함수}
\begin{enumerate}[label=(\emph{\alph*})]
\item
\((x^a)'=ax^{a-1}\)\qquad(\(a\)는 실수)
\item
\((e^x)'=e^x\)
\item
\((a^x)'=a^x\ln a\)\qquad(\(a>0\))
\item
\((\ln x)'=\frac1x\)
\item
\((\log_ax)'=\frac1{x\ln a}\)\qquad(\(a>0,\quad a\neq0\))
\item
\((\sin x)'=\cos x\)
\item
\((\cos x)'=-\sin x\)
\item
\((\tan x)'=\sec^2x\)
\item
\((\cot x)'=-\csc^2x\)
\item
\((\sec x)'=\tan x\sec x\)
\item
\((\csc x)'=-\cot x\csc x\)
\end{enumerate}
\end{mdframed}

\begin{mdframed}
%
\theo{합성함수의 미분법}
두 함수 \(f\), \(g\)가 미분가능할 때 합성함수 \((f\circ g)\)의 도함수는
\[
\left\{f(g(x))\right\}'=f'(g(x))g'(x)
\]
이다.
\(y=f(u)\), \(u=g(x)\)라고 하면
\[
\frac{dy}{dx}=\frac{dy}{du}\cdot\frac{du}{dx}\]
로 쓸 수도 있다.
\end{mdframed}

\begin{mdframed}
%
\theo{역함수의 미분법}
함수 \(f\)와 그 역함수 \(f^{-1}\)가 모두 미분가능할 때,
\[
\frac{dy}{dx}=\frac1{\frac{dx}{dy}}
\]
이다.
\(y=f(x)\)라고 하면
\[
\left(f^{-1}(x)\right)=\frac1{f'(y)}
\]
로 쓸 수도 있다.
\end{mdframed}

\prob{}
다음 함수의 도함수를 구하여라.
\begin{enumerate}
\item
\(y=\sqrt x\)
\begin{mdframed}
\(y'=(x^{\frac12})'=\frac12x^{1-\frac12}=\frac12x^{-\frac12}=\frac1{2\sqrt x}\)
\end{mdframed}
%\ans
{\par
\raggedleft\textbf{답 : (\qquad\qquad\(\frac1{2\sqrt x}\)\qquad\qquad)}
\par}\bigskip\bigskip
\item
\(y=\frac1{x-1}\)
\begin{mdframed}
\vspace{0.1\textheight}
\end{mdframed}
\ans
\clearpage
\item
\(y=3\sin^3 x\)
\begin{mdframed}
\vspace{0.08\textheight}
\end{mdframed}
\ans
\item
\(y=e^{-2x}\)
\begin{mdframed}
\vspace{0.08\textheight}
\end{mdframed}
\ans
\item
\(y=\left(\log_2 x\right)^3\)
\begin{mdframed}
\vspace{0.08\textheight}
\end{mdframed}
\ans
\item
\(y=\frac{\cos x}x\)
\begin{mdframed}
\vspace{0.08\textheight}
\end{mdframed}
\ans
\end{enumerate}

%%
\section{접선의 방정식}
\begin{mdframed}
%
\theo{}\label{formula}
점 \((x_1,y_1)\)을 지나고 기울기가 \(m\)인 직선의 방정식은
\[y=m(x-x_1)+y_1\]
이다.
\end{mdframed}
\proo
기울기가 \(m\)인 직선의 방정식은
\[y=mx+n\]
이다.
\(n\)을 구하기 위해 \((x_1,y_1)\)을 대입하면
\[y_1=mx_1+n\]
이고, 따라서
\[n=y_1-mx_1\]
이다.
그러므로 구하는 방정식은 
\[y=mx+y_1-mx_1\]
이다.
이것을 정리하면 위의 식이 된다.

\begin{mdframed}
%
\theo{}
함수 \(y=f(x)\)가 \(x=a\)에서 미분가능할 때, 곡선 \(y=f(x)\) 위의 점 \(P(a,f(a))\)에서의 접선의 방정식은
\[y=f'(a)(x-a)+f(a)\]
이다.
\end{mdframed}

\vspace{-10pt}
\proo
접선의 기울기는 \(f'(a)\)이다.
따라서 정리 \ref{formula}에 대입하면 위의 결과를 얻는다.

\end{document}