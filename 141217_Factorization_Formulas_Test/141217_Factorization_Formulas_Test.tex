\documentclass{article}
\usepackage{amsmath,amssymb,amsthm,kotex,paralist,mathrsfs,centernot,marvosym}
%\newcounter{num}[section]
%\newcommand{\defi}[1]
%{\bigskip\noindent\refstepcounter{num}\textbf{정의 \arabic{section}. \arabic{num}) #1}\par}
%\newcommand{\theo}[1]
%{\bigskip\noindent\refstepcounter{num}\textbf{정리 \arabic{section}. \arabic{num}) #1}\par}
%\newcommand{\axio}[1]
%{\bigskip\noindent\refstepcounter{num}\textbf{공리 \arabic{section}. \arabic{num}) #1}\par}
%
%\newcommand{\notiff}{%
%  \mathrel{{\ooalign{\hidewidth$\not\phantom{"}$\hidewidth\cr$\iff$}}}}
%\newcommand{\LHS}{\text{LHS}}
%\newcommand{\RHS}{\text{RHS}}
%\newcommand{\irange}{\ensuremath{1\le i\le n}}
%\newcommand{\jrange}{\ensuremath{1\le j\le n}}
%\newcommand{\bb}[2]{\ensuremath{(^{#1}_{#2})}}
%\newcommand{\cc}[2]{\ensuremath{_{#1}C_{#2}}}
%
%\renewcommand{\figurename}{그림.}
%\renewcommand{\proofname}{증명.}
%\renewcommand{\contentsname}{목차}
%\renewcommand\emph{\textbf}

%%%
\begin{document}

\title{인수분해 공식 테스트}
\author{}
\date{\today}
\maketitle
%\tableofcontents

%%

%
1.
실수 \(a\), \(b\), \(c\), \(d\), \(x\)에 대해 다음을 전개하여라.
\begin{enumerate}[(1)]
\item
\((3a+\frac12b)^2=\)
\item
\((ax+b)(cx+d)=\)
\item
\((x-\frac1x)^2=\)
\item
\((x+a)(x+b)(x+c)=\)
\item
\((x-a)(x-b)(x-c)=\)
\item
\((a+b+c)^2=\)
\item
\((a-b+c)^2=\)
\item
\((a+b)^3=\)
\item
\((a-b)^3=\)
\item
\((x-1)(x^4+x^3+x^2+x+1)=\)
\item
\((x+1)(x^4-x^3+x^2-x+1)=\)
\end{enumerate}

2.
실수 \(a\), \(b\), \(c\), \(x\)에 대해 다음을 인수분해하여라.
\begin{enumerate}[(1)]
\item
\(a^3+b^3=\)
\item
\(a^3-b^3=\)
\item
\(x^4-1=\)
\item
\(x^6-1=\)
\item
\(x^7-1=\)
\item
\(a^3+b^3+c^3-3abc=\)
\item
\(a^4+a^2b^2+b^4=\)
\end{enumerate}

3.
실수 \(a\), \(b\), \(c\), \(x\), \(y\)에 대하여 다음을 인수분해하여라.
\begin{enumerate}[(1)]
\item
\(x^3-3x^2+3x-2=\)
\item
\(2x^3+x^2+x-1=\)
\item
\(x^3-4x^2+x+6=\)
\item
\(3x^3-5x^2-34x+24=\)
\item
\(x^4+4=\)
\item
\(x^4+5x^2+9=\)
\item
\(a^2(b-c)+b^2(c-a)+c^2(a-b)=\)
\item
\(x^2+xy-6y^2+x+13y-6=\)
\end{enumerate}

4.
서로 다른 세 정수 \(a\), \(b\), \(c\)(\(c<b<a\))에 대해
\[a^2+b^2+c^2-ab-bc-ca=2\]
이고 \(c=1\)일 때, 세 정수의 합은?

\bigskip
5.
세 변의 길이가 각각 \(a\), \(b\), \(c\)인 삼각형에서
\[(a-b)c^4+(a+b)(a^4-b^4)=2(a^3-b^3)c^2\]
이 성립할 때 이 삼각형은 어떤 삼각형인가?
\end{document}