\documentclass{oblivoir}
\usepackage{amsmath,amssymb,amsthm,kotex,tabu,graphicx,pifont}

\usepackage[skipabove=10pt,innertopmargin=10pt,nobreak=true]{mdframed}

\newcounter{num}
\newcommand{\theo}[1]
{\noindent\refstepcounter{num}\textbf{정리 \arabic{num}) #1}\par\noindent}
\newcommand{\prob}[1]
{\bigskip\bigskip\noindent\refstepcounter{num}\textbf{문제 \arabic{num})} #1\par\noindent}
\newcommand{\proo}
{\bigskip\noindent\textsf{증명)}}

\newcommand{\pb}[1]%\Phantom + fBox
{\fbox{\phantom{\ensuremath{#1}}}}

\renewcommand{\arraystretch}{1.5}

\newcommand{\procedure}[1]{\begin{mdframed}\vspace{#1\textheight}\end{mdframed}}

\newcommand\ovv[1]{\ensuremath{\overline{#1}}}
\newcommand\ov[2]{\ensuremath{\overline{#1#2}}}

%%% Title
\title{자연수의 거듭제곱의 합}
\date{\today}
\author{}

\begin{document}
\maketitle

\begin{mdframed}[frametitle=자연수의 거듭제곱의 합]
\begin{enumerate}[(1)]
\item
\(\displaystyle\sum_{k=1}^nk=\frac{n(n+1)}2\)
\item
\(\displaystyle\sum_{k=1}^nk^2=\frac{n(n+1)(2n+1)}6\)
\item
\(\displaystyle\sum_{k=1}^nk^3=\left\{\frac{n(n+1)}2\right\}^2\)
\end{enumerate}
\end{mdframed}

\proo
\begin{enumerate}[(1)]
\item
첫항이 1이고 공차가 1인 등비수열의 합이므로
\[
\sum_{k=1}^nk=\frac{n\{2a+(n-1)d\}}2
=
\frac{n\{2+(n-1)\cdot1\}}2
=
\frac{n(n+1)}2
\]
\item
항등식 \((k+1)^3=k^3+3k^2+3k+1\)에 \(k=1,2,3,\cdots,n\)을 차례로 대입하면
\begin{align*}
2^3&=1^3+3\times1^2+3\times1+1\\
3^3&=2^3+3\times2^2+3\times2+1\\
4^3&=3^3+3\times3^2+3\times3+1\\
%5^2&=4^2+2\times4+1\\
&\vdots\\
(n+1)^3&=n^3+3\times n^2+3\times n+1
\end{align*}
이 식들을 모두 더하면
\[(n+1)^3=1^3+3(1^2+2^2+\cdots+n^2)+3(1+2+\cdots+n)+(1+1+\cdots+1)\]
이다.
이것을 정리하면
\begin{gather*}
n^3+3n^2+3n+1=1+3\sum_{k=1}^nk^2+3\cdot\frac{n(n+1)}2+n\\
2n^3+6n^2+6n+2=2+6\sum_{k=1}^nk^2+3n^2+3n+2n\\
6\sum_{k=1}^nk^2=2n^3+3n^2+n\\
\sum_{k=1}^nk^2=\frac{n(n+1)(2n+1)}6
\end{gather*}
이 된다.
\end{enumerate}

\prob{항등식 \((k+1)^4=k^4+4k^3+6k^2+4k+1\)을 이용하여 (3)을 증명하여라.}
\\[-30pt]
\procedure{.41}
\end{document}