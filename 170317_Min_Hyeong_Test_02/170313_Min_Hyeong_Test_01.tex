\documentclass[a4paper]{oblivoir}
\usepackage{amsmath,amssymb,kotex,kswrapfig,mdframed,paralist}
\usepackage{fapapersize}
\usefapapersize{210mm,297mm,20mm,*,20mm,*}

\usepackage{tabto,pifont}
\TabPositions{0.2\textwidth,0.4\textwidth,0.6\textwidth,0.8\textwidth}
\newcommand\tabb[5]{\par\noindent
\ding{172}\:{\ensuremath{#1}}
\tab\ding{173}\:\:{\ensuremath{#2}}
\tab\ding{174}\:\:{\ensuremath{#3}}
\tab\ding{175}\:\:{\ensuremath{#4}}
\tab\ding{176}\:\:{\ensuremath{#5}}}

\usepackage{tabu}

\pagestyle{empty}
%%% Counters
\newcounter{num}

%%% Commands
\newcommand\defi[1]
{\bigskip\par\noindent\stepcounter{num} \textbf{정의 \thenum) #1}\par\noindent}
\newcommand\theo[1]
{\bigskip\par\noindent\stepcounter{num} \textbf{정리 \thenum) #1}\par\noindent}
\newcommand\exam[1]
{\bigskip\par\noindent\stepcounter{num} \textbf{예시 \thenum) #1}\par\noindent}
\newcommand\prob[1]
{\bigskip\par\noindent\stepcounter{num} \textbf{문제 \thenum) #1}\par\noindent}

\newcommand\pb[1]{\ensuremath{\fbox{\phantom{#1}}}}

\newcommand\an[1]{\bigskip\par\noindent\textbf{문제 #1)}\par\noindent}

\newcommand\ba{\ensuremath{\:|\:}}

\newcommand\vs[1]{\vspace{60pt}}

%%% Meta Commands
\let\oldsection\section
\renewcommand\section{\clearpage\oldsection}

\let\emph\textsf

\begin{document}
\begin{center}
\LARGE민형, 미니테스트 02
\end{center}
\begin{flushright}
날짜 : 2017년 \(\pb3\)월 \(\pb{10}\)일 \(\pb{월}\)요일
,\qquad
제한시간 : \pb{17년}분
,\qquad
점수 : \pb{20} / \pb{20}
\end{flushright}

%미3확8
\prob{}
\(0\le x<2\pi\)에서 \(4\cos x=-3\)을 만족시키는 모든 \(x\)의 값의 합을 \(\theta\)라 할 때, \(\sin\theta\)의 값은?
\tabb{-\frac{\sqrt2}2}{-\frac12}0{\frac12}{\frac{\sqrt2}2}
\vs

%미3확9
\prob{}
\(0\le x<\pi\)에서 부등식 \(2\cos\left(2x+\frac\pi3\right)<-1\)의 해가 \(\alpha<x<\beta\)일 때, \(\sin(\beta-\alpha)\)의 값을 구하시오.
\vs

%미3서1
\prob{}
이차방정식 \(2x^2-2ax+1=0\)의 두 근이 \(\sin\theta\), \(\cos\theta\)일 때, \(\sin^3\theta+\cos^3\theta\)의 값을 구하시오.
\big(단, \(0<\theta<\frac\pi2\), \(a\)는 상수\big)
\vs

%미3서3
\prob{}
\(\sin\left(\frac\pi2-x\right)+\sin(\pi-x)=\sin\left(\frac32\pi-x\right)+\sin(2\pi-x)\)
를 만족시키는 모든 \(x\)의 값을 \(\theta\)라 할 때, \(\cos\theta\)의 값을 구하시오. (단, \(0\le x<2\pi\))
\vs

%미3서4
\prob{}
\(0\le\theta\le2\pi\)일 때, 모든 실수 \(x\)에 대하여 부등식
\[\frac12x^2-(2\sin\theta+1)x+2>0\]
이 항상 성립하도록 하는 \(\theta\)의 값의 범위를 구하시오.
\vs

%미3고2
\prob{}
\(0\le x<2\pi\)일 때, 함수 \(f(x)=\cos^2\left(\frac\pi2-x\right)+2\cos^2x+2\sin(\pi+x)+1\)의 최댓값을 \(M\), 최솟값을 \(m\)이라 할 때, \(M+m\)의 값을 구하시오.
\vs


\clearpage
\begin{tabu}{|c|c|c|c|c|c|}
\hline
문제1&문제2&문제3&문제4&문제5&문제6
\\\hline
\ding{174}&\(\frac{\sqrt3}2\)&\(\frac{\sqrt2}2\)&\(0\)cm&\(0\le\theta<\frac\pi6\) 또는 \(\frac56\pi<\theta\le2\pi\)&\ding{175}
\\\hline
\end{tabu}

\end{document}