\documentclass{oblivoir}
\usepackage{amsmath,amssymb,amsthm,kotex,paralist,kswrapfig,tabu}

\usepackage[skipabove=10pt,skipbelow=10pt,innertopmargin=10pt]{mdframed}

\usepackage{tabto,pifont}
\TabPositions{0.2\textwidth,0.4\textwidth,0.6\textwidth,0.8\textwidth}
\newcommand\tabb[5]{\par\bigskip\noindent
\ding{172}\:{\ensuremath{#1}}
\tab\ding{173}\:\:{\ensuremath{#2}}
\tab\ding{174}\:\:{\ensuremath{#3}}
\tab\ding{175}\:\:{\ensuremath{#4}}
\tab\ding{176}\:\:{\ensuremath{#5}}}

\usepackage{enumitem}
\setlist[enumerate]{label=(\arabic*)}

\newcounter{num}
\newcommand{\defi}[1]
{\noindent\refstepcounter{num}\textbf{정의 \arabic{num}) #1}\par\noindent}
\newcommand{\theo}[1]
{\noindent\refstepcounter{num}\textbf{정리 \arabic{num}) #1}\par\noindent}
\newcommand{\exam}[1]
{\bigskip\bigskip\noindent\refstepcounter{num}\textbf{예시 \arabic{num}) #1}\par\noindent}
\newcommand{\prob}[1]
{\bigskip\bigskip\noindent\refstepcounter{num}\textbf{문제 \arabic{num}) #1}\par\noindent}
\newcommand{\proo}
{\bigskip\textsf{증명)}\par}

\newcommand{\procedure}[1]{\begin{mdframed}\vspace{#1\textwidth}\end{mdframed}}
\newcommand{\ans}{
{\par\raggedleft\textbf{답 : (\qquad\qquad\qquad\qquad\qquad\qquad)}\par}\bigskip\bigskip}
\newcommand\an[1]{\par\bigskip\noindent\textbf{문제 #1)}\\}

\newcommand{\pb}[1]%\Phantom + fBox
{\fbox{\phantom{\ensuremath{#1}}}}

\newcommand\ba{\,|\,}

\let\oldsection\section
\renewcommand\section{\clearpage\oldsection}
\counterwithout{subsection}{section}

\newenvironment{talign}
 {\let\displaystyle\textstyle\align}
 {\endalign}
\newenvironment{talign*}
 {\let\displaystyle\textstyle\csname align*\endcsname}
 {\endalign}

\let\emph\textsf

%\usepackage{fapapersize}
%\usefapapersize{210mm,297mm,15mm,15mm,15mm,15mm}
%%%%
\begin{document}

\title{민형 : 07 연속확률변수와 확률밀도함수}
\author{}
\date{\today}
\maketitle

%%
\subsection{이산확률변수}

%
\exam{}
동전을 두 번 던졌을 때, 앞면이 나온 횟수를 \(X\)라고 하면, \(X\)는 \(0\), \(1\), \(2\)의 세 값을 가질 수 있다.
이때,

\begin{figure}[h!]
\centering
\begin{tabu}spread 2in{X[2,C]||X[1,C]|X[1,C]|X[1,C]}\hline
\(X\)		&0		&1		&2	\\\hline
\(P(X=x)\)	&\(\frac14\)	&\(\frac12\)	&\(\frac14\)\\
\hline
\end{tabu}
\end{figure}
이다.
평균(기댓값)과 분산을 구하면
\begin{align*}
E(X)
&=\sum_{i=0}^2x_ip_i\\
&=0\times\frac14+1\times\frac12+2\times\frac14=1\\
V(X)
&=\sum_{i=0}^2(x_i-m)^2p_i\\
&=(0-1)^2\times\frac14+(1-1)^2\times\frac12+(1-0)^2\times\frac14=\frac12
\end{align*}

\clearpage
%%
\subsection{연속확률변수}

%
\exam{}
\(A=(2,0)\), \(B=(2,2)\)에 대해 \(\triangle OAB\) 내부의 한 점 \(Q\)를 임의로 잡을 때, \(Q\)의 \(x\)좌표를 \(X\)라고 하자.
\(X\)가 취할 수 있는 값의 범위는 \(0\le X\le 2\)이며, 예제 1과 달리 무한히 많은 값이 될 수 있다.
이때 다음 물음에 답하여라.
\begin{enumerate}
\item
\(P(X=1)\)을 구하여라.
\item
\(P(0\le X\le 1)\)를 구하여라.
\item
\(0\le a\le b\le 1\)인 \(a\), \(b\)에 대하여 \(P(a\le X\le b)\)를 구하여라.
\item
\(P(a\le x\le b)=\int_a^bf(x)\,dx\)가 되기 위해서는 \(f(x)=\pb{\frac12x}\)이다.
\end{enumerate}

%
\exam{}
철수는 12시에서 1시 사이의 아무때나 은행에 방문하기로 하였다.
철수가 은행에 방문한 시각을 12시 \(X\)분이라고 할 때, 다음 물음에 답하여라.
\begin{enumerate}
\item
\(P(X=10)\)을 구하여라.
\item
\(P(20\le X\le 30)\)를 구하여라.
\item
\(0\le a\le b\le 60\)인 \(a\), \(b\)에 대하여 \(P(a\le X\le b)\)를 구하여라.
\item
\(P(a\le x\le b)=\int_a^bf(x)\,dx\)가 되기 위해서는 \(f(x)=\pb{\frac1{60}}\)이다.
\end{enumerate}

%
\prob{}
예시 2, 3에서 \(X\)의 기댓값 \(E(X)\)를 구하여라.

\end{document}