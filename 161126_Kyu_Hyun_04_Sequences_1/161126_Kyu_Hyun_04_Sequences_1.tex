\documentclass{oblivoir}
\usepackage{amsmath,amssymb,amsthm,kotex,paralist,kswrapfig}

\usepackage[skipabove=10pt]{mdframed}

\usepackage{tabto,pifont}
\TabPositions{0.2\textwidth,0.4\textwidth,0.6\textwidth,0.8\textwidth}
\newcommand\tabb[5]{\par\noindent
\ding{172}{#1}
\tab\ding{173}{#2}
\tab\ding{174}{#3}
\tab\ding{175}{#4}
\tab\ding{176}{#5}}

\usepackage{enumitem}
%\setlist{noitemsep}
\setlist[enumerate]{label=(\arabic*)}

\newcounter{num}
\newcommand{\defi}[1]
{\bigskip\noindent\refstepcounter{num}\textbf{정의 \arabic{num}) #1}\par\noindent}
\newcommand{\theo}[1]
{\bigskip\noindent\refstepcounter{num}\textbf{정리 \arabic{num}) #1}\par\noindent}
\newcommand{\exam}[1]
{\bigskip\noindent\refstepcounter{num}\textbf{예시 \arabic{num}) #1}\par\noindent}
\newcommand{\prob}[1]
{\bigskip\noindent\refstepcounter{num}\textbf{문제 \arabic{num}) #1}\par\noindent}
\newcommand{\proo}
{\bigskip\textsf{증명)}\par}

\newcommand{\ans}{
{\par
\raggedleft\textbf{답 : (\qquad\qquad\qquad\qquad\qquad\qquad)}
\par}\bigskip\bigskip}

\newcommand{\pb}[1]%\Phantom + fBox
{\fbox{\phantom{\ensuremath{#1}}}}

%%%%
\begin{document}

\title{규현 : 03 수열(1)}
\author{}
\date{\today}
\maketitle
\tableofcontents
\newpage


%%
\section{수열}

%
\prob{}
다음 빈 칸에 알맞은 수를 넣어라.
\begin{enumerate}
\item
\(4\quad6\quad8\quad10\quad12\quad\pb{14}\quad16\) \tab\tab\tab\(\cdots\cdots\quad \{a_n\}\)
\item
\(1\quad3\quad9\quad27\quad81\quad\pb{243}\quad729\) \tab\tab\tab\(\cdots\cdots\quad \{b_n\}\)
\item
\(7\quad3\quad1\quad7\quad3\quad\pb{1}\quad7\)
\item
\(0\quad2\quad6\quad12\quad20\quad\pb{30}\quad42\) \tab\tab\tab\(\cdots\cdots\quad \{c_n\}\)
\item
\(\frac13\quad\frac14\quad\frac15\quad\frac16\quad\frac17\quad\pb{\frac18}\quad\frac19\)
\tab\tab\tab\(\cdots\cdots\quad\{d_n\}\)
\item
\(6\quad3\quad2\quad\frac32\quad\frac65\quad\pb{1}\quad\frac67\)
 \tab\tab\tab\(\cdots\cdots\quad\{e_n\}\)
\item
\(1\quad1\quad2\quad3\quad5\quad\pb{8}\quad13\)
\end{enumerate}

\begin{mdframed}[innertopmargin=-5pt]
%
\defi{수열}
위 문제에서처럼, 숫자들이 일정한 규칙에 의해 나열되어 있는 것을 \textbf{수열}이라고 한다.
그리고 수열을 구성하는 각각의 숫자들을 \textbf{항}이라고 한다.
예를 들어 첫번째 수열의 첫째항은 \(4\)이고 둘째항은 \(6\)이고 셋째항은 \(8\)이다.

수열을 나타낼 때에는 중괄호를 써서 \(\{a_n\}\), \(\{b_n\}\), \(\cdots\)와 같이 나타낸다.
첫번째 수열을 \(\{a_n\}\)이라고 나타낸다면 \(a_1=4\), \(a_2=6\), \(a_3=8\) 등으로 나타낼 수 있다.
\end{mdframed}

%
\prob{}
문제 1의 (2)에 나타난 수열을 \(\{b_n\}\)으로 나타내고, (4)에 나타난 수열을 \(\{c_n\}\)으로 나타낼 때, 다음 빈칸을 채우시오.
\begin{gather*}
b_1=\pb{1}\quad b_3=\pb{9}\quad b_5=\pb{81}\quad b_7=\pb{729}\\
c_2=\pb{3}\quad b_4=\pb{7}\quad b_6=\pb{1}\phantom{\quad b_7=72900}
\end{gather*}

\begin{mdframed}[innertopmargin=-5pt]
%
\defi{수열의 일반항}
문제 1의 (1)에서 
\[
a_1=4,\quad a_2=6,\quad a_3=8\quad a_4=10,\quad a_5=12,\quad a_6=14,\quad a_7=16
\]
이다.
따라서 자연수 \(n\)에 대해
\[a_n=2n+2\]
임을 유추할 수 있다.
이와 같은 \(a_n\)을 수열 \(\{a_n\}\)의 \textbf{일반항}이라고 부른다.
\end{mdframed}

%
\prob{}
문제 1의 (2), (4), (5), (6)의 일반항을 구하시오.
\begin{align*}
&b_1=1,\quad b_2=3,\quad b_3=9\quad b_4=27,\quad b_5=81,\quad b_6=243,\quad b_7=729\\
\therefore\:&b_n=\pb{3^{n-1}}\\
&c_1=0,\quad c_2=2,\quad c_3=6\quad c_4=12,\quad c_5=20,\quad c_6=30,\quad c_7=42\\
\therefore\:&c_n=\pb{n(n-1)}\\
&d_1=\frac13,\quad d_2=\frac14,\quad d_3=\frac15\quad d_4=\frac16,\quad d_5=\frac17,\quad
d_6=\frac18,\quad d_7=\frac19\\
\therefore\:&d_n=\pb{\frac1n}\\
&e_1=6,\quad e_2=3,\quad e_3=2\quad e_4=\frac32,\quad e_5=\frac65,\quad e_6=1,\quad e_7=\frac67\\
\therefore\:&e_n=\pb{\frac6n}\\
\end{align*}

\clearpage
%%
\section{등차수열}
문제 1의 (1)번 수열을 다시 보자.

\bigskip
\(4\quad6\quad8\quad10\quad12\quad14\quad16\) \tab\tab\(\cdots\cdots\quad\{a_n\}\)
%\[
%a_1=4,\quad a_2=6,\quad a_3=8\quad a_4=10,\quad a_5=12,\quad a_6=14,\quad a_7=16,\quad\cdots
%\]

\bigskip\noindent
이 수열은 항 사이의 차가 2로 일정하다;
\[a_2-a_1=2,\quad a_3-a_2=2,\quad a_4-a_3=2,\quad a_5-a_4=2,\quad a_6-a_5=2,\quad\cdots\]
%혹은, 첫째항부터 일정한 수를 더해서 얻어지는 수열이다 ;
%\[a_2=a_1+2\quad, a_3=a_2+2\quad, a_4=a_3+2\quad, a_5=a_4+2\quad, a_6=a_5+2\quad\cdots\]

이처럼, 인접한 항 사이의 차가 일정한 수열을 \textbf{등차수열}이라고 부른다.
%다시 말해, 등차수열이란 첫째항부터 일정한 수를 더해서 얻어지는 수열이다.
이때, 등차수열에서 인접한 항 사이의 차를 \textbf{공차}라고 부른다.
공차는 보통 \(d\)로 쓴다.

%
\begin{mdframed}[innertopmargin=-5pt]
\defi{등차수열}
수열 \(\{a_n\}\)이 다음 조건을 만족시키면 이 수열은 등차수열이다.
\[a_{n+1}-a_n=d.\quad(n\text{은 자연수})\]
\end{mdframed}

%
\prob{}
다음 수열들 중 등차수열인 것을 고르고, 등차수열인 경우 공차 \(d\)를 구하여라.
\begin{enumerate}
\item
\(1\quad3\quad5\quad7\quad9\quad11\quad13\)
\tab\qquad\qquad등차수열이다/아니다 : \(d=\pb{11}\)
\item
\(2\quad4\quad8\quad16\quad32\quad64\quad128\)
\tab\qquad\qquad등차수열이다/아니다 : \(d=\pb{11}\)
\item
\(-10\quad-7\quad-4\quad-1\quad2\quad5\quad8\)
\qquad\quad등차수열이다/아니다 : \(d=\pb{11}\)
\item
\(5\quad5\quad5\quad5\quad5\quad5\quad5\)
\tab\qquad\qquad등차수열이다/아니다 : \(d=\pb{11}\)
\item
\(1\quad0\quad1\quad0\quad1\quad0\quad1\)
\tab\qquad\qquad등차수열이다/아니다 : \(d=\pb{11}\)
\item
\(200\quad300\quad400\quad500\quad600\quad700\quad800\)
\quad등차수열이다/아니다 : \(d=\pb{11}\)
\item
\(2\quad4\quad6\quad2\quad4\quad6\quad2\)
\tab\qquad\qquad등차수열이다/아니다 : \(d=\pb{11}\)
\item
\(100\quad99\quad98\quad97\quad96\quad95\quad94\)
\qquad\qquad등차수열이다/아니다 : \(d=\pb{11}\)
\item
\(1\quad\frac12\quad\frac13\quad\frac14\quad\frac15\quad\frac16\quad\frac17\)
\tab\qquad\qquad등차수열이다/아니다 : \(d=\pb{11}\)
\item
\(0\quad-\frac13\quad-\frac23\quad-1\quad-\frac43\quad-\frac53\quad-2\)
등차수열이다/아니다 : \(d=\pb{11}\)
\end{enumerate}

%
\prob{}
다음 등차수열의 열번째 항을 구하여라.
\begin{enumerate}
\item
\(4\quad6\quad8\quad10\quad12\quad14\quad16\)
\tab\tab\(\cdots\cdots\quad \{a_n\}\)
\item
\(10\quad20\quad30\quad40\quad50\quad60\quad70\)
\tab\tab\(\cdots\cdots\quad \{b_n\}\)
\item
\(7\quad4\quad1\quad-2\quad-5\quad-8\quad-11\)
\tab\(\cdots\cdots\quad \{c_n\}\)
\item
\(50\quad43\quad36\quad29\quad22\quad15\quad8\)
\tab\tab\(\cdots\cdots\quad \{d_n\}\)
\item
\(3\quad\frac92\quad6\quad\frac{15}2\quad9\quad\frac{21}2\quad12\)
\tab\tab\(\cdots\cdots\quad \{e_n\}\)
\end{enumerate}

{\par
\raggedleft\textbf{답 :}
(1) \(a_{10}=\pb{22}\),\quad
(2) \(b_{10}=\pb{100}\),\quad
(3) \(c_{10}=\pb{-20}\),\\
(4) \(d_{10}=\pb{-13}\),\quad
(5) \(e_{10}=\pb{\frac{33}2}\)
\par}\bigskip\bigskip

%
\prob{}
문제 8에 제시된 등차수열의 일반항을 구하여라.\\
(1) \(a_n=\pb{2n+2}\)\\
(2) \(b_n=\pb{10n}\)\\
(3) \(c_n=\pb{10-3n}\),\\
(4) \(d_n=\pb{57-7n}\),\\
(5) \(e_n=\pb{\frac32+\frac32n}\)

%
\begin{mdframed}[innertopmargin=-5pt]
\theo{}
첫번째 항(=\(a_1\))이 \(a\)이고 공차가 \(d\)인 등차수열의 일반항은
\[a_n=a+(n-1)d\]
이다.
\end{mdframed}

\clearpage
\proo
초항이 \(a\)이고 공차가 \(d\)인 등차수열의 항을 나열해보면
\begin{align*}
a_1&=a\\
a_2&=a_1+d=a+d\\
a_3&=a_2+d=a+2d\\
a_4&=a_3+d=a+3d\\
a_5&=a_4+d=a+4d\\
&\vdots
\end{align*}
이다.
따라서
\[a_n=a+(n-1)d\]
이다.
\qed

%
\prob{}
문제 8에서\\
(1) \(a=4\), \(d=2\)이므로
\[a_n=4+(n-1)\times2=2n+2\]
이다.\\
(2) \(a=\pb{10}\), \(d=10\)이므로
\[a_n=\pb{10}+(n-1)\times10=10n\]
이다.\\
(3) \(a=7\), \(d=\pb{-3}\)이므로
\[a_n=7+(n-1)\times\pb{-3}=-3n+10\]
이다.\\
(4) \(a=\pb{50}\), \(d=\pb{-7}\)이므로
\[a_n=\pb{50}+(n-1)\times\pb{-7}=\pb{57-7n}\]
이다.\\
(5) \(a=3\), \(d=\pb{\frac32}\)이므로
\[a_n=\pb{3+(n-1)\times\frac32}=\pb{\frac32+\frac32n}\]
이다.
(문제 9의 결과와 비교해보자.)

%
\prob{}
다음 등차수열들의 일반항 \(a_n\)을 구하시오.
\begin{enumerate}
\item
\(-11,\quad-8,\quad-5,\quad-2,\quad\cdots\)
\item
\(6,\quad3,\quad0,\quad-3,\quad\cdots\)
\item
\(3,\quad6,\quad9,\quad12,\quad\cdots\)
\item
\(\frac13,\quad\frac12,\quad\frac23,\quad\frac56,\cdots\)
\end{enumerate}

%
\theo{등차중항}
세 숫자 \(a\), \(b\), \(c\)가 등차수열을 이룰 때, \(b\)를 \(a\)와 \(c\)의 \textbf{등차중항}이라고 한다.
이때 등차중항 \(b\)는 다음 조건을 만족한다.
\[b=\frac{a+c}2.\]

\proo
\(a\), \(b\), \(c\)가 등차수열을 이루므로, 인접한 항 사이의 차가 같다.
즉
\[b-a=c-b\]
이다.
이것을 \(b\)에 관한 식으로 정리하면
\[b=\frac{a+c}2.\]
이다.
\qed

\clearpage
%
\exam{}
\begin{enumerate}
\item
세 숫자
\[1,\quad x,\quad9\]
가 등차수열을 이룬다면, \(x=\frac{1+9}2=5\)이다.
\item
다섯 숫자
\[3,\quad x,\quad y,\quad z,\quad 19\]
가 등차수열을 이룬다고 하면,
\[3,\quad y,\quad19\]
가 등차수열을 이루므로 \(y=\frac{3+19}2=11\)이다.
또,
\[3,\quad x,\quad y(=11)\]
가 등차수열을 이루므로 \(x=\frac{3+11}2=7\)이고
\[y(=11),\quad z,\quad19\]
가 등차수열을 이루므로 \(z=\frac{11+19}2=15\)이다.
따라서 \(x=7\), \(y=11\), \(z=15\)이다.
\end{enumerate}

%
\prob{}
\begin{enumerate}
\item
세 숫자
\[2,\quad 8,\quad x\]
가 등차수열을 이룰 때, \(x\)의 값을 구하시오.
\item
다섯 숫자
\[4,\quad x,\quad 18,\quad y,\quad z\]
가 등차수열을 이룰 때, \(x\), \(y\), \(z\)의 값을 구하시오.
\end{enumerate}
\textbf{답 :} (1) \(x=\pb{14}\), (2) \(x=\pb{11}\), \(y=\pb{25}\), \(z=\pb{32}\)



%%
\section{등차수열의 합}

%
\prob{}
다음을 계산하시오.
\begin{enumerate}
\item
\(3+4+5+6+7=\pb{25}\)
\item
\(2+4+6+8+10+12=\pb{42}\)
\item
\(1+2+3+\cdots+10=\pb{55}\)
\end{enumerate}

%
\exam{}
문제 16은 다음과 같이 계산할 수도 있다.
(3)을 다시 계산해보자.
먼저 구하려는 값을 \(S=1+2+3+\cdots+10\)라고 놓자.
이제 이 식과 이 식을 거꾸로 쓴 식을 나란히 놓고,
\begin{gather*}
S=1+2+3+4+\cdots+9+10\\
S=10+9+8+7+\cdots+2+1
\end{gather*}
두 식을 더하자.
\begin{align*}
2S
&=(1+10)+(2+9)+(3+8)+(4+7)+\cdots+(9+2)+(10+1)\\
&=11+11+11+11+\cdots+11+11\\
&=11\times 10=110
\end{align*}
따라서 \(S=\frac{110}2=55\)이다.

\clearpage
%
\prob{}
예시 17의 방법을 이용해 다음 계산을 하여라.
\begin{enumerate}
\item
\(1+2+3+\cdots+99+100=\pb{5050}\)
\item
\(1+3+5+\cdots+17+19=\pb{100}\)
\end{enumerate}

\begin{mdframed}
\textbf{풀이 : }
\vspace{0.7\textheight}
\end{mdframed}

%
\begin{mdframed}[innertopmargin=-5pt]
\theo{등차수열의 합}
등차수열 \(\{a_n\}\)의 첫번째 항을 \(a\), 공차를 \(d\)라고 할 때, 첫째항부터 제\(n\)항까지의 합 \(S(=a_1+a_2+\cdots+a_n)\)은
\[S=\frac{n\{2a+(n-1)d\}}2\]
이다.
마지막 항을 \(l(=a_n)\)이라고 할 때,
\[S=\frac{n(a+l)}2\]
이라고 쓸 수도 있다.
\end{mdframed}

\proo
예시 17와 같이 \(S\)를 나열한 식과, 그 식을 거꾸로 쓴 식을 나란히 놓으면
\[
\begin{array}{c@{\;\;=\;\;}c@{\;\;+\;\;}c@{\;\;+\cdots+\;\;}c@{\;\;+\;\;}c}
S	&a_1	&a_2	&a_{n-1}	&a_n\\
S	&a_n	&a_{n-1}	&a_2	&a_1
\end{array}
\]
이다.
좀 더 자세하게 쓰면
\[
\begin{array}{c@{\;\;=\;\;}c@{\;\;+\;\;}c@{\;\;+\cdots+\;\;}c@{\;\;+\;\;}c}
S	&a	&(a+d)	&\left(a+(n-2)d\right)	&\left(a+(n-1)d\right)\\
S	&\left(a+(n-1)d\right)	&\left(a+(n-2)d\right)	&(a+d)	&a
\end{array}
\]
이다.
두 식을 더하면
\begin{align*}
2S
&=(2a+(n-1)d)+(2a+(n-1)d)+\cdots+(2a+(n-1)d)+(2a+(n-1)d)\\
&=(2a+(n-1)d)\times n.
\end{align*}
따라서
\[S=\frac{n\{2a+(n-1)d\}}2\]
이다.

또한,
\[l=a_n=a+(n-1)d\]
이므로
\[S=\frac{n\{2a+(n-1)d\}}2=\frac{n\left[a+\{a+(n-1)d\}\right]}2=\frac{n(a+l)}2\]
이다.
\qed

%
\exam{}
문제 16의 (1)에서 \(a=3\), \(d=1\), \(n=5\)이므로
\[S=\frac{5\times\{2\times3+(5-1)\times1\}}2=25\]
이다.
혹은 \(l=7\)이므로
\[S=\frac{5(3+7)}2=25\]
이다.

%
\prob{}
등차수열의 합 공식을 이용하여 다음 계산을 하여라.
\begin{enumerate}
\item
\(2+4+6+8+10+12=\pb{42}\)
\item
\(1+2+3+\cdots+10=\pb{55}\)
\item
\(1+2+3+\cdots+100=\pb{5050}\)
\item
\(1+3+5+7+\cdots+19=\pb{100}\)
\end{enumerate}

\begin{mdframed}
\textbf{풀이 : }
\begin{enumerate}
\item
\(a=\pb{2}\), \(d=\pb{2}\), \(n=6\)이므로
\[S=\frac{6\times\{2\times\pb{2}+(6-1)\times\pb{2}\}}2=\pb{42}\]
이다.
혹은 \(l=12\)이므로
\[S=\frac{6(\pb{2}+12)}2=\pb{42}\]
이다.
\end{enumerate}
\end{mdframed}

\begin{mdframed}
\begin{enumerate}
\item[(2)]
\(a=1\), \(d=1\), \(n=\pb{10}\)이므로
\[S=\frac{\pb{10}\times\{2\times1+(\pb{10}-1)\times1\}}2=\pb{55}\]
이다.
혹은 \(l=\pb{10}\)이므로
\[S=\frac{10(1+\pb{10})}2=\pb{55}\]
이다.
\item
\(a=\pb{1}\), \(d=\pb{1}\), \(n=\pb{100}\)이므로
\[S=\frac{\pb{100}\times\{2\times\pb{1}+(\pb{100}-1)\times\pb{1}\}}2=\pb{5050}\]
이다.
혹은 \(l=\pb{100}\)이므로
\[S=\frac{\pb{100}(\pb{1}+\pb{100})}2=\pb{5050}\]
이다.
\item
\(a=\pb{1}\), \(d=\pb{2}\), \(n=\pb{10}\)이므로
\[S=\frac{\pb{100\times\{2\times1+(100-1)\times1\}}}2=\pb{100}\]
이다.
혹은 \(l=\pb{19}\)이므로
\[S=\frac{\pb{100(1+100)}}2=\pb{100}\]
이다.
\end{enumerate}
%\vspace{0.45\textheight}
(문제 16, 18의 결과와 비교해보자.)
\end{mdframed}

\clearpage
%%
\section{보충·심화 문제}

%수력충전, 이하 '수'
\prob{}
다음 수열의 제9항을 구하여라.
\begin{enumerate}
\item
\(1\), \(4\), \(9\), \(16\), \(25\), \(\cdots\)
\item
\(\frac13\), \(\frac15\), \(\frac17\), \(\frac19\), \(\frac1{11}\), \(\cdots\)
\end{enumerate}

%수
\prob{}
다음 수열의 일반항 \(a_n\)을 구하여라.
\begin{enumerate}
\item
\(1\), \(8\), \(27\), \(64\), \(81\), \(125\), \(\cdots\)
\item
\(1\cdot2\), \(2\cdot3\), \(3\cdot4\), \(4\cdot5\), \(5\cdot6\), \(\cdots\)
\item
\(-1\), \(1\), \(-1\), \(1\), \(-1\), \(\cdots\)
\end{enumerate}

%개념유형, 이하 '유'
\prob{}
다음 수열의 일반항 \(a_n\)을 구하여라.
\begin{enumerate}
\item
\(\frac12\), \(\frac23\), \(\frac34\), \(\frac45\), \(\cdots\)
\item
\(10\), \(100\), \(1000\), \(10000\), \(\cdots\)
\item
\(9\), \(99\), \(999\), \(9999\), \(\cdots\)
\item
\(\frac3{1\cdot2}\), \(\frac4{2\cdot3}\), \(\frac5{3\cdot4}\), \(\frac6{4\cdot5}\), \(\cdots\)
\end{enumerate}


%수
\prob{}
다음 수열이 등차수열을 이루도록 \(\pb{10}\) 안에 알맞은 수를 써넣어라
\begin{enumerate}
\item
\(2\), \(5\), \pb{8}, \pb{11}, \(14\), \(\cdots\)
\item
\(30\), \pb{28}, \pb{26}, \(24\), \(22\), \(\cdots\)
\end{enumerate}

\clearpage
%수
\prob{}
다음 등차수열 \(\{a_n\}\)의 공차를 구하여라.
\begin{enumerate}
\item
\(a_1=5\), \(a_7=23\)
\item
\(a_1=10\), \(a_{10}=-8\)
\end{enumerate}

%수
\prob{}
다음 등차수열의 일반항 \(a_n\)을 구하여라.
\begin{enumerate}
\item
첫째항이 \(10\), 공차 \(-4\)
\item
\(4\), \(6\), \(8\), \(10\), \(12\), \(\cdots\)
\item
\(1\), \(4\), \(7\), \(10\), \(13\), \(\cdots\)
\item
\(-7\), \(-4\), \(-1\), \(2\), \(5\), \(\cdots\)
\item
\(1\), \(-\frac12\), \(-2\), \(-\frac72\), \(-5\), \(\cdots\)
\end{enumerate}

%수
\prob{}
다음을 구하여라.
\begin{enumerate}
\item
제3항이 \(5\), 제8항이 \(-5\)인 등차수열의 일반항
\begin{mdframed}
\textbf{풀이 : }
\(a+2d=5\), \(a+7d=-5\)이므로 두 식을 연립하면 \(a=\pb9\), \(d=\pb{-2}\)이다.
따라서 일반항 \(a_n\)은
\[a_n=9+(n-1)(-2)=-2n+7\]
이다.
\end{mdframed}

\item
제3항이 \(7\)이고, 제8항이 \(27\)인 등차수열의 일반항
\vspace{0.2\textwidth}
\item
등차수열 \(\{a_n\}\)에서 \(a_2=3\), \(a_7=13\)일 때, \(a_{30}\)의 값
\end{enumerate}

%유
\prob{}
다음 조건을 만족하는 등차수열의 일반항 \(a_n\)을 구하여라.
\begin{enumerate}
\item
\(a_1=2\), \(a_3=\frac23\)
\vspace{0.1\textheight}
\item
\(a_2=-10\), \(a_7=20\)
\vspace{0.1\textheight}
\end{enumerate}

%유
\prob{}
등차수열 \(\{a_n\}\)에서 \(a_4+a_8=24\), \(a_{15}+a_{19}=68\)일 때, 일반항 \(a_n\)을 구하여라.
\vspace{0.13\textheight}

%유
\prob{}
등차수열 \(\{a_n\}\)에서 \(a_7=10\), \(a_{11}=4\)일 때, 처음으로 음수가 되는 항은 제 몇 항인지 구하여라.
\begin{mdframed}
\textbf{풀이 : }
\(a+6d=10\), \(a+10d=4\)이다.
두 식을 연립하면 \(a=\pb{19}\), \(d=\pb{-\frac32}\)이다.
따라서 \[a_n=19+(n-1)(-\frac32)=\pb{-\frac32n+\frac{41}2}\]이고
\(a_n<0\)을 풀면
\begin{gather*}
-\frac32n+\frac{41}2<0\\
n>\frac{41}3=13.666\cdots
\end{gather*}
따라서 \(n\)의 최솟값은 \(14\)이고, 처음으로 음수가 되는 항은 제\pb{14}항이다.
\end{mdframed}

%유
\prob{}
등차수열 \(\{a_n\}\)에서 제5항이 \(72\), 제10항이 \(37\)일 때, 처음으로 음수가 되는 항은 제 몇 항인지 구하여라.
\vspace{0.25\textheight}

%유
\prob{}
등차수열 \(\{a_n\}\)에서 \(a_1+a_2+a_3=-12\), \(a_4+a_5+a_6=33\)일 때, 처음으로 100보다 크게 되는 항은 제 몇 항인지 구하여라.
\vspace{0.25\textheight}

%유
\prob{}
등차수열 \(4\), \(x_1\), \(x_2\), \(x_3\), \(\cdots\), \(x_m\), \(34\)의 공차가 \(2\)일 때, \(m\)의 값을 구하여라.
\vspace{0.25\textheight}

%수
\prob{}
다음 수열이 등차수열을 이룰 때, \(x\), \(y\), \(z\)의 값을 구하여라.
\begin{enumerate}
\item
\(32\), \(x\), \(22\), \(y\), \(12\), \(\cdots\)
\item
\(-1\), \(x\), \(5\), \(y\), \(11\), \(\cdots\)
\item
\(x\), \(13\), \(y\), \(5\), \(z\), \(\cdots\)
\item
\(x\), \(-1\), \(y\), \(11\), \(z\), \(\cdots\)
\end{enumerate}

%유
\prob{}
네 수 \(28\), \(a\), \(b\), \(13\)이 이 순서대로 등차수열을 이룰 때, \(a\), \(b\)의 값을 구하여라.
\vspace{0.05\textheight}

%유
\prob{}
\kswrapfig[Pos=r,Width=60pt]{magic_square}{
오른쪽 그림에서 가로줄과 세로줄에 있는 세 수가 각각 등차수열을 이룬다.
예를 들어 \(-2\), \(a\), \(b\)가 이 순서대로 등차수열을 이루고, \(b\), \(5\), \(f\)가 이 순서대로 등차수열을 이룰 때, \((b-a)+(f-e)\)의 값을 구하여라.}
\vspace{0.1\textheight}

%수
\prob{}
세 실수 \(\frac1{a+b}\), \(\frac1{b+c}\), \(\frac1{c+a}\)가 이 순서로 등차수열을 이룰 때, 세 실수 \(a\), \(b\), \(c\) 사이의 관계식은?
\par\noindent
\ding{172}\(a^2=b^2+c^2\)
\tab\tab\ding{173}\(b^2=a^2+c^2\)
\tab\tab\ding{174}\(2a^2=b^2+c^2\)
\tab\ding{175}\(2b^2=a^2+b^2\)
\tab\ding{176}\(2c^2=a^2+b^2\)
\vspace{0.4\textheight}


%수, 유
\prob{}
등차수열을 이루는 세 수가 있다.
다음 물음에 답하여라.
\begin{enumerate}
\item
세 수의 합이 \(15\)이고, 곱이 \(105\)일 때, 이들 세 수를 구하여라.
\begin{mdframed}
\textbf{풀이 : }
세 수를 \(a-d\), \(a\), \(a+d\)로 두면
\[
\left\{\begin{aligned}
&(a-d)+a+(a+d)=15\\
&(a-d)\times a\times(a+d)=105
\end{aligned}
\right.
\]
이고, 첫 번째 식을 정리하면 \(a=\pb{5}\).
이것을 두 번째 식에 대입하면,
\[
(5-d)\times 5\times(5+d)=105
\]
따라서 \(25-d^2=21\), \(d^2=4\), \(d=\pm\pb{2}\).
그러므로 구하는 세 수는 \(3\), \(5\), \(\pb{7}\)이다.
\end{mdframed}

\item
세 수의 합이 \(12\)이고, 곱이 \(28\)일 떄, 이들 세 수를 구하여라.
\vspace{0.3\textheight}

\item
세 수의 합이 \(15\)이고, 제곱의 합이 \(83\)일 때, 이들 세 수를 구하여라.
\vspace{0.3\textheight}
\end{enumerate}

%수
\prob{}
다음 계산을 하시오.
\begin{enumerate}
\item
\(1+3+5+\cdots+99\)
\item
\(3+8+13+18+\cdots+48\)
\item
\((-2)+2+6+10+\cdots+394\)
\end{enumerate}

%수
\prob{}
등차수열 \(\{a_n\}\)에 대해 공차 \(d\)를 구하시오.
(단 \(a\)는 첫째항이고, \(S_{10}=a_1+a_2+a_3+\cdots+a_{10}\)이다.)
\begin{enumerate}
\item
\(a=30\), \(S_{10}=210\)
\begin{mdframed}
\textbf{풀이 : }
\[\frac{10(2\times30+9d)}2=210\]
이므로 \(d=\pb{-2}\)이다.
\end{mdframed}
\item
\(a=40\), \(S_{10}=175\)
\vspace{0.2\textheight}
\item
\(a=-3\), \(S_{10}=285\)
\vspace{0.2\textheight}
\end{enumerate}


%유
\prob{}
\(a_2=4\), \(a_5=22\)인 등차수열 \(\{a_n\}\)의 첫째항부터 제30항까지의 합 \(S_{30}\)의 값을 구하여라.
\vspace{0.2\textheight}

%유
\prob{}
\(a_3=8\), \(a_7=20\)인 등차수열 \(\{a_n\}\)의 첫째항부터 제20항까지의 합 \(S_{20}\)의 값을 구하여라.
\vspace{0.2\textheight}

%유
\prob{}
\(10\)과 \(30\) 사이에 \(n\)개의 수를 넣어 만든 등차수열 \(10\), \(x_1\), \(x_2\), \(\cdots\), \(x_n\), \(30\)의 모든 항의 합이 \(820\)일 때, \(n\)의 값과 공차 \(d\)를 구하여라.
\vspace{0.3\textheight}

%수
\prob{}
\(100\)부터 \(300\)까지의 자연수에 대하여 다음을 구하여라.
\begin{enumerate}
\item
\(3\)의 배수의 총합
\begin{mdframed}
\textbf{풀이 : }
%\(3\)의 배수를 \pb{3k}(\(k\)는 자연수)라고 하면,
%\[100\le\pb{3k}\le300\]
%이고, 이것을 정리하면
%\[\therefore \frac{100}3\le k\le100\]
%\(k\)는 자연수이므로 \(k=34,\:35,\:36,\:\cdots,\:100\)이다.
%따라서 \(100\)부터 \(300\)까지의 자연수 중 \(3\)의 배수는
%\[3\times 34,\:3\times 35,\:3\times 36,\:\cdots,\:3\times 100\]
%이다.
\(100\)보다 크고 \(300\)보다 작은 수는
\[102,\:\pb{105},\:108\:,\cdots,300\]
이다.
이것은 \(a=\pb{102}\), \(d=3\), \(l=\pb{300}\)인 등차수열이다.
항수는 \(n=100-34+1=67\)이므로
\[102+105+108+\cdots+300=\frac{67(102+300)}2=\pb{13467}\]
%\[S_{67}=\frac{67(102+300)}2=\pb{13467}\]
\end{mdframed}
\item
\(4\)의 배수의 총합
\vspace{0.3\textheight}
\item
\(7\)의 배수의 총합
\vspace{0.3\textheight}
\end{enumerate}

%유
\prob{}
\(100\) 이하의 자연수 중에서 \(4\)로 나누면 \(3\)이 남는 자연수의 총합을 구하여라.
\vspace{0.25\textheight}

%유
\prob{}
\(100\) 이상, \(200\) 이하의 자연수 중에서 \(5\)로 나누면 \(3\)이 남는 자연수의 총합을 구하여라.
\vspace{0.25\textheight}

%유
\prob{}
세 자리의 자연수 중에서 \(9\)의 배수의 총합을 구하여라.
\vspace{0.25\textheight}

%유
\prob{}
등차수열 \(\{a_n\}\)의 첫째항부터 제\(n\)항까지의 합을 \(S_n\)이라고 할 때, 다음 물음에 답하여라.
\begin{enumerate}
\item
\(S_{10}=120\), \(S_{20}=440\)이다.
이때 \(S_{30}\)의 값을 구하여라.
\begin{mdframed}
\textbf{풀이 : }
첫항을 \(a\), 공차를 \(d\)로 놓고, 공식 \(S_n=\frac{n\{2a+(n-1)d\}}2\)을 사용하면
\[\frac{10(2a+9d)}2=120,\quad\frac{20(2a+19d)}2=440\]
따라서
\[2a+9d=\pb{24},\quad 2a+19d=\pb{44}\]
이다.
두 식을 연립하면 \(a=\pb3\), \(d=\pb2\).
따라서
\[S_{30}=\frac{30(2a+29d)}2=\pb{960}\]
\end{mdframed}
\item
\(S_5=10\), \(S_{10}=45\)이다.
이때 \(S_{20}\)의 값을 구하여라.
\end{enumerate}
\vspace{0.25\textheight}

%유
\prob{}
공차가 \(3\)인 등차수열 \(\{a_n\}\)에서 \(a_1+a_2+a_3+\cdots+a_{100}=200\)일 때, \(a_2+a_3+a_4+\cdots+a_{101}\)의 값을 구하여라.
\vspace{0.25\textheight}

%%유
%\prob{}
%첫째항부터 제10항까지의 합은 \(110\)이고, 제11항부터 제20항까지의 합이 \(310\)인 등차수열의 제21항부터 제30항까지의 합을 구하여라.

%%유
%\prob{}
%수열 \(\{a_n\}\)의 첫째항부터 제\(n\)항까지의 합 \(S_n\)이 다음과 같이 주어질 때, \(a_1+a_{10}\)을 구하여라.
%\begin{enumerate}
%\item
%\(S_n=2n^2-3n\)
%\item
%\(S_n=n^2+3n-1\)
%\end{enumerate}
\end{document}