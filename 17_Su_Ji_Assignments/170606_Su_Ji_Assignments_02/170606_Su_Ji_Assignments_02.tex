\documentclass[a4paper]{oblivoir}
\usepackage{amsmath,amssymb,kotex,kswrapfig,mdframed,paralist}
\usepackage{fapapersize}
\usefapapersize{210mm,297mm,10mm,*,10mm,*}

\usepackage{tabto,pifont}
\TabPositions{0.2\textwidth,0.4\textwidth,0.6\textwidth,0.8\textwidth}
\newcommand\tabb[5]{\par\noindent
\ding{172}\:{\ensuremath{#1}}
\tab\ding{173}\:\:{\ensuremath{#2}}
\tab\ding{174}\:\:{\ensuremath{#3}}
\tab\ding{175}\:\:{\ensuremath{#4}}
\tab\ding{176}\:\:{\ensuremath{#5}}}

\usepackage{graphicx}

%\pagestyle{empty}

%%% Counters
\newcounter{num}

%%% Commands
\newcommand\prob[1]
{\bigskip\par\noindent\stepcounter{num} \textbf{문제 \thenum) #1}\par\noindent}

\newcommand\pb[1]{\ensuremath{\fbox{\phantom{#1}}}}

\newcommand\ba{\ensuremath{\:|\:}}

\newcommand\vs[1]{\vspace{25pt}}

\newcommand\an[1]{\bigskip\par\noindent\textbf{문제 #1)}\par\noindent}

%%% Meta Commands
\let\oldsection\section
\renewcommand\section{\clearpage\oldsection}

\let\emph\textsf

\begin{document}
\begin{center}
\LARGE수지, 추가과제 02
\end{center}
\begin{flushright}
날짜 : 2017년 \(\pb3\)월 \(\pb{10}\)일 \(\pb{월}\)요일
,\qquad
제한시간 : \pb{17년}분
,\qquad
점수 : \pb{20} / \pb{20}
\end{flushright}

\prob{}
다음 빈칸에 알맞은 것을 써넣어라.
\begin{mdframed}[rightmargin=100pt,leftmargin=100pt]
\centering
\(\displaystyle\lim_{x\to 2}f(x)\text{가 존재한다.}\iff
\lim_{\pb{aaaa}}f(x)=\lim_{\pb{aaaa}}f(x)\)
\end{mdframed}

%
\prob{}
\begin{minipage}{0.45\textwidth}
함수 \(y=\frac{x-2}{x+2}\)에 대하여 다음 물음에 답하여라.
\begin{enumerate}[(1)]
\item
이 함수의 정의역은 \(\{x\,|\,x\text{는  }\fbox{\phantom{\(x\neq0\)}}\text{인 실수}\}\) 이다.
\item
이 함수의 그래프를 그려라(오른쪽 모눈).
\item
다음 극한값들을 구하여라. (존재하지 않으면 `\(\times\)' 표시 하여라)
\par\bigskip
\(\displaystyle\lim_{x\to0+}\frac{x-2}{x+2}=\pb{3}\)\qquad
\(\displaystyle\lim_{x\to2+}\frac{x-2}{x+2}=\pb{3}\)
\\\\
\(\displaystyle\lim_{x\to0-}\frac{x-2}{x+2}=\pb{3}\)\qquad
\(\displaystyle\lim_{x\to2-}\frac{x-2}{x+2}=\pb{3}\)
\\\\
\(\displaystyle\lim_{x\to0}\frac{x-2}{x+2}=\pb{3}\)\qquad
\(\displaystyle\lim_{x\to2}\frac{x-2}{x+2}=\pb{3}\)
\end{enumerate}
\end{minipage}
\begin{minipage}{0.45\textwidth}
\par\bigskip\includegraphics[width=0.9\textwidth]{55}
\end{minipage}\bigskip\bigskip\par


%
\prob{}
\begin{minipage}{0.45\textwidth}
함수 \(y=\frac{x^2-3x}{|x-3|}\)에 대하여 다음 물음에 답하여라.
\begin{enumerate}[(1)]
\item
이 함수의 정의역은 \(\{x\,|\,x\text{는  }\fbox{\phantom{\(x\neq0\)}}\text{인 실수}\}\) 이다.
\item
이 함수의 그래프를 그려라(오른쪽 모눈).
\item
다음 극한값들을 구하여라. (존재하지 않으면 `\(\times\)' 표시 하여라)
\par\bigskip
\(\displaystyle\lim_{x\to0+}\frac{x^2-3x}{|x-3|}=\pb{3}\)\qquad
\(\displaystyle\lim_{x\to3+}\frac{x^2-3x}{|x-3|}=\pb{3}\)
\\\\
\(\displaystyle\lim_{x\to0-}\frac{x^2-3x}{|x-3|}=\pb{3}\)\qquad
\(\displaystyle\lim_{x\to3-}\frac{x^2-3x}{|x-3|}=\pb{3}\)
\\\\
\(\displaystyle\lim_{x\to0}\frac{x^2-3x}{|x-3|}=\pb{3}\)\qquad
\(\displaystyle\lim_{x\to3}\frac{x^2-3x}{|x-3|}=\pb{3}\)
\end{enumerate}
\end{minipage}
\begin{minipage}{0.45\textwidth}
\par\bigskip\includegraphics[width=0.9\textwidth]{55}
\end{minipage}\bigskip\bigskip\par

\clearpage
%
\prob{}
\begin{minipage}{0.45\textwidth}
함수 \(y=\frac{2x^2+x-1}{|x+1|}\)에 대하여 다음 물음에 답하여라.
\begin{enumerate}[(1)]
\item
이 함수의 정의역은 \(\{x\,|\,x\text{는  }\fbox{\phantom{\(x\neq0\)}}\text{인 실수}\}\) 이다.
\item
이 함수의 그래프를 그려라(오른쪽 모눈).
\item
다음 극한값들을 구하여라. (존재하지 않으면 `\(\times\)' 표시 하여라)
\par\bigskip
\(\displaystyle\lim_{x\to-1+}\frac{x^2-1}{|x+1|}=\pb{3}\)\qquad
\(\displaystyle\lim_{x\to1+}\frac{x^2-1}{|x+1|}=\pb{3}\)
\\\\
\(\displaystyle\lim_{x\to-1-}\frac{x^2-1}{|x+1|}=\pb{3}\)\qquad
\(\displaystyle\lim_{x\to1-}\frac{x^2-1}{|x+1|}=\pb{3}\)
\\\\
\(\displaystyle\lim_{x\to-1}\frac{x^2-1}{|x+1|}=\pb{3}\)\qquad
\(\displaystyle\lim_{x\to1}\frac{x^2-1}{|x+1|}=\pb{3}\)
\end{enumerate}
\end{minipage}
\begin{minipage}{0.45\textwidth}
\par\bigskip\includegraphics[width=0.9\textwidth]{55}
\end{minipage}\bigskip\bigskip\par

%
\prob{}
\begin{minipage}{0.45\textwidth}
함수 \(y=[x]\)에 대하여 다음 물음에 답하여라.
\begin{enumerate}[(1)]
\item
이 함수의 정의역은 \(\{x\,|\,x\text{는  }\fbox{\phantom{임의의 실수}}\}\) 이다.
\item
이 함수의 그래프를 그려라(오른쪽 모눈).
\item
다음 극한값들을 구하여라. (존재하지 않으면 `\(\times\)' 표시 하여라)
\par\bigskip
\(\displaystyle\lim_{x\to0+}[x]=\pb{3}\)\qquad
\(\displaystyle\lim_{x\to2+}[x]=\pb{3}\)
\\\\
\(\displaystyle\lim_{x\to0-}[x]=\pb{3}\)\qquad
\(\displaystyle\lim_{x\to2+}[x]=\pb{3}\)
\\\\
\(\displaystyle\lim_{x\to0}[x]=\pb{3}\)\qquad
\(\displaystyle\lim_{x\to2}[x]=\pb{3}\)
\end{enumerate}
\end{minipage}
\begin{minipage}{0.45\textwidth}
\par\bigskip\includegraphics[width=0.9\textwidth]{55}
\end{minipage}\bigskip\bigskip\par

%
\prob{}
다음 극한값을 구하여라. (존재하지 않으면 `\(\times\)' 표시 하여라)
\begin{enumerate}[(1)]
\item
\(\displaystyle\lim_{x\to4+}\frac{[x]-4}{x-4}\pb{3}\)
\item
\(\displaystyle\lim_{x\to0-}\frac{[x+1]}{x+1}\pb{3}\)
\item
\(\displaystyle\lim_{x\to0}\frac{x-1}{[x-1]}\pb{3}\)
\end{enumerate}

%
\prob{}
함수 \(f(x)\)에 대하여 \(\displaystyle\lim_{x\to0}\frac{f(x)}x=2\)일 때, \(\displaystyle\lim_{x\to0}\frac{2x^2+5f(x)}{3x^2-f(x)}\)의 값을 구하여라.

\clearpage

%
\prob{}
함수 \(f(x)\)에 대하여 \(\displaystyle\lim_{x\to7}f(x-7)=2\)가 성립할 때, \(\displaystyle\lim_{x\to0}\frac{2f(x)+1}{3f(x)-1}\)의 값을 구하여라.
\vspace{0.1\textheight}

%
\prob{}
다음 식을 만족시키는 상수 \(a\), \(b\)의 값을 각각 구하여라.
\begin{enumerate}[(1)]
\item
\(\displaystyle\lim_{x\to1}\frac{\sqrt{a+x}-b}{x-1}=\frac14\)
\item
\(\displaystyle\lim_{x\to2}\frac{x-2}{x^2+ax+b}=\frac13\)
\item
\(\displaystyle\lim_{x\to a}\frac{x^3-a^3}{x^2-a^2}=3\)
\item
\(\displaystyle\lim_{x\to-1}\frac{x^2+ax+b}{x+1}=2\)
\item
\(\displaystyle\lim_{x\to2}\frac{a\sqrt{x-1}+b}{x-2}=1\)
\end{enumerate}

\vspace{0.1\textheight}
%
\prob{}
\(x\)에 대한 다항식 \(f(x)\)가 \(\displaystyle\lim_{x\to\infty}\frac{f(x)}{2x^2+x+1}=1\),
\(\displaystyle\lim_{x\to2}\frac{f(x)}{x^2-x-2}=1\)을 만족시킬 때, \(f(0)\)의 값을 구하여라.

\vspace{0.1\textheight}
%
\prob{}
임의의 실수 \(x\)에 대하여 함수 \(f(x)\)가 다음을 만족시킬 때, \(\displaystyle\lim_{x\to\infty}f(x)\)의 값을 구하여라.
\[\frac{x^2+x-1}{3x^2+2}\le f(x)\le\frac{x^2+x+4}{3x^2+2}\]

\vspace{0.1\textheight}
%
\prob{}
두 함수 \(f(x)\), \(g(x)\)가 \(f(x)=2x+1\), \(g(x)=x^2+2\)일 때, 함수 \(h(x)\)가 모든 실수 \(x\)에 대하여 \(f(x)\le h(x)\le g(x)\)를 만족시킨다.
이때, \(\displaystyle\lim_{x\to1}f(x)\)의 값을 구하여라.
\end{document}