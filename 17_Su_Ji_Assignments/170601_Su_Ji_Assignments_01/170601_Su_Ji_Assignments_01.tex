\documentclass[a4paper]{oblivoir}
\usepackage{amsmath,amssymb,kotex,kswrapfig,mdframed,paralist}
\usepackage{fapapersize}
\usefapapersize{210mm,297mm,10mm,*,10mm,*}

\usepackage{tabto,pifont}
\TabPositions{0.2\textwidth,0.4\textwidth,0.6\textwidth,0.8\textwidth}
\newcommand\tabb[5]{\par\noindent
\ding{172}\:{\ensuremath{#1}}
\tab\ding{173}\:\:{\ensuremath{#2}}
\tab\ding{174}\:\:{\ensuremath{#3}}
\tab\ding{175}\:\:{\ensuremath{#4}}
\tab\ding{176}\:\:{\ensuremath{#5}}}

\usepackage{graphicx}

\pagestyle{empty}

%%% Counters
\newcounter{num}

%%% Commands
\newcommand\prob[1]
{\bigskip\par\noindent\stepcounter{num} \textbf{문제 \thenum) #1}\par\noindent}

\newcommand\pb[1]{\ensuremath{\fbox{\phantom{#1}}}}

\newcommand\ba{\ensuremath{\:|\:}}

\newcommand\vs[1]{\vspace{25pt}}

\newcommand\an[1]{\bigskip\par\noindent\textbf{문제 #1)}\par\noindent}

%%% Meta Commands
\let\oldsection\section
\renewcommand\section{\clearpage\oldsection}

\let\emph\textsf

\begin{document}
\begin{center}
\LARGE수지, 추가과제 01
\end{center}
\begin{flushright}
날짜 : 2017년 \(\pb3\)월 \(\pb{10}\)일 \(\pb{월}\)요일
,\qquad
제한시간 : \pb{17년}분
,\qquad
점수 : \pb{20} / \pb{20}
\end{flushright}

\begin{mdframed}[rightmargin=100pt,leftmargin=100pt]
\centering
\(\displaystyle\lim_{x\to a}f(x)\text{가 존재한다.}\iff
\lim_{x\to a+}f(x)=\lim_{x\to a-}f(x)\)
\end{mdframed}

%
\prob{}
\begin{minipage}{0.45\textwidth}
함수 \(y=\frac{x}{|x|}\)에 대하여 다음 물음에 답하여라.
\begin{enumerate}[(1)]
\item
이 함수의 정의역은 \(\{x\,|\,x\text{는  }\fbox{\phantom{\(x\neq0\)}}\text{인 실수}\}\) 이다.
\item
이 함수의 그래프를 그려라(오른쪽 모눈).
\item
다음 극한값들을 구하여라. (존재하지 않으면 `\(\times\)' 표시 하여라)
\par\bigskip
\(\displaystyle\lim_{x\to0+}\frac{x}{|x|}=\pb{3}\)\qquad
\(\displaystyle\lim_{x\to3+}\frac{x}{|x|}=\pb{3}\)
\\
\(\displaystyle\lim_{x\to0-}\frac{x}{|x|}=\pb{3}\)\qquad
\(\displaystyle\lim_{x\to3+}\frac{x}{|x|}=\pb{3}\)
\\
\(\displaystyle\lim_{x\to0}\frac{x}{|x|}=\pb{3}\)\qquad
\(\displaystyle\lim_{x\to3}\frac{x}{|x|}=\pb{3}\)
\end{enumerate}
\end{minipage}
\begin{minipage}{0.45\textwidth}
\par\bigskip\includegraphics[width=0.9\textwidth]{55}
\end{minipage}\bigskip\bigskip\par


%
\prob{}
\begin{minipage}{0.45\textwidth}
함수 \(y=|x-1|\)에 대하여 다음 물음에 답하여라.
\begin{enumerate}[(1)]
\item
이 함수의 정의역은 \(\{x\,|\,x\text{는 임의의 실수}\}\) 이다.
\item
이 함수의 그래프를 그려라(오른쪽 모눈).
\item
다음 극한값들을 구하여라. (존재하지 않으면 `\(\times\)' 표시 하여라)
\par\bigskip
\(\displaystyle\lim_{x\to0+}|x-1|=\pb{3}\)\qquad
\(\displaystyle\lim_{x\to1+}|x-1|=\pb{3}\)
\\
\(\displaystyle\lim_{x\to0-}|x-1|=\pb{3}\)\qquad
\(\displaystyle\lim_{x\to1+}|x-1|=\pb{3}\)
\\
\(\displaystyle\lim_{x\to0}|x-1|=\pb{3}\)\qquad
\(\displaystyle\lim_{x\to1}|x-1|=\pb{3}\)
\end{enumerate}
\end{minipage}
\begin{minipage}{0.45\textwidth}
\par\bigskip\includegraphics[width=0.9\textwidth]{55}
\end{minipage}\bigskip\bigskip\par

\clearpage
%
\prob{}
\begin{minipage}{0.45\textwidth}
함수 \(y=\frac{x^2-1}{|x+1|}\)에 대하여 다음 물음에 답하여라.
\begin{enumerate}[(1)]
\item
이 함수의 정의역은 \(\{x\,|\,x\text{는  }\fbox{\phantom{\(x\neq0\)}}\text{인 실수}\}\) 이다.
\item
이 함수의 그래프를 그려라(오른쪽 모눈).
\item
다음 극한값들을 구하여라. (존재하지 않으면 `\(\times\)' 표시 하여라)
\par\bigskip
\(\displaystyle\lim_{x\to-1+}\frac{x^2-1}{|x+1|}=\pb{3}\)\qquad
\(\displaystyle\lim_{x\to1+}\frac{x^2-1}{|x+1|}=\pb{3}\)
\\
\(\displaystyle\lim_{x\to-1-}\frac{x^2-1}{|x+1|}=\pb{3}\)\qquad
\(\displaystyle\lim_{x\to1+}\frac{x^2-1}{|x+1|}=\pb{3}\)
\\
\(\displaystyle\lim_{x\to-1}\frac{x^2-1}{|x+1|}=\pb{3}\)\qquad
\(\displaystyle\lim_{x\to1}\frac{x^2-1}{|x+1|}=\pb{3}\)
\end{enumerate}
\end{minipage}
\begin{minipage}{0.45\textwidth}
\par\bigskip\includegraphics[width=0.9\textwidth]{55}
\end{minipage}\bigskip\bigskip\par

%
\prob{}
\begin{minipage}{0.45\textwidth}
함수 \(y=x|x|\)에 대하여 다음 물음에 답하여라.
\begin{enumerate}[(1)]
\item
이 함수의 정의역은 \(\{x\,|\,x\text{는  }\fbox{\phantom{임의의 실수}}\}\) 이다.
\item
이 함수의 그래프를 그려라(오른쪽 모눈).
\item
다음 극한값들을 구하여라. (존재하지 않으면 `\(\times\)' 표시 하여라)
\par\bigskip
\(\displaystyle\lim_{x\to0+}x|x|=\pb{3}\)\qquad
\(\displaystyle\lim_{x\to2+}x|x|=\pb{3}\)
\\
\(\displaystyle\lim_{x\to0-}x|x|=\pb{3}\)\qquad
\(\displaystyle\lim_{x\to2+}x|x|=\pb{3}\)
\\
\(\displaystyle\lim_{x\to0}x|x|=\pb{3}\)\qquad
\(\displaystyle\lim_{x\to2}x|x|=\pb{3}\)
\end{enumerate}
\end{minipage}
\begin{minipage}{0.45\textwidth}
\par\bigskip\includegraphics[width=0.9\textwidth]{55}
\end{minipage}\bigskip\bigskip\par

%
\prob{}
다음 극한값을 구하여라.
\begin{enumerate}[(1)]
\item
\(\displaystyle\lim_{x\to2}\frac{x^2-4}{x-2}\)
\item
\(\displaystyle\lim_{x\to3}\frac{x^3-3}{x-3}\)
\item
\(\displaystyle\lim_{x\to1}\frac{\sqrt{x^2+1}-2}{x-1}\)
\item
\(\displaystyle\lim_{x\to4}\frac{x-4}{\sqrt x-2}\)
\item
\(\displaystyle\lim_{x\to\infty}\frac{x^2-4}{x^3+3x^2+3x+1}\)
\item
\(\displaystyle\lim_{x\to\infty}\frac{x^2-4}{3x^2+3x+1}\)
\item
\(\displaystyle\lim_{x\to\infty}\frac{x^2-4}{3x+1}\)
\end{enumerate}

\end{document}