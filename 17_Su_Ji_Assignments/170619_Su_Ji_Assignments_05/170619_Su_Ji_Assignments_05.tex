\documentclass[a4paper]{oblivoir}
\usepackage{amsmath,amssymb,kotex,kswrapfig,mdframed,paralist}
\usepackage{fapapersize}
\usefapapersize{210mm,297mm,10mm,*,10mm,*}

\usepackage{tabto,pifont}
\TabPositions{0.2\textwidth,0.4\textwidth,0.6\textwidth,0.8\textwidth}
\newcommand\tabb[5]{\par\noindent
\ding{172}\:{\ensuremath{#1}}
\tab\ding{173}\:\:{\ensuremath{#2}}
\tab\ding{174}\:\:{\ensuremath{#3}}
\tab\ding{175}\:\:{\ensuremath{#4}}
\tab\ding{176}\:\:{\ensuremath{#5}}}

\usepackage{graphicx}

\pagestyle{empty}

%%% Counters
\newcounter{num}

%%% Commands
\newcommand\prob[1]
{\bigskip\par\noindent\stepcounter{num} \textbf{문제 \thenum) #1}\par\noindent}

\newcommand\pb[1]{\ensuremath{\fbox{\phantom{#1}}}}

\newcommand\ba{\ensuremath{\:|\:}}

\newcommand\vs[1]{\vspace{25pt}}

\newcommand\an[1]{\bigskip\par\noindent\textbf{문제 #1)}\par\noindent}

%%% Meta Commands
\let\oldsection\section
\renewcommand\section{\clearpage\oldsection}

\let\emph\textsf

\begin{document}
\begin{center}
\LARGE수지, 추가과제 05
\end{center}
\begin{flushright}
날짜 : 2017년 \(\pb3\)월 \(\pb{10}\)일 \(\pb{월}\)요일
,\qquad
제한시간 : \pb{17년}분
,\qquad
점수 : \pb{20} / \pb{20}
\end{flushright}

%%
%\prob{}
%함수 \(f(x)=\begin{cases}x^2-4x+3&(x\neq2)\\0&(x=2)\end{cases}\)에 대하여 다음 극한을 조사하여라.
%\par\bigskip\noindent
%(1) \(\displaystyle\lim_{x\to0}f(x)\)
%\tabto{0.25\textwidth}
%(2) \(\displaystyle\lim_{x\to1}f(x)\)
%\tabto{0.5\textwidth}
%(3) \(\displaystyle\lim_{x\to2}f(x)\)
%\tabto{0.75\textwidth}
%(4) \(\displaystyle\lim_{x\to\infty}f(x)\)
%\par\bigskip

%
\prob{}
다음 빈칸에 알맞은 것을 써넣어라
(극한값이 존재하지 않으면 \(\times\) 표시 하여라).

%
\prob{}
다음 극한값을 구하여라.
\begin{enumerate}[(1)]
\item
\(\displaystyle\lim_{x\to1}\frac{(x-1)(x^2+x+2)}{x^2-1}\)
\item
\(\displaystyle\lim_{x\to2}\frac{x^3-8}{x-2}\)
\item
\(\displaystyle\lim_{x\to9}\frac{\sqrt x-3}{x-9}\)
\item
\(\displaystyle\lim_{x\to3}\frac{2x-6}{\sqrt{x+1}-2}\)
\end{enumerate}

%
\prob{}
다음 극한을 조사하여라.
\begin{enumerate}[(1)]
\item
\(\displaystyle\lim_{x\to\infty}\frac{2x}{\sqrt{x^2+3}+4}\)
\item
\(\displaystyle\lim_{x\to\infty}\left(x^3+3x^2+2x-1\right)\)
\item
\(\displaystyle\lim_{x\to-\infty}\left(x^3+3x^2+2x-1\right)\)
\item
\(\displaystyle\lim_{x\to\infty}\left(\sqrt{x^2+1}-x\right)\)
\item
\(\displaystyle\lim_{x\to\infty}\left(\sqrt{x^2-3x}-\sqrt{x^2+3x}\right)\)
\end{enumerate}

%
\prob{}
다음 극한값을 구하여라.
\begin{enumerate}[(1)]
\item
\(\displaystyle\lim_{x\to0}\frac1x\left(\frac1{x+\sqrt2}-\frac1{\sqrt2}\right)\)
\item
\(\displaystyle\lim_{x\to2}\frac1{x-2}\left(2x-\frac{5x+2}{x+1}\right)\)
\end{enumerate}

%
\prob{}
다음 등식이 성립하도록 상수 \(a\)의 값을 정하여라.
\begin{enumerate}[(1)]
\item
\(\displaystyle\lim_{x\to1}\frac{ax-2}{x-1}=2\)
\item
\(\displaystyle\lim_{x\to-1}\frac{x+1}{2x^2+3x+a}=-1\)
\end{enumerate}

%
\prob{}
다음 중 그 값이 가장 큰 것은? 
(단, \([x]\)는 \(x\)보다 크지 않은 최대의 정수이다.)
\tabb
{\displaystyle\lim_{x\to0-}\frac{[x-2]}{x-2}}
{\displaystyle\lim_{x\to0+}\frac{x}{[x-1]}}
{\displaystyle\lim_{x\to-1+}\frac{[x]^2-1}{[x^2-1]}}
{\displaystyle\lim_{x\to1-}\frac{[x-2]}{x+1}}
{\displaystyle\lim_{x\to\infty}\left[\frac{2x+1}{x+1}\right]}


\end{document}