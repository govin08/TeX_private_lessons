\documentclass[a4paper]{oblivoir}
\usepackage{amsmath,amssymb,kotex,kswrapfig,mdframed,paralist}
\usepackage{fapapersize}
\usefapapersize{210mm,297mm,10mm,*,10mm,*}

\usepackage{tabto,pifont}
\TabPositions{0.2\textwidth,0.4\textwidth,0.6\textwidth,0.8\textwidth}
\newcommand\tabb[5]{\par\noindent
\ding{172}\:{\ensuremath{#1}}
\tab\ding{173}\:\:{\ensuremath{#2}}
\tab\ding{174}\:\:{\ensuremath{#3}}
\tab\ding{175}\:\:{\ensuremath{#4}}
\tab\ding{176}\:\:{\ensuremath{#5}}}

\usepackage{graphicx}

%\pagestyle{empty}

%%% Counters
\newcounter{num}

%%% Commands
\newcommand\prob[1]
{\bigskip\par\noindent\stepcounter{num} \textbf{문제 \thenum) #1}\par\noindent}

\newcommand\pb[1]{\ensuremath{\fbox{\phantom{#1}}}}

\newcommand\ba{\ensuremath{\:|\:}}

\newcommand\vs[1]{\vspace{25pt}}

\newcommand\an[1]{\bigskip\par\noindent\textbf{문제 #1)}\par\noindent}

%%% Meta Commands
\let\oldsection\section
\renewcommand\section{\clearpage\oldsection}

\let\emph\textsf

\begin{document}
\begin{center}
\LARGE수지, 추가과제 03
\end{center}
\begin{flushright}
날짜 : 2017년 \(\pb3\)월 \(\pb{10}\)일 \(\pb{월}\)요일
,\qquad
제한시간 : \pb{17년}분
,\qquad
점수 : \pb{20} / \pb{20}
\end{flushright}

%
\prob{}
다항식 \(8x^2+4x-3\)을 \(x+1\)로 나누었을 때의 나머지를 구하시오.
\tabb12345

%
\prob{}
다항식 \(x^3+3x^2-2x+1\)을 \(x-1\)로 나누었을 때의 나머지를 구하시오.
\tabb12345

%
\prob{}
다항식 \(5x^2-3x+4\)을 \(2x-1\)로 나누었을 때의 나머지를 구하시오.
\tabb3{\frac{13}4}{\frac72}{\frac{15}4}4

%
\prob{}
다항식 \(x^3-8x^2+kx+1\)을 \(x-2\)로 나누었을 때의 나머지가 \(3\)일 때, 상수 \(k\)의 값을 구하시오.
\tabb{-26}{-13}{0}{13}{26}

%
\prob{}
다항식 \(2x^2+ax-6\)을 \(2x+1\)로 나누었을 때의 나머지가 \(1\)일 때, 상수 \(a\)의 값을 구하시오.
\tabb{-26}{-13}{0}{13}{26}

%
\prob{}
다항식 \(x^3+kx^2-4x+9\)가 \(x-1\)로 나누어떨어질 때, 상수 \(k\)의 값을 구하시오.
\tabb{-11}{-6}{-1}{4}{9}

%
\prob{}
다항식 \(2x^3+5x^2-ax-3\)이 \(x+3\)으로 나누어떨어질 때, 상수 \(a\)의 값을 구하시오.
\tabb{-11}{-6}{-1}{4}{9}

%
\prob{}
다항식 \(f(x)\)를 \(x+3\)으로 나누었을 때의 나머지가 \(4\)이고, 다항식 \(g(x)\)를 \(x+3\)으로 나누었을 때의 나머지가 \(-7\)일 때, \(3f(x)+2g(x)\)를 \(x+3\)으로 나누었을 때의 나머지는?
\tabb{-4}{-2}{1}{2}{4}

%
\prob{}
다항식 \(f(x)\)를 \(x-3\)으로 나눈 나머지가 \(2\)일 때, \((x-2)f(x)\)를 \(x-3\)을 나눈 나머지는?
\tabb01234

%
\prob{}
다항식 \(x^3+3x^2-2x-10\)를 \(x-2\)으로 나눈 나머지는?
\tabb{-2}0246

%
\prob{}
다항식 \(f(x)\)를 \(x+1\)로 나누었을 때의 나머지는 \(3\), \(x-4\)로 나누었을 때의 나머지는 \(-6\)이라고 한다.
\(f(x)\)를 이차식 \(x^2-3x-4\)로 나누었을 때의 나머지는?
\tabb
{-9x+6}
{-\frac95x-\frac65}
{-\frac95x+\frac65}
{\frac95x-\frac65}
{9x-6}

%
\prob{}
다항식 \(f(x)\)를 \(x+1\), \(x-2\)로 나누었을 때의 나머지가 각각 \(-5\), \(1\)일 때, \(f(x)\)를 \(x^2-x-2\)로 나누었을 때의 나머지는?
\tabb
{2x-3}
{2x+3}
{3x-2}
{3x+2}
{20}

%
\prob{}
다항식 \(x^3+ax^2-5x+4\)를 \(x-2\)로 나눌 때의 나머지와 \(x+1\)로 나눌 때의 나머지가 같을 때, 상수 \(a\)의 값은?
\tabb12345

%
\prob{}
다음 식을 인수분해하여라.
\begin{enumerate}[(1)]
\item
\((a-b)x+(b-a)y\)
\item
\(ad-bc-ac+bd\)
\item
\(4x^2+4x+1\)
\item
\(16a^2+8ab+b^2\)
\item
\((x+1)^2-(y-1)^2\)
\item
\((x+1)^2-5(x+1)+6\)
\item
\(x^4-5x^2+4\)
\item
\(x^3+2x^2-11x-12\)
\end{enumerate}

%
\prob{}
다음 직선의 방정식을 구하시오.
\begin{enumerate}[(1)]
\item
기울기가 \(-2\)이고 \(y\)절편이 \(3\)인 직선
\item
기울기가 \(3\)이고 \((0,-4)\)를 지나는 직선
\item
기울기가 \(2\)이고 점 \((-1,1)\)을 지나는 직선
\item
기울기가 \(-\frac14\)이고 점 \((8,-3)\)을 지나는 직선
\end{enumerate}

%
\prob{}
직선 \(2x+ay+b=0\)은 직선 \(x-2y+3=0\)과 기울기가 같고 점 \((4,3)\)을 지난다.
이때 상수 \(a\), \(b\)에 대하여 \(a-b\)의 값은?
\tabb{-8}{-4}248

\noindent
\begin{minipage}{0.45\textwidth}
%
\prob{}
다음 <보기>의 직선 중 서로 평행한 것을 찾으시오.
\begin{mdframed}[frametitle=<보기>]
\begin{enumerate}
\item[ㄱ.]
\(y=3x-1\)
\item[ㄴ.]
\(y=-3x+5\)
\item[ㄷ.]
\(y=\frac13x+4\)
\item[ㄹ.]
\(y=-\frac13x-1\)
\item[ㅁ.]
\(x-3y+6=0\)
\end{enumerate}
\end{mdframed}
\end{minipage}
~~~~
\begin{minipage}{0.45\textwidth}
%
\prob{}
다음 <보기>의 직선 중 서로 수직인 것을 찾으시오.
\begin{mdframed}[frametitle=<보기>]
\begin{enumerate}
\item[ㄱ.]
\(y=-x+2\)
\item[ㄴ.]
\(y=2x+7\)
\item[ㄷ.]
\(y=-\frac12x+1\)
\item[ㄹ.]
\(y=\frac12x-4\)
\item[ㅁ.]
\(x-y=10\)
\item[ㅂ.]
\(2x+y+8=0\)
\end{enumerate}
\end{mdframed}
\end{minipage}


\end{document}