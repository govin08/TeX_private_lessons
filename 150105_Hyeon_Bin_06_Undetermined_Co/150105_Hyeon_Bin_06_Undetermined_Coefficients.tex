\documentclass{memoir}
\usepackage{amsmath,amssymb,amsthm,kotex,paralist,mathrsfs}

\begin{document}

\title{현빈 : 06 미정계수법}
\author{}
\date{\today}
\maketitle

--------------------- 예 제 ---------------------

01.
다음 등식이 \(x\)에 대한 항등식일 때, 상수 \(a\), \(b\), \(c\)의 값을 구하여라.

\quad\:
(1) \(x^2-6x+16=ax^2+bx+c\)

\quad\:
(2) \(a(x-1)(x-2)+b(x-2)+c=x^2\)

\bigskip
02.
등식 \(2x^2-6x-2=a(x+1)(x-2)+bx(x-2)+cx(x+1)\)이 \(x\)에 대한 항등식일 때, 상수 \(a\), \(b\), \(c\)의 값을 구하여라.

\bigskip
03.
등식 \((2k-1)x+(k+1)y-k-7=0\)이 \(k\)의 값에 관계없이 항상 성립할 때, 상수 \(x\), \(y\)의 값을 구하여라.

\bigskip
04.
모든 실수 \(x\), \(y\)에 대하여 등식
\[(x-2y)a+(3y-x)b+2x-3y=0\]
이 성립할 때, 상수 \(a\), \(b\)의 값을 구하여라.

\bigskip
05.
다음 물음에 답하여라.

\hangfrom{\qquad\quad
(1) }다항식 \(x^3+ax+b\)를 \(x^2-3x+2\)로 나누었을 때의 나머지가 \(2x+1\)이 되도록 하는 상수 \(a\), \(b\)의 값을 구하여라.

\hangfrom{\qquad\quad
(2) }다항식 \(x^3+ax^2+bx+2\)가 \(x^2+x+1\)로 나누어떨어지도록  하는 상수 \(a\), \(b\)의 값을 구하여라.

\bigskip
06.
다항식 \(x^3+ax^2+bx-4\)는 \(x-2\)로 나누어떨어지고, \(x+1\)로 나누었을 때의 나머지가 \(6\)이다.
이때 상수 \(a\), \(b\)의 값을 구하여라.

\bigskip
07.
다항식 \(x^3-ax^2+bx-2\)가 \((x-1)(x+2)\)로 나누어떨어지도록 하는 상수 \(a\), \(b\)의 값을 구하여라.

\bigskip
08.
다항식 \(f(x)\)를 \(x-2\)로 나누었을 때의 나머지가 \(5\)이고, \(x-3\)으로 나누었을 때의 나머지가 \(9\)이다.
\(f(x)\)를 \((x-2)(x-3)\)으로 나누었을 때의 나머지를 구하여라.

\bigskip
09.
다항식 \(f(x)\)를 \((x-1)^2\)으로 나누었을 때의 나머지가 \(3x+2\)이고, \(x+1\)로 나누었을 때의 나머지가 \(3\)이다.
\(f(x)\)를 \((x-1)^2(x+1)\)로 나누었을 때의 나머지를 구하여라.

%\bigskip
%10.
%다항식 \(f(x)\)를 \(x^2+x-2\)로 나누었을 때의 나머지가 \(3x-2\)일 때, 다항식 \(f(2x-3)\)을 \(x-2\)로 나누었을 때의 나머지를 구하여라.

%\bigskip
%11.
%다항식 \(f(x)\)를 \(x-2\)로 나누었을 때의 몫이 \(Q(x)\), 나머지가 \(5\)이고, \(Q(x)\)를 \(x+3\)으로 나누었을 때의 나머지가 \(3\)일 때, 다항식 \(xf(x)\)를 \(x+3\)으로 나누었을 때의 나머지를 구하여라.

\bigskip
14.
등식 \(3x^3-x+2=a(x-1)^3+b(x-1)^2+c(x-1)+d\)가 \(x\)에 대한 항등식일 때, 상수 \(a\), \(b\), \(c\), \(d\)의 값을 구하여라.

\bigskip
--------------------- 연습 문제 ---------------------

26.
다항식 \(f(x)\)에 대해서 등식 \((x+1)(x^2-2)f(x)=x^4+ax^2-b\)가 \(x\)에 대한 항등식일 때, 상수 \(a\), \(b\)의 합 \(a+b\)의 값을 구하여라.

\bigskip
27.
등식 \((k-2)x+(k-1)y=4k+1\)이 \(k\)에 관계없이 항상 성립할 때, 상수 \(x\), \(y\)의 값을 구하여라.

\bigskip
28.
임의의 실수 \(x\), \(y\)에 대하여 등식 \(a(x+y)+b(x-y)+2=3x-5y+c\)가 성립하도록 하는 상수 \(a\), \(b\), \(c\)의 곱 \(abc\)의 값을 구하여라.

\bigskip
29.
다음 물음에 답하여라.

\hangfrom{\qquad\quad
(1) }다항식 \(x^3+ax^2+bx+2\)를 \(x^2+2x-3\)으로 나누었을 때의 나머지가 \(2x+1\)이 되도록 하는 상수 \(a\), \(b\)의 값을 구하여라.

\hangfrom{\qquad\quad
(2) }다항식 \(x^3+ax-8\)이 \(x^2+4x+b\)로 나누어떨어질 때, 상수 \(a\), \(b\)의 값을 구하여라.

\bigskip
33.
다항식 \(f(x)\)를 \(x+2\), \(x-6\)으로 나누었을 때의 나머지가 각각 \(6\), \(-10\)이다.
\(f(x)\)를 \(x^2-4x-12\)로 나누었을 때의 나머지를 구하여라.

\bigskip
35.
다항식 \(f(x)\)는 \((x-1)^2\)으로 나누어떨어지고, \(x-3\)으로 나누었을 때의 나머지가 \(5\)이다.
\(f(x)\)를 \((x-1)^2(x-3)\)으로 나누었을 때의 나머지를 구하여라.

\bigskip
43.
등식 \(x^3+2x+4=a(x-1)^3+b(x-1)^2+c(x-1)+d\)가 \(x\)에 관한 항등식일 때, 상수 \(a\), \(b\), \(c\), \(d\)의 값을 구하여라.

--------------------- 추가 문제 ---------------------

18.
다항식 \(x^3+ax^2-2x+1\)을 \(x^2+x+2\)로 나누었을 때의 몫이 \(x-1\)일 때, 상수 \(a\)의 값과 나머지를 구하여라.

\bigskip
19.
\[\frac{4x+ay+b}{x+y+1}\]
가 \(x\), \(y\)의 값에 관계없이 항상 일정한 값을 갖도록 하는 상수 \(a\), \(b\)의 값을 구하여라. (단, \(x+y\neq1\))

\bigskip
24.
등식 \(ax^3+bx^2+cx+d=3(x-1)^3-2(x-1)^2-4\)가 \(x\)에 대한 항등식일 때, 상수 \(a\), \(b\), \(c\), \(d\)의 값을 구하여라.

\bigskip
25.
다항식 \(x^3+ax^2-7x+b\)가 \(x-1\), \(x+2\), \(x-c\)를 인수로 가질 때, 상수 \(a\), \(b\), \(c\)의 합 \(a+b+c\)의 값을 구하여라.
\end{document}