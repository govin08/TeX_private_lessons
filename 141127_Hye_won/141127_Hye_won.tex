\documentclass{article}
\usepackage[a4paper,margin=1cm,footskip=.5cm]{geometry}
\usepackage{amsmath,amssymb,amsthm,amsfonts,mdframed,kotex}
\newcounter{problem}
\newcommand{\bp}
{\stepcounter{problem}\begin{mdframed}
[frametitle={\theproblem},skipabove=10pt,skipbelow=10pt,innerbottommargin=100pt]}
\newcommand{\ep}{\end{mdframed}}
\newcommand{\parall}{\mathbin{\!/\mkern-5mu/\!}}
\begin{document}

\section{평행사변형}

평행사변형은 두 쌍의 대변이 각각 평행한 사각형이다.

\bp

평행사변형의 두 대변의 길이는 각각 같다.
\ep

\bp

두 쌍의 대변의 길이가 각각 평행한 사각형은 평행사변형이다.
\ep

\bp

평행사변형의 두 대각의 크기는 각각 같다.
\ep

\bp

두 쌍의 대각의 크기가 각각 같은 사각형은 평행사변형이다.
\ep

\bp

한 쌍의 대변이 평행하고 그 길이가 같은 사각형은 평행사변형이다.
\ep

\bp

평행사변형의 두 대각선은 서로 다른 것을 이등분한다.
\ep


\bp

두 대각선이 서로를 이등분하는 사각형은 평행사변형이다.
\ep

\section{직사각형}

\bp

직사각형의 두 대각선의 길이는 같고 서로를 이등분한다.
\ep

\bp

한 내각이 직각인 평행사변형은 직사각형이다.
\ep

\bp

두 대각선의 길이가 같은 평행사변형은 직사각형이다.
\ep

\section{마름모}
마름모는 네 변의 길이가 모두 같은 사각형이다.
\bp

마름모의 두 대각선은 서로를 수직이등분한다.
\ep

\bp

이웃하는 두 변의 길이는 같은 평행사변형은 마름모이다.
\ep

\bp

두 대각선이 서로를 수직이등분하는 평행사변형은 마름모이다.
\ep

\section{정사각형}
정사각형은 네 내각의 크기가 모두 같고 네 변의 길이가 모두 같은 사각형이다.


\end{document}