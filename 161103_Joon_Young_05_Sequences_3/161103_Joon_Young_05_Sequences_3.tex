\documentclass{oblivoir}
\usepackage{amsmath,amssymb,amsthm,kotex,paralist,kswrapfig}

\usepackage[skipabove=10pt]{mdframed}

\usepackage{tabto,pifont}
\TabPositions{0.2\textwidth,0.4\textwidth,0.6\textwidth,0.8\textwidth}
\newcommand\tabb[5]{\par\bigskip\noindent
\ding{172}\:{\ensuremath{#1}}
\tab\ding{173}\:\:{\ensuremath{#2}}
\tab\ding{174}\:\:{\ensuremath{#3}}
\tab\ding{175}\:\:{\ensuremath{#4}}
\tab\ding{176}\:\:{\ensuremath{#5}}}

\usepackage{enumitem}
\setlist[enumerate]{label=(\arabic*)}

\newcounter{num}
\newcommand{\defi}[1]
{\bigskip\noindent\refstepcounter{num}\textbf{정의 \arabic{num}) #1}\par\noindent}
\newcommand{\theo}[1]
{\bigskip\noindent\refstepcounter{num}\textbf{정리 \arabic{num}) #1}\par\noindent}
\newcommand{\exam}[1]
{\bigskip\noindent\refstepcounter{num}\textbf{예시 \arabic{num}) #1}\par\noindent}
\newcommand{\prob}[1]
{\bigskip\noindent\refstepcounter{num}\textbf{문제 \arabic{num}) #1}\par\noindent}
\newcommand{\proo}
{\bigskip\textsf{증명)}\par}

\newcommand{\ans}{
{\par
\raggedleft\textbf{답 : (\qquad\qquad\qquad\qquad\qquad\qquad)}
\par}\bigskip\bigskip}

\newcommand{\pb}[1]%\Phantom + fBox
{\fbox{\phantom{\ensuremath{#1}}}}

\newcommand\summ[4]{\ensuremath{\displaystyle\sum_{#1=#2}^{#3}#4}}

\let\oldsection\section
\renewcommand\section{\clearpage\oldsection}
%%%%
\begin{document}

\title{준영 : 05 수열(3)}
\author{}
\date{\today}
\maketitle
\tableofcontents
\newpage

%%

%%
\section{합의 기호 \(\Sigma\)}
\begin{mdframed}[innertopmargin=-5pt]
%
\defi{합의 기호 \(\Sigma\)}
수열 \(\{a_n\}\)의 첫째항부터 제 \(n\)항까지의 합 \(a_1+a_2+\cdots+a_n\)은 합의 기호 \(\Sigma\)를 통해 다음과 같이 나타낸다.

\[a_1+a_2+\cdots+a_n=\summ k1n{a_k}\]
이때 \(\summ k1n{a_k}\)는 \(a_k\)의 \(k\)에 \(1\), \(2\), \(3\), \(\cdots\), \(n\)을 차례로 대입하여 얻은 항 \(a_1\), \(a_2\), \(a_3\), \(\cdots\), \(a_n\)의 합을 뜻한다.
\end{mdframed}

%
\exam{}
\begin{enumerate}
\item
\summ k1{10}{(2k-1)}은 \(2k-1\)의 \(k\)에 \(1\), \(2\), \(3\), \(\cdots\), \(n\)을 차례로 대입하여 얻은 항의 합이므로
\[\summ k1{10}{(2k-1)}=1+3+5+\cdots+19\]
\item
\(2+4+8+\cdots+2^{10}\)은 수열의 제 \(i\)항 \(2^i\)의 \(i\)에 \(1\), \(2\), \(3\), \(\cdots\), \(10\)을 차례로 대입하여 얻은 항을 모두 더한 것이므로 기호 \(\Sigma\)를 사용하여 나타내면,
\[2+4+8+\cdots+2^{10}=\summ i1{10}2^i\]
\end{enumerate}

%
\prob{}
다음을 합의 기호 \(\Sigma\)를 사용하지 않은 합의 꼴로 나타내어라.
\begin{enumerate}
\item
\(\summ k1{10}{2k}=2+4+6+8+\cdots+20\)
\item
\(\summ k17{3^k}=\)
\item
\(\summ k38{\sqrt k}=\)
\item
\(\summ m15{\frac1{2m+1}}=\)
\end{enumerate}

%
\prob{}
다음을 합의 기호 \(\Sigma\)를 사용하여 나타내시오.
\begin{enumerate}
\item
\(4+7+10+\cdots+31=\summ k1{10}(3k+1)\)
\item
\(3+3^2+3^3+\cdots+3^8=\)
\item
\(1+\frac12+\frac13+\frac14+\cdots+\frac1{15}=\)
\item
\(1^2+2^2+3^2+4^2+\cdots+n^2=\)
\item
\(\frac1{1\cdot2}+\frac1{2\cdot3}+\frac1{3\cdot4}+\frac1{4\cdot5}
+\cdots+\frac1{15\cdot16}=\)
\end{enumerate}

%
\prob{}\label{property_example}
다음을 계산하시오.
\begin{enumerate}
\item
\(\summ k1{10}(4k+2)
=6+10+14+18+\cdots+42
=\frac{10(6+42)}2=240\)
\item
\(\summ n1{10}n=\)
\item
\(\summ j1{10}2^j=\)
\end{enumerate}

%%
\section{수열의 성질}

%
\exam{}
다음 계산을 해보자.
\begin{enumerate}
\item
\(\summ k13{2^k}=\)
\item
\(\summ k13{k}=\)
\item
\(\summ k13{(2^k+k)}=\)
\item
\(\summ k13{(2^k-k)}=\)
\item
\(\summ k13{(3\cdot2^k)}=\)
\item
\(\summ k13{4}=\)
\end{enumerate}
위의 계산에서 (1)의 결과와 (2)의 결과를 더하면 (3)의 결과가 나오고, 빼면 (4)의 결과가 나온다.
또 (5)는 (1)의 결과에 3을 곱한 값이다.
이상을 정리하면 다음 정리를 얻을 수 있다.

\begin{mdframed}[innertopmargin=-5pt]
%
\theo{수열의 기본성질}\label{sequence_property}
수열 \(\{a_n\}\)과 \(\{b_n\}\), 실수 \(c\)에 대해 다음이 성립한다.
\begin{enumerate}[label=(\emph{\alph*})]
\item
\(\summ k1n{(a_k+b_k)}=\summ k1n{a_k}+\summ k1n{b_k}\)
\item
\(\summ k1n{(a_k-b_k)}=\summ k1n{a_k}-\summ k1n{b_k}\)
\item
\(\summ k1n{ca_k}=c\summ k1n{a_k}\)
\item
\(\summ k1nc=cn\)
\end{enumerate}
\end{mdframed}

\proo
\begin{enumerate}[label=(\emph{\alph*})]
\item
\begin{align*}
\summ k1n{(a_k+b_k)}
&=(a_1+a_2+\cdots+a_n)+(b_1+b_2+\cdots+b_n)\\
&=(a_1+b_1)+(a_2+b_2)+\cdots+(a_n+b_n)\\
&=\summ k1n{a_k}+\summ k1n{b_k}
\end{align*}
\item
\begin{align*}
\summ k1n{(a_k+b_k)}
&=(a_1+a_2+\cdots+a_n)-(b_1+b_2+\cdots+b_n)\\
&=(a_1-b_1)+(a_2-b_2)+\cdots+(a_n-b_n)\\
&=\summ k1n{a_k}-\summ k1n{b_k}
\end{align*}
\item
\begin{align*}
\summ k1n{ca_k}
&=ca_1+ca_2+\cdots+ca_n\\
&=c(a_1+a_2+\cdots+a_n)\\
&=c\summ k1n{a_k}
\end{align*}
\item
\(\summ k1nc=\underbrace{c+c+\cdots+c}_\text{\(n\)개}=cn\)
\end{enumerate}

\clearpage
%
\prob{}
\(\summ k1{10}{a_k}=30\), \(\summ k1{10}{b_k}=50\)일 때, 다음을 계산하여라.
\begin{enumerate}
\item
\(\summ k1{10}{(2a_k+b_k)}\stackrel{(a)}=\summ k1{10}{2a_k}+\summ k1{10}{b_k}
\stackrel{(c)}=2\summ k1{10}{a_k}+\summ k1{10}{b_k}=2\cdot30+50=110\)
\item
\(\summ k1{10}{(3a_k-b_k)}=\)
\item
\(\summ k1n{(2a_k+5)}=\)
\end{enumerate}

%
\prob{}\label{caution}
다음 \(a_n=n\), \(b_n=2^{n-1}\)일 때, 다음을 각각 계산하고 물음에 답하여라.
\begin{enumerate}
\item
\(\summ k13{{a_k}}=\)
\item
\(\summ k13{{a_k}^2}=\)
\item
\(\summ k13{{a_k}^2}\)와 \(\left(\summ k13a_k\right)^2\)는 서로 (같다, 다르다).
\item
\(\summ k13{a_kb_k}=\)
\item
\(\summ k13{b_k}=\)
\item
\(\summ k13{a_kb_k}\)와 \(\summ k13a_k\times\summ k13b_k\)는 서로 (같다, 다르다).
\end{enumerate}

%%
\section{자연수의 거듭제곱의 합}
%
\begin{mdframed}[innertopmargin=-5pt]
\theo{}\label{sequence_formula}
\begin{enumerate}[label=(\emph{\alph*})]
\item
\(\summ k1nk=\frac{n(n+1)}2\)
\item
\(\summ k1n{k^2}=\frac{n(n+1)(2n+1)}6\)
\item
\(\summ k1n{k^3}=\left\{\frac{n(n+1)}2\right\}^2\)
\end{enumerate}
\end{mdframed}
\proo
\begin{enumerate}
\item
등차수열의 합 공식을 이용하면
\[\summ k1nk=1+2+3+\cdots+n=\frac{n(a+l)}2=\frac{n(n+1)}2\]
\item
식 \((k+1)^3-k^3=3k^2+3k+1\)에 \(k\)대신 \(1\), \(2\), \(\cdots\), \(n\)을 차례로 대입하고
\begin{align*}
2^3-1^3&=3\cdot1^2+3\cdot1+1\\
3^3-2^3&=3\cdot2^2+3\cdot2+1\\
4^3-3^3&=3\cdot3^2+3\cdot3+1\\
&\vdots\\
(n+1)^3-n^3&=3\cdot n^2+3\cdot n+1
\end{align*}
이것들을 모두 더하면,
\begin{align*}
(n+1)^3-1^3=&3(1^2+2^2+3^2+\cdots+n^2)+3(1+2+3+\cdots+n)\\
&+(1+1+1+\cdots+1)\\
n^3+3n^2+3n=&3\summ k1n{k^2}+3\summ k1n{k}+\summ k1n1\\
n^3+3n^2+3n=&3\summ k1n{k^2}+3\cdot\frac{n(n+1)}2+n
\end{align*}
이다.
이것을 정리하면
\[\summ k1n{k^2}=\frac{n(n+1)(2n+1)}6\]
이 된다.
\item
(생략, (2)와 같은 방법을 적용하면 된다.)
\end{enumerate}

%
\prob{}
다음을 구하여라.
\begin{enumerate}
\item
\(1+2+3+\cdots+10=\frac{10\times11}2=55\)
\item
\(1^2+2^2+3^2+\cdots+10^2=\)
\item
\(1^3+2^3+3^3+\cdots+10^3=\)
\end{enumerate}

%
\prob{}
\begin{enumerate}
\item
\(2^2+4^2+6^2+\cdots+14^2=\summ k17{(2k)^2}=\summ k174k^2=4\summ k17k^2\\\)
\(=4\times\frac{7\times8\times15}6=560\)
\item
\(2^3+4^3+6^3+\cdots+12^3=\)
\end{enumerate}

%
\prob{}\label{sequence_formula_example}
다음을 간단히 하여라.
\par\bigskip\noindent
(1) \summ k1n{(2k-1)}
\begin{mdframed}
\(=2\summ k1nk-\summ k1n1=2\cdot\frac{n(n+1)}2-n=n^2\)
\end{mdframed}

\clearpage\noindent
(2) \summ k1n{k(k+1)}
\begin{mdframed}
\(=\:\summ k1n{(k^2+k)}=\summ k1nk^2+\summ k1nk\)\\
\(=\:\frac{n(n+1)(2n+1)}6+\frac{n(n+1)}2=\frac{n(n+1)}6\left\{(2n+1)+3\right\}\)\\
\(=\:\frac{n(n+1)(n+2)}3\)
\end{mdframed}

\par\bigskip\noindent
(3) \summ k1n{(k+1)^2}
\begin{mdframed}
\vspace{0.25\textheight}
\end{mdframed}

\par\bigskip\noindent
(4) \summ k1n{k(k+1)(k-1)}
\begin{mdframed}
\vspace{0.25\textheight}
\end{mdframed}

%%
\section{여러 가지 수열의 합}
%
\begin{mdframed}[innertopmargin=-5pt]
\theo{부분분수}\label{partial_fractions}
자연수 \(k\)와, 실수 \(A\), \(B\)에 대해(\(A\neq B\))
\begin{enumerate}
\item
\(\frac1{k(k+1)}=\frac1k-\frac1{k+1}\)
\item
\(\frac1{AB}=\frac1{B-A}\left(\frac1A-\frac1B\right)\)
\end{enumerate}
\end{mdframed}

%
\exam{}
\begin{enumerate}
\item
\(\frac16\)은 \(\frac1{2\times3}\)이므로 \(\frac12-\frac13\)이다.
마찬가지로 \(\frac1{12}=\frac13-\frac14\), \(\frac1{20}=\frac14-\frac15\) 이다.
\item
\(\frac1{15}\)은 \(\frac1{3\times5}\)이므로 \(\frac12\left(\frac13-\frac15\right)\)이다.
\end{enumerate}

%
\prob{}
다음 합을 구하여라.
\par\bigskip\noindent
(1) \summ k1n{\frac1{(2k-1)(2k+1)}}
\begin{mdframed}
\(=\frac1{1\cdot3}+\frac1{3\cdot5}+\frac1{5\cdot7}+\cdots+\frac1{(2n-1)(2n+1)}\)\\
\(=2\left(\frac11-\frac13\right)+2\left(\frac13-\frac15\right)
+2\left(\frac15-\frac17\right)+\cdots+2\left(\frac1{2n-1}-\frac1{2n+1}\right)\)\\
\(=2\left[\left(\frac11-\frac13\right)+\left(\frac13-\frac15\right)
+\left(\frac15-\frac17\right)+\cdots+\left(\frac1{2n-1}-\frac1{2n+1}\right)\right]\)\\
\(=2\left[\frac11-\frac1{2n+1}\right]=2\times \frac{2n}{2n+1}=\frac{4n}{2n+1}\)
\end{mdframed}

\clearpage\noindent
(2) \summ k1n{\frac1{k(k+1)}}
\begin{mdframed}
\vspace{0.3\textheight}
\end{mdframed}

\par\bigskip\noindent
(3) \summ k1n{\frac1{k(k+2)}}
\begin{mdframed}
\vspace{0.3\textheight}
\end{mdframed}

%
\clearpage
\prob{}
다음 합을 구하여라.
\par\bigskip\noindent
(1) \summ k1n{\frac1{\sqrt{k+1}+\sqrt k}}
\begin{mdframed}
\(=\frac1{\sqrt2+\sqrt1}+\frac1{\sqrt3+\sqrt2}
+\frac1{\sqrt4+\sqrt3}+\cdots+\frac1{\sqrt{n+1}+\sqrt n}\)\\
\(=\frac{1\times(\sqrt2-\sqrt1)}{(\sqrt2+\sqrt1)\times(\sqrt2-\sqrt1)}
+\frac{1\times(\sqrt3-\sqrt2)}{(\sqrt3+\sqrt2)\times(\sqrt3-\sqrt2)}\)\\
\(+\frac{1\times(\sqrt4-\sqrt3)}{(\sqrt4+\sqrt3)\times(\sqrt4-\sqrt3)}
+\cdots+\frac{1\times(\sqrt{n+1}-\sqrt n)}{(\sqrt{n+1}+\sqrt n)\times(\sqrt{n+1}-\sqrt n)}\)\\
\(=(\sqrt2-\sqrt1)+(\sqrt3-\sqrt2)+(\sqrt4-\sqrt3)+\cdots+(\sqrt{n+1}-\sqrt n)\)\\
\(=\sqrt{n+1}-1\)
\end{mdframed}

\par\bigskip\noindent
(2) \(\frac1{\sqrt3-1}+\frac1{\sqrt5-\sqrt3}+\frac1{\sqrt7-\sqrt5}
+\cdots+\frac1{\sqrt{121}-\sqrt{119}}\)
\begin{mdframed}
\vspace{0.3\textheight}
\end{mdframed}

\clearpage\noindent
(3) \(\summ k1{32}{\frac1{\sqrt{3k+4}-\sqrt{3k+1}}}\)
\begin{mdframed}
\vspace{0.3\textheight}
\end{mdframed}

%%
\section{군수열}

%
\prob{}
다음 수열 \(\{a_n\}\)의 110번째 항을 구하여라.
\[1,\:\:1,\:\:2,\:\:1,\:\:2,\:\:3,\:\:1,\:\:2,\:\:3,\:\:4,\:\:1,\:\:2,\:\:3,\:\:4,\:\:5,\:\:\cdots\]
\begin{mdframed}
이 수열을 다음과 같이 제 1군, 제 2군, 제 3군, \(\cdots\) 등으로 나누자.
\[
\underbrace{\stackrel{a_1}1}_{제 1군},\:\:
\underbrace{\stackrel{a_2}1,\:\:\stackrel{a_3}2}_{제 2군},\:\:
\underbrace{\stackrel{a_4}1,\:\:\stackrel{a_5}2,\:\:\stackrel{a_6}3}_{제 3군},\:\:
\underbrace{\stackrel{a_7}1,\:\:\stackrel{a_8}2,\:\:\stackrel{a_9}3,\:\:\stackrel{a_{10}}4}_{제 4군},\:\:
\underbrace{\stackrel{a_{11}}1,\:\:\stackrel{a_{12}}2,\:\:\stackrel{a_{13}}3,\:\:\stackrel{a_{14}}4,\:\:\stackrel{a_{15}}5}_{제 5군},\:\:\cdots
\]
제 1군의 마지막 항은 \(1\)번째, 제 2군의 마지막 항은 \((1+2)\)번째, 제 3군의 마지막 항은 \((1+2+3)\)번째, \(\cdots\) 제 \(n\)군의 마지막 항은 \(1+2+3+\cdots+n=\frac{n(n+1)}2\)번째이다.

그 중 \(\frac{13\times14}2=91<110<105=\frac{14\times15}2\)이므로 110번째 항은 제 13군의 마지막 항보다 뒤에 있고 제 14군의 마지막 항보다 앞쪽에 있다.
따라서 110번째 항은 제 14군에 속해있으며, 제 14군의 \(110-91=19\)번째 항이다.
제 14군의 19번째 항은 19이므로 \(a_{110}=19\)이다.
\end{mdframed}

%
\prob{}
다음 수열 \(\{a_n\}\)의 100번째 항을 구하여라.
\[\textstyle\frac12,\:\:\frac13,\:\:\frac23,\:\:\frac14,\:\:\frac24,\:\:\frac34,\:\:\frac15,\:\:\frac25,\:\:\frac35,\:\:\frac45,\:\:\cdots\]

\begin{mdframed}
\vspace{0.2\textheight}
\end{mdframed}

%%
\section{보충·심화 문제}

%
\prob{}
다음 중 옳은 것은?
\par\bigskip\noindent
\ding{172}
\(\summ k1n(a_k+1)=\summ k1na_k+1\)
\par\medskip\noindent
\ding{173}
\(\summ k1nk^2=\summ k0nk^2\)
\par\medskip\noindent
\ding{174}
\(\summ k1n2^k=\summ k0n2^k\)
\par\medskip\noindent
\ding{175}
\(\left(\summ k1nk\right)^2=\summ k1nk^2\)
\par\bigskip\noindent
\ding{176}
\(\left(\summ k1na_k\right)\left(\summ k1nb_k\right)=\left(\summ k1na_kb_k\right)\)
\vspace{0.2\textheight}

%
\prob{}
\(\summ k1{10}{a_k}=6\), \(\summ k1{10}{{a_k}^2}=10\)일 때, \summ k1{10}{\left(2a_k-3\right)^2}의 값을 구하여라.
\tabb{46}{50}{54}{58}{62}
\vspace{0.2\textheight}

%
\prob{}
다음 합을 구하여라.
\[\summ k1{10}{(2^k+4k)}\]
\tabb{2260}{2266}{2272}{2278}{2284}
\vspace{0.15\textheight}

%
\prob{}
다음 합을 구하여라.
\[\summ k1{10}{(k+1)^2}-\summ k1{10}{(2k+1)}\]
\tabb{55}{165}{275}{385}{495}
\vspace{0.15\textheight}

%
\prob{}
다음 합을 구하여라.
\[1\cdot2+2\cdot3+3\cdot4+\cdots+9\cdot10\]
\tabb{240}{330}{440}{560}{660}
\vspace{0.15\textheight}

%
\prob{}
다음 합을 구하여라.
\[1\cdot14+2\cdot13+3\cdot12+\cdots+14\cdot1\]
\tabb{240}{330}{440}{560}{660}
\vspace{0.15\textheight}

%
\prob{}
수열 \(\{a_n\}\)에 대하여 \(\summ k1n{(a_{2k-1}+a_{2k})}=n^2+3n\)일 때, \summ k1{12}{a_k}의 값을 구하여라.
\tabb{46}{50}{54}{58}{62}
\vspace{0.15\textheight}

%
\prob{}
자연수 \(n\)에 대하여 이차방정식 \(x^2-nx+2n-1=0\)의 두 근을 \(a_n\), \(b_n\)이라고 할 때, \summ k1{10}{\left({a_k}^2+{b_k}^2\right)}의 값을 구하여라.
\tabb{175}{180}{185}{190}{195}
\vspace{0.2\textheight}

%
\prob{}
다음 합을 구하여라.
\[\summ k1{20}{\frac1{(k+2)(k+3)}}\]
\tabb{\frac{21}{23}}{\frac{22}{23}}{\frac{19}{66}}{\frac{23}{66}}{\frac{20}{69}}
\vspace{0.15\textheight}

%
\prob{}
다음 합을 구하여라.
\[\summ k1{20}{\frac1{1+2+3+\cdots+k}}\]
\tabb{\frac{10}{21}}{\frac{20}{21}}{\frac{30}{21}}{\frac{40}{21}}{\frac{50}{21}}
\vspace{0.15\textheight}

%
\prob{}
\(\summ k1n{a_k}=n^2+2n\)일 때, \summ k1n{k\cdot a_{2k}}의 값을 구하여라.
\par\bigskip\noindent
\ding{172}\:\:\(\frac13n(n+1)(4n+7)\)
\tab\ding{173}\:\:\(\frac13n(n+1)(8n+7)\)
\tab\tab\ding{174}\:\:\(\frac16n(n+1)(4n+7)\)
\tab\ding{175}\:\:\(\frac16n(n+1)(8n+7)\)
\tab\tab\ding{176}\:\:\(\frac1{12}n(n+1)(4n+7)\)
\vspace{0.15\textheight}

%
\prob{}
\(\summ k1n{ka_k}=n(n+1)(n+2)\)일 때, 수열 \(\{a_n\}\)의 일반항을 구하여라.
\tabb{n+1}{2n+2}{3n+3}{4n+4}{5n+5}
\vspace{0.15\textheight}

%
\prob{}
다음 식의 값을 구하여라.
\[S=1\cdot3+2\cdot3^2+3\cdot3^3+\cdots+n\cdot3^n\]
\begin{mdframed}
\textbf{풀이 : }
주어진 식과 이 식의 양 변에 3를 곱한 식을 나란히 놓고,
\[
\begin{array}{c@{\:\:=\:\:}c@{\:\:}c@{\:\:+\:\:}c@{\:\:+\cdots+\:\:}c@{\:\:+\:\:}c@{\:\:}c}
S 	&1\cdot3\:\:+	&2\cdot3^2	&3\cdot3^3	&(n-1)\cdot3^{n-1}	&n\cdot3^n	&\\
3S 	&			&1\cdot3^2	&2\cdot3^3	&(n-2)\cdot3^{n-1}	&(n-1)\cdot3^n		&+\:\:n\cdot3^{n+1}
\end{array}
\]
두 식을 빼자.
\begin{align*}
S-3S&=3+3^2+3^3+\cdots+3^{n-1}+3^n-n\cdot3^{n+1}\\
-2S&=\frac{3(3^n-1)}{3-1}-n\cdot3^{n+1}\\
S&=\frac n2\cdot3^{n+1}-\frac34(3^n-1)\\
S&=\frac n2\cdot3^{n+1}-\frac14\cdot3^{n+1}+\frac34\\
S&=\frac{2n-1}4\cdot3^{n+1}+\frac34
\end{align*}
따라서 \(S=\frac{2n-1}4\cdot3^{n+1}+\frac34\)이다
\end{mdframed}
\vspace{0.2\textheight}

%
\prob{}
다음 식의 값을 구하여라.
\[S=3\cdot2+6\cdot2^2+9\cdot2^3+12\cdot2^4+\cdots+3n\cdot2^n\]
\par\bigskip\noindent
\ding{172}\:\:\((3n-3)\times2^n+3\)
\tab\ding{173}\:\:\((3n-6)\times2^n+3\)
\tab\tab\ding{174}\:\:\((3n-3)\times2^n+6\)
\tab\ding{175}\:\:\((3n-6)\times2^n+6\)
\tab\tab\ding{176}\:\:\((3n-3)\times2^n+9\)
\vspace{0.2\textheight}

%
\prob{}
수열 \(\frac11\), \(\frac12\), \(\frac22\), \(\frac13\), \(\frac23\), \(\frac33\), \(\frac14\), \(\frac24\), \(\frac34\), \(\frac44\), \(\cdots\)에서 \(\frac58\)은 몇 번째 항인지 구하여라.
\tabb{30번째}{31번째}{32번째}{33번째}{34번째}
\vspace{0.15\textheight}

%
\prob{}
수열 \(\frac11\), \(\frac12\), \(\frac21\), \(\frac13\), \(\frac22\), \(\frac31\), \(\frac14\), \(\frac23\), \(\frac32\), \(\frac41\), \(\cdots\)에서 50번째 항을 구하여라.
\tabb{\frac47}{\frac56}{\frac65}{\frac74}{\frac83}
\vspace{0.15\textheight}

%
\prob{}
다음 식을 간단히 하여라.
\[1+(1+2)+(1+2+3)+(1+2+3+4)+\cdots+(1+2+3+\cdots+n)\]
\tabb{\frac13{n(n+1)(n+2)}}{\frac16{n(n+1)(n+2)}}{\frac1{12}{n(n+1)(n+2)}}
{\frac13{n(n-1)(n+1)}}{\frac16{n(n-1)(n+1)}}
\vspace{0.15\textheight}

%
\prob{}
다음 식을 간단히 하여라.
\[1+(1+2)+(1+2+2^2)+(1+2+2^2+2^3)+\cdots+(1+2+2^2+\cdots+2^{n-1})\]
\tabb{2^{n+1}-(n+1)}{2^{n+1}-(n+2)}{2^{n+1}-(n+3)}
{2^{n+2}-(n+1)}{2^{n+2}-(n+2)}
\vspace{0.15\textheight}

%
\prob{}
다음을 계산하여라.
\[\frac12+\frac13+\frac23+\frac14+\frac24+\frac34+\cdots
+\frac1{50}+\cdots+\frac{49}{50}\]
\tabb{600}{\frac{1225}2}{625}{\frac{1275}2}{650}

\kswrapfig[Pos=r,Width=5cm]{lattices}{
\vspace{0.07\textheight}
(문제 \ref{start}--\ref{end})\\
좌표 평면 위에 점 \(P_1(1,1)\), \(P_2(1,2)\), \(P_3(2,1)\), \(P_4(1,3)\), \(P_5(2,2)\), \(P_6(3,1)\), \(\cdots\)을 순서대로 나타낼 때, 다음 물음에 답하여라.}

%
\prob{}\label{start}
\(P_{62}\)의 좌표를 구하여라.
\tabb{(7,5)}{(8,4)}{(9,3)}{(10,2)}{(11,1)}
\vspace{0.2\textheight}

%
\prob{}\label{end}
\(P_m=(4,11)\)을 만족시키는 \(m\)의 값을 구하여라.
\tabb{101}{102}{103}{104}{105}
\end{document}