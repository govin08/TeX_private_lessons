\documentclass{oblivoir}
\usepackage{amsmath,amssymb,amsfonts,amsthm,mathrsfs,kotex,mdframed,paralist}

\newcounter{num}
\newcommand{\prob}
{\bigskip\noindent\refstepcounter{num}\textbf{문제 \arabic{num})}\par}

\newcommand{\uu}{\ensuremath{\boldsymbol{u}}}
\newcommand{\xx}{\ensuremath{\boldsymbol{x}}}
\newcommand{\vv}{\ensuremath{\boldsymbol{v}}}
\newcommand{\ww}{\ensuremath{\boldsymbol{w}}}
\newcommand{\zz}{\ensuremath{\boldsymbol{0}}}

%%%
\begin{document}

\title{동락 04}
\author{}
\date{\today}
\maketitle
\newpage

\prob
\(B\)의 열벡터들을 각각 \(B_1\), \(B_2\), \(\cdots\), \(B_p\)이라고 하자 ;
\[
B=
\begin{bmatrix}
B_1&\cdots&B_p
\end{bmatrix},
\quad B_i\in\mathbb R^n.
\]
또 \(r=\text{rank}(A)\)라고 하자.

\(AB=\boldsymbol0\)에서 \(AB_i=\boldsymbol0\)이다(\(1\le i\le p\)).
%\begin{align*}
%AB_1&=\boldsymbol0\\
%&\vdots\\
%AB_n&=\boldsymbol0\\
%\end{align*}
따라서 \(B_i\in N(A)\) (\(N(A)\)는 \(A\)의 퇴화공간, \(1\le i\le p\)).
이때
\[\text{rank}(A)+\dim(N(A))=n\]
이므로 \(\dim(N(A))=n-r\).
\(B_i\)들이 \(n-r\)차원의 부분공간 내에 있으므로 \(B_i\)들은 최대 \(n-r\)개의 일차독립인 열들을 가진다.
즉 \(\text{rank}(B)\le n-r\).
따라서
\[\text{rank}(A)+\text{rank}(B)\le n.\]

\prob
\begin{gather*}
(가)=\begin{bmatrix}
1&0&0&-1\\
0&1&0&2\\
0&0&1&0
\end{bmatrix}\\
(나) : Rx=\boldsymbol0,
\left(R=\begin{bmatrix}
1&0&0&-1\\
0&1&0&2\\
0&0&1&0
\end{bmatrix}
\right)\\
(다)=\begin{bmatrix}
1\\-2\\0\\1
\end{bmatrix}
\\
(라)=1\\
(마)=3\\
(바): 1,2,3(번째)\\
(사)=2
\end{gather*}

\prob
(문제가 이상한 것 같은데..?)

반례)
\(\vv_1=(1,0,0)\), \(\vv_2=(0,1,0)\), \(\vv_3=(1,1,0)\)이면, \(\vv_1\)와 \(\vv_2\)는 일차독립, \(\vv_1\)와 \(\vv_3\)도 일차독립, \(\vv_2\)와 \(\vv_3\)도 일차독립이다.
하지만 \(\vv_1\), \(\vv_2\), \(\vv_3\)는 일차종속이다.

\prob
\[c_1\ww_1+\cdots+c_r\ww_r=\boldsymbol0\]
을 가정하자.
양변의 왼쪽에 \(A\)를 곱하면
\begin{gather*}
A\times\left(c_1\ww_1+\cdots+c_r\ww_r\right)=A\times\boldsymbol0,\\
c_1A\ww_1+\cdots+c_rA\ww_r=\boldsymbol0,\\
c_1\vv_1+\cdots+c_r\vv_r=\boldsymbol0.
\end{gather*}
\(\vv_1\), \(\vv_2\), \(\cdots\), \(\vv_r\)은 일차독립이므로 \(c_1=c_2=\cdots c_r=0\).
따라서 \(\ww_1\), \(\ww_2\), \(\cdots\), \(\ww_r\)은 일차독립이다.
\end{document}