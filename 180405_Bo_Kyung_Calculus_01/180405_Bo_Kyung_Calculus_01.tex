\documentclass[twocolumn]{article}
\usepackage{amsmath,amssymb,kotex,mdframed,paralist}
\usepackage{chngcntr}
\usepackage{geometry}
\geometry{a4paper,margin=1in}

\usepackage{tabto,pifont}
\TabPositions{0.2\textwidth,0.4\textwidth,0.6\textwidth,0.8\textwidth}
\newcommand\tabb[5]{\par\noindent
\ding{172}\:{\ensuremath{#1}}
\tab\ding{173}\:\:{\ensuremath{#2}}
\tab\ding{174}\:\:{\ensuremath{#3}}
\tab\ding{175}\:\:{\ensuremath{#4}}
\tab\ding{176}\:\:{\ensuremath{#5}}}

\newcounter{num}
\newcommand\pb[1]{\ensuremath{\fbox{\phantom{#1}}}}
\newcommand\lin{\displaystyle\ensuremath{\lim_{n\to\infty}}}

\pagestyle{empty}

\newcommand\prob[1]
{\vspace{40pt}\par\noindent\stepcounter{num} \textbf{문제 \thenum) #1}\par\medskip\noindent}

\counterwithout{subsection}{section}

\begin{document}
\begin{center}
\LARGE보경, 01 수열의 극한
\end{center}
\begin{flushright}
\today
\end{flushright}

%%
%\section{수열의 극한}

%%%
\subsection{극한값의 계산}

%
\prob{}
\(\lin\frac1n=0\)

%
\prob{}
\begin{align*}
\lin\frac{5n+4}{2n^2+3n+1}
&=\lin\frac{\frac5n+\frac4{n^2}}{2+\frac3n+\frac1{n^2}}\\
&=\frac{0+0}{2+0+0}=0
\end{align*}

%
\prob{}
\begin{align*}
\lin\frac{n^2+5n+4}{2n^2+3n+1}
&=\lin\frac{1+\frac5n+\frac4{n^2}}{2+\frac3n+\frac1{n^2}}\\
&=\frac{1+0+0}{2+0+0}=\frac12
\end{align*}

%
\prob{}
\begin{align*}
\lin\frac{-n^3+n^2+5n+4}{2n^2+3n+1}
&=\lin\frac{-n+1+\frac5n+\frac4{n^2}}{2+\frac3n+\frac1{n^2}}\\
&=-\infty
\end{align*}

%
\prob{}
\begin{align*}
\lin\frac{2n+7}{\sqrt{n^2+1}-1}
&=\lin\frac{2+\frac7n}{\sqrt{1+\frac1{n^2}}-\frac1n}\\
&=\frac{2+0}{\sqrt{1+0}-0}=2
\end{align*}

%
\prob{}
\begin{align*}
\lin\frac1{\sqrt{n^2+4n}-n}
&=\lin\frac{\sqrt{n^2+4n}+n}{4n}\\
&=\lin\frac{\sqrt{1+\frac4n}+1}4=\frac12
\end{align*}

%
\prob{}
\begin{align*}
\lin\frac{4^{n+2}}{2^{n+1}-4^n}
&=\lin\frac{16\times4^n}{2\times2^n-4^n}\\
&=\lin\frac{16}{2\times\left(\frac12\right)^n-1}\\
&=\frac{16}{0-1}=-16
\end{align*}

\subsection{극한의 성질}
\vspace{-20pt}
%
\prob{}
\(\lin a_n=2\), \(\lin(a_n-b_n)=3\)일 때,\\[10pt]
\begin{align*}
\lin b_n
&=\lin\left(a_n-(a_n-b_n)\right)\\
&=\lin a_n-\lin(a_n-b_n)=2-3=-1
\end{align*}

%
\prob{}
\(\lin(n+4)a_n=4\)일 때,\\[10pt]
\begin{align*}
\lin(2n+5)a_n
&=\lin\left\{(n+4)a_n\times\frac{2n+5}{n+4}\right\}\\
&=\lin(n+4)a_n\times\lin\frac{2n+5}{n+4}\\
&=4\times2=8
\end{align*}

%
\prob{}
\begin{mdframed}[frametitle=방법 1]
\(b_n=\frac{3a_n-2}{2a_n+1}\)라고 하면
\begin{gather*}
nb_n+4b_n=2n+5
\end{gather*}
\end{mdframed}
\end{document}