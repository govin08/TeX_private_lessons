\documentclass[twocolumn]{article}
\usepackage{amsmath,amssymb,kotex,mdframed,paralist}
\usepackage{chngcntr}
\usepackage{geometry}
\geometry{a4paper,margin=1in}

\usepackage{tabto,pifont}
\TabPositions{0.2\textwidth,0.4\textwidth,0.6\textwidth,0.8\textwidth}
\newcommand\tabb[5]{\par\noindent
\ding{172}\:{\ensuremath{#1}}
\tab\ding{173}\:\:{\ensuremath{#2}}
\tab\ding{174}\:\:{\ensuremath{#3}}
\tab\ding{175}\:\:{\ensuremath{#4}}
\tab\ding{176}\:\:{\ensuremath{#5}}}

\newcounter{num}
\newcommand\pb[1]{\ensuremath{\fbox{\phantom{#1}}}}
\newcommand\lin{\displaystyle\ensuremath{\lim_{n\to\infty}}}

\pagestyle{empty}

\newcommand\prob[1]
{\vspace{40pt}\par\noindent\stepcounter{num} \textbf{문제 \thenum) #1}\par\medskip\noindent}

\counterwithout{subsection}{section}

\begin{document}
\begin{center}
\LARGE보경, 01 수열의 극한
\end{center}
\begin{flushright}
\today
\end{flushright}

%%
%\section{수열의 극한}

%%%
\subsection{극한값의 계산}

\vspace{-20pt}

%
\prob{}
\(\lin\frac1n=\)

%
\prob{}
\(\lin\frac{5n+4}{2n^2+3n+1}=\)

%
\prob{}
\(\lin\frac{n^2+5n+4}{2n^2+3n+1}=\)

%
\prob{}
\(\lin\frac{-n^3+n^2+5n+4}{2n^2+3n+1}=\)

%
\prob{}
\(\lin\frac{2n+7}{\sqrt{n^2+1}-1}=\)

%
\prob{}
\(\lin\frac1{\sqrt{n^2+4n}-n}=\)

%
\prob{}
\(\lin\frac{4^{n+2}}{2^{n+1}-4^n}=\)

\newpage
\subsection{극한의 성질}
\vspace{-20pt}
%
\prob{}
\(\lin a_n=2\), \(\lin(a_n-b_n)=3\)일 때,\\[10pt]
\(\lin b_n=\)

%
\prob{}
\(\lin(n+4)a_n=4\)일 때,\\[10pt]
\(\lin(2n+5)a_n=\)

%
\prob{}
\(\lin\frac{3a_n-2}{2a_n+1}=3\)일 때,\\[10pt]
\(\lin a_n=\)
\end{document}