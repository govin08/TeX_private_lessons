\documentclass{oblivoir}
\usepackage{amsmath,amssymb,amsthm,kotex,paralist,kswrapfig,tabu}

\usepackage[skipabove=10pt,skipbelow=10pt,innertopmargin=10pt]{mdframed}

\usepackage{tabto,pifont}
\TabPositions{0.2\textwidth,0.4\textwidth,0.6\textwidth,0.8\textwidth}
\newcommand\tabb[5]{\par\bigskip\noindent
\ding{172}\:{\ensuremath{#1}}
\tab\ding{173}\:\:{\ensuremath{#2}}
\tab\ding{174}\:\:{\ensuremath{#3}}
\tab\ding{175}\:\:{\ensuremath{#4}}
\tab\ding{176}\:\:{\ensuremath{#5}}}

\usepackage{enumitem}
\setlist[enumerate]{label=(\arabic*)}

\newcounter{num}
\newcommand{\defi}[1]
{\noindent\refstepcounter{num}\textbf{정의 \arabic{num}) #1}\par\noindent}
\newcommand{\theo}[1]
{\noindent\refstepcounter{num}\textbf{정리 \arabic{num}) #1}\par\noindent}
\newcommand{\exam}[1]
{\bigskip\bigskip\noindent\refstepcounter{num}\textbf{예시 \arabic{num}) #1}\par\noindent}
\newcommand{\prob}[1]
{\bigskip\bigskip\noindent\refstepcounter{num}\textbf{문제 \arabic{num}) #1}\par\noindent}
\newcommand{\proo}
{\bigskip\textsf{증명)}\par}

\newcommand\summ[3]{\ensuremath{\displaystyle\sum_{#1=#2}^{#3}}}

\newcommand{\procedure}[1]{\begin{mdframed}\vspace{#1\textwidth}\end{mdframed}}
\newcommand{\ans}{
{\par\raggedleft\textbf{답 : (\qquad\qquad\qquad\qquad\qquad\qquad)}\par}\bigskip\bigskip}
\newcommand\an[1]{\par\bigskip\noindent\textbf{문제 #1)}\\}

\newcommand{\pb}[1]%\Phantom + fBox
{\fbox{\phantom{\ensuremath{#1}}}}

\newcommand\ba{\,|\,}

\let\oldsection\section
\renewcommand\section{\clearpage\oldsection}
\counterwithout{subsection}{section}

\newenvironment{talign}
 {\let\displaystyle\textstyle\align}
 {\endalign}
\newenvironment{talign*}
 {\let\displaystyle\textstyle\csname align*\endcsname}
 {\endalign}

\let\emph\textsf

\newcommand\blfootnote[1]{%
  \begingroup
  \renewcommand\thefootnote{}\footnote{#1}%
  \addtocounter{footnote}{-1}%
  \endgroup
}

%\usepackage{fapapersize}
%\usefapapersize{210mm,297mm,45mm,45mm,15mm,15mm}
%%%

\begin{document}

\title{윤영 : 14 수열의 합 계산 연습}
\author{}
\date{\today}
\maketitle

\prob{}
다음 합을 구하여라.
\begin{enumerate}
\item
\(\summ k1n(2k-1)\)
\item
\(\summ k1nk(k+1)\)
\item
\(\summ k1n(k+1)^2\)
\item
\(\summ k1nk(k+1)(k-1)\)
\end{enumerate}
%{\par\raggedleft\textbf{답 : } (1) \(n^2\), (2) \(\frac{n(n+1)(n+3)}3\), 
%(3) \(\frac16n(2n^2+9n+13)\), (4) \(\frac14n(n-1)(n+1)(n+2)\)
%}\bigskip\bigskip


\prob{}
다음 수열의 첫째항부터 제 \(n\)항까지의 합을 구하여라.
\begin{enumerate}
\item
\(2\cdot3\), \(3\cdot4\), \(4\cdot5\), \(\cdots\), \((n+1)(n+2)\), \(\cdots\)
\item
\(1^2\), \(3^2\), \(5^2\), \(\cdots\), \((2n-1)^2\), \(\cdots\)
\end{enumerate}
\blfootnote{\par
문제 1 : (1) \(n^2\), (2) \(\frac{n(n+1)(n+2)}3\), (3) \(\frac16n(2n^2+9n+13)\), (4) \(\frac14n(n-1)(n+1)(n+2)\)

문제 2 : (1) \(\frac13n(n^2+6n+11)\), (2) \(\frac13n(2n+1)(2n-1)\)
}

\thispagestyle{empty}
\end{document}