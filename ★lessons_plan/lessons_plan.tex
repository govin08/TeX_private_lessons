\documentclass{oblivoir}
\usepackage{kotex,amsmath,mdframed,tabu,multirow}

%\renewcommand\section{\clearpage\oldsection}
\renewcommand{\arraystretch}{1.5}
%\counterwithout{subsection}{section}
\renewcommand{\thesubsection}{(\arabic{subsection})}
%\renewcommand{\thesubsection}{\Alph{subsection}.}

\makeatletter
\def\@fnsymbol#1{\ensuremath{
	\ifcase#1\or*\or **\or ***\or\star\or\star\star\or\star\star\star\or\dagger\or\dagger\dagger\or\dagger\dagger\dagger\else\@ctrerr\fi}}
\renewcommand{\thefootnote}{\fnsymbol{footnote}}
\makeatother

\begin{document}
\title{교습 계획서}
\author{김선중}
\date{
작성일자 : 2018년 9월 19일\\
최종수정 : \today}
\maketitle
\tableofcontents

\newpage

%%
\section{간단한 이력}
\begin{tabu}{X[2]X[1]|X[12]}
\multicolumn{2}{c|}{기간}	&\multicolumn{1}{c}{학력 / 경력}			\\\hline
1988년	&9월			&출생(강원도 춘천)	\\
1995년	&3월			&호반초등학교 입학	\\
2001년	&2월			&우석초등학교 졸업	\\
2001년	&3월			&소양중학교 입학	\\
2003년	&				&교육청 주관 수학경시대회 참가(도대회 동상)\\
2003년	&				&과학고등학교 지원 - 1차 전형(내신)에서 불합격\\
2004년	&2월			&소양중학교 졸업		\\
2001년	&3월			&춘천고등학교 입학	\\
2006년	&				&첫 수능을 치르고 한양대학교 등을 지원\\
2007년	&2월			&춘천고등학교 졸업	\\
2007년	&				&강남종로학원에서 재수\\
2007년	&				&고려대학교 수학과  합격(수시, 논술전형)\\
2008년	&3월			&고려대학교 수학과 입학\\
\multicolumn{2}{c|}{2009--2010년}	&군복무(육군, 양평)\\
\multicolumn{2}{c|}{2011--2012년}	&TeX 공부모임 참여\\
\multicolumn{2}{c|}{2011--2012년}	&URP(학부생 연구 프로그램) 참여\\
%\multicolumn{2}{c|}{2012년 -- 2013년}	&QoLT, 시각장애인 학생을 위한 미적분학\\
%2014년	&				&대학원 진학 유예\\
2015년	&2월			&고려대학교 수학과 졸업\\
\multicolumn{2}{c|}{2014년 -- 2018년}	&수학 과외\\
2018년	&8월			&개인 사업자 등록\\
\end{tabu}

%%
\section{인사말}
안녕하세요.
몇 년째 수학 과외 일에 종사하고 있는 김선중이라고 합니다.

저는 중학교 3학년 때, 수학 경시대회에 참여했던 것을 계기로 수학에 많은 흥미를 가지게 되었습니다.
물론 그전부터도 수학을 좋아했었지만, 경시대회를 계기로 본격적으로 수학을 진지하게 생각하게 되었습니다.

고등학교에 입학해서도 수학에 대한 흥미는 깊었습니다.
친구 한 명과 함께 그룹과외를 받으며, <수학의 정석> 문제들을 천천히 공부하는 것을 좋아했습니다.
고등학교에서도 포항공대 경시대회, KMO 등에 참여했지만 입상하지는 못했습니다.

수능은 두 번 치렀지만, 대단한 성적을 거두진 못했습니다.
첫 수능에서는 한양대학교 합격권에 들었으나 입학하지 않고 재수를 하게 되었습니다.
재수에서도 고려대학교를 들어갈 만큼의 성적은 나오지 않았으나 수리 논술 위주의 수시 전형에 합격하여 고려대학교 수학과에 입학하게 되었습니다.

두 수능에서의 수학 등급은 모두 2등급으로, (두 번째 수능에서는 한 문제를 틀려 2등급이라고 하더라도) 전혀 자랑할 만한 것이 못됩니다.
하지만, 수학에 대한 저의 애정은 그 누구보다도 깊고, 또 대학 시절 동안에도 진지하게 수학을 공부했기 때문에, 그 점에서는 제가 장점을 가진다고 생각합니다.
이는 대학 시절, 수학 전공과목의 학점이 나쁘지 않았는 점으로도 증명할 수 있을 거라고 생각합니다.

이것이 제가 고2--고3 학생을 맡기를 꺼리는 이유입니다.
제가 대입에 있어서 크게 강점을 가지지 않는다는 점, 또 시험을 위한 수학 공부는 제 스스로도 싫어했다는 점 등때문에 수능이나 대입을 앞둔 고2--고3 학생에게는 제 존재가 큰 도움이 되지 않을 거라고 생각합니다.
그 학생들에게는 더 전문적인 분의 도움을 받는 것이 옳다고 봅니다.
다만, 제가 할 수 있는 일은, 수학 개념을 처음 배우는 중학교 학생들과 고1학생들에게 자세하고 친절하게 수학공부를 도와주는 것이라고 생각합니다.

거듭 말씀드리지만, 수학에 관해서는 자신이 있습니다.
수능 수학을 풀면 다 맞진 못하더라도 1등급 점수는 충분히 나옵니다.
모든 고등학교 수학문제와 심지어는 대학 수준의 수학에 대해서도 어느 정도 질문에 답할 수 있습니다.
하지만 고3 학생들을 대상으로 체계적인 커리큘럼을 제공하는 것에는 한계를 느낍니다.

따라서 맡은 학생이 고등학교 2학년이 되면, 저는 미련없이 학생을 더 맡지 않을 생각입니다.
나중에 조금 더 경험이 쌓이고 제 스스로도 준비가 완전히 되었을 때 고등학교 2학년 학생과 3학년 학생도 맡을 생각입니다.

학부를 졸업하고 나서 대학원에 진학하지 않은 것에 대해서는 저로서도 많은 아쉬움이 있습니다.
이 선택에 관해서는 여러 이야기를 할 수 있지만, 결국 제 능력이 부족했기 때문이라고 말할 수밖에 없을 것 같습니다.
그래도 대학에 다니는 동안 하나 얻은 것이 있다면, 몇몇 교수님들과 교류하면서 \TeX이라는 문서편집기의 사용법을 배운 것입니다.

이것은 주로 논문이나 책을 조판할 때 쓰이는 문서편집기로, MS word나 아래아 한글과 같은 다른 워드프로세서에 비해 수식을 깔끔하게 출력합니다.
또한 같은 시기에 Geogebra라는 프로그램도 배웠는데, 이것은 중·고등학교 교과서에도 자주 등장하는 도형-그래프 소프트웨어입니다.
저는 과외 자료들을 만들 때, 이  \TeX과 Geogebra를 사용하여 만듭니다.
필요한 프린트물을 만들거나, 시험 대비 자료를 만들 때에 주로 사용합니다.
자료들을 직접 만드는 작업은 시간이 꽤 들기 때문에 많은 양을 작업하지 못할 수는 있습니다.

저는 지난 여름동안, 이러한 도구들을 사용해 작게나마 개념 소책자들을 만들었습니다.%
\footnote{고등학교 1학년 수학에 대해서만 만들었고, 앞으로 고등학교 2학년 수학이나 중학교 수학에 대해서도 만들 예정입니다.
따라서 현재로서는 프로그램 B에만 적용됩니다.
생각보다 시간이 굉장히 많이 걸리는 작업이기 때문에 앞으로의 작업에 얼만큼의 진척이 있을지는 예상하기 힘듭니다.}
이 소책자들을 활용하면 학생들이 처음 고등학교 수학을 접하는 데 있어 도움이 될 것이라고 생각합니다.
물론 기본적으로는, 중학교 개념서 <개념유형>과 고등학교 개념서 <수학의 정석> (혹은 <개념유형>)을 주로 사용하게 됩니다.

\bigskip
다음은 중학교 학생들과 고등학교 1학년 학생을 대상으로 제공할 수 있는 두 개의 프로그램 A, B를 간략히 표로 나타낸 것입니다.

%%
\section{프로그램}
\begin{center}
\centering\small
\begin{tabu}{|X[c]|X[c]X[1.3,c]X[c]X[2,c]|}
\hline
구분	&대상			&교습시간	&월 교습료	&내용		\\\hline
A	&중1 \(\sim\) 중3	&1시간 30분	&30만원		&내신대비 위주\\
B	&중3 \(\sim\) 고1	&2시간		&40만원		&선행학습 위주\\\hline
\end{tabu}
\end{center}

%고등학교 수학의 선행학습을 시작하는 때를 기준으로 하여 두 프로그램으로 나뉩니다.
고등학교 수학의 선행학습을 시작하기 전에는 A프로그램을, 시작한 이후로는 B프로그램을 적용하게 됩니다.
B프로그램으로 넘어가는 시기는 중학교 3학년 여름방학이 가장 적절하지만  학생에 따라 더 일찍 시작할 수도 있고 더 늦게 시작할 수도 있습니다.

\newpage
%
\subsection{프로그램 A}
\begin{center}
\centering\small
\begin{tabu}[t]{|X[c]|X[c]X[c]X[c]X[c]X[c]|}
\hline
구분	&선행학습	&유형문제	&심화문제	&내신대비	&모의고사\\\hline
A	&\(\triangle\)	&\(\bigcirc\)	&\(\triangle\)	&\(\bigcirc\)	&\(\times\)\\\hline
\end{tabu}
\end{center}
\paragraph{선행학습}
여기서 말하는 선행학습은 중학교 수학에 대한 선행학습입니다.
중학교 2학년 1학기까지는 웬만하면 선행학습을 하지 않습니다.
중학교 2학년 2학기와 중학교 3학년 1학기에, 중학교 3학년 수학에 대해서 선행학습을 진행합니다.
\begin{itemize}
\item
<개념유형> - 개념편
\item
<개념유형> - 유형편 light
\end{itemize}
\paragraph{유형문제}
각 단원의 유형별 문제를 풉니다.
\begin{itemize}
\item
<개념유형> - 유형편 power
\end{itemize}
\paragraph{심화문제}
유형문제들을 푸는 데 어려움이 없다면, 조금 더 어려운 문제를 도전해봅니다.
이 과정은 생략될 수 있습니다.
\begin{itemize}
\item
<최상위수학>
\item
<A급수학>
\end{itemize}

\paragraph{내신대비}
중간고사 및 기말고사 5개년 기출문제와 교과서의 연습문제를 풉니다.
만약 두 명 이상의 학생이 같은 학년이고 같은 교과서를 사용한다면, 교과서 연습문제를 변형한 문제들도 제공하여 풉니다.
\begin{itemize}
\item
내신 5개년 기출문제
\item
교과서 연습문제
\item
(교과서 변형문제)
\end{itemize}

\newpage
%
\subsection{프로그램 B}
\vspace{-10pt}
\begin{center}
\centering
\begin{tabu}[t]{|X[c]|X[c]X[c]X[c]X[c]X[c]|}
\hline
구분	&선행학습	&유형문제	&심화문제	&내신대비	&모의고사\\\hline
B	&\(\bigcirc\)	&\(\bigcirc\)	&\(\triangle\)	&\(\bigcirc\)	&\(\triangle\)\\\hline
\end{tabu}
\end{center}
\paragraph{선행학습}
중학교 3학년 과정까지 선행학습이 완료되고, 고등학교 과정을 공부하는 데 어려움이 없을 거라고 예상되면 프로그램 B로 넘어갑니다.
자체 제작한 소책자를 이용해 먼저 기본적인 개념을 다지고, <수학의 정석> 혹은 <개념유형>을 이용해 기본적인 문제들을 풀어가며 진도를 나갑니다.
\begin{itemize}\tightlist
\item
<수학의 정석> - 기본 / 기본문제 및 유제
\item
<개념유형>
\end{itemize}
\paragraph{유형문제}
각 단원의 유형별 문제를 풉니다.
<RPM>과 <쎈수학> 중 하나의 문제집을 골라 풉니다.
\begin{itemize}\tightlist
\item
<RPM> / 모든 문제
\item
<쎈수학> / B단계
\end{itemize}
\paragraph{심화문제}
유형문제들을 푸는 데 어려움이 없다면, 조금 더 어려운 문제를 도전해봅니다.
이 과정은 생략될 수 있습니다.
\begin{itemize}\tightlist
\item
<수학의 정석> - 기본 / 연습문제
\item
<수학의 정석> - 실력 / 연습문제
%\item
%<쎈수학> / C단계
\item
<EBS 올림포스>
\end{itemize}

\paragraph{내신대비}
중간고사 및 기말고사 5개년 기출문제와 교과서의 연습문제를 풉니다.
만약 두 명 이상의 학생이 같은 학년이고 같은 교과서를 사용한다면, 교과서 연습문제를 변형한 문제들도 제공하여 풉니다.
\begin{itemize}\tightlist
\item
내신 5개년 기출문제
\item
교과서 연습문제
\item
(교과서 유사문제)
\end{itemize}

\paragraph{모의고사 대비}
교육청 모의고사 기출문제를 최대 5개년까지 풉니다.
해야 하는 다른 일들이 많은 경우에는 생략될 수 있습니다.
\begin{itemize}\tightlist
\item
모의고사 5개년 기출문제
\end{itemize}

\newpage

%%
\section{참고사항}
%
\subsection{수업 노트}
학생의 방에 수업 노트 한 권을 비치합니다.
여기에는 각 수업에서 어떤 것을 배웠고, 다음 시간까지의 과제가 무엇인지에 대한 사항들이 간략하게 적히게 됩니다.

%
\subsection{가능한 교습시간}
다음과 같이 지정된 시간에만 교습합니다.
단, 토요일 시간은 학생 사정에 따라 변동될 수 있습니다.
\begin{center}
\small
\begin{tabu}to.8\textwidth{|X[c]|X[c]|X[c]|}
\hline
요일							&시작시간	&비고\\\hline
\multirow{2}{*}{월요일, 목요일}	&오후 5시	&프로그램 A만 가능\\\cline{2-3}
							&오후 8시	&교습 진행중\\\hline
\multirow{2}{*}{화요일, 금요일}	&오후 5시	&교습 가능\\\cline{2-3}
							&오후 8시	&교습 가능\\\hline
\multirow{2}{*}{수요일, 토요일}	&오후 5시	&교습 가능\\\cline{2-3}
							&오후 8시	&교습 가능\\\hline
\end{tabu}
\normalsize
\end{center}

%
\subsection{교습비 납입}
매월 교습을 시작한 날에 하나은행 288-910282-83007(예금주 : 김선중)으로 입금해주시면 됩니다.
예를 들어 10월 9일에 프로그램 B가 시작되었다면, 매월 9일마다 40만원씩 입금해주시면 됩니다.

%%
%\subsection{교재구입비용}
%제 교재는 제가 구입하고 학생의 교재는 학생이 구입합니다.

%
\subsection{상담전화}
매월 1일(해당 월의 1일이 주말이나 공휴일인 경우에는 2일이나 3일)에 정기적으로 상담전화를 드립니다.
궁금하신 점이나 건의사항이 있으시면 말씀해주시면 됩니다.
특이사항이 없을 때에는 간단히 용건만 전하고 끊습니다.

%
\subsection{단체교습(그룹과외)}
두 명 혹은 세 명의 학생에 대해 그룹과외를 진행할 수 있습니다.
교습시간은 위와 동일하며, 월 교습비는 한 명이 추가될 때마다 5만원씩을 뺀 금액으로 합니다.
근처 스터디카페나 일반 카페에서 진행하며, 음료값은 각자 부담하면 좋을 것 같습니다.%
\footnote{개인교습의 경우에도 근처 카페에서 진행할 수 있습니다.}
\begin{center}
\small
\begin{tabu}to.8\textwidth{|X[c]|X[c]|X[c]|X[c]|}
\hline
프로그램			&인원	&월 교습료	&장소\\\hline
\multirow{2}{*}{A}	&2명	&각각 25만원	&\multirow{4}{*}{근처 카페}\\\cline{2-3}
				&3명	&각각 20만원	&\\\cline{1-3}
\multirow{2}{*}{B}	&2명	&각각 35만원	&\\\cline{2-3}
				&3명	&각각 30만원	&\\\hline
\end{tabu}
\normalsize
\end{center}

%
\subsection{환불규정}
아래와 같은 환불규정에 따라 교습비의 일부 또는 전부가 환불될 수 있습니다.
이 환불규정은 교육지원청에서 제시하고 있는 규정을 따른 것입니다.
\begin{center}
\small
\begin{tabu}to.8\textwidth{|X[c]|X[c]|}
\hline
반환사유 발생일			&반환금액\\\hline
교습시작 전				&교습비 전액\\
8회 중 1회 \(\sim\) 2회 교습	&교습비의 2/3\\
8회 중 3회 \(\sim\) 4회 교습	&교습비의 1/2\\
8회 중 4회 이상 교습		&반환하지 않음\\\hline
\end{tabu}
\normalsize
\end{center}


%%
\cftaddtitleline{toc}{section}{\;\;* 첨부문서}{9}
\cftaddtitleline{toc}{subsection}{(1) 교습비 등 영수증}{9}
\cftaddtitleline{toc}{subsection}{(2) 사업자등록증 사본}{10}
\cftaddtitleline{toc}{subsection}{(3) 졸업증명서}{11}
\cftaddtitleline{toc}{subsection}{(4) 성적증명서}{12}
\cftaddtitleline{toc}{subsection}{(5) 주민등록증 사본}{14}

\end{document}