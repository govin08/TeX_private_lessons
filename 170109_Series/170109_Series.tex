\documentclass{oblivoir}
\usepackage{amsmath,amssymb,amsthm,kotex,paralist,kswrapfig}

\usepackage[skipabove=10pt,skipbelow=10pt]{mdframed}

\usepackage{tabto,pifont}
\TabPositions{0.2\textwidth,0.4\textwidth,0.6\textwidth,0.8\textwidth}
\newcommand\tabb[5]{\par\bigskip\noindent
\ding{172}\:{\ensuremath{#1}}
\tab\ding{173}\:\:{\ensuremath{#2}}
\tab\ding{174}\:\:{\ensuremath{#3}}
\tab\ding{175}\:\:{\ensuremath{#4}}
\tab\ding{176}\:\:{\ensuremath{#5}}}

\usepackage{enumitem}
\setlist[enumerate]{label=(\arabic*)}

\newcounter{num}
\newcommand{\defi}[1]
{\bigskip\noindent\refstepcounter{num}\textbf{정의 \arabic{num}) #1}\par\noindent}
\newcommand{\theo}[1]
{\bigskip\noindent\refstepcounter{num}\textbf{정리 \arabic{num}) #1}\par\noindent}
\newcommand{\exam}[1]
{\bigskip\noindent\refstepcounter{num}\textbf{예시 \arabic{num}) #1}\par\noindent}
\newcommand{\prob}[1]
{\bigskip\noindent\refstepcounter{num}\textbf{문제 \arabic{num}) #1}\par\noindent}
\newcommand{\proo}
{\bigskip\textsf{증명)}\par}

\newcommand{\ans}{
{\par
\raggedleft\textbf{답 : (\qquad\qquad\qquad\qquad\qquad\qquad)}
\par}\bigskip\bigskip}
\newcommand\an[1]{\par\bigskip\noindent\textbf{문제 #1)}\\}

\newcommand{\pb}[1]%\Phantom + fBox
{\fbox{\phantom{\ensuremath{#1}}}}

\newcommand{\procedure}[1]{\begin{mdframed}\vspace{#1\textheight}\end{mdframed}}

\let\oldsection\section
\let\emph\textsf
\renewcommand\section{\clearpage\oldsection}
%%%%
\begin{document}

\title{미적분 1 : 02 급수}
\author{}
\date{\today}
\maketitle
\tableofcontents
\newpage

%%
\section{부분합과 급수의 합}

\begin{mdframed}[innertopmargin=-5pt]
%
\defi{부분합과 급수의 합}
수열 \(\{a_n\}\)에서 첫항부터 제 \(n\)항까지의 합
\[S_n=\sum_{k=1}^na_k=a_1+a_2+\cdots+a_n\]
을 \emph{부분합}이라고 부른다.

%또
%\[S=\sum_{k=1}^\infty a_k=a_1+a_2+\cdots+a_n+\cdots\]
%를 \emph{급수}라고 부른다.

이때 수열 \(\{S_n\}\)이 수렴하면 그 극한값
\begin{align*}
\lim_{n\to\infty}S_n
&=\lim_{n\to\infty}\sum_{k=1}^na_k=\sum_{n=1}^\infty a_n=a_1+a_2+\cdots+a_n+\cdots
\end{align*}
을 \emph{급수의 합}이라고 부른다.
\end{mdframed}

%
\exam{}
다음 수열의 부분합과 급수의 합을 구하여라.
\begin{enumerate}
\item
\(\frac1{1\cdot2}\), \(\frac1{2\cdot3}\), \(\frac1{3\cdot4}\), \(\cdots\), \(\frac1{n(n+1)}\), \(\cdots\)
\item
\(\frac12\), \(\left(\frac12\right)^2\), \(\left(\frac12\right)^3\), \(\cdots\), \(\left(\frac12\right)^n\), \(\cdots\)
\end{enumerate}

\begin{mdframed}
(1)
부분합 \(S_n\)은
\begin{align*}
S_n
&=\frac1{1\cdot2}+\frac1{2\cdot3}+\frac1{3\cdot4}+\cdots+\frac1{n(n+1)}\\
&=\left(\frac11-\frac12\right)+\left(\frac12-\frac13\right)+\left(\frac13-\frac14\right)
+\cdots+\left(\frac1n-\frac1{n+1}\right)\\
&=1-\frac1{n+1}=\frac n{n+1}
\end{align*}
이고, 급수의 합 \(S\)는
\[S=\lim_{n\to\infty}S_n=\lim_{n\to\infty}\frac n{n+1}=1\]
이다.
\end{mdframed}

\begin{mdframed}
(2)
부분합 \(S_n\)은
\begin{align*}
S_n
&=\frac12+\left(\frac12\right)^2+\left(\frac12\right)^3+\cdots+\left(\frac12\right)^n\\
&=\frac{\frac12\left\{1-\left(\frac12\right)^n\right\}}{1-\frac12}\\
&=1-\left(\frac12\right)^n
\end{align*}
이고, 급수의 합 \(S\)는
\[S=\lim_{n\to\infty}S_n=\lim_{n\to\infty}\left\{1-\left(\frac12\right)^n\right\}=1\]
이다.
\end{mdframed}
{\par
\raggedleft\textbf{답 : (1) 부분합=\(\frac n{n+1}\), 급수의 합=\(1\),\quad(2) 부분합=\(1-\left(\frac12\right)^n\), 급수의 합=\(1\)}
\par}\bigskip\bigskip

%
\prob{}
다음 수열의 부분합과 급수의 합을 구하여라.
\begin{enumerate}
\item
\(\frac1{1\cdot3}\), \(\frac1{3\cdot5}\), \(\frac1{5\cdot7}\), \(\cdots\), \(\frac1{(2n-1)(2n+1)}\), \(\cdots\)
\item
\(\frac23\), \(\left(\frac23\right)^2\), \(\left(\frac23\right)^3\), \(\cdots\), \(\left(\frac23\right)^n\), \(\cdots\)
\end{enumerate}
\procedure{0.3}
\procedure{0.2}
{\par
\raggedleft\textbf{답 : (1) \qquad\qquad\qquad\qquad\qquad\qquad(2) \qquad\qquad\qquad\qquad\qquad\qquad}
\par}\bigskip\bigskip

%
\exam{}
다음 급수의 합을 구하여라.
\[1+\frac1{1+2}+\frac1{1+2+3}+\cdots\]

\vspace{-10pt}
\begin{mdframed}
제 \(n\)항을 \(a_n\)이라고 하면
\begin{align*}
a_n
&=\frac1{1+2+\cdots+n}\\
&=\frac1{\frac12n(n+1)}=\frac2{n(n+1)}\\
&=2\left(\frac1n-\frac1{n+1}\right)
\end{align*}
이다.
부분합을 \(S_n\)이라고 하면,
\begin{align*}
S_n
&=2\left(\frac11-\frac12\right)+2\left(\frac12-\frac13\right)
+\cdots+2\left(\frac1n-\frac1{n+1}\right)\\
&=2\left(1-\frac1{n+1}\right)
\end{align*}
따라서 급수의 합 \(S\)는
\[
S=\lim_{n\to\infty}S_n=\lim_{n\to\infty}2\left(1-\frac1{n+1}\right)
=2\lim_{n\to\infty}\left(1-\frac1{n+1}\right)=2
\]
\end{mdframed}
{\par
\raggedleft\textbf{답 : \(2\)}
\par}\bigskip\bigskip

\clearpage
%
\exam{}
다음 급수의 합을 구하여라.
\[\frac1{1^2+2}+\frac1{2^2+4}+\frac1{3^2+6}+\frac1{4^2+8}+\cdots\]
\procedure{0.7}
\ans

%%
\section{급수의 수렴과 발산}
\begin{mdframed}[innertopmargin=-5pt]
%
\defi{급수의 수렴과 발산}
부분합 \(\{S_n\}\)이 수렴하여 급수의 합
\[\lim_{n\to\infty}S_n
=\lim_{n\to\infty}\sum_{k=1}^na_k=\sum_{n=1}^\infty a_n=a_1+a_2+\cdots+a_n+\cdots\]
이 존재하면
\begin{center}
급수 \(\displaystyle\sum_{n=1}^\infty a_n\)은 수렴한다
\end{center}

\vspace{10pt}
\noindent라고 말한다.
급수의 합이 존재하지 않으면
\begin{center}
급수 \(\displaystyle\sum_{n=1}^\infty a_n\)은 발산한다
\end{center}
라고 말한다.
\end{mdframed}

%
\exam{}\label{harmonic_series_1}
다음 급수의 수렴과 발산을 조사하여라.
\begin{enumerate}
\item
\(\frac12+\frac23+\frac34+\frac45+\cdots\)
\item
\(1+\frac12+\frac13+\frac14+\cdots\)
\end{enumerate}

\begin{mdframed}
\begin{enumerate}
\item
\(a_1=\frac12\), \(a_2=\frac23\), \(a_3=\frac34\), \(\cdots\)라고 하면 \(a_n=\frac n{n+1}\)이다.
이때,
\[\lim_{n\to\infty}a_n=1\]
이므로 \(n\)이 충분히 크면 \(a_n\)은 \(1\)과 비슷한 값이다.
따라서 주어진 급수
\[a_1+a_2+a_3+\cdots\]
는 \(1\)과 비슷한 값을 무한히 더하여 얻어진다.
그러므로 주어진 급수 \(\displaystyle\sum_{n=1}^\infty a_n\)는 발산한다.
\end{enumerate}
\end{mdframed}

\begin{mdframed}
\begin{enumerate}
\item[(2)]
\begin{align*}
&1+\frac12+\frac13+\cdots\\
=&1+\frac12+\left(\frac13+\frac14\right)+\left(\frac15+\frac16+\frac17+\frac18\right)\\
&+\left(\frac19+\frac1{10}+\frac1{11}+\frac1{12}+\frac1{13}+\frac1{14}+\frac1{15}+\frac1{16}\right)
+\cdots\\
>&1+\frac12+\left(\frac14+\frac14\right)+\left(\frac18+\frac18+\frac18+\frac18\right)\\
&+\left(\frac1{16}+\frac1{16}+\frac1{16}+\frac1{16}+\frac1{16}+\frac1{16}+\frac1{16}+\frac1{16}\right)
+\cdots\\
=&1+\frac12+\frac12+\frac12+\frac12+\cdots=\infty
\end{align*}
따라서 주어진 급수는 발산한다.
\end{enumerate}
\end{mdframed}
{\par
\raggedleft\textbf{답 : (1) 발산(양의 무한대로 발산), (2) 발산(양의 무한대로 발산)}
\par}\bigskip\bigskip

%
\prob{}\label{harmonic_series_2}
다음 급수의 수렴과 발산을 조사하여라.
\begin{enumerate}
\item
\(\frac21+\frac33+\frac45+\frac57+\frac69+\frac7{11}+\cdots\)
\item
\(1+\frac13+\frac15+\frac17+\frac19+\cdots\)
\end{enumerate}
\procedure{0.3}
\procedure{0.4}
{\par
\raggedleft\textbf{답 : (1) \qquad\qquad\qquad\qquad (2) \qquad\qquad\qquad\qquad}
\par}\bigskip\bigskip

예시 \ref{harmonic_series_1}, \ref{harmonic_series_2}의 (1)로부터 다음과 같은 결과를 얻을 수 있다.
\begin{mdframed}
\center
\(\displaystyle\lim_{n\to\infty}a_n\neq0\)이면
급수 \(\displaystyle\sum_{n=1}^\infty a_n\)은 발산한다.
\end{mdframed}
또한 이 명제의 대우인 다음 명제도 성립한다.
\begin{mdframed}
\center
급수 \(\displaystyle\sum_{n=1}^\infty a_n\)이 수렴하면
\(\displaystyle\lim_{n\to\infty}a_n=0\)이다.
\end{mdframed}
하지만 
\begin{mdframed}
\center
\(\displaystyle\lim_{n\to\infty}a_n=0\)이면 급수 \(\displaystyle\sum_{n=1}^\infty a_n\)이 수렴한다.
\end{mdframed}
는 성립하지 않는다.
예시 \ref{harmonic_series_1}, \ref{harmonic_series_2}의 (2)가 그 반례이다.

%%
\section{등비급수의 수렴과 발산}

\vspace{-10pt}
%
\exam{}\label{geometric_series_1}
다음 급수의 수렴과 발산을 조사하고, 수렴하면 그 합을 구하여라.
\begin{enumerate}[topsep=0pt,itemsep=0pt]
\item
\(3+\frac32+\frac34+\frac38+\cdots\)
\item
\(2+2+2+2+\cdots\)
\item
\(1+(-2)+(-2)^2+(-2)^3+\cdots\)
\end{enumerate}

\begin{mdframed}
\begin{enumerate}
\item
\(a=3\), \(r=\frac12\)인 등비수열이므로 일반항은 \(a_n=ar^{n-1}=3\times\left(\frac12\right)^{n-1}\)이고 부분합은
\[S_n=\frac{a(1-r^n)}{1-r}=\frac{3\left\{1-\left(\frac12\right)^n\right\}}{1-\frac12}
=6\left\{1-\left(\frac12\right)^n\right\}\]
이다.
따라서 급수의 합은
\[S=\lim_{n\to\infty}S_n
=\lim_{n\to\infty}6\left\{1-\left(\frac12\right)^n\right\}
=6\lim_{n\to\infty}\left\{1-\left(\frac12\right)^n\right\}=6\]
\item
\(a=2\), \(r=1\)인 등비수열로 일반항은 \(a_n=2\times1^{n-1}=2\)이고 부분합은
\[S_n=2+2+2+\cdots+2=2n\]
이다.
따라서
\[\lim_{n\to\infty}S_n
=\lim_{n\to\infty}2n=\infty\]
이고 이 급수는 발산한다.
\item
\(a=1\), \(r=-2\)인 등비수열이므로 일반항은 \(a_n=ar^{n-1}=(-2)^{n-1}\)이고 부분합은
\[S_n=\frac{a(1-r^n)}{1-r}=\frac{1\times\left\{1-(-2)^n\right\}}{1-(-2)}
=\frac{1-(-2)^n}3\]
이다.
따라서
\[\lim_{n\to\infty}S_n
=\lim_{n\to\infty}\frac{1-(-2)^n}3\]
는 진동하고 이 급수는 발산한다.
\end{enumerate}
\end{mdframed}
{\par
\raggedleft\textbf{답 : (1) 수렴(6으로 수렴), (2) 발산(양의 무한대로 발산), (3) 발산(진동)}
\par}\bigskip\bigskip

%
\prob{}\label{geometric_series_2}
다음 급수의 수렴과 발산을 조사하고, 수렴하면 그 합을 구하여라.
\begin{enumerate}
\item
\(4+(-2)+1+\left(-\frac12\right)+\frac14+\cdots\)
\item
\(1+(-1)+1+(-1)+1+\cdots\)
\item
\(2+6+18+54+\cdots\)
\end{enumerate}
\procedure{0.63}
{\par
\raggedleft\textbf{답 : (1) \qquad\qquad\qquad\qquad (2) \qquad\qquad\qquad\qquad (3) \qquad\qquad\qquad}
\par}\bigskip\bigskip

\clearpage
\begin{mdframed}[innertopmargin=-5pt]
%
\defi{등비급수}
등비수열의 급수를 \emph{등비급수}라고 부른다.
\end{mdframed}
따라서, 예시 \ref{geometric_series_1}\과 문제 \ref{geometric_series_2}에서 계산한 급수들이 모두 등비급수이다.
이때 다음 정리가 성립한다.

\begin{mdframed}[innertopmargin=-5pt]
%
\theo{등비급수의 수렴과 발산}
첫째항이 \(a(\neq0)\)이고, 공비가 \(r\)인 등비수열 \(a_n=ar^{n-1}\)에 대해
\begin{enumerate}
\item
\(|r|<1\)이면 등비급수가 수렴하고
\[\sum_{n=1}^\infty a_n=\frac a{1-r}\]
이다.
\item
\(|r|\ge1\)이면 등비급수는 발산한다.
\end{enumerate}
\end{mdframed}

%%
\section{급수의 성질}
\begin{mdframed}[innertopmargin=-5pt]
%
\theo{급수의 성질}
두 급수 \(\displaystyle\sum_{n=1}^\infty a_n\), \(\displaystyle\sum_{n=1}^\infty b_n\)이 수렴한다고 가정하면
\begin{enumerate}
\item
\(\displaystyle\sum_{n=1}^\infty ca_n=c\sum_{n=1}^\infty a_n\)
\item
\(\displaystyle\sum_{n=1}^\infty(a_n+b_n)=\sum_{n=1}^\infty a_n+\sum_{n=1}^\infty b_n\)
\item
\(\displaystyle\sum_{n=1}^\infty(a_n-b_n)=\sum_{n=1}^\infty a_n-\sum_{n=1}^\infty b_n\)
\end{enumerate}
이다.
\end{mdframed}

\proo{}
\begin{align*}
(1)\qquad
&\sum_{n=1}^\infty ca_n=\lim_{n\to\infty}\sum_{k=1}^nca_k=\lim_{n\to\infty}c\sum_{k=1}^na_k
=c\lim_{n\to\infty}\sum_{k=1}^na_k=c\sum_{n=1}^\infty a_n\\
(2), (3)\:\:
&\sum_{n=1}^\infty(a_n\pm b_n)=\lim_{n\to\infty}\sum_{k=1}^n(a_n\pm b_n)
=\lim_{n\to\infty}\left(\sum_{k=1}^na_n\pm\sum_{k=1}^nb_n\right)\\
&=\lim_{n\to\infty}\sum_{k=1}^na_n\pm\lim_{n\to\infty}\sum_{k=1}^nb_n
=\sum_{n=1}^\infty a_n\pm\sum_{n=1}^\infty b_n
\end{align*}

%%
\section*{답}

%
\an{3}
\begin{enumerate}[itemsep=0pt,topsep=0pt]
\item
\(\frac12\)
\item
\(2\)
\end{enumerate}

%
\an{5}
\(\frac34\)

%
\an{8}
\begin{enumerate}[itemsep=0pt,topsep=0pt]
\item
발산
\item
발산
\end{enumerate}

%
\an{10}
\begin{enumerate}[itemsep=0pt,topsep=0pt]
\item
수렴, \(\frac83\)
\item
발산
\item
발산
\end{enumerate}
\end{document}