\documentclass{oblivoir}
\usepackage{amsmath,amssymb,amsthm,kotex,paralist,kswrapfig}

\usepackage[skipabove=10pt,skipbelow=10pt,innertopmargin=10pt]{mdframed}

\usepackage{tabto,pifont}
\TabPositions{0.2\textwidth,0.4\textwidth,0.6\textwidth,0.8\textwidth}
\newcommand\tabb[5]{\par\bigskip\noindent
\ding{172}\:{\ensuremath{#1}}
\tab\ding{173}\:\:{\ensuremath{#2}}
\tab\ding{174}\:\:{\ensuremath{#3}}
\tab\ding{175}\:\:{\ensuremath{#4}}
\tab\ding{176}\:\:{\ensuremath{#5}}}

\newcommand\tabfive[5]{\par\medskip\noindent
\ding{172}\:\:{\ensuremath{#1}}\\
\ding{173}\:\:{\ensuremath{#2}}\\
\ding{174}\:\:{\ensuremath{#3}}\\
\ding{175}\:\:{\ensuremath{#4}}\\
\ding{176}\:\:{\ensuremath{#5}}}

\usepackage{enumitem}
\setlist[enumerate]{label=(\arabic*)}

\newcounter{num}
\newcommand{\defi}[1]
{\noindent\refstepcounter{num}\textbf{정의 \arabic{num}) #1}\par\noindent}
\newcommand{\theo}[1]
{\noindent\refstepcounter{num}\textbf{정리 \arabic{num}) #1}\par\noindent}
\newcommand{\exam}[1]
{\bigskip\bigskip\noindent\refstepcounter{num}\textbf{예시 \arabic{num}) #1}\par\noindent}
\newcommand{\prob}[1]
{\bigskip\bigskip\noindent\refstepcounter{num}\textbf{문제 \arabic{num}) #1}\par\noindent}
\newcommand{\proo}
{\bigskip\textsf{증명)}\par}

\newcommand{\procedure}[1]{\begin{mdframed}\vspace{#1\textwidth}\end{mdframed}}
\newcommand{\ans}{
{\par\raggedleft\textbf{답 : (\qquad\qquad\qquad\qquad\qquad\qquad)}\par}\bigskip\bigskip}
\newcommand\an[1]{\par\bigskip\noindent\textbf{문제 #1)}\\}

\newcommand{\pb}[1]%\Phantom + fBox
{\fbox{\phantom{\ensuremath{#1}}}}

\newcommand\ba{\,|\,}

\let\oldsection\section
\renewcommand\section{\clearpage\oldsection}

\let\emph\textsf
%%%%
\begin{document}

\title{01 다항식의 연산}
\author{}
\date{\today}
\maketitle
\tableofcontents
\newpage

%%
\section{항등식과 미정계수}

\begin{mdframed}
%
\defi{}
주어진 식의 문자에 어떤 값을 대입하여도 항상 성립하는 등식을 그 문자에 관한 \emph{항등식}이라고 한다.
\end{mdframed}

%
\exam{}
\begin{enumerate}
\item
\(x^2+2x+1=(x+1)^2\)은 \(x\)에 어떤 값을 대입하여도 성립하는 등식이므로 \(x\)에 관한 항등식이다.
\item
\((2a+b)^2=4a^2+4ab+b^2\)은 \(a\)와 \(b\)에 어떤 값을 대입하여도 성립하는 등식이므로 \(a\)와 \(b\)에 관한 항등식이다.
\item
\(x^2-2x-3=0\)는 \(x\)에 \(-1\)과 \(2\)를 대입하면 성립하지만, \(x\)에 \(0\)이나 \(1\)을 대입하면 성립하지 않으므로 항등식이 아니다.
몇몇 \(x\)에 대해서만 등식이 성립하는 이와 같은 등식을 \emph{방정식}이라고 한다.
\item
\(x^2-2x+3=0\)는 \(x\)에 어떤 값을 대입하여도 성립하지 않는다.
따라서 항등식이 아니다.
\end{enumerate}


%
\prob{}
다음 등식 중에서 항등식인 것을 찾아라.
\tabfive
{\text{\(x^2-2=0\)}}
{\text{\(x^3+2x^2+3x=x^3+2x^2+x+2\)}}
{\text{\(xy=2x+2y-4\)}}
{\text{\(\frac12\left\{(x-3)^2-(x-1)^2\right\}=-2x+4\)}}
{\text{\(y=2x+1\)}}

\clearpage

\begin{mdframed}
%
\theo{}\label{identity}
\begin{enumerate}
\item
\(ax+b=0\)이 \(x\)에 관한 항등식이면 \(a=0\)이고 \(b=0\)이다.
\item
\(ax+b=a'x+b'\)이 \(x\)에 관한 항등식이면 \(a=a'\)이고 \(b=b'\)이다.
\item
\(ax^2+bx+c=0\)가 \(x\)에 관한 항등식이면 \(a=0\), \(b=0\), \(c=0\)이다.
\item
\(ax^2+bx+c=a'x^2+b'x+c'\)가 \(x\)에 관한 항등식이면 \(a=a'\), \(b=b'\), \(c=c'\)이다.
\item
\(ax+by+c=0\)이 \(x\), \(y\)에 관한 항등식이면 \(a=0\), \(b=0\), \(c=0\)이다.
\item
\(ax+by+c=a'x+b'y+c'=0\)이 \(x\), \(y\)에 관한 항등식이면 \(a=a'\), \(b=b'\), \(c=c'\)이다.
\end{enumerate}
\end{mdframed}

%
\exam{미정계수법}
\(x\)에 관한 등식
\[a(x-1)+b(x-2)=2x-3\]
이 항등식이 되도록 \(a\)와 \(b\)의 값을 정하여라.
\begin{mdframed}
\textbf{(계수비교법)}\par\noindent
주어진 식을 정리하면
\[(a+b)x+(-a-2b)=2x-3\]
이다.
여기에 정리 \ref{identity}의 (2)를 적용하면 \(a+b=2\), \(a+2b=3\)이다.
두 식을 연립하면 \(a=1\), \(b=1\)이다.
\par\bigskip\noindent
\textbf{(수치대입법)}\par\noindent
\(x\)에 관한 항등식이므로 \(x\)에 어떤 값을 대입하여도 성립하여야 한다.
그중 \(x=1\)를 대입하면 \(-b=-1\)이 된다.
따라서 \(b=1\)이다.
또 \(x=2\)를 대입하면 \(a=1\)이 된다.
따라서 \(a=1\)이다.
\end{mdframed}
{\par\raggedleft\textbf{답 : \(a=1\), \(b=1\)}\par}\bigskip

%%
\section{나머지정리와 인수정리}

%
\exam{}
다항식 \(f(x)=x^3-2x+3\)을 일차식 \(x-2\)로 나누었을 때의 나머지를 구하여라.
\begin{mdframed}
나누는 수가 \(1\)차 다항식이므로 나머지는 상수항이어야 한다.
몫을 \(Q(x)\)라고 하고, 나머지를 \(R\)이라고 하면
\[f(x)=(x-2)Q(x)+R\]
이다.
이 식은 항등식이므로 여기에 \(x=2\)를 대입하여도 성립하여야 한다.
따라서 \(f(2)=R\)이다.
\(f(2)=2^3-2\cdot2+3=7\)이므로 \(R=7\)
\end{mdframed}
{\par\raggedleft\textbf{답 : \(7\)}\par}\bigskip

\begin{mdframed}
%
\theo{나머지 정리}
다항식 \(f(x)\)를 일차식 \(x-\alpha\)로 나눈 나머지는 \(f(\alpha)\)이다.
\end{mdframed}

%
\prob{}
다항식 \(f(x)=2x^3-4x^2+5x+8\)을 다음 일차식으로 나누었을 때의 나머지를 구하여라.
\par\noindent
(1)\:\:\(x+1\)
\tabto{.5\textwidth}
(2)\:\:\(x-3\)
\procedure{0.3}
{\par\raggedleft\textbf{답 :
(1)\:\:\(-3\)\qquad
(2)\:\:\(41\)\qquad\qquad}\par}

\clearpage
나머지 정리의 특별한 경우로, 다음과 같은 인수정리도 얻을 수 있다.
\begin{mdframed}
%
\theo{인수정리}
\[
\text{다항식 \(f(x)\)가 일차식 \(x-\alpha\)로 나누어 떨어진다.}
\iff
f(\alpha)=0\text{이다}
\]
\end{mdframed}

다음은 모두 같은 말들이다 ;

\begin{align*}
&\text{다항식 \(f(x)\)가 일차식 \(x-\alpha\)로 나누어 떨어진다.}\\
&\iff R=0\\
&\iff f(\alpha)=0\text{이다}\\
&\iff\text{다항식 \(f(x)\)가 \(x-\alpha\)라는 인수를 가진다.}\\
&\iff\text{다항식 \(f(x)\)를 \(x-\alpha\)로 묶을 수 있다.}
\end{align*}

%
\exam{}
\(x^3-7x-6\)을 인수분해하여라.
\begin{mdframed}
주어진 식을 \(f(x)=x^3-7x-6\)라고 하자.
\(f(x)\)에 여러 값을 대입해서 \(f(a)=0\)이 되는 \(a\)의 값을 찾으면 \(f(-1)=0\)이다.
따라서 \(f(x)\)는 \(x+1\)이라는 인수를 가진다.
조립제법을 사용하여  \(x+1\)로 나눈 몫을 구하면 \(x^2-x-6\)이고,
\[f(x)=(x+1)(x^2-x-6)\]
이다.
\(x^2-x-6\)도 인수분해하여 정리하면
\[f(x)=(x+1)(x-3)(x+2)\]
이다.
\end{mdframed}
{\par\raggedleft\textbf{답 : \(f(x)=(x+1)(x-3)(x+2)\)}\par}\bigskip
\end{document}