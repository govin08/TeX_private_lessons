\documentclass{oblivoir}
\usepackage{amsmath,amssymb,amsthm,kotex,paralist,kswrapfig,tabu}

\usepackage[skipabove=10pt,innertopmargin=10pt,nobreak]{mdframed}

\usepackage{tabto,pifont}
\TabPositions{0.2\textwidth,0.4\textwidth,0.6\textwidth,0.8\textwidth}
\newcommand\tabb[5]{\par\bigskip\noindent
\ding{172}\:{\ensuremath{#1}}
\tab\ding{173}\:\:{\ensuremath{#2}}
\tab\ding{174}\:\:{\ensuremath{#3}}
\tab\ding{175}\:\:{\ensuremath{#4}}
\tab\ding{176}\:\:{\ensuremath{#5}}}

\usepackage{enumitem}
\setlist[enumerate]{label=(\arabic*)}

\newcounter{num}
\newcommand{\defi}[1]
{\noindent\refstepcounter{num}\textbf{정의 \arabic{num}) #1}\par\noindent}
\newcommand{\theo}[1]
{\noindent\refstepcounter{num}\textbf{정리 \arabic{num}) #1}\par\noindent}
\newcommand{\exam}[1]
{\bigskip\bigskip\noindent\refstepcounter{num}\textbf{예시 \arabic{num}) #1}\par\noindent}
\newcommand{\prob}[1]
{\bigskip\bigskip\noindent\refstepcounter{num}\textbf{문제 \arabic{num}) #1}\par\noindent}
\newcommand{\proo}
{\bigskip\textsf{증명)}\par}

\newcommand{\procedure}[1]{\begin{mdframed}\vspace{#1\textheight}\end{mdframed}}

\newcommand{\ans}{
{\par
\raggedleft\textbf{답 : (\qquad\qquad\qquad\qquad\qquad\qquad)}
\par}\bigskip}

\newcommand{\pb}[1]%\Phantom + fBox
{\fbox{\phantom{\ensuremath{#1}}}}

\newcommand\ba{\,|\,}

\newcommand\an[1]{\par\bigskip\noindent\textbf{문제 #1)}\\}

\let\oldsection\section
\renewcommand\section{\clearpage\oldsection}

\renewcommand{\arraystretch}{1.5}

\let\emph\textsf
%%%%
\begin{document}

\title{준영 : 08 로그}
\author{}
\date{\today}
\maketitle
\tableofcontents
\newpage

%%
\section{복습}

%
\prob{}
다음 빈칸에 들어갈 알맞은 수를 구하여라.
\begin{enumerate}
\item
\(\displaystyle2^\square=1\)
\item
\(\displaystyle\left(\frac13\right)^\square=9\)
\item
\(\displaystyle(0.1)^\square=10\)
\item
\(\displaystyle4^\square=\sqrt2\)
\end{enumerate}
{\par\raggedleft\textbf{답 :
(1)\:\:\qquad\qquad
(2)\:\:\qquad\qquad
(3)\:\:\qquad\qquad
(4)\:\:\qquad\qquad\qquad}\par}\bigskip

%%
\section{로그의 뜻}

%
\exam{}
어느 실험실에서 박테리아가 시간당 \(2\)배의 속도로 증식했다고 한다.
다음은 처음 \(1\)g이던 박테리아가 \(x\)시간이 지난 후 증식한 양 \(2^x\)g을 표로 나타낸 것이다.
%빈칸에 알맞은 수를 써넣어보자.
\begin{figure}[h!]
\begin{tabu}{X[c,4]|X[c]|X[c]|X[c]|X[c]|X[c]|X[c]|X[c]}
\hline
\(x\)시간		&1	&2	&3	&\(\cdots\)	&6	&\(\cdots\)	&	\\\hline
박테리아의 양(g)	&2	&4	&8	&\(\cdots\)	&	&\(\cdots\)	&512	\\\hline
\end{tabu}
\end{figure}

\begin{enumerate}[itemsep=0pt]
\item
6시간이 지난 후의 박테리아의 양은 몇 g인가?
\item
박테리아의 양이 512g이 된 때는 처음으로부터 몇 시간이 흐른 때인가?
\item
박테리아의 양이 5g이 된 때는 처음으로부터 몇 시간이 흐른 때인가?
\end{enumerate}

\begin{mdframed}
\begin{enumerate}[itemsep=0pt]
\item
6시간이 지난 후의 박테리아의 양 \(2^6\)g으로 \(64\)g이다.
\item
박테리아의 양이 512g이 된 때가 처음으로부터 \(x\) 시간이 흐른 때라고 가정하면
\[2^x=512\]
이다.
이를 만족하는 \(x\)는 단 하나 존재하며, \(x=9\)이면 식이 성립된다.
따라서 \(x=9\)이다.
\item
박테리아의 양이 5g이 된 때가 처음으로부터 \(x\)시간이 흐른 때라고 가정하면
\[2^x=5\]
이다.
이를 만족하는 \(x\)는 단 하나 존재한다.
하지만 정확히 어떤 값인지는 알 수 없고, 그 값을 우리가 알고 있는 사칙연산이나 루트를 사용해 표현할 수는 없다.
따라서 새로운 기호를 도입해 \(2^x=5\)를 만족하는 \(x\)를
\[x=\log_25\]
로 쓴다.
\end{enumerate}

\end{mdframed}

%첫 번째 빈칸을 채우기 위해서는 다음 방정식을 풀어야 한다.
%\[2^6=N\]
%따라서 \(N=64\)이다.
%
%두 번째 빈칸을 채우기 위해서는 다음 방정식을 풀어야 한다.
%\[2^x=512\]
%이를 만족하는 수는 \(x=9\)가 있다.
%
%\bigskip
%첫 번째와 같은 방정식은 전단원인 `지수'단원에서 다뤘다.
%\(a>0\)이고 \(x\)가 실수이면 \(a^x\)의 값을 정할 수 있었다.
%
%두 번째와 같은 방정식을 이번 단원에서 다룬다.
%위와 같이 \(2^x=512\)는 쉽게 \(x\)를 구할 수 있고, \(x=9\)라는 사실이 명백하게 드러난다.
%하지만 \(2^x=5\)와 같은 방정식의 경우에는 \(x\)의 값을 쉽게 정할 수 없다.
%
%\[2^x=5\]
%를 만족시키는 \(x\)의 값은 하나 존재하며, 자연수도 아니고 유리수도 아니다.
%또 루트라든지 우리가 알고 있는 기호를 가지고 표현하는 것이 불가능하므로 새로운 기호를 도입해야 한다.
%저 방정식을 만족시키는 \(x\)를
%\[\log_25\]
%라고 쓴다.

\begin{mdframed}
%
\defi{로그의 정의}
\(a>0\), \(a\neq1\), \(N>0\)일 때,
\[a^x=N\iff x=\log_aN\]
이다.
이때, \(a\)를 \emph{밑}, \(N\)을 \emph{진수}라고 한다.
\end{mdframed}

%
\exam{}
\begin{enumerate}
\item
\[2^3=8\iff3=\log_28\]
\item
\[3^{-2}=\frac19\iff-2=\log_3{\frac19}\]
\end{enumerate}

%
\prob{}
다음 등식을 로그를 사용하여 나타내어라.
\begin{enumerate}[itemsep=7pt]
\item
\(5^3=125\)
\item
\(2^0=1\)
\item
\(2^{-2}=0.25\)
\item
\(3^{\frac12}=\sqrt3\)
\end{enumerate}

%
\prob{}
다음 등식을 \(a^x=N\)의 꼴로 나타내어라.
\begin{enumerate}[itemsep=7pt]
\item
\(\log_216=4\)
\item
\(\log_31=0\)
\item
\(\log_{\frac13}9=-2\)
\item
\(\log_93=\frac12\)
\end{enumerate}

%
\exam{}
다음 값을 구하여라.
\par\noindent
(1)\:\:\(\log_232\)
\tabto{.5\textwidth}
(2)\:\:\(\log_{\frac13}27\)
\begin{mdframed}
\begin{enumerate}
\item
\(\log_232=x\)로 놓으면
\[2^x=32\]
이다.
이때 \(32=2^5\)이므로 \(x=5\)이다.
\item
\(\log_{\frac13}27=x\)로 놓으면
\[\left(\frac13\right)^x=27\]
이다.
이때 \(27=3^3=(3^{-1})^{-3}=\left(\frac13\right)^{-3}\)이므로 \(x=-3\)이다.
\end{enumerate}
\end{mdframed}

%
\exam{}
다음 값을 구하여라.
\par\noindent
(1)\:\:\(\log_525\)
\tabto{.5\textwidth}
(2)\:\:\(\log_7\frac1{49}\)
\procedure{0.11}
{\par\raggedleft\textbf{답 :
(1)\:\:\qquad\qquad
(2)\:\:\qquad\qquad}\par}\bigskip

%
\exam{}
다음 등식을 만족하는 \(x\)의 값을 구하여라.
\par\noindent
(1)\:\:\(\log_2x=-5\)
\tabto{.5\textwidth}
(2)\:\:\(\log_x27=3\)
\procedure{0.11}
{\par\raggedleft\textbf{답 :
(1)\:\:\qquad\qquad
(2)\:\:\qquad\qquad}\par}\bigskip


%%
\section{로그의 성질}
로그의 정의와 지수법칙을 이용하여 로그의 성질을 알아보자.

\(a>0\), \(a\neq1\)일 때, \(a^0=1\), \(a^1=a\)이므로 로그의 정의에 의하여
\[\log_a1=0,\qquad\log_aa=1\]
이다.

또 \(a>0\), \(a\neq1\), \(M>0\), \(N>0\)일 때,
\[\log_aM=p,\qquad\log_aN=q\]
로 놓으면 로그의 정의에 의해 \(a^p=M\), \(a^q=N\)이다.
이때 지수법칙을 이용하면
\[MN=a^pa^q=a^{p+q}\]
이므로 로그의 정의를 이용하면 다음이 성립한다.
\[\log_aMN=p+q=\log_aM+\log_aN\]

\clearpage
%
\prob{}
\(a>0\), \(a\neq0\), \(M>0\), \(N>0\)일 때, 다음 등식이 성립함을 보여라.
\par\noindent
(1)\:\:\(\log_a{\frac MN}=\log_aM-\log_aN\)
\tabto{.5\textwidth}
(2)\:\:\(\log_aM^k=k\log_aM\)
\procedure{0.8}

\clearpage
일반적으로 다음과 같은 로그의 성질이 성립한다.
\begin{mdframed}
%
\theo{로그의 성질(1)}
\(a>0\), \(a\neq0\), \(M>0\), \(N>0\)일 때,
\begin{enumerate}
\item
\(\log_a1=0\),\quad\(\log_aa=1\)
\item
\(\log_aMN=\log_aM+\log_aN\)
\item
\(\log_a\frac MN=\log_aM-\log_aN\)
\item
\(\log_aM^k=k\log_aM\)
\end{enumerate}
\end{mdframed}

%
\exam{}
\begin{enumerate}
\item
\(\log_26=\log_2(2\times3)=\log_22+\log_23=1+\log_23\).
\item
\(\log_2\frac52=\log_25-\log_22=\log_25-1\).
\item
\(\log_3\sqrt{27}=\log_3\sqrt{3^3}=\log_3(3^3)^{\frac12}=\log_33^{\frac32}=\frac32\log_33=
\frac32\).
\end{enumerate}

%
\prob{}
다음 값을 구하여라.
\par\noindent
(1)\:\:\(\log_525\sqrt5\)
\tabto{.5\textwidth}
(2)\:\:\(\log_{10}{\frac1{\sqrt{0.0001}}}\)
\procedure{0.2}
{\par\raggedleft\textbf{답 :
(1)\:\:\qquad\qquad
(2)\:\:\qquad\qquad}\par}\bigskip

\clearpage
%
\exam{}
다음을 간단히 하여라.
\par\noindent
(1)\:\:\(\log_2{12}+\log_2{\frac13}\)
\tabto{.5\textwidth}
(2)\:\:\(\log_26-2\log_2\sqrt3\)
\begin{mdframed}
\begin{enumerate}
\item
\(\log_2{12}+\log_2{\frac13}=\log_2\left(12\times\frac13\right)=\log_24=\log_22^2=2\log_22=2\)
\item
\(\log_26-2\log_2\sqrt3=\log_26-\log_2\sqrt3^2=\log_26-\log_23=\log_2{\frac63}=\log_22=1\)
\end{enumerate}
\end{mdframed}
{\par\raggedleft\textbf{답 :
(1)\:\:\(2\)\quad(2)\:\:\(1\)}\par}

\vspace{-20pt}
%
\prob{}
다음을 간단히 하여라.
\par\noindent
(1)\:\:\(\log_36+\log_3\frac12\)
\tabto{.5\textwidth}
(2)\:\:\(\log_345-\log_35\)
\par\noindent
(3)\:\:\(2\log_5\sqrt{15}-\log_53\)
\tabto{.5\textwidth}
(4)\:\:\(\log_212+\log_26-2\log_23\)
\procedure{0.15}
{\par\raggedleft\textbf{답 :
(1)\:\:\qquad\qquad
(2)\:\:\qquad\qquad}\par}
{\par\raggedleft\textbf{
(3)\:\:\qquad\qquad
(4)\:\:\qquad\qquad}\par}\bigskip

\vspace{-20pt}
%
\prob{}
\(\log_52=a\), \(\log_53=b\)일 때, 다음을 \(a\), \(b\)로 나타내어라.
\par\noindent
(1)\:\:\(\log_54\)
\tabto{.5\textwidth}
(2)\:\:\(\log_510\)
\par\noindent
(3)\:\:\(\log_512\)
\tabto{.5\textwidth}
(4)\:\:\(\log_5\sqrt{15}\)
\procedure{0.15}
{\par\raggedleft\textbf{답 :
(1)\:\:\qquad\qquad\qquad\qquad
(2)\:\:\qquad\qquad\qquad\qquad}\par}
{\par\raggedleft\textbf{
(3)\:\:\qquad\qquad\qquad\qquad
(4)\:\:\qquad\qquad\qquad\qquad}\par}\bigskip

\clearpage
\begin{mdframed}
%
\theo{로그의 성질(2), 밑의 변환공식}
\(a>0\), \(a\neq1\), \(b>0\), \(c>0\), \(\c\neq1\)일 때,
\[\log_ab=\frac{\log_cb}{\log_ca}\]
\end{mdframed}

\proo
%\(a>0\), \(a\neq1\), \(b>0\)라고 하자.
\(\log_ab=x\), \(\log_ca=y\)로 놓으면 \(a^x=b\), \(c^y=a\)이므로
\[b=a^x=(c^y)^x=c^{xy}\]
따라서, 로그의 정의에 의해
\[\log_cb=xy=\log_ab\cdot\log_ca\]
이다.
그러므로
\[\log_ab=\frac{\log_cb}{\log_ca}\]

%
\exam{}
\begin{enumerate}
\item
\(\log_23=\frac{\log_53}{\log_52}=\frac{\log_73}{\log_72}=\frac{\log_{10}3}{\log_{10}2}\)
\item
\(\log_23=\frac{\log_33}{\log_32}=\frac1{\log_32}\)
\end{enumerate}

\clearpage
%
\prob{}
다음 값을 구하여라.
\par\noindent
(1)\:\:\(\log_23\cdot\log_34\)
\tabto{.5\textwidth}
(2)\:\:\(\log_45\cdot\log_56\cdot\log_64\)
\procedure{0.3}
{\par\raggedleft\textbf{답 :
(1)\:\:\qquad\qquad
(2)\:\:\qquad\qquad}\par}\bigskip

%
\prob{}
\(\log_72=a\), \(\log_73=b\)일 때, 다음을 \(a\), \(b\)로 나타내어라.
\par\noindent
(1)\:\:\(\log_23\)
\tabto{.5\textwidth}
(2)\:\:\(\log_3\sqrt{8}\)
\procedure{0.3}
{\par\raggedleft\textbf{답 :
(1)\:\:\qquad\qquad\qquad\qquad
(2)\:\:\qquad\qquad\qquad\qquad}\par}\bigskip

%
\prob{}
\(a>0\), \(a\neq1\), \(b>0\), \(b\neq1\), \(c>0\), \(\c\neq1\)일 때 다음 등식이 성립함을 보여라.
\begin{enumerate}
\item
\(\log_{a^m}b^n=\frac nm\log_ab\)
\item
\(a^{\log_bc}=c^{\log_ba}\)
\end{enumerate}
\procedure{0.6}

\begin{mdframed}
%
\theo{로그의 성질(3)}
\(a>0\), \(a\neq1\), \(b>0\), \(b\neq1\), \(c>0\), \(\c\neq1\)일 때,
\begin{enumerate}
\item
\(\log_{a^m}b^n=\frac nm\log_ab\)
\item
\(a^{\log_bc}=c^{\log_ba}\)
\end{enumerate}
\end{mdframed}

%%
\section{계산 문제 연습}
%
\prob{}
다음 등식을 로그를 사용하여 나타내어라.
\begin{enumerate}
\item
\(2^5=32\)
\item
\(10^{-3}=0.001\)
\item
\(5^{\frac12}=\sqrt5\)
\item
\(\left(\frac15\right)^{-3}=125\)
\end{enumerate}

%
\prob{}
다음 등식을 지수를 사용하여 나타내어라.
\begin{enumerate}
\item
\(\log_381=4\)
\item
\(\log_{10}0.0001=-4\)
\item
\(\log_3\sqrt3=\frac12\)
\item
\(\log_{\frac12}8=-3\)
\end{enumerate}

%
\prob{}
다음 값을 구하여라.
\begin{enumerate}
\item
\(\log_216\)
\item
\(\log_{0.5}16\)
\item
\(\log_{0.25}4\)
\item
\(\log_{125}\sqrt[3]{25}\)
\item
\(\log_{2\sqrt2}\sqrt[4]{32}\)
\item
\(\log_{49}\sqrt{343}\)
\end{enumerate}

%
\prob{}
다음 식을 만족하는 \(x\)의 값을 구하여라.
\begin{enumerate}
\item
\(\log_3x=4\)
\item
\(\log_{\frac12}x=3\)
\item
\(\log_x49=2\)
\item
\(\log_x\frac1{100}=-2\)
\item
\(\log_3(\log_2x)=1\)
\item
\(\log_2(\log_5x)=2\)
\end{enumerate}

%
\prob{}
다음 값을 구하여라.
\begin{enumerate}
\item
\(\log_31\)
\item
\(\log_55\)
\item
\(\log_3\frac1{81}\)
\item
\(\log_20.25\)
\end{enumerate}

\clearpage
%
\prob{}
다음을 간단히 하여라.
\begin{enumerate}
\item
\(\log_216+\log_2\frac18\)
\item
\(\log_63+\log_6\sqrt{12}\)
\item
\(\log_2\frac43+2\log_212\)
\item
\(\log_2\frac29+4\log_2\sqrt{12}\)
\item
\(\log_2\sqrt8-\log_2\sqrt2\)
\item
\(\log_3\sqrt{27}-\log_3{\frac1{\sqrt3}}\)
\item
\(\log_36+\log_32-\log_34\)
\item
\(\log_23-\log_2\frac92+\log_212\)
\end{enumerate}

%
\prob{}
다음을 간단히 하여라.
\begin{enumerate}
\item
\(9^{\frac32}+\log_381\)
\item
\(\sqrt[3]{27}-\log_3\sqrt{81}\)
\item
\(\frac1{\sqrt[3]{8}}\times\log_381\)
\item
\(3^{\frac23}\times27^{\frac19}+\log_28\)
\end{enumerate}

\clearpage
%
\prob{}
\(\log_{10}2=a\), \(\log_{10}3=b\)일 때, 다음을 \(a\), \(b\)로 나타내어라.
\begin{enumerate}
\item
\(\log_{10}48\)
\item
\(\log_{10}5\)
\item
\(\log_{10}\frac1{25}\)
\item
\(\log_{10}45\)
\item
\(\log_{10}0.072\)
\item
\(\log_{10}\left(\frac45\right)^3\)
\item
\(\log_{10}\sqrt[4]{15}\)
\end{enumerate}

%
\prob{}
다음을 간단히 하여라
\begin{enumerate}
\item
\(\log_23\cdot\log_32\)
\item
\(\log_{25}9\cdot\log_{27}5\)
\item
\(\log_53\cdot\log_3\sqrt5\)
\item
\(\log_35\cdot\log_57\cdot\log_79\)
\end{enumerate}

%
\prob{}
\(\log_{10}2=a\), \(\log_{10}3=b\)일 때, 다음을 \(a\), \(b\)로 나타내어라.
\begin{enumerate}
\item
\(\log_23\)
\item
\(\log_612\)
\item
\(\log_2\sqrt{27}\)
\item
\(\log_3\sqrt{18}\)
\end{enumerate}

%
\prob{}
다음을 간단히 하여라.
\begin{enumerate}
\item
\(\log_220-\frac1{\log_52}\)
\item
\(\log_2(\log_23)+\log_2(\log_34)\)
\item
\(\log_23\times\log_34\times\cdots\times\log_{31}32\)
\end{enumerate}

%
\prob{}
다음을 간단히 하여라.
\begin{enumerate}
\item
\(\log_{5^3}5^4\)
\item
\(\log_82\sqrt2\)
\item
\(2^{\log_210}\)
\item
\(27^{\log_35}\)
\end{enumerate}

%
\prob{}
\begin{enumerate}
\item
\(\log_42+\log_{16}2\)
\item
\(\log_{\frac12}2+\log_7{\frac17}\)
\item
\((\log_23+\log_83)(\log_32+\log_92)\)
\item
\(\log_53\times(\log_3\sqrt5-\log_{\frac19}125)\)
\end{enumerate}

%%
\section{보충 · 심화 문제}

%
\prob{}
다음이 정의되도록 실수 \(x\)의 값의 범위를 정하여라.
\[\log_{x-2}{5-x}\]
\procedure{0.2}
\ans

%
\prob{}
다음을 간단히 하여라.
\par\noindent
(1)\:\:\(\frac1{\log_26}+\frac1{\log_36}\)
\tabto{.5\textwidth}
(2)\:\:\((\log_38+\log_94)\cdot\log_23\)
\procedure{0.22}
{\par\raggedleft\textbf{답 :
(1)\:\:\qquad\qquad\qquad\qquad
(2)\:\:\qquad\qquad\qquad\qquad}\par}\bigskip

%
\prob{}
다음 값을 계산하여라.
\[\log_2\left(1-\frac12\right)+\log_2\left(1-\frac13\right)+\log_2\left(1-\frac14\right)+\cdots
+\log_2\left(1-\frac1{32}\right)\]
\procedure{0.27}
\ans

%
\prob{}
이차방정식 \(x^2+4x+1=0\)의 두 근이 \(\log_2a\), \(\log_2b\)일 때, \(\log_ab+\log_ba\)의 값을 구하여라.
\procedure{0.27}
\ans

%
\prob{}
\(a>0\), \(a\neq1\)인 실수 \(a\)와 세 양의 실수 \(x\), \(y\), \(z\)에 대하여 \(x^2=y^3=z^4=a\)일 때, \(\log_a{xyz}\)의 값을 구하여라.

\section*{답}
\begin{minipage}{0.49\textwidth}
%
\an{1}
\noindent
(1)\:\:\(0\)\\
(2)\:\:\(-2\)\\
(3)\:\:\(-1\)\\
(4)\:\:\(\frac14\)

%
\an{5}
\noindent
(1)\:\:\(3=\log_5125\)\\
(2)\:\:\(0=\log_21\)\\
(3)\:\:\(-2=\log_20.25\)\\
(4)\:\:\(\frac12=\log_3\sqrt3\)

%
\an{6}
\noindent
(1)\:\:\(2^4=16\)\\
(2)\:\:\(3^0=1\)\\
(3)\:\:\(\left(\frac13\right)^{-2}=9\)\\
(4)\:\:\(9^{\frac12}=3\)

%
\an{8}
\noindent
(1)\:\:\(2\)\\
(2)\:\:\(-2\)

%
\an{9}
\noindent
(1)\:\:\(\frac1{32}\)\\
(2)\:\:\(3\)

%
\an{10}
(1) 
\[\log_aM=p,\qquad\log_aN=q\]
로 놓으면 로그의 정의에 의해 \(a^p=M\), \(a^q=N\)이다.
\end{minipage}
\begin{minipage}{0.49\textwidth}

이때 지수법칙을 이용하면
\[\frac MN=\frac{a^p}{a^q}=a^{p-q}\]
이므로 로그의 정의를 이용하면
\[\log_a\frac MN=p-q=\log_aM-\log_aN\]
이다.

(2)
\[\log_aM=p\]
로 놓으면 로그의 정의에 의해 \(a^p=M\)이다.
양변에 \(k\)제곱을 하면 \[(a^p)^k=M^k\] \[a^{pk}=M^k\]가 된다.
여기에 로그의 정의를 이용하면
\[\log_aM^k=pk=k\log_aM\]
이다.

%
\an{13}
\noindent
(1)\:\:\(\frac52\)\\
(2)\:\:\(2\)

%
\an{15}
\noindent
(1)\:\:\(1\)\\
(2)\:\:\(2\)\\
(3)\:\:\(1\)\\
(4)\:\:\(3\)

\end{minipage}

\begin{minipage}{0.49\textwidth}

%
\an{16}
\noindent
(1)\:\:\(2a\)\\
(2)\:\:\(a+1\)\\
(3)\:\:\(2a+b\)\\
(4)\:\:\(\frac12b+\frac12\)

%
\an{19}
\noindent
(1)\:\:\(2\)\\
(2)\:\:\(1\)

%
\an{20}
\noindent
(1)\:\:\(\frac ba\)\\
(2)\:\:\(\frac{3a}{2b}\)

%
\an{21}
(1)
\begin{align*}
\log_{a^m}b^n
=&\frac{\log_cb^n}{\log_ca^m}=\frac{n\log_cb}{m\log_ca}\\
=&\frac mn\frac{\log_cb}{\log_ca}=\frac mn\log_ab
\end{align*}
(2)
\(a^{\log_bc}=x\)로 놓으면,
\[\log_bc=\log_ax=\frac{\log_bx}{\log_ba}\]
이고 이것을 정리하면
\[\log_ba=\frac{\log_bx}{\log_bc}=\log_cx\]
따라서
\[x=c^{\log_ba}\]
이고,
\[a^{\log_bc}=c^{\log_ba}\]
\end{minipage}
\begin{minipage}{0.49\textwidth}
%
\an{23}
\noindent
(1)\:\:\(\log_232=5\)\\
(2)\:\:\(\log_{10}0.001=-3\)\\
(3)\:\:\(\log_5\sqrt5=\frac12\)\\
(4)\:\:\(\log_{\frac15}125=-3\)

%
\an{24}
\noindent
(1)\:\:\(3^4=81\)\\
(2)\:\:\(10^{-4}=0.0001\)\\
(3)\:\:\(3^{\frac12}=\sqrt3\)\\
(4)\:\:\(\left(\frac12\right)^{-3}=8\)

%
\an{25}
\noindent
(1)\:\:\(4\)
(2)\:\:\(-4\)\\
(3)\:\:\(-1\)\\
(4)\:\:\(\frac29\)\\
(5)\:\:\(\frac56\)\\
(6)\:\:\(\frac34\)

%
\an{26}
\noindent
(1)\:\:\(81\)\\
(2)\:\:\(\frac18\)\\
(3)\:\:\(7\)\\
(4)\:\:\(10\)\\
(5)\:\:\(8\)\\
(6)\:\:\(625\)

%
\an{27}
\noindent
(1)\:\:\(0\)\\
(2)\:\:\(1\)\\
(3)\:\:\(-4\)\\
(4)\:\:\(-2\)
\end{minipage}

\begin{minipage}{0.49\textwidth}
%
\an{28}
\noindent
(1)\:\:\(1\)\\
(2)\:\:\(2\)\\
(3)\:\:\(4\)\\
(4)\:\:\(5\)\\
(5)\:\:\(1\)\\
(6)\:\:\(2\)\\
(7)\:\:\(1\)\\
(8)\:\:\(3\)

%
\an{29}
\noindent
(1)\:\:\(31\)\\
(2)\:\:\(1\)\\
(3)\:\:\(2\)\\
(4)\:\:\(6\)

%
\an{30}
\noindent
(1)\:\:\(4a+b\)\\
(2)\:\:\(1-a\)\\
(3)\:\:\(2a-2\)\\
(4)\:\:\(-a+2b+1\)\\
(5)\:\:\(3a+2b-3\)\\
(6)\:\:\(9a-3\)\\
(7)\:\:\(\frac14(-a+b+1)\)

%
\an{31}
\noindent
(1)\:\:1\\
(2)\:\:\(\frac13\)\\
(3)\:\:\(\frac12\)\\
(4)\:\:\(2\)

%
\an{32}
\noindent
(1)\:\:\(\frac ba\)\\
(2)\:\:\(\frac{2a+b}{a+b}\)\\
(3)\:\:\(\frac{3b}{2a}\)\\
(4)\:\:\(\frac{a+2b}{2b}\)
\end{minipage}
%%%
\begin{minipage}{0.49\textwidth}
%
\an{33}
\noindent
(1)\:\:\(2\)\\
(2)\:\:\(1\)\\
(3)\:\:\(5\)

%
\an{34}
\noindent
(1)\:\:\(\frac43\)\\
(2)\:\:\(\frac12\)\\
(3)\:\:\(10\)\\
(4)\:\:\(125\)

%
\an{35}
\noindent
(1)\:\:\(\frac34\)\\
(2)\:\:\(-2\)\\
(3)\:\:\(2\)\\
(4)\:\:\(2\)

%
\an{36}
\noindent
\(2< x< 3\) 또는 \(3< x< 5\)

%
\an{37}
\noindent
(1)\:\:\(1\)\\
(2)\:\:\(4\)

%
\an{38}
\noindent
\(-5\)

%
\an{39}
\noindent
\(14\)

%
\an{40}
\noindent
\(\frac{13}{12}\)

\end{minipage}
\end{document}