\documentclass{oblivoir}
\usepackage{amsmath,amssymb,kotex,paralist,graphicx}
\usepackage{mdframed}
\usepackage{../kswrapfig}
\usepackage{fapapersize}
\usefapapersize{210mm,297mm,20mm,*,20mm,*}
%\pagestyle{empty}
\usepackage{multicol}
\setlength{\columnsep}{30pt}
\setlength{\columnseprule}{1pt}
%\def\columnseprulecolor{\color{blue}}

%%% 객관식 선지

\usepackage{tabto,pifont}
\TabPositions{0.2\textwidth,0.4\textwidth,0.6\textwidth,0.8\textwidth}

\newcommand\one{\ding{172}}
\newcommand\two{\ding{173}}
\newcommand\three{\ding{174}}
\newcommand\four{\ding{175}}
\newcommand\five{\ding{176}}

\newcommand\taba[5]{\par\bigskip\noindent
\one\:{\ensuremath{#1}}
\tab\two\:\:{\ensuremath{#2}}
\tab\three\:\:{\ensuremath{#3}}
\tab\four\:\:{\ensuremath{#4}}
\tab\five\:\:{\ensuremath{#5}}}

\newcommand\tabb[5]{\par\bigskip\noindent
\one\:{\ensuremath{#1}}
\tabto{0.16\textwidth}\two\:\:{\ensuremath{#2}}
\tabto{0.33\textwidth}\three\:\:{\ensuremath{#3}}\medskip\par\noindent
\four\:\:{\ensuremath{#4}}.
\tabto{0.16\textwidth}\five\:\:{\ensuremath{#5}}}

\newcommand\tabc[5]{\par\bigskip\noindent
\one\:{\ensuremath{#1}}
\tabto{0.25\textwidth}\two\:\:{\ensuremath{#2}}\medskip\par\noindent
\three\:\:{\ensuremath{#3}}
\tabto{0.25\textwidth}\four\:\:{\ensuremath{#4}}\medskip\par\noindent
\five\:\:{\ensuremath{#5}}}

\newcommand\tabd[5]{\par\bigskip\noindent
\one\:{#1}\medskip\par\noindent
\two\:\:{#2}\medskip\par\noindent
\three\:\:{#3}\medskip\par\noindent
\four\:\:{#4}\medskip\par\noindent
\five\:\:{#5}}

%%% Counters
\newcounter{num}

%%% Commands
\newcommand{\prob}[1]
{\vs\par\noindent\refstepcounter{num}\textbf{문제 \arabic{num})}\label{#1}\par\noindent}

\newcommand\vs[1]{\vspace{70pt}}

\newcommand\inc[1]{\begin{center}\includegraphics[width=0.95\columnwidth]{#1}\end{center}}

\newcommand\pb[1]{\ensuremath{\fbox{\phantom{#1}}}}

\newcommand\ba{\ensuremath{\:|\:}}

\newcommand\an[2]{\par\bigskip\noindent\textbf{문제 \ref{#1})} #2\\}

\newcommand\ans[1]{\begin{flushright}\textbf{답 : }#1\end{flushright}}

\renewcommand{\arraystretch}{1.5}

%%% Meta Commands
\let\oldsection\section
\renewcommand\section{\clearpage\oldsection}
\let\emph\textsf

%%%%
\begin{document}

\title{수학(상) : 06 원의 방정식}
\author{}
\date{\today}
\maketitle
\tableofcontents
\newpage

%%trace
\section{자취문제}
\begin{center}
\fbox{\(\cdots\)를 만족시키는 점 \(P\)의 자취를 구하여라.}\\[10pt]
\fbox{\(\cdots\)를 만족시키는 점 \(P\)의 자취의 방정식을 구하여라.}\\[10pt]
\fbox{\(\cdots\)를 만족시키는 점 \(P\)가 그리는 도형의 방정식을 구하여라.}\\[10pt]
\fbox{\(\cdots\)를 만족시키는 점 \(P\)가 그리는 궤적을 구하여라.}
\end{center}
와 같은 수학문제가 있다.
이것을 `자취문제'라고 하자.

%trace1
\exam{}\label{trace1}
예를 들어,
\vspace{-15pt}
\begin{center}
\fbox{두 점 \(A(1,0)\), \(B(5,0)\)에 대해 \(\ov{PA}=\ov{PB}\)을 만족시키는 점 \(P\)의 자취를 구하여라.}
\end{center}
라는 문제가 있다고 하자.

\(P\)가 어떤 점일 때 \(\ov{PA}=\ov{PB}\)가 성립할까?
만약 \(P=(2,0)\)이거나 \(P=(5,2)\)이면 \(\ov{PA}\neq\ov{PB}\)이 되어 주어진 조건이 성립하지 않는다.
하지만 \(P=(3,0)\)이거나 \(P=(3,2)\) 같은 점이면 주어진 조건 \(\ov{PA}=\ov{PB}\)가 성립한다.
조금만 더 생각해보면 점 \(P\)가 \(A\)와 \(B\)의 수직이등분선인 \(x=3\) 위에 있으면 된다는 것을 알 수 있다.

이 문제는 다음 두 방법으로 설명할 수 있다.
\begin{mdframed}[frametitle=풀이1]
점 \(P\)가 \(A\)와 \(B\)의 수직이등분선 \(l\) 위에 있으면 주어진 조건 \(\ov{PA}=\ov{PB}\)가 성립한다.
하지만 점 \(P\)가 직선 \(l\)보다 왼쪽에 있으면 \(\ov{PA}<\ov{PB}\)이고 오른쪽에 있으면 \(\ov{PA}<\ov{PB}\)이다.

즉 \(P\)가 \(l\) 위에 있지 않으면 주어진 조건 \(\ov{PA}=\ov{PB}\)가 성립하지 않는다.
따라서 \(P\)의 자취는 \(A\)와 \(B\)의 수직이등분선이다.

점 \(P\)가 \(A\)와 \(B\)의 수직이등분선 \(l\) 위에 있으면 주어진 조건 \(\ov{PA}=\ov{PB}\)가 성립한다.
하지만 점 \(P\)가 직선 \(l\)보다 왼쪽에 있으면 \(\ov{PA}<\ov{PB}\)이고 오른쪽에 있으면 \(\ov{PA}<\ov{PB}\)이다.

즉 \(P\)가 \(l\) 위에 있지 않으면 주어진 조건 \(\ov{PA}=\ov{PB}\)가 성립하지 않는다.
따라서 \(P\)의 자취는 \(A\)와 \(B\)의 수직이등분선이다.

점 \(P\)가 \(A\)와 \(B\)의 수직이등분선 \(l\) 위에 있으면 주어진 조건 \(\ov{PA}=\ov{PB}\)가 성립한다.
하지만 점 \(P\)가 직선 \(l\)보다 왼쪽에 있으면 \(\ov{PA}<\ov{PB}\)이고 오른쪽에 있으면 \(\ov{PA}<\ov{PB}\)이다.

즉 \(P\)가 \(l\) 위에 있지 않으면 주어진 조건 \(\ov{PA}=\ov{PB}\)가 성립하지 않는다.
따라서 \(P\)의 자취는 \(A\)와 \(B\)의 수직이등분선이다.

\end{mdframed}

%\begin{mdframed}[frametitle=풀이2]
%구하는 점 \(P\)를
%\[P=(x,y)\]
%라고 두자.
%그러면 \(\ov{PA}=\ov{PB}\)는
%\[\sqrt{(x-1)^2+(y-0)^2}=\sqrt{(x-5)^2+(y-0)^2}\]
%이고 이것을 제곱하여 정리하면
%\begin{gather*}
%(x-1)^2+(y-0)^2=(x-5)^2+(y-0)^2\\
%x^2-2x+1+y^2=x^2-10x+25+y^2\\
%8x=24\\
%x=3
%\end{gather*}
%이다.
%따라서 점 \(P\)의 자취의 방정식은 \(x=3\)이다.
%\end{mdframed}
%\ans{\(A\)와 \(B\)의 수직이등분선 \quad또는\quad \(x=3\)}

\end{document}