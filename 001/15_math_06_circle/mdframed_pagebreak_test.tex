\documentclass{oblivoir}
%%%Default packages
\usepackage{amsmath,amssymb,amsthm,kotex,tabu,graphicx,pifont}
\usepackage{../kswrapfig}

\usepackage{gensymb} %\degree

%%%More packages
%\usepackage{caption,subcaption}
%\usepackage[perpage]{footmisc}
%
\usepackage[skipabove=10pt,innertopmargin=10pt,nobreak=true]{mdframed}

\usepackage[inline]{enumitem}
\setlist[enumerate,1]{label=(\arabic*)}
\setlist[enumerate,2]{label=(\alph*)}

\usepackage{multicol}
\setlength{\columnsep}{30pt}
\setlength{\columnseprule}{1pt}
%
%\usepackage{forest}
%\usetikzlibrary{shapes.geometric,arrows.meta,calc}
%
%%%defi theo exam prob rema proo
%이 환경들 아래에 문단을 쓸 경우 살짝 들여쓰기가 되므로 \hspace{-.7em}가 필요할 수 있다.

\newcounter{num}
\newcommand{\defi}[1]
{\noindent\refstepcounter{num}\textbf{정의 \arabic{num})} #1\par\noindent}
\newcommand{\theo}[1]
{\noindent\refstepcounter{num}\textbf{정리 \arabic{num})} #1\par\noindent}
\newcommand{\revi}[1]
{\noindent\refstepcounter{num}\textbf{복습 \arabic{num})} #1\par\noindent}
\newcommand{\exam}[1]
{\bigskip\bigskip\noindent\refstepcounter{num}\textbf{예시 \arabic{num})} #1\par\noindent}
\newcommand{\prob}[1]
{\bigskip\bigskip\noindent\refstepcounter{num}\textbf{문제 \arabic{num})} #1\par\noindent}
\newcommand{\rema}[1]
{\bigskip\bigskip\noindent\refstepcounter{num}\textbf{참고 \arabic{num})} #1\par\noindent}
\newcommand{\proo}
{\bigskip\noindent\textsf{증명)}}

\newenvironment{talign}
 {\let\displaystyle\textstyle\align}
 {\endalign}
\newenvironment{talign*}
 {\let\displaystyle\textstyle\csname align*\endcsname}
 {\endalign}
%
%%%Commands

\newcommand{\procedure}[1]{\begin{mdframed}\vspace{#1\textheight}\end{mdframed}}

\newcommand\an[1]{\par\bigskip\noindent\textbf{문제 \ref{#1})}\par\noindent}

\newcommand\ann[2]{\par\bigskip\noindent\textbf{문제 \ref{#1})}\:\:#2\par\medskip\noindent}

\newcommand\ans[1]{\begin{flushright}\textbf{답 : }#1\end{flushright}}

\newcommand\anssec[1]{\bigskip\bigskip\noindent{\large\bfseries#1}}

\newcommand{\pb}[1]%\Phantom + fBox
{\fbox{\phantom{\ensuremath{#1}}}}

\newcommand\ba{\,|\,}

\newcommand\ovv[1]{\ensuremath{\overline{#1}}}
\newcommand\ov[2]{\ensuremath{\overline{#1#2}}}
%
%%%% Settings
%\let\oldsection\section
%
%\renewcommand\section{\clearpage\oldsection}
%
%\let\emph\textsf
%
%\renewcommand{\arraystretch}{1.5}
%
%%%% Footnotes
%\makeatletter
%\def\@fnsymbol#1{\ensuremath{\ifcase#1\or
%*\or **\or ***\or
%\star\or\star\star\or\star\star\star\or
%\dagger\or\dagger\dagger\or\dagger\dagger\dagger
%\else\@ctrerr\fi}}
%
%\renewcommand{\thefootnote}{\fnsymbol{footnote}}
%\makeatother
%
%\makeatletter
%\AtBeginEnvironment{mdframed}{%
%\def\@fnsymbol#1{\ensuremath{\ifcase#1\or
%*\or **\or ***\or
%\star\or\star\star\or\star\star\star\or
%\dagger\or\dagger\dagger\or\dagger\dagger\dagger
%\else\@ctrerr\fi}}%
%}   
%\renewcommand\thempfootnote{\fnsymbol{mpfootnote}}
%\makeatother
%
%%% 객관식 선지
\newcommand\one{\ding{172}}
\newcommand\two{\ding{173}}
\newcommand\three{\ding{174}}
\newcommand\four{\ding{175}}
\newcommand\five{\ding{176}}
\usepackage{tabto,pifont}
%\TabPositions{0.2\textwidth,0.4\textwidth,0.6\textwidth,0.8\textwidth}

\newcommand\taba[5]{\par\noindent
\one\:{#1}
\tabto{0.2\textwidth}\two\:\:{#2}
\tabto{0.4\textwidth}\three\:\:{#3}
\tabto{0.6\textwidth}\four\:\:{#4}
\tabto{0.8\textwidth}\five\:\:{#5}}

\newcommand\tabb[5]{\par\noindent
\one\:{#1}
\tabto{0.33\textwidth}\two\:\:{#2}
\tabto{0.67\textwidth}\three\:\:{#3}\medskip\par\noindent
\four\:\:{#4}
\tabto{0.33\textwidth}\five\:\:{#5}}

\newcommand\tabc[5]{\par\noindent
\one\:{#1}
\tabto{0.5\textwidth}\two\:\:{#2}\medskip\par\noindent
\three\:\:{#3}
\tabto{0.5\textwidth}\four\:\:{#4}\medskip\par\noindent
\five\:\:{#5}}

\newcommand\tabd[5]{\par\noindent
\one\:{#1}\medskip\par\noindent
\two\:\:{#2}\medskip\par\noindent
\three\:\:{#3}\medskip\par\noindent
\four\:\:{#4}\medskip\par\noindent
\five\:\:{#5}}
%
%%%% fonts
%
%\usepackage{fontspec, xunicode, xltxtra}
%\setmainfont[]{은 바탕}
%\setsansfont[]{은 돋움}
%\setmonofont[]{은 바탕}
%\XeTeXlinebreaklocale "ko"
%%%%
\begin{document}

\title{수학(상) : 06 원의 방정식}
\author{}
\date{\today}
\maketitle
\tableofcontents
\newpage

%%trace
\section{자취문제}
\begin{center}
\fbox{\(\cdots\)를 만족시키는 점 \(P\)의 자취를 구하여라.}\\[10pt]
\fbox{\(\cdots\)를 만족시키는 점 \(P\)의 자취의 방정식을 구하여라.}\\[10pt]
\fbox{\(\cdots\)를 만족시키는 점 \(P\)가 그리는 도형의 방정식을 구하여라.}\\[10pt]
\fbox{\(\cdots\)를 만족시키는 점 \(P\)가 그리는 궤적을 구하여라.}
\end{center}
와 같은 수학문제가 있다.
이것을 `자취문제'라고 하자.

%trace1
\exam{}\label{trace1}
예를 들어,
\vspace{-15pt}
\begin{center}
\fbox{두 점 \(A(1,0)\), \(B(5,0)\)에 대해 \(\ov{PA}=\ov{PB}\)을 만족시키는 점 \(P\)의 자취를 구하여라.}
\end{center}
라는 문제가 있다고 하자.

\(P\)가 어떤 점일 때 \(\ov{PA}=\ov{PB}\)가 성립할까?
만약 \(P=(2,0)\)이거나 \(P=(5,2)\)이면 \(\ov{PA}\neq\ov{PB}\)이 되어 주어진 조건이 성립하지 않는다.
하지만 \(P=(3,0)\)이거나 \(P=(3,2)\) 같은 점이면 주어진 조건 \(\ov{PA}=\ov{PB}\)가 성립한다.
조금만 더 생각해보면 점 \(P\)가 \(A\)와 \(B\)의 수직이등분선인 \(x=3\) 위에 있으면 된다는 것을 알 수 있다.

이 문제는 다음 두 방법으로 설명할 수 있다.
\begin{mdframed}[frametitle=풀이1]
점 \(P\)가 \(A\)와 \(B\)의 수직이등분선 \(l\) 위에 있으면 주어진 조건 \(\ov{PA}=\ov{PB}\)가 성립한다.
하지만 점 \(P\)가 직선 \(l\)보다 왼쪽에 있으면 \(\ov{PA}<\ov{PB}\)이고 오른쪽에 있으면 \(\ov{PA}<\ov{PB}\)이다.

즉 \(P\)가 \(l\) 위에 있지 않으면 주어진 조건 \(\ov{PA}=\ov{PB}\)가 성립하지 않는다.
따라서 \(P\)의 자취는 \(A\)와 \(B\)의 수직이등분선이다.

점 \(P\)가 \(A\)와 \(B\)의 수직이등분선 \(l\) 위에 있으면 주어진 조건 \(\ov{PA}=\ov{PB}\)가 성립한다.
하지만 점 \(P\)가 직선 \(l\)보다 왼쪽에 있으면 \(\ov{PA}<\ov{PB}\)이고 오른쪽에 있으면 \(\ov{PA}<\ov{PB}\)이다.

즉 \(P\)가 \(l\) 위에 있지 않으면 주어진 조건 \(\ov{PA}=\ov{PB}\)가 성립하지 않는다.
따라서 \(P\)의 자취는 \(A\)와 \(B\)의 수직이등분선이다.

점 \(P\)가 \(A\)와 \(B\)의 수직이등분선 \(l\) 위에 있으면 주어진 조건 \(\ov{PA}=\ov{PB}\)가 성립한다.
하지만 점 \(P\)가 직선 \(l\)보다 왼쪽에 있으면 \(\ov{PA}<\ov{PB}\)이고 오른쪽에 있으면 \(\ov{PA}<\ov{PB}\)이다.

즉 \(P\)가 \(l\) 위에 있지 않으면 주어진 조건 \(\ov{PA}=\ov{PB}\)가 성립하지 않는다.
따라서 \(P\)의 자취는 \(A\)와 \(B\)의 수직이등분선이다.

\end{mdframed}

%\begin{mdframed}[frametitle=풀이2]
%구하는 점 \(P\)를
%\[P=(x,y)\]
%라고 두자.
%그러면 \(\ov{PA}=\ov{PB}\)는
%\[\sqrt{(x-1)^2+(y-0)^2}=\sqrt{(x-5)^2+(y-0)^2}\]
%이고 이것을 제곱하여 정리하면
%\begin{gather*}
%(x-1)^2+(y-0)^2=(x-5)^2+(y-0)^2\\
%x^2-2x+1+y^2=x^2-10x+25+y^2\\
%8x=24\\
%x=3
%\end{gather*}
%이다.
%따라서 점 \(P\)의 자취의 방정식은 \(x=3\)이다.
%\end{mdframed}
%\ans{\(A\)와 \(B\)의 수직이등분선 \quad또는\quad \(x=3\)}

\end{document}