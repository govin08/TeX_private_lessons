\documentclass{oblivoir}
\usepackage{amsmath,amssymb,kotex,paralist,graphicx}
\usepackage{mdframed}
\usepackage{../kswrapfig}
\usepackage{fapapersize}
\usefapapersize{210mm,297mm,20mm,*,20mm,*}
%\pagestyle{empty}
\usepackage{multicol}
\setlength{\columnsep}{30pt}
\setlength{\columnseprule}{1pt}
%\def\columnseprulecolor{\color{blue}}

%%% 객관식 선지

\usepackage{tabto,pifont}
\TabPositions{0.2\textwidth,0.4\textwidth,0.6\textwidth,0.8\textwidth}

\newcommand\one{\ding{172}}
\newcommand\two{\ding{173}}
\newcommand\three{\ding{174}}
\newcommand\four{\ding{175}}
\newcommand\five{\ding{176}}

\newcommand\taba[5]{\par\bigskip\noindent
\one\:{\ensuremath{#1}}
\tab\two\:\:{\ensuremath{#2}}
\tab\three\:\:{\ensuremath{#3}}
\tab\four\:\:{\ensuremath{#4}}
\tab\five\:\:{\ensuremath{#5}}}

\newcommand\tabb[5]{\par\bigskip\noindent
\one\:{\ensuremath{#1}}
\tabto{0.16\textwidth}\two\:\:{\ensuremath{#2}}
\tabto{0.33\textwidth}\three\:\:{\ensuremath{#3}}\medskip\par\noindent
\four\:\:{\ensuremath{#4}}.
\tabto{0.16\textwidth}\five\:\:{\ensuremath{#5}}}

\newcommand\tabc[5]{\par\bigskip\noindent
\one\:{\ensuremath{#1}}
\tabto{0.25\textwidth}\two\:\:{\ensuremath{#2}}\medskip\par\noindent
\three\:\:{\ensuremath{#3}}
\tabto{0.25\textwidth}\four\:\:{\ensuremath{#4}}\medskip\par\noindent
\five\:\:{\ensuremath{#5}}}

\newcommand\tabd[5]{\par\bigskip\noindent
\one\:{#1}\medskip\par\noindent
\two\:\:{#2}\medskip\par\noindent
\three\:\:{#3}\medskip\par\noindent
\four\:\:{#4}\medskip\par\noindent
\five\:\:{#5}}

%%% Counters
\newcounter{num}

%%% Commands
\newcommand{\prob}[1]
{\vs\par\noindent\refstepcounter{num}\textbf{문제 \arabic{num})}\label{#1}\par\noindent}

\newcommand\vs[1]{\vspace{70pt}}

\newcommand\inc[1]{\begin{center}\includegraphics[width=0.95\columnwidth]{#1}\end{center}}

\newcommand\pb[1]{\ensuremath{\fbox{\phantom{#1}}}}

\newcommand\ba{\ensuremath{\:|\:}}

\newcommand\an[2]{\par\bigskip\noindent\textbf{문제 \ref{#1})} #2\\}

\newcommand\ans[1]{\begin{flushright}\textbf{답 : }#1\end{flushright}}

\renewcommand{\arraystretch}{1.5}

%%% Meta Commands
\let\oldsection\section
\renewcommand\section{\clearpage\oldsection}
\let\emph\textsf

%%%%
\begin{document}

\title{수학 : 01 다항식의 연산}
\author{}
\date{\today}
\maketitle
\tableofcontents
\newpage

%%%
\section{다항식}

%%
\subsection{다항식}

%
\exam{\(x^2-3x+4\)}
\begin{enumerate}
\item
\(x\)에 대한 \emph{다항식} \(x^2-3x+4\)은 세 개의 \emph항 \(x^2\), \(-3x\), \(4\)로 이루어져 있다.
\item
\(x^2\)의 \emph{차수}는 \(2\)이고 \emph{이차항}이라고 부른다.
이차항의 \emph{계수}는 \(1\)이다.
\item
\(-3x\)의 차수는 \(1\)이고 \emph{일차항}이라고 부른다.
일차항의 계수는 \(-3\)이다.
\item
\(4\)의 차수는 \(0\)이고 \emph{상수항}이라고 부른다.
\item
\emph{최고차항}의 차수가 \(2\)이므로 이 다항식 \(x^2+3x+4\)는 \emph{이차식}이다.
\item
이 다항식은 \emph{내림차순}으로 정리되어 있다.
이것을 \emph{오름차순}으로 정리하면 \(4+3x+x^2\)이다.
\end{enumerate}

%
\exam{\(x^3+2x^2y-x-2y\)}
\begin{enumerate}
\item
\(x\), \(y\)에 대한 다항식 \(x^3+2x^2y-x-2y\)은 네 개의 항으로 이루어져 있다.
\item
\(x^3\)의 차수는 \(x\)에 대하여 3차이고, 계수는 1이다.
\item
\(2x^2y\)의 차수는 \(x\)에 대하여 2차, \(y\)에 대하여 1차이며 계수는 2이다.
\item
이 다항식을 \(x\)에 대해 내림차순으로 정리하면 \(x^3+2yx^2-x-2y\)이고, \(y\)에 대해 내림차순으로 정리하면 \((2x^2-2)y+(x^3-x)\)이다.
\item
이 다항식은 \(x\)에 대하여 삼차식, \(y\)에 대하여 일차식이다.
\end{enumerate}

\newpage
%
\prob{다항식 \(x^3-6x+4\)에 대한 다음 설명 중 틀린 것을 고르시오.}\label{poly1}
\vspace{-20pt}
\tabd
{\text{세 개의 항으로 이루어져 있다.}}
{\text{일차항의 계수는 \(6\)이다.}}
{\text{상수항은 \(4\)이다.}}
{\text{3차 다항식이다.}}
{\text{오름차순으로 정리하면 \(4-6x+x^3\)이다.}}

%
\prob{다음 다항식에 대한 설명 중 틀린 것을 고르시오.}\label{poly2}
\vspace{-20pt}
\tabd
{\(\sqrt{x+1}\), \(\frac1{x^2}+3\)은 다항식이 아니다.}
{\(x^2+y^2+z^2-xy-yz-zx\)는 \(x\), \(y\), \(z\)에 대한 다항식이다.}
{\(x^2+3xy+2y^2\)은 \(x\), \(y\)에 대한 이차식이다.}
{\(x^4-4x^2y^2+3y^4\)에서 \(x^2y^2\)의 계수는 \(-4\)이다.}
{\(2x^2-3xy+3y^2-2x+4y-3\)을 \(x\)에 대한 내림차순으로 정리하면\\ \(2x^2-(3y-2)x+3y^2-4y-3\)이다.}

\clearpage
%%
\subsection{다항식의 덧셈, 뺄셈}

다항식 \(A\), \(B\), \(C\)에 대하여 다음 법칙들이 성립한다.
\begin{mdframed}
%
\defi{}
\begin{enumerate}
\item
\emph{교환법칙} : \(A+B=B+A\)			\tabto{0.6\textwidth}\(AB=BA\)
\item
\emph{결합법칙} : \((A+B)+C=A+(B+C)\)	\tabto{0.6\textwidth}\((AB)C=A(BC)\)
\item
\emph{분배법칙} : \(A(B+C)=AB+AC\)		
\end{enumerate}
\end{mdframed}

%
\exam{}
\vspace{-25pt}
\begin{align*}
(1)&\:\:
(3x+2y)+(4x-3y)
=(3x+4x)+(2y-3y)
=(3+4)x+(2-3)y=7x-y\\
(2)&\:\:
3(x^2-x+1)+2(-x^2+2x-3)
=(3x^2-3x+3)+(-2x^2+4x-6)\\
&=(3x^2-2x^2)+(-3x+4x)+(3-6)
=(3-2)x^2+(-3+4)x+(-3)\\
&=x^2+x-3.
\end{align*}

%
\prob{\(A=3x^2+3xy-5y^2\), \(B=x^2-xy-3y^2\)일 때, 다음 물음에 답하여라.}
\begin{enumerate}\label{poly3}
\item
\(2A-B\)를 계산하여라.
\item
\(A-2X=B\)를 만족시키는 다항식 \(X\)를 구하여라.
\end{enumerate}

%%%
\section{다항식의 나눗셈}

%
\subsection{정수의 나눗셈}

\exam{}
\(32\)을 \(5\)로 나누면 몫은 \(6\)이고 나머지는 \(2\)이다.
\begin{table}[h!]
\centering
\begin{tabular}{cc@{}c}
&&6\\
\cline{2-3}
5	&\multicolumn{1}{|c}{3}	&2\\
	&3							&0\\
\hline
	&							&2
\end{tabular}
\end{table}

이것을 \[32=5\times6+2\]로 표현할 수 있다.
하지만
\[32=5\times5+7\]
이라고 해서 몫이 \(5\)이고 나머지가 \(7\)이라고 말하지는 않는다.
또한,
\[32=5\times7+(-3)\]
라고 해서 몫이 \(7\)이고 나머지가 \(-3\)이라고 말하지는 않는다.
%\[28=5\times q+r\]
%으로 표현되었을 때, \(r\)의 범위가 \(0\le r<5\)인 경우에만 \(r\)을 나머지라고 부른다.

\begin{mdframed}
%
\defi{}
\(a\)가 정수이고, \(b\)가 자연수일 때,
\[a=bq+r\qquad(0\le r<b)\]
가 성립하면, \(a\)를 \(b\)로 나누었을 떄의 \emph몫은 \(q\), \emph{나머지}는 \(r\)이다.
\end{mdframed}

\clearpage
%
\exam{}
\begin{enumerate}
\item
\(32\)\를 \(4\)\으로 나누면(\(32=4\times8+0\)) 몫이 \(8\)이고 나머지가 \(0\)이다.
이때 \(32\)\은 \(4\)\으로 \emph{나누어떨어진다}고 한다.
또한 \(4\)\는 \(32\)의 약수, \(32\)\은 \(4\)의 배수이다.
\item
\(-32\)\은 \(5\)\로 나누면(\(-32=5\times(-7)+3\)) 몫이 \(-7\)이고 나머지가 \(3\)이다.
\item
\(2\)로 나누어떨어지는 정수를 \emph{짝수}, \(2\)로 나누었을 때 나머지가 \(1\)인 정수를 \emph{홀수}라고 한다.
따라서 \(5\)는 홀수, \(0\)은 짝수, \(-4\)는 짝수이다.
\end{enumerate}

%
\prob{}\label{div1}
주어진 \(a\), \(b\)에 대하여, \(a\)를 \(b\)로 나누었을 때의 몫 \(q\)와 나머지 \(r\)을 각각 구하여라.\begin{enumerate}
\item
\(a=50,\quad b=4\)
\item
\(a=50,\quad b=5\)
\item
\(a=50,\quad b=12\)
\item
\(a=0,\quad b=4\)
\item
\(a=-14,\quad b=5\)
\item
\(a=-14,\quad b=2\)
\end{enumerate}

\clearpage
%%%
\subsection{다항식의 나눗셈}

정수와 마찬가지로 다항식도 나눌 수 있다.

%
\exam{}
다항식 \(x^3+2x^2+5x-3\)을 \(x^2-2x-1\)으로 나누면
\begin{equation*}
\begin{array}{c@{\:}c@{\:}cc@{\:}c@{\:}c@{\:}c}
&&&x&+4\\
\cline{4-7}
x^2&-2x&-1	&\multicolumn{1}{|c}{x^3}&+2x^2&+5x&-3\\
&&							&x^3	&-2x^2	&-x\\
\cline{4-7}
&&							&			&4x^2	&+6x		&-3\\
&&							&			&4x^2	&-8x	&-4\\
\cline{4-7}
&&							&			&			&14x	&+1
\end{array}
\end{equation*}
몫이 \(x+4\)이고 나머지가 \(14x+1\)이다.

이것을
\[x^3+2x^2+5x-3=(x^2-2x-1)(x+4)+14x+1\]
로 표현한다.
한편
\[x^3+2x^2+5x-3=(x^2-2x-1)(x+3)+x^2+12x\]
도 성립한다.
하지만 몫이 \(x+3\)이고 나머지가 \(x^2+12x\)이라고 말하지는 않는다.

\begin{mdframed}
%
\defi{}
\(A\), \(B\)가 다항식일 때,
\[A=BQ+R\qquad(R\text{의 차수}<B\text{의 차수})\]
가 성립하면, \(A\)를 \(B\)로 나누었을 떄의 \emph몫은 \(Q\), \emph{나머지}는 \(R\)이다.
\end{mdframed}

\clearpage
%
\exam{}
\begin{enumerate}
\item
\(x^3+11=(x+2)(x^2-2x+4)+3\)이므로 \(x^3+11\)을 \(x+2\)로 나누었을 때의 몫은 \(x^2-2x+4\), 나머지는 \(3\)이다.
\item
\(x^4+4x^2+16=(x^2+2x+4)(x^2-2x+4)\)이므로 \(x^4+4x^2+16\)를 \(x^2+2x+4\)로 나누었을 때의 몫은 \(x^2-2x+4\)이고 나머지는 \(0\)이다.
이때, \(x^4+4x^2+16\)는 \(x^2+2x+4\)로 \emph{나누어떨어진다}고 말한다.
\end{enumerate}

%
\prob{다음 나눗셈의 몫과 나머지를 구하여라.}
\begin{enumerate}\label{div2}
\item
\((3x^3-2x^2-5x+1)\div(x-2)\)
\item
\((2x^3-5x^2+4x-1)\div(x^2-2x-2)\)
\end{enumerate}

%
\prob{}
\label{div3}
다항식 \(A\)는 \(x^3+1\)으로 나누어떨어지고 이 때의 몫이 \(x+1\)이다.
다항식 \(A\)를 구하여라.

\clearpage
%%
\subsection{조립제법}
나누는 수 \(B\)가 일차식일 때, 다음의 \emph{조립제법}을 사용할 수 있다.

%
\exam{}
\(3x^3-2x^2-5x+1\)를 \(x-2\)로 나누는 과정은
\begin{equation*}
\begin{array}{ccccc}
\multicolumn{1}{c|}{2}	&3	&-2	&-5	&1\\
\multicolumn{1}{c|}{}		&	&6	&8	&6\\
\cline{2-5}
							&3	&4	&3	&\multicolumn1{|c}{7}\\
\cline{5-5}
\end{array}
\end{equation*}
로 간단히 표시할 수 있다.
이때, 몫은 \(3x^2+4x+3\)이고, 나머지는 \(7\)이다.
이것은 문제 \ref{div2})의 (1)에서 구한 답과 일치한다.

%
\prob{조립제법을 사용하여 다음 나눗셈의 몫과 나머지를 구하여라.}
\begin{enumerate}\label{div4}
\item
\((x^3-2x^2-5x+3)\div(x+2)\)
\item
\((2x^3+3x^2-6x+1)\div(x-\frac12)\)
\end{enumerate}

%
\prob{}\label{div5}
위의 (2)의 결과를 이용하여 \(2x^3+3x^2-6x+1\)를 \(2x-1\)로 나눈 몫과 나머지를 구하여라.


%%%
\section{다항식의 곱셈}

%
\begin{mdframed}
\theo{곱셈공식(1)}
\begin{enumerate}
\item\label{mult_01}
\((a+b)^2=a^2+2ab+b^2\)
\item\label{mult_02}
\((a-b)^2=a^2-2ab+b^2\)
\item\label{mult_03}
\((a+b)(a-b)=a^2-b^2\)
\item\label{mult_04}
\((x+a)(x+b)=x^2+(a+b)x+ab\)
\item\label{mult_05}
\((ax+b)(bx+d)=acx^2+(ad+bc)x+bd\)
\end{enumerate}
\end{mdframed}

%
\prob{다음 식을 전개하여라.}
\begin{enumerate}\label{mult1}
\item
\((2x+1)^2=\)
\item
\((3x-1)^2=\)
\item
\((x+3)(x+4)=\)
\item
\((x+2)(3x+2)=\)
\end{enumerate}

%
\prob{}
\begin{enumerate}\label{mult2}
\item
\(x+y=5\), \(xy=6\)일 때 \(x^2+y^2\)의 값을 구하여라.
\item
\(x-y=2\), \(x^2+y^2=34\)일 때, \(xy\)의 값을 구하여라.
\end{enumerate}

%
\prob{\(x+\frac1x=4\)일 때, 다음 식의 값을 구하여라(단, \(x>1\)).}
\begin{enumerate}\label{mult3}
\item
\(x^2+\frac1{x^2}=\)
\item
\(x-\frac1x=\)
\end{enumerate}

\newpage
%
\begin{mdframed}
\theo{곱셈공식(2)}
\begin{enumerate}
\setcounter{enumi}{5}
\item\label{mult_06}
\((a+b)^3=a^3+3a^2b+3ab^2+b^3\)
\item\label{mult_07}
\((a-b)^3=a^3-3a^2b+3ab^2-b^3\)
\item\label{mult_08}
\((a+b)(a^2-ab+b^2)=a^3+b^3\)
\item\label{mult_09}
\((a-b)(a^2+ab+b^2)=a^3-b^3\)
\item\label{mult_10}
\((a+b+c)^2=a^2+b^2+c^2+2ab+2bc+2ca\)
\item\label{mult_11}
\((x+a)(x+b)(x+c)=x^3+(a+b+c)x^2+(ab+bc+ca)x+abc\)
\end{enumerate}
\end{mdframed}

%
\exam{(\ref{mult_06}), (\ref{mult_08}), (\ref{mult_10})의 식을 유도해보면}
\begin{enumerate}
\item[(\ref{mult_06})]
\((a+b)^3=(a+b)(a+b)^2=(a+b)(a^2+2ab+b^2)\\
=a(a^2+2ab+b^2)+b(a^2+2ab+b^2) =(a^3+2a^2b+ab^2)+(a^2b+2ab^2+b^3)\\
=a^3+3a^2b+3ab^2+b^3\)
\item[(\ref{mult_08})]
\((a+b)(a^2-ab+b^2)=a(a^2-ab+b^2)+b(a^2-ab+b^2)\\
=(a^3-a^2b+ab^2)+(a^2b-ab^2+b^3)=a^3+b^3\)
\item[(\ref{mult_10})]
\((a+b+c)^2=\{(a+b)+c\}^2=(a+b)^2+2(a+b)c+c^2\\
=(a^2+2ab+b^2)+(2ac+2bc)+c^2=a^2+b^2+c^2+2ab+2bc+2ca\)
\end{enumerate}

%
\prob{(\ref{mult_07}), (\ref{mult_09}), (\ref{mult_11})의 식을 유도하여라.}\label{mult4}
\procedure{0.2}

%
\prob{다음 식을 전개하여라.}
\begin{enumerate}\label{mult5}
\item
\((x+2)^3=\)
\item
\((x-1)^3=\)
\item
\((x+2y)(x^2-2xy+4y^2)=\)
\item
\((3a-1)(9a^2+3a+1)=\)
\item
\((a+2b-c)^2=\)
\item
\((x+2)(x-4)(x+5)=\)
\end{enumerate}

%
\prob{}\label{mult6}
\(a+b=3\), \(ab=-2\)일 때, \(a^3+b^3\)의 값을 구하여라.

%
\prob{}\label{mult7}
\(a-b=1\), \(ab=4\)일 때, \(a^3-b^3\)의 값을 구하여라.

%
\prob{}\label{mult8}
\(a+b+c=9\), \(ab+bc+ca=8\)일 때, \(a^2+b^2+c^2\)의 값을 구하여라.

\newpage
%
\begin{mdframed}
\theo{곱셈공식(3)}
\begin{enumerate}
\setcounter{enumi}{11}
\item\label{mult_12}
\((a+b+c)(a^2+b^2+c^2-ab-bc-ca)=a^3+b^3+c^3-3abc\)
\item\label{mult_13}
\((a^2+ab+b^2)(a^2-ab+b^2)=a^4+a^2b^2+b^4\)
\end{enumerate}
\end{mdframed}

%
\exam{(\ref{mult_13})의 식을 유도해보면}
\((a^2+ab+b^2)(a^2-ab+b^2)=\{(a^2+b^2)+ab\}\{(a^2+b^2)-ab\}\\
=(a^2+b^2)^2-(ab)^2=(a^4+2a^2b^2+b^4)-a^2b^2=a^4+a^2b^2+b^4\)

%
\prob{(\ref{mult_12})의 식을 유도하여라.}\label{mult9}
\procedure{0.2}

%
\prob{다음 식을 전개하여라.}
\begin{enumerate}\label{mult10}
\item
\((2a+b-c)(4a^2+b^2+c^2-2ab+bc+2ca)=\)
\item
\((x-y-1)(x^2+y^2+xy+x-y+1)=\)
\item
\((4x^2+2xy+y^2)(4x^2-2xy+y^2)=\)
\end{enumerate}

%
\prob{}\label{mult11}
\(a+b+c=3\), \(a^2+b^2+c^2=9\), \(abc=-4\)일 때, \(a^3+b^3+c^3\)의 값을 구하여라.

%%%
\section{인수분해}

%
\exam{}
하나의 다항식을 두 개 이상의 다항식의 곱으로 나타내는 것을 인수분해라고 한다.
예를 들어, \(x^2-x-2\)를 \emph{인수분해}하면 \((x-2)(x+1)\)이다.
반면 \((x-2)(x+1)\)를 \emph{전개}하면 \(x^2-x-2\)이다.
\begin{align*}
x^2-x-2&\xrightarrow{인수분해}(x-2)(x+1)\\
(x-2)(x+1)&\xrightarrow{\phantom{인}전개\phantom{인}}x^2-x-2
\end{align*}

%\[x^3-x^2+2x-2=(x-1)(x^2+2)\]에서 좌변을 우변으로 바꾸는 것이 \emph{인수분해}이다.
%반면에 우변을 좌변으로 바꾸는 것은 \emph{전개}라고 한다.
%\((x-1)(x^2+2)\)를 \emph{전개}하면 \(x^3-x^2+2x-2\)가 된다.
%반대로 \(x^3

%
\begin{mdframed}
\theo{인수분해공식(1)}
\begin{enumerate}
\item\label{mult_01}
\(a^2+2ab+b^2=(a+b)^2\)
\item\label{mult_02}
\(a^2-2ab+b^2=(a-b)^2\)
\item\label{mult_03}
\(a^2-b^2=(a+b)(a-b)\)
\item\label{mult_04}
\(x^2+(a+b)x+ab=(x+a)(x+b)\)
\item\label{mult_05}
\(acx^2+(ad+bc)x+bd=(ax+b)(cx+d)\)
\end{enumerate}
\end{mdframed}

%
\prob{다음 식을 인수분해하여라.}
\begin{enumerate}\label{fact01}
\item
\(2ab^2+6b=\)
\item
\(x^2+6x+9=\)
\item
\(25a^2-10ab+b^2=\)
\item
\(16x^2-y^2=\)
\item
\(x^2+10x+21=\)
\item
\(6a^2-13a-5=\)
\item
\(3a^2+4ab-7b^2=\)
\end{enumerate}

\newpage
%
\begin{mdframed}
\theo{인수분해 공식(2)}
\begin{enumerate}
\setcounter{enumi}{5}
\item\label{mult_06}
\(a^3+3a^2b+3ab^2+b^3=(a+b)^3\)
\item\label{mult_07}
\(a^3-3a^2b+3ab^2-b^3=(a-b)^3\)
\item\label{mult_08}
\(a^3+b^3=(a+b)(a^2-ab+b^2)\)
\item\label{mult_09}
\(a^3-b^3=(a-b)(a^2+ab+b^2)\)
\item\label{mult_10}
\(a^2+b^2+c^2+2ab+2bc+2ca=(a+b+c)^2\)
\end{enumerate}
\end{mdframed}

%
\prob{다음 식을 인수분해하여라.}
\begin{enumerate}\label{fact02}
\item
\(x^3+9x^2+27x+27=\)
\item
\(x^3-6x^2+12x-8=\)
\item
\(-8a^3+36a^2b-54ab^2+27b^3=\)
\item
\(a^3+8=\)
\item
\(27a^3-64b^3=\)
\item
\(a^2+b^2+4c^2+2ab+4bc+4ca=\)
\item
\(x^2+y^2+z^2-2xy-2yz+2zx=\)
\end{enumerate}

\newpage
%
\begin{mdframed}
\theo{인수분해 공식(3)}
\begin{enumerate}
\setcounter{enumi}{10}
\item\label{mult_11}
\(a^3+b^3+c^3-3abc=(a+b+c)(a^2+b^2+c^2-ab-bc-ca)\)
\item\label{mult_12}
\(a^4+a^2b^2+b^4=(a^2+ab+b^2)(a^2-ab+b^2)\)
\end{enumerate}
\end{mdframed}

%
\prob{다음 식을 인수분해하여라}
\begin{enumerate}\label{fact03}
\item
\(a^3-b^3+c^3+3abc=\)
\item
\(x^3+y^3-3xy+1=\)
\item
\(a^4+a^2+1=\)
\item
\(x^4+4x^2y^2+16y^4=\)
\end{enumerate}

%%%
\section{항등식과 미정계수법}

%%
\subsection{항등식}
\begin{mdframed}
%
\defi{항등식}
주어진 등식의 문자에 어떤 값을 대입해도 항상 성립하는 등식을 \emph{항등식}이라고 한다.
\end{mdframed}

%
\exam{}
\begin{enumerate}
\item
등식 \(x^2-2x-3=0\)에 \(x=3\)을 대입하면 \(0=0\)이 되어 성립한다.
하지만 \(x=4\)을 대입하면 \(5\neq0\)이 되어 성립하지 않는다.
따라서 이 등식은 항등식이 아니다.
\item
등식 \(x^2-2x-3=(x-3)(x+1)\)에 \(x=3\)을 대입하면 \(0=0\)이 되고 \(x=4\)를 대입하면 \(5=5\)가 되어 성립한다.
그밖에 \(x\)에 어떤 값을 대입하더라도 항상 성립한다.
따라서 항등식이 맞다.
실제로 좌변을 잘 정리하면 우변이 되므로, 항등식인 것이 당연하다.
\item
\((x-3y)^2=x^2-6xy+9y^2\)에서 좌변을 전개하면 우변이 된다.
따라서 \(x\)와 \(y\)에 각각 어떤 값을 대입하더라도 이 등식은 항상 성립하며, 항등식이 맞다.
\item
\(x^3+2x^2+5x-3\)를 \(x^2-2x-1\)로 나누어 생기는 몫과 나머지로 만든 
\[x^3+2x^2+5x-3=(x^2-2x-1)(x+4)+14x+1\]
에서도, 우변을 전개하면 좌변이 된다.
따라서 항등식이다.
\end{enumerate}

\clearpage
%
\exam{}
\begin{enumerate}
\item
등식 \(ax+b=0\)이 항등식이면 \(a=0\), \(b=0\)임을 설명하여라.
\item
등식 \(ax+b=a'x+b'\)이 항등식이면 \(a=a'\), \(b=b'\)임을 설명하여라.
\end{enumerate}

\begin{mdframed}
\begin{enumerate}
\item
등식 \(ax+b=0\)에서 \(x\)에 어떤 값을 대입하더라도 항상 성립해야 한다.\\
%\(ax+b=0\)이 \(x\)에 대한 항등식이므로 \(x\)에 어떤 값을 대입하더라도 항상 성립해야 한다.
\(x=0\)을 대입하면 \(a\cdot0+b=0\), 즉 \(b=0\)이다.\\
\(x=1\)을 대입하면 \(a\cdot1+0=0\), 즉 \(a=0\)이다.
\item
주어진 식의 우변을 좌변으로 이항하면 \[(a-a')x+(b-b')=0\]이 되는데, 이 식이 항등식이 되려면 (1)에 의해 \(a-a'=0\), \(b-b'=0\)이어야 한다.
따라서 \(a=a'\)이고 \(b=b'\)이다.
\end{enumerate}
\end{mdframed}

%
\exam{}
등식 \((m-1)x+(n+2)=3x+7\)이 항등식이 되려면 \(m-1=3\), \(n+2=7\)이어야 한다.
따라서 \(m=4\), \(n=5\)이다.

\clearpage
%
\prob{}
\begin{enumerate}\label{iden1}
\item
등식 \(ax^2+bx+c=0\)가 항등식이면 \(a=0\), \(b=0\), \(c=0\)임을 설명하여라.
\item
등식 \(ax^2+bx+c=a'x^2+b'x+c'\)이 항등식이면 \(a=a'\), \(b=b'\), \(c=c'\)임을 설명하여라.
\end{enumerate}

\procedure{0.3}

%
\prob{다음 등식이 \(x\)에 대한 항등식일 때, \(a\), \(b\), \(c\)의 값을 구하여라.}
\begin{enumerate}\label{iden2}
\item
\((a+2)x^2+(b-2)x+3-c=0\)
\item
\(1-2x+ax^2=5x^2+bx+c\)
\end{enumerate}

\clearpage
%
\prob{}
\begin{enumerate}\label{iden3}
\item
등식 \(ax+by+c=0\)가 항등식이면 \(a=0\), \(b=0\), \(c=0\)임을 설명하여라.
\item
등식 \(ax+by+c=a'x+b'y+c'\)이 항등식이면 \(a=a'\), \(b=b'\), \(c=c'\)임을 설명하여라.
\end{enumerate}

\procedure{0.3}

%%
%\prob{다음 등식이 \(x\), \(y\)에 대한 항등식일 때, \(a+b+c\)의 값을 구하여라.}
%\[(a+b)x+(b+c)y+(c+a)=3x+6y-5\]
\bigskip\bigskip
이상에서 다음의 항등식의 성질을 얻을 수 있다.
\begin{mdframed}
%
\theo{항등식의 성질}
\begin{enumerate}[label=(\emph{\alph*})]
\item
\(ax+b=0\)이 항등식이면 \(a=0\), \(b=0\)이다.
\item
\(ax+b=a'x+b'\)이 항등식이면 \(a=a'\), \(b=b'\)이다.
\item
\(ax^2+bx+c=0\)이 항등식이면 \(a=0\), \(b=0\), \(c=0\)이다.
\item
\(ax^2+bx+c=a'x^2+b'x+c'\)이 항등식이면 \(a=a'\), \(b=b'\), \(c=c'\)이다.
\item
\(ax+by+c=0\)이 항등식이면 \(a=0\), \(b=0\), \(c=0\)이다.
\item
\(ax+by+c=a'x+b'y+c'\)이 항등식이면 \(a=a'\), \(b=b'\), \(c=c'\)이다.
\end{enumerate}
\end{mdframed}

\clearpage
%%
\subsection{미정계수법}

%
\exam{다음 등식이 \(x\)에 대한 항등식일 때, 상수 \(a\), \(b\), \(c\)의 값을 구하여라.}
\[a(x-1)^2+b(x-1)+c=2x^2-3x+4\]
\begin{mdframed}[frametitle=<풀이1>]
좌변을 전개하여 정리하면
\begin{align*}
a(x-1)^2+b(x-1)+c
&=ax^2-2ax+a+bx-b+c\\
&=ax^2+(-2a+b)x+(a-b+c)
\end{align*}
이므로 주어진 항등식은
\[ax^2+(-2a+b)x+(a-b+c)=2x^2-3x+4\]
가 되고, 따라서 \(a=2\), \(-2a+b=-3\), \(a-b+c=4\)이다.
이 식을 연립하여 풀면 \(a=2\), \(b=1\), \(c=3\)이 된다.
\end{mdframed}

\begin{mdframed}[frametitle=<풀이2>]
주어진 식이 항등식이므로 \(x\)에 \(x=1\), \(x=0\), \(x=2\)를 넣어도 성립해야 한다.
\begin{align*}
x=1	\:\:;	&\:\:c=3\\
x=0	\:\:;	&\:\:a-b+c=4\\
x=2\:\:;	&\:\:a+b+c=6
\end{align*}
이 식들을 연립하면 \(a=2\), \(b=1\), \(c=3\)이다.
\end{mdframed}

이처럼 항등식에서 아직 정해지지 않은 계수(=미정계수)의 값을 정하는 방법을 \emph{미정계수법}이라고 한다.
<풀이 1>에서처럼 양변의 계수를 비교하는 방법을 \emph{계수비교법},
<풀이 2>에서처럼 \(x\)에 여러 값들을 대입하는 방법을 \emph{수치대입법}이라고 한다.

%
\prob{계수비교법을 이용하여 다음 항등식에서 \(a+b+c\)\을 구하여라.}\label{iden4}
\[x^3+ax^2-36=(x+c)(x^2+bx-12)\]

%
\prob{수치대입법을 이용하여 다음 항등식에서 \(a^2-b^2\)의 값을 구하여라.}\label{iden5}
\[(x-1)^4=x^4-4x^3+ax^2+bx+1\]

%
\prob{다음 등식이 \(x\)에 대한 항등식일 때, 상수 \(a\), \(b\)의 값을 구하여라.}
\begin{enumerate}\label{iden6}
\item
\(2x^2-2=(x+1)(ax+b)\)
\item
\(a(x-1)+b(x-2)=2x-3\)
\end{enumerate}

%%
\subsection{나머지정리와 인수정리}
\(3x^3-2x^2-5x+1\)를 \(x-2\)로 나누었을 때 몫과 나머지를 구하는 과정에는 다음의 두 방법이 있었다.
\begin{equation*}
\begin{array}{c@{\:}cc@{\:}c@{\:}c@{\:}c}
&&3x^2&+4x&+3\\
\cline{3-6}
x&-2	&\multicolumn{1}{|c}{3x^3}	&-2x^2&-5x&+1\\
&							&3x^3		&-6x^2\\
\cline{3-6}
&							&			&4x^2	&-5x	&+1\\
&							&			&4x^2	&-8x	&\\
\cline{3-6}
&							&			&		&3x		&+1\\
&							&			&		&3x 	&-6\\
\cline{3-6}
&							&			&		&	 	&7
\end{array}
\qquad\qquad
\begin{array}{ccccc}
\multicolumn{1}{c|}{2}	&3	&-2	&-5	&1\\
\multicolumn{1}{c|}{}		&	&6	&8	&6\\
\cline{2-5}
							&3	&4	&3	&\multicolumn1{|c}{7}\\
\cline{5-5}
\end{array}
\end{equation*}

하지만 몫은 제외하고, 나머지만 구하려고 한다면 훨씬 쉽게 구하는 방법이 있다.

\begin{mdframed}
%
\theo{나머지정리}
다항식 \(f(x)\)를 \(x-\alpha\)로 나누었을 때의 나머지는 \(f(\alpha)\)이다.
\end{mdframed}

%
\proo
\(f(x)\)를 \(x-\alpha\)로 나누었을 때의 몫을 \(Q(x)\), 나머지를 \(R\)이라고 하면
\[f(x)=(x-\alpha)Q(x)+R\]
이다.
이 식은 항등식이므로 \(x=\alpha\)를 대입해도 성립한다.
따라서 \(f(\alpha)=R\)이다.
\qed

%
\exam{}
\begin{enumerate}
\item
위의 예에서 나머지정리를 쓰면, \(f(x)=3x^3-2x^2-5x+1\)을 \(x-2\)로 나누었을 때의 나머지는
\[R=f(2)=3\cdot2^3-2\cdot2^2-5\cdot2+1=7\]
이다.
\item
\(x^3-2x^2+3x+4\)를 \(x+3\)으로 나누었을 때의 나머지는
\((-3)^3-2(-3)^2+3(-3)+4=-50\)
이다.
\end{enumerate}

\clearpage
%
\prob{\(2x^3-2x^2+1\)을 다음 일차식으로 나누었을 때의 나머지를 구하여라.}
\label{iden7}
\par\noindent
(1) \(x-2\)
\tabto{0.5\textwidth}
(2) \(x+\frac12\)

%
\prob{}
\label{iden8}
\(x^3-x^2+ax+4\)을 \(x+2\)로 나누었을 때의 나머지가 \(2\)일 때, 상수 \(a\)의 값을 구하여라.

%
\exam{}
다항식 \(f(x)\)를 일차식 \(ax+b\)로 나누었을 때의 나머지가 \(f\left(-\frac ba\right)\)임을 보여라.
\begin{mdframed}
\(f(x)\)를 \(ax+b\)로 나누었을 때의 몫을 \(Q(x)\), 나머지를 \(R\)이라고 하면
\[f(x)=(ax+b)Q(x)+R\]
이다.
이 식은 항등식이므로 \(x=-\frac ba\)를 대입해도 성립한다.
따라서 \(f\left(-\frac ba\right)=R\)이다.
\qed
\end{mdframed}

%
\prob{\(3x^2+x+2\)를 다음 일차식으로 나누었을 때의 나머지를 구하여라.}\label{iden9}
\par\noindent
(1) \(2x+1\)
\tabto{0.5\textwidth}
(2) \(3x-2\)

\clearpage
%
\exam{}
다항식 \(f(x)\)를 \(x-1\)로 나누었을 때의 나머지가 \(3\)이고, \(x+2\)로 나누었을 때의 나머지가 \(-3\)이다.
이때, \(f(x)\)를 \((x-1)(x+2)\)로 나누었을 때의 나머지를 구하여라.
\begin{mdframed}
\(f(x)\)를 \((x-1)(x+2)\)로 나누었을 때의 몫을 \(Q(x)\), 나머지를 \(R(x)\)이라고 하자.
나누는 식 \((x-1)(x+2)\)가 이차식이므로, \(R(x)\)는 일차 이하의 다항식 \(ax+b\)이다.
그러면
\[f(x)=(x-1)(x+2)Q(x)+ax+b\]
이다.
문제의 조건에서 \(f(1)=3\)이므로
\[a+b=3\]
또, \(f(-2)=-3\)이므로
\[-2a+b=-3\]
이다.
두 식을 빼면 \(3a=6\), \(a=2\)이다.
따라서 \(b=1\)이다.
그러므로 \(R(x)=2x+1\)이다.
\qed
\end{mdframed}

%
\prob{}\label{iden10}
다항식 \(f(x)\)를 \(x+1\)로 나누었을 때의 나머지가 \(4\)이고, \(x-3\)으로 나누었을 때의 나머지가 \(8\)이다.
이때 \(f(x)\)를 \((x+1)(x-3)\)으로 나누었을 때의 나머지를 구하여라.

\clearpage
\(f(\alpha)\)는 \(x-\alpha\)로 나누었을 때의 나머지이다.
따라서 \(f(\alpha)=0\)이면 \(f(x)\)는 \(x-\alpha\)로 나누어떨어진다.
\begin{mdframed}
%
\theo{인수정리}
\(f(\alpha)=0\)이면 \(f(x)\)는 \(x-\alpha\)로 나누어떨어진다.
\end{mdframed}

%
\exam{}
다항식 \(f(x)=x^3+x^2+4\)에서 \(f(-2)=(-2)^3+(-2)^2+4=0\)이므로, \(f(x)\)는 \(x+2\)로 나누어 떨어진다.
조립제법을 써서 \(f(x)\)를 인수분해하면
\[f(x)=(x+2)(x^2-x+2)\]
가 된다.
이때 \(f(x)\)가 \(x+2\)를 \emph{인수로 가진다}라고 말한다.

%
\prob{다음 일차식 중에서 다항식 \(x^3+5x^2+2x-8\)의 인수인 것을 모두 찾아라.}\label{iden11}
\begin{mdframed}[skipabove=-10pt,innertopmargin=-3pt,leftmargin=60pt,rightmargin=60pt]
\[x,\qquad x-1,\qquad x+1,\qquad x+2\]
\end{mdframed}

%
\prob{}\label{iden12}
다항식 \(x^3+x^2+ax+a\)가 \(x-4\)로 나누어떨어지도록 상수 \(a\)의 값을 정하여라.

%%
%\prob{}
%다항식 \(x^3+ax^2+Bx+2\)가 \((x-1)(x+2)\)로 나누어떨어지도록 상수 \(a\), \(b\)의 값을 정하여라.


%%%
\section*{답}
\addcontentsline{toc}{chapter}{\protect\numberline{*}답}
\begin{minipage}{0.49\textwidth}
%
\an{poly1}
\ding{173}

%
\an{poly2}
\ding{176}

%
\an{poly3}
(1) \(5x^2+7xy-7y^2\)\\
(2) \(x^2+2xy-y^2\)

%
\an{div1}
(1) \(q=12,\quad r=2\)\\
(2) \(q=10,\quad r=0\)\\
(3) \(q=4,\quad r=2\)\\
(4) \(q=0,\quad r=0\)\\
(5) \(q=-3,\quad r=1\)\\
(6) \(q=-7,\quad r=0\)

%
\an{div2}
(1) 몫 : \(3x^2+4x+3\), 나머지 : \(7\)\\
(2) 몫 : \(2x-1\), 나머지 : \(6x-3\)

\end{minipage}
\begin{minipage}{0.49\textwidth}

%
\an{div3}
\(A=x^4+x^3+x+1\)

%
\an{div4}
(1) 몫 : \(x^2-4x+3\), 나머지 : \(-3\)\\
(2) 몫 : \(2x^2+4x-4\), 나머지 : \(-1\)

%
\an{div5}
몫 : \(x^2+2x-2\), 나머지 : \(-1\)

%
\an{mult1}
(1) \(4x^2+4x+1\)\\
(2) \(9x^2-6x+1\)\\
(3) \(x^2+7x+12\)\\
(4) \(3x^2+8x+4\)

%
\an{mult2}
(1) \(13\)\\
(2) \(15\)

%
\an{mult3}
(1) \(14\)\\
(2) \(2\sqrt3\)

\end{minipage}

\clearpage
%
\an{mult4}
\begin{enumerate}
\item[(\ref{mult_07})]
\((a-b)^3=(a-b)(a-b)^2=(a-b)(a^2-2ab+b^2)\\
=a(a^2-2ab+b^2)-b(a^2-2ab+b^2)=(a^3-2a^2b+ab^2)-(a^2b-2ab^2+b^3)\\
=a^3-3a^2b+3ab^2-b^3\)
\item[(\ref{mult_09})]
\((a-b)(a^2+ab+b^2)=a(a^2+ab+b^2)-b(a^2+ab+b^2)\\
=(a^3+a^2b+ab^2)-(a^2b+ab^2+b^3)=a^3-b^3\)
\item[(\ref{mult_11})]
\((x+a)(x+b)(x+c)=\{x^2+(a+b)x+ab\}(x+c)\\
=\{x^2+(a+b)x+ab\}x+\{x^2+(a+b)x+ab\}c\\
=\{x^3+(a+b)x^2+abx\}+\{cx^2+(ca+bc)x+abc\}\\
=x^3+(a+b+c)x^2+(ab+bc+ca)x+abc\)
\end{enumerate}

%
\an{mult5}
(1) \(x^3+6x^2+12x+8\)\\
(2) \(x^3-3x^2+3x-1\)\\
(3) \(x^3+8y^3\)\\
(4) \(27a^3-1\)\\
(5) \(a^2+4b^2+c^2+4ab-4bc-2ca\)\\
(6) \(x^3+3x^2-18x-40\)

%
\an{mult6}
\(45\)

%
\an{mult7}
\(13\)

%
\an{mult8}
\(65\)

\clearpage
%
\an{mult9}
\vspace{-20pt}
\begin{align*}
&(a+b+c)(a^2+b^2+c^2-ab-bc-ca)\\
=&a(a^2+b^2+c^2-ab-bc-ca)+b(a^2+b^2+c^2-ab-bc-ca)\\
&+c(a^2+b^2+c^2-ab-bc-ca)\\
=&(a^3+ab^2+c^2a-a^2b-abc-ca^2)+(a^2b+b^3+bc^2-ab^2-b^2c-abc)\\
&+(ca^2+b^2c+c^3-abc-bc^2-c^2a)\\
=&a^3+b^3+c^3-3abc
\end{align*}

\begin{minipage}{0.49\textwidth}
%
\an{mult10}
(1) \(8a^3+b^3+c^3+6abc\)\\
(2) \(x^3-y^3+xy+x-y-1\)\\
(3) \(16x^4+4x^2y^2+y^4\)

%
\an{mult11}
\(15\)

%
\an{fact01}
(1) \(2b(3ab+3)\)\\
(2) \((x+3)^2\)\\
(3) \((5a-b)^2\)\\
(4) \((4x+y)(4x-y)\)\\
(5) \((x+3)(x+7)\)\\
(6) \((2a-5)(3a+1)\)\\
(7) \((a-b)(3a+7b)\)

\end{minipage}
\begin{minipage}{0.49\textwidth}

%
\an{fact02}
(1) \((x+3)^3\)\\
(2) \((x-2)^3\)\\
(3) \((-2a+3b)^3\)\\
(4) \((a+2)(a^2-2a+4)\)\\
(5) \((3a-4b)(9a^2+12ab+16b^2)\)\\
(6) \((a+b+2c)^2\)\\
(7) \((x-y+z)^2\)

%
\an{fact03}
(1) \((a-b+c)(a^2+b^2+c^2+ab+bc-ca)\)\\
(2) \((x+y+1)(x^2+y^2-xy-x-y+1)\)\\
(3) \((a^2+a+1)(a^2-a+1)\)\\
(4) \((x^2+2xy+4y^2)(x^2-2xy+4y^2)\)
\end{minipage}

\clearpage
%
\an{iden1}
\begin{mdframed}
\begin{enumerate}
\item
\(x=0\)을 대입하면 \(c=0\)이다.
\(x=1\)을 대입하면 \(a+b=0\)이다.
\(x=-1\)을 대입하면 \(a-b=0\)이다.
두 식을 연립하면 \(a=0\), \(b=0\)를 얻는다.
\item
주어진 식의 우변을 좌변으로 이항하면 \[(a-a')x^2+(b-b')x+(c-c')=0\]이다.
따라서 (1)에 의해 \(a=a'\), \(b=b'\), \(c=c'\)를 얻는다.
\end{enumerate}
\end{mdframed}

%
\an{iden2}
(1) \(a=-2\), \(b=2\), \(c=3\)\\
(2) \(a=5\), \(b=-2\), \(c=1\)

\an{iden3}
\begin{mdframed}
\begin{enumerate}
\item
\(x=0\), \(y=0\)을 대입하면 \(c=0\)이다.
\(x=1\), \(y=0\)을 대입하면 \(a=0\)이다.
\(x=0\), \(y=1\)을 대입하면 \(b=0\)이다.
\item
주어진 식의 우변을 좌변으로 이항하면 \[(a-a')x+(b-b')y+(c-c')=0\]이다.
따라서 (1)에 의해 \(a=a'\), \(b=b'\), \(c=c'\)를 얻는다.
\end{enumerate}
\end{mdframed}

\clearpage
%
\an{iden4}
\(14\)

%
\an{iden5}
\(20\)

%
\an{iden6}
(1) \(a=2\), \(b=-2\)\\
(2) \(a=1\), \(b=-6\), \(c=10\)

%
\an{iden7}
(1) \(9\)\\
(2) \(\frac14\)

%
\an{iden8}
\(-5\)

%
\an{iden9}
(1) \(\frac94\)\\
(2) \(4\)

%
\an{iden10}
\(x+5\)

%
\an{iden11}
\(x-1\), \(x+2\)

%
\an{iden12}
\(-16\)

\end{document}