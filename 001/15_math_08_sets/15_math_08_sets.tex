\documentclass{oblivoir}
\usepackage{amsmath,amssymb,kotex,paralist,graphicx}
\usepackage{mdframed}
\usepackage{../kswrapfig}
\usepackage{fapapersize}
\usefapapersize{210mm,297mm,20mm,*,20mm,*}
%\pagestyle{empty}
\usepackage{multicol}
\setlength{\columnsep}{30pt}
\setlength{\columnseprule}{1pt}
%\def\columnseprulecolor{\color{blue}}

%%% 객관식 선지

\usepackage{tabto,pifont}
\TabPositions{0.2\textwidth,0.4\textwidth,0.6\textwidth,0.8\textwidth}

\newcommand\one{\ding{172}}
\newcommand\two{\ding{173}}
\newcommand\three{\ding{174}}
\newcommand\four{\ding{175}}
\newcommand\five{\ding{176}}

\newcommand\taba[5]{\par\bigskip\noindent
\one\:{\ensuremath{#1}}
\tab\two\:\:{\ensuremath{#2}}
\tab\three\:\:{\ensuremath{#3}}
\tab\four\:\:{\ensuremath{#4}}
\tab\five\:\:{\ensuremath{#5}}}

\newcommand\tabb[5]{\par\bigskip\noindent
\one\:{\ensuremath{#1}}
\tabto{0.16\textwidth}\two\:\:{\ensuremath{#2}}
\tabto{0.33\textwidth}\three\:\:{\ensuremath{#3}}\medskip\par\noindent
\four\:\:{\ensuremath{#4}}.
\tabto{0.16\textwidth}\five\:\:{\ensuremath{#5}}}

\newcommand\tabc[5]{\par\bigskip\noindent
\one\:{\ensuremath{#1}}
\tabto{0.25\textwidth}\two\:\:{\ensuremath{#2}}\medskip\par\noindent
\three\:\:{\ensuremath{#3}}
\tabto{0.25\textwidth}\four\:\:{\ensuremath{#4}}\medskip\par\noindent
\five\:\:{\ensuremath{#5}}}

\newcommand\tabd[5]{\par\bigskip\noindent
\one\:{#1}\medskip\par\noindent
\two\:\:{#2}\medskip\par\noindent
\three\:\:{#3}\medskip\par\noindent
\four\:\:{#4}\medskip\par\noindent
\five\:\:{#5}}

%%% Counters
\newcounter{num}

%%% Commands
\newcommand{\prob}[1]
{\vs\par\noindent\refstepcounter{num}\textbf{문제 \arabic{num})}\label{#1}\par\noindent}

\newcommand\vs[1]{\vspace{70pt}}

\newcommand\inc[1]{\begin{center}\includegraphics[width=0.95\columnwidth]{#1}\end{center}}

\newcommand\pb[1]{\ensuremath{\fbox{\phantom{#1}}}}

\newcommand\ba{\ensuremath{\:|\:}}

\newcommand\an[2]{\par\bigskip\noindent\textbf{문제 \ref{#1})} #2\\}

\newcommand\ans[1]{\begin{flushright}\textbf{답 : }#1\end{flushright}}

\renewcommand{\arraystretch}{1.5}

%%% Meta Commands
\let\oldsection\section
\renewcommand\section{\clearpage\oldsection}
\let\emph\textsf

%%%%
\begin{document}

\title{수학(하) : 08 집합}
\author{}
\date{\today}
\maketitle
\tableofcontents
\newpage

%%sets
\section{집합과 원소}

%
\exam{}
\begin{enumerate}\label{sets1}
\item
`어떤 조건이나 기준에 의하여 그 대상을 분명히 알 수 있는 것들의 모임'을 \emph{집합}이라고 한다.
또, 집합을 이루는 대상 하나하나를 \emph{원소}라고 한다.
\item
예를 들어 `\(6\)의 약수의 모임', `성북구에 위치한 초등학교의 모임'은 집합이다.
하지만 `\(6\)에 가까운 수들의 모임', `착한 학생들이 다니는 초등학교의 모임'은 집합이 아니다.
\item
원소가 하나도 없는 집합을 \emph{공집합}이라고 하고 기호로는  \(\varnothing\)로 나타낸다.
\item
\(A\)를 `\(6\)의 약수의 모임'이라고 하자.
그러면 \(2\)는 \(A\)의 원소이다.
이것을
\[2\in A\]
으로 표현한다.
반면 \(4\)는 \(A\)의 원소가 아니다.
이것을
\[4\notin A\]
으로 표현한다.
마찬가지로 \(B\)를 `성북구에 위치한 초등학교의 모임'이라고 하면
\[\text{일신초등학교}\in B\]
이고
\[\text{영훈초등학교}\notin B\]
이다.
\item
\(A\)의 원소에는 \(1\), \(2\), \(3\), \(6\)의 네 개가 있다.
이것을
\[A=\{1,2,3,6\}\]
으로 표현한다.
혹은
\[A=\{x\ba x\text{는 6의 약수}\}\]
으로 표현하기도 한다.
\end{enumerate}

%
\begin{mdframed}
\defi{원소나열법, 조건제시법}\label{sets2}
\[A=\{1,2,3,6\}\]
와 같이 표현하는 방법을 `원소나열법'이라고 한다.
\[A=\{x\ba x\text{는 6의 약수}\}\]
와 같이 표현하는 방법을 `조건제시법'이라고 한다.
\end{mdframed}

%
\prob{}
\begin{enumerate}\label{sets3}
\item
\(10\)보다 작은 자연수의 모임		\tabto{0.6\textwidth}(집합이다, 집합이 아니다)
\item
큰 수들의 모임							\tabto{0.6\textwidth}(집합이다, 집합이 아니다)
\item
\(2\)보다 작은 소수의 모임			\tabto{0.6\textwidth}(집합이다, 집합이 아니다)
\item
예쁘게 생긴 꽃들의 모임				\tabto{0.6\textwidth}(집합이다, 집합이 아니다)
\item
키가 \(170\)cm 이상인 학생들의 모임	\tabto{0.6\textwidth}(집합이다, 집합이 아니다)
\end{enumerate}

%
\prob{}\label{sets4}
\(A\)를 `\(10\)보다 작은 소수들의 모임'이라고 할 때, 다음 빈 칸에 \(\in\), \(\notin\) 중 알맞은 기호를 써넣어라.
\\[10pt]
\begin{enumerate*}[itemjoin={,\qquad\qquad}]
\item
\(1\:\:\pb{\in}\:\:A\)
\item
\(3\:\:\pb{\in}\:\:A\)
\item
\(7\:\:\pb{\in}\:\:A\)
\item
\(9\:\:\pb{\in}\:\:A\)
\end{enumerate*}

%
\prob{}\label{sets5}
\(B\)가 `\(18\)의 약수들의 집합'일 때, \(B\)를 원소나열법과 조건제시법으로 나타내어라.

%
\prob{}\label{sets6}
\(C\)가 `\(4\)의 배수들의 모임'일 때, \(C\)를 원소나열법과 조건제시법으로 나타내어라.

%%subset
\section{부분집합}
%
\exam{}
\begin{enumerate}\label{subset1}
\item
\(A=\{1,3,5\}\), \(B=\{1,2,3,4,5\}\)이라고 하자.
이것을 그림\footnotemark으로
\footnotetext{이러한 그림을 벤 다이어그램(Venn Diagram)이라고 부른다.}
\begin{center}
\includegraphics[width=0.27\textwidth]{subset_1}
\end{center}
와 같이 나타낼 수 있다.
이처럼 \(A\)가 \(B\)안에 포함되면
\footnote{정확하게는 ``집합 \(A\)의 모든 원소가 집합 \(B\)에 포함되면''}
\[A\subset B\]
로 나타내고 `\(A\)가 \(B\)의 부분집합이다'라고 말한다.
\item
\(A=\{2,3,4\}\), \(B=\{3,4,5\}\)이면
\begin{center}
\includegraphics[width=0.25\textwidth]{subset_2}
\end{center}
이다.
\(A\)는 \(B\)에 포함되지 않고 \(B\)도 \(A\)에 포함되지 않으므로
\[A\not\subset B,\qquad B\not\subset A\]
이다.
\item
만약 두 집합 \(A\), \(B\)가 \(A\subset B\)이고 \(B\subset A\)이면
\[A=B\]
로 나타내고 `집합 \(A\), \(B\)가 같다'고 말한다.
\item
만약 두 집합 \(A\), \(B\)가 \(A\subset B\)이고 \(A\neq B\)이면
\[A\subsetneq B\]
로 나타내고 `\(A\)가 \(B\)의 \emph{진부분집합}이다'라고 말한다.
\end{enumerate}

%
\prob{}\label{subset2}
\(\subset\), \(=\)을 사용하여 \(A\)와 \(B\) 사이의 포함관계를 나타내어라.

\begin{enumerate}
\item
\(A=\{x\ba x\text{는 3의 약수}\}\), \tabto{0.48\textwidth}\(B=\{x\ba x\text{는 6의 약수}\}\)
\item
\(A=\{x\ba x\text{는 3의 배수}\}\), \tabto{0.48\textwidth}\(B=\{x\ba x\text{는 6의 배수}\}\)
\item
\(A=\{x\ba x^2-4x+3=0\}\), \tabto{0.48\textwidth}\(B=\{1,3\}\)
\end{enumerate}

%
\prob{}\label{subset3}
다음 중 틀린 것을 고르시오.
\tabd
{\(\{2,4\}\)는 \(\{2,4,6\}\)의 부분집합이다.}
{\(\{2,4\}\)는 \(\{2,4,6\}\)의 진부분집합이다.}
{\(\{2,4\}\)는 \(\{2,4\}\)의 부분집합이다.}
{\(\{2,4\}\)는 \(\{2,4\}\)의 진부분집합이다.}
{\(\{2,4\}=\{4,2\}\)이다.}

%
\prob{}\label{subset4}
두 집합 \(A=\{2,3,5,7\}\), \(B=\{4,5,6,7,8\}\)를 벤다이어그램으로 나타내어라.

%
\section{부분집합의 개수}

%
\exam{}\label{ssubset1}
집합 \(B=\{a,b\}\)의 부분집합을 모두 구하고, 그 개수를 말하여라.
\begin{mdframed}
\(\varnothing\subset B\),\quad \(\{a\}\subset B\),\quad \(\{b\}\subset B\),\quad \(\{a,b\}\subset B\)
이므로\\ \(B\)의 부분집합의 개수는 4개이다.
\end{mdframed}
\ans{\(\varnothing\), \(\{a\}\), \(\{b\}\), \(\{a,b\}\); 4개}

%
\prob{}\label{ssubset2}
다음 집합들의 부분집합을 모두 구하고, 그 개수를 말하여라.
\begin{enumerate}
\item
\(A=\{a\}\)
\item
\(C=\{a,b,c\}\)
\item
\(D=\{a,b,c,d\}\)
\end{enumerate}

%
\prob{}\label{ssubset3}
예시 \ref{ssubset1})과 문제 \ref{ssubset2})로부터
다음 집합들의 부분집합의 개수를 유추하여라.
\begin{enumerate}
\item
\(E=\{a,b,c,d,e\}\)
\item
\(F=\{a,b,c,d,e,f\}\)
\end{enumerate}

%
\begin{mdframed}
\theo{}\label{ssubset4}
원소의 개수가 \(k\)개인 집합의 부분집합의 개수는 \pb{2^k}개이다.
\end{mdframed}

%
\prob{}\label{ssubset5}
\(P=\{1,3,5,7\}\)일 때, \(P\)의 부분집합의 개수를 구하여라.

\newpage
%
\exam{집합 \(C=\{a,b,c\}\)에 대하여 다음 물음에 답하여라.}
\begin{enumerate}\label{ssubset6}
\item
\(a\)를 원소로 가지지 않는 부분집합의 개수를 구하여라.
\item
\(a\)를 원소로 가지는 부분집합의 개수를 구하여라.
\end{enumerate}

\begin{mdframed}[skipabove=0pt]
\(C\)의 부분집합은
\[\varnothing,\:\{a\},\:\{b\},\:\{c\},\:\{a,b\},\:\{a,c\},\:\{b,c\},\:\{a,b,c\}\]
의 \(8\)개가 있다.
이중 \(a\)를 원소로 가지지 않는 부분집합은
\[\varnothing,\:\{b\},\:\{c\},\:\{b,c\}\]
의 4개이고, \(a\)를 를 원소로 가지는 부분집합은
\[\{a\},\:\{a,b\},\:\{a,c\},\:\{a,b,c\}\]
의 4개이다.
\end{mdframed}
\ans{(1) \(4\)개,\qquad (2) \(4\)개}

%
\prob{}\label{ssubset7}
집합 \(D=\{a,b,c,d\}\)에 대하여 다음 물음에 답하여라.
\begin{enumerate}
\item
\(a\)를 원소로 가지지 않는 부분집합의 개수를 구하여라.
\item
\(a\)를 원소로 가지는 부분집합의 개수를 구하여라.
\item
\(a,b\)를 원소로 가지지 않는 부분집합의 개수를 구하여라.
\item
\(a,b\)를 원소로 가지는 부분집합의 개수를 구하여라.
\end{enumerate}

%
\begin{mdframed}
\theo{\normalfont{원소의 개수가 \(k\)개인 집합의 부분집합 중}}
\begin{enumerate}\label{ssubset8}
\item
\(m\)개를 원소로 가지지 않는 것의 개수는 \pb{2^{k-m}}개이다.
\item
\(m\)개를 원소로 가지는 것의 개수는 \pb{2^{k-m}}개이다.
\end{enumerate}
\end{mdframed}


%%opreations
\section{집합의 연산}
%
\exam{합집합, 교집합, 차집합}
\begin{minipage}{0.65\textwidth}\label{operations1}
(1)\:\(A=\{1,2,3,4\}\), \(B=\{4,5,6\}\)일 때, 이것을 벤다이어그램으로 표현하면 오른쪽 그림과 같다.
\end{minipage}
\begin{minipage}{0.3\textwidth}
\begin{center}
\includegraphics[width=\textwidth]{operations_1-1}
\end{center}
\end{minipage}

\noindent
\begin{minipage}{0.65\textwidth}
(2)\:\(A\)와 \(B\)를 합친 부분을
\(A\)와 \(B\)의 \emph{합집합}이라고 부르고 기호로 \(A\cup B\)로 표현한다.
따라서
\[A\cup B=\{1,2,3,4,5,6\}\]
\end{minipage}
\begin{minipage}{0.3\textwidth}
\begin{center}
\includegraphics[width=\textwidth]{operations_1-2}
\end{center}
\end{minipage}

\noindent
\begin{minipage}{0.65\textwidth}
(3)\:\(A\)와 \(B\)를 겹치는 부분을
\(A\)와 \(B\)의 \emph{교집합}이라고 부르고 기호로 \(A\cap B\)로 표현한다.
따라서
\[A\cap B=\{4\}\]
\end{minipage}
\begin{minipage}{0.3\textwidth}
\begin{center}
\includegraphics[width=\textwidth]{operations_1-3}
\end{center}
\end{minipage}

\noindent
\begin{minipage}{0.65\textwidth}
(4)\:\(A\)에만 해당되고 \(B\)에는 해당되지 않는 부분을
\(A\)에 대한 \(B\)의 \emph{차집합}이라고 부르고 기호로 \(A-B\)로 표현한다.
따라서
\[A-B=\{1,2,3\}\]
또한 반대로 생각하면
\[B-A=\{5,6\}\]
이다.
\end{minipage}
\begin{minipage}{0.3\textwidth}
\begin{center}
\includegraphics[width=\textwidth]{operations_1-4}\\
\includegraphics[width=\textwidth]{operations_1-5}\\
\end{center}
\end{minipage}

\newpage
%
\begin{mdframed}
\defi{합집합, 교집합, 차집합}\label{operations2}
두 집합 \(A\), \(B\)에 대하여
\begin{align*}
A\cup B&=\{x\ba x\in A\text{ 또는 }x\in B\}\\
A\cap B&=\{x\ba x\in A\text{ 그리고 }x\in B\}\\
A-B&=\{x\ba x\in A\text{ 그리고 }x\notin B\}
\end{align*}
\end{mdframed}

%
\prob{}\label{operations3}
다음 두 집합 \(A\), \(B\)에 대해 \(A\cup B\), \(A\cap B\), \(A-B\), \(B-A\)를 구하여라.
\begin{enumerate}
\item
\(A=\{1,3,5,7,9\}\), \(B=\{2,3,5,7\}\)
\item
\(A=\{x\ba x\text{는 }6\text{의 약수}\}\), \(B=\{x\ba x\text{는 }12\text{의 약수}\}\)
\end{enumerate}

%
\prob{}\label{operations4}
세 집합 \(A=\{1,4,7,10\}\), \(B=\{2,4,6,8,10\}\), \(C=\{4,5,7,9\}\)에 대하여 다음을 차례대로 구하여라.
\par\medskip\noindent
\begin{enumerate*}[itemjoin={\tabto{0.5\textwidth}}]
\item
\(B\cup C\)
\item
\(A\cup(B\cup C)\)
\end{enumerate*}

\begin{mdframed}
%
\defi{서로소}
두 집합 \(A\)와 \(B\)에 대해, \(A\)와 \(B\)가 공통된 원소를 가지고 있지 않으면, 즉
\[A\cap B=\varnothing\]
이면, `\(A\), \(B\)가 \textbf{서로소}이다'라고 한다.
\end{mdframed}

%
\prob{}
세 집합
\[A=\{5,10\},\quad B=\{1,3,5,7,9\},\quad C=\{1,4,7\}\]
중에서 서로소인 두 집합을 찾아라.

\newpage
%
\exam{여집합}\label{operations5}
\begin{minipage}{0.65\textwidth}
(1)\:전체집합 \(U=\{x\ba x\text{는 \(10\) 이하의 자연수}\}\)와\\ \(U\)의 부분집합 \(A\)를 
\[A=\{3,6,9\}\]
라고 하자.
\end{minipage}
\begin{minipage}{0.3\textwidth}
\begin{center}
\includegraphics[width=\textwidth]{operations_5-1}
\end{center}
\end{minipage}

\noindent
\begin{minipage}{0.65\textwidth}
(2)\:이때, \(A\)의 바깥쪽에 있는 부분을 \(A\)의 \emph{여집합}이라고 부르고 기호로 \(A^c\)로 표현한다.
따라서
\[A^c=\{1,2,4,5,7,8,10\}\]
\end{minipage}
\begin{minipage}{0.3\textwidth}
\begin{center}
\includegraphics[width=\textwidth]{operations_5-2}
\end{center}
\end{minipage}

%
\begin{mdframed}
\defi{여집합}\label{operations6}
전체집합 \(U\)의 부분집합 \(A\)에 대하여
\[A^c=\{x\ba x\in U\text{ 그리고 }x\notin A\}\]
이다.
즉 \(A^c=U-A\)이다.
\end{mdframed}

%
\exam{}\label{operations7}
\begin{minipage}{0.65\textwidth}
(1)\:한편, \(B\)를 \(B=\{2,4,6,8,10\}\)이라고 하면
\[B^c=\{1,3,5,7,9\}\]
이다.
\end{minipage}
\begin{minipage}{0.3\textwidth}
\begin{center}
\includegraphics[width=\textwidth]{operations_5-3}
\end{center}
\end{minipage}

\noindent
\begin{minipage}{0.65\textwidth}
(2)\:따라서 \(A\cap B^c\)는
\[A\cap B^c=\{3,9\}\]
이다.
즉
\[A-B=A\cap B^c\]
이 성립한다.
\end{minipage}
\begin{minipage}{0.3\textwidth}
\begin{center}
\includegraphics[width=\textwidth]{operations_5-4}
\end{center}
\end{minipage}

%
\begin{mdframed}
\theo{차집합의 성질}\label{operations8}
두 집합 \(A\), \(B\)에 대하여
\[A-B=A\cap B^c\]
이다.
\end{mdframed}

%
\prob{}\label{operations9}
전체집합 \(U=\{x\ba x\text{는 7 이하의 자연수}\}\)의 두 부분집합
\[A=\{1,2,3,4,5\},\quad B=\{1,4,7\}\]
에 대하여 \(A^c\), \(B^c\), \(A\cap B^c\)를 구하여라.

%%
%\prob{}\label{operations10}
%전체집합 \(U=\{x\ba x\text{는 10 이하의 자연수}\}\)의 두 부분집합
%\[A=\{x\ba x\text{는 10 이하의 짝수}\},\quad B=\{x\ba x\text{는 10 이하의 소수}\}\]
%에 대하여 \(A\cup B\), \((A\cup B)^c\), \(A\cap B\), \((A\cap B)^c\)를 구하여라.

%
\prob{집합 \(A\)에 대하여 다음 중 틀린 것을 고르시오.}\label{operations11}
\par\noindent
\begin{minipage}{0.65\textwidth}
\tabd
{\((A^c)^c=A\)}
{\(\varnothing^c=U\)}
{\(A\cap\varnothing=\varnothing\)}
{\(A\cup U=\varnothing\)}
{\(A\)와 \(A^c\)는 서로소이다.}
\end{minipage}
\begin{minipage}{0.3\textwidth}
\begin{center}
\includegraphics[width=\textwidth]{operations_11}
\end{center}
\end{minipage}

%%properties
\section{집합의 연산법칙}

%
\begin{mdframed}
\theo{집합의 연산법칙}
\begin{enumerate}\label{properties1}
\item
\(A\cup B=B\cup A\)
\tabto{0.67\textwidth}[교환법칙]\\
\(A\cap B=B\cap A\)
\item
\((A\cup B)\cup C=A\cup(B\cup C)\)
\tabto{0.67\textwidth}[결합법칙]\\
\((A\cap B)\cap C=A\cap(B\cap C)\)
\item
\(A\cup(B\cap C)=(A\cup B)\cap(A\cup C)\)
\tabto{0.67\textwidth}[분배법칙]\\
\(A\cap(B\cup C)=(A\cap B)\cup(A\cap C)\)
\item
\(A-B=A\cap B^c\)
\tabto{0.67\textwidth}[차집합의 성질]
\item
\((A\cup B)^c=A^c\cap B^c\)
\tabto{0.67\textwidth}[드 모르간의 법칙]\\
\((A\cap B)^c=A^c\cup B^c\)
\end{enumerate}
\end{mdframed}

%
\exam{교환법칙}
\begin{minipage}{0.65\textwidth}\label{properties2}
(1)\:\(A=\{2,4,5,7,9\}\), \(B=\{5,6,7,8\}\)이라고 하면
\begin{align*}
A\cup B&=\{2,4,5,6,7,8,9\}\\
B\cup A&=\{2,4,5,6,7,8,9\}
\end{align*}
이다.
즉
\[A\cup B=B\cup A\]
\end{minipage}
\begin{minipage}{0.3\textwidth}
\begin{center}
\includegraphics[width=\textwidth]{properties_2-1}
\end{center}
\end{minipage}

\bigskip\bigskip\noindent
\begin{minipage}{0.65\textwidth}
(2)\:또한
\begin{align*}
A\cap B&=\{5,7\}\\
B\cap A&=\{5,7\}
\end{align*}
이다.
즉
\[A\cap B=B\cap A\]
\end{minipage}
\begin{minipage}{0.3\textwidth}
\begin{center}
\includegraphics[width=\textwidth]{properties_2-2}
\end{center}
\end{minipage}

\newpage
%
\prob{결합법칙}
\begin{enumerate}\label{properties3}
\item
\((A\cup B)\cup C=A\cup(B\cup C)\)
\begin{mdframed}
좌변과 우변을 각각 벤 다이어그램으로 표현하면
\par
\includegraphics[width=.9\textwidth]{properties_3-1}
\par\vspace{-10pt}
\(\qquad\:A\cup B\qquad\qquad\qquad\quad\:\:C
\qquad\qquad\qquad\:\:(A\cup B)\cup C\)
\par
\includegraphics[width=.9\textwidth]{properties_3-2}
\par\vspace{-10pt}
\(\qquad\quad A\qquad\qquad\qquad\quad\:\:B\cup C
\qquad\qquad\quad\:\: A\cup (B\cup C)\)
\par
이다.
따라서 좌변과 우변이 같다.
\end{mdframed}
\item
\((A\cap B)\cap C=A\cap(B\cap C)\)
\begin{mdframed}
좌변과 우변을 각각 벤 다이어그램으로 표현하면
\par
\includegraphics[width=.9\textwidth]{three_set_rule_cap}
\par\vspace{-10pt}
\(\qquad\:A\cap B\qquad\qquad\qquad\quad\:\:C
\qquad\qquad\qquad\:\:(A\cap B)\cap C\)
\par
\includegraphics[width=.9\textwidth]{three_set_rule_cap}
\par\vspace{-10pt}
\(\qquad\quad A\qquad\qquad\qquad\quad\:\:B\cap C
\qquad\qquad\quad\:\: A\cap (B\cap C)\)
\par
이다.
따라서 좌변과 우변이 같다.
\end{mdframed}
\end{enumerate}
\newpage
%
\prob{분배법칙}
\begin{enumerate}\label{properties4}
\item
\(A\cup(B\cap C)=(A\cup B)\cap(A\cup C)\)
\begin{mdframed}
좌변과 우변을 각각 벤 다이어그램으로 표현하면
\par
\includegraphics[width=.9\textwidth]{properties_4-1}
\par\vspace{-10pt}
\(\qquad\quad\:A\qquad\qquad\qquad\quad\:\:B\cap C
\qquad\qquad\quad\:A\cup(B\cap C)\)
\par
\includegraphics[width=.9\textwidth]{properties_4-2}
\par\vspace{-10pt}
\(\qquad A\cup B\qquad\qquad\qquad\:\:A\cup C
\qquad\qquad(A\cup B)\cap(A\cup C)\)
\par
이다.
따라서 좌변과 우변이 같다.
\end{mdframed}
\item
\(A\cap(B\cup C)=(A\cap B)\cup(A\cap C)\)
\begin{mdframed}
좌변과 우변을 각각 벤 다이어그램으로 표현하면
\par
\includegraphics[width=.9\textwidth]{three_set_rule_cap}
\par\vspace{-10pt}
\(\qquad\quad\:A\qquad\qquad\qquad\quad\:\:B\cup C
\qquad\qquad\quad\:A\cap(B\cup C)\)
\par
\includegraphics[width=.9\textwidth]{three_set_rule_cup}
\par\vspace{-10pt}
\(\qquad A\cap B\qquad\qquad\qquad\:\:A\cap C
\qquad\qquad(A\cap B)\cup(A\cap C)\)
\par
이다.
따라서 좌변과 우변이 같다.
\end{mdframed}
\end{enumerate}

\clearpage
%
\prob{차집합에 대한 교환법칙과 결합법칙은 성립하는지 확인하여라.}
교환법칙\::\:\(A-B\stackrel?=B-A\)\\\label{properties5}
결합법칙\::\:\(A-(B-C)\stackrel?=(A-B)-C\)
\begin{mdframed}
교환법칙 :
\par\vspace{-20pt}
\begin{center}
\includegraphics[width=.5\textwidth]{two_set_equality}
\par\(A-B\qquad\qquad\qquad\:\:B-A\)
\end{center}
\par\noindent
결합법칙 :
\par
\includegraphics[width=.9\textwidth]{properties_5}
\par\vspace{-10pt}
\(\qquad\quad\:A\qquad\qquad\qquad\qquad B-C
\qquad\qquad\qquad\:A-(B-C)\)
\par
\includegraphics[width=.9\textwidth]{three_set_rule_setminus}
\par\vspace{-10pt}
\(\qquad A-B\qquad\qquad\qquad\qquad\:\: C
\qquad\qquad\qquad\quad(A-B)-C\)
\par
\end{mdframed}
\ans{
교환법칙이 (성립한다 / 성립하지 않는다.)\\
결합법칙이 (성립한다 / 성립하지 않는다.)
}
\newpage

%\vspace{-10pt}
%%
%\prob{차집합의 성질 : \(A\cap B^c=A-B\)}\label{properties6}
%\begin{mdframed}[skipabove=0pt,innertopmargin=5pt]
%좌변을 벤 다이어그램으로 표현하면
%\par
%\includegraphics[width=.9\textwidth]{two_set_rule-cap}
%\par\vspace{-10pt}
%\(\qquad\qquad A\qquad\qquad\qquad\qquad\quad\:B^c
%\qquad\qquad\qquad\quad\:\:A\cap B^c\)
%\par\noindent
%이다.
%따라서 우변인 \(A-B\)와 같다.
%\end{mdframed}

%
\prob{드 모르간의 법칙}
\begin{enumerate}\label{properties7}
\item
\((A\cup B)^c=A^c\cap B^c\)
\begin{mdframed}
좌변과 우변을 각각 벤 다이어그램으로 표현하면
\par
\includegraphics[width=.9\textwidth]{properties_7-1}
\par\vspace{-10pt}
\(\qquad\qquad\:\: A\cup B\qquad\qquad\qquad\qquad\qquad\quad\:\:(A\cup B)^c\)
\par
\includegraphics[width=.9\textwidth]{properties_7-2}
\par\vspace{-10pt}
\(\qquad\quad\:\:A^c\qquad\qquad\qquad\qquad B^c
\qquad\qquad\qquad A^c\cap B^c\)
\par
이다.
따라서 좌변과 우변이 같다.
\end{mdframed}
\item
\((A\cap B)^c=A^c\cup B^c\)
\begin{mdframed}
좌변과 우변을 각각 벤 다이어그램으로 표현하면
\par
\includegraphics[width=.9\textwidth]{two_set_rule-complement}
\par\vspace{-10pt}
\(\qquad\qquad\:\: A\cap B\qquad\qquad\qquad\qquad\qquad\quad\:\:(A\cap B)^c\)
\par
\includegraphics[width=.9\textwidth]{two_set_rule-cup}
\par\vspace{-10pt}
\(\qquad\quad\:\:A^c\qquad\qquad\qquad\qquad B^c
\qquad\qquad\qquad A^c\cup B^c\)
\par
이다.
따라서 좌변과 우변이 같다.
\end{mdframed}
\end{enumerate}

%%cardinal
\section{유한집합의 원소의 개수}

%
\begin{mdframed}
\defi{}\label{cardinal1}
집합\(A\) 의 원소의 개수는 \(n(A)\)로 나타낸다.
\end{mdframed}

%
\exam{}
\begin{enumerate}\label{cardinal2}
\item
\(n(\varnothing)=0\)이다.
\item
\(A=\{1,2,3\}\)이면 \(n(A)=3\)이다.
\item
\(B=\{x\ba x\text{는 15 이하의 짝수}\}\)이면 \(n(B)=7\)이다.
\end{enumerate}

\begin{mdframed}
%
\theo{\(n(A\cup B)=n(A)+n(B)-n(A\cap B)\)}\label{cardinal3}
\begin{center}
\includegraphics[width=0.9\textwidth]{cardinal_3-2}
\end{center}
\end{mdframed}

%
\exam{}\label{cardinal4}
\(n(A)=7,\:\:n(B)=4,\:\:n(A\cap B)=3\)일 때, \(n(A\cup B)\)의 값을 구하여라.
\begin{mdframed}
\(n(A\cup B)=n(A)+n(B)-n(A\cap B)=7+4-3=8\)
\end{mdframed}
\ans{\(8\)}

%
\prob{}\label{cardinal5}
\(n(A)=5,\:\:n(B)=11,\:\:n(A\cup B)=13\)일 때, \(n(A\cap B)\)의 값을 구하여라.

%
\prob{}\label{cardinal6}
\(A\cap B=\varnothing,\:\:n(A)=10,\:\:n(A\cup B)=27\)일 때, \(n(B)\)의 값을 구하여라.

%%
\section*{답}
\addcontentsline{toc}{chapter}{\protect\numberline{*}답}
\begin{multicols*}{2}

%
\an{sets3}
\begin{enumerate}
\item
집합이다.
\item
집합이 아니다.
\item
집합이다.
\item
집합이 아니다.
\item
집합이다.
\end{enumerate}

%
\an{sets4}
\begin{enumerate*}[itemjoin={,\quad}]
\item
\(\notin\)
\item
\(\in\)
\item
\(\in\)
\item
\(\notin\)
\end{enumerate*}

%
\an{sets5}
\begin{itemize}
\item
원소나열법\\
\(A=\{1,2,3,6,9,18\}\)
\item
조건제시법\\
\(A=\{x\ba x\text{는 \(18\)의 약수}\}\)
\end{itemize}

%
\an{sets6}
\begin{itemize}
\item
원소나열법\\
\(A=\{4,8,12,16,\cdots\}\)
\item
조건제시법\\
\(A=\{x\ba x\text{는 \(4\)의 배수}\}\)\\
\(A=\{4k\ba\text{k는 자연수}\}\)
\end{itemize}

%
\an{subset2}
\begin{enumerate*}[itemjoin={,\:\:}]
\item
\(A\subset B\)
\item
\(A\supset B\)
\item
\(A=B\)
\end{enumerate*}

%
\ann{subset3}\four

\columnbreak

%
\an{subset4}
\begin{center}
\includegraphics[width=0.2\textwidth]{subset_4}
\end{center}

%
\an{ssubset2}
\begin{enumerate}
\item
\(2\)개\\
\(\varnothing\), \(\{a\}\)
\item
\(8\)개\\
\(\varnothing\), \(\{a\}\), \(\{b\}\), \(\{c\}\),
\(\{a,b\}\), \(\{a,c\}\), \(\{b,c\}\), \(\{a,b,c\}\)
\item
\(16\)개\\
{\raggedright
\(\varnothing\), \(\{a\}\), \(\{b\}\), \(\{c\}\), \(\{d\}\),
\(\{a,b\}\), \(\{a,c\}\), \(\{a,d\}\),  \(\{b,c\}\),  \(\{b,d\}\), \(\{c,d\}\),
\(\{a,b,c\}\), \(\{a,b,d\}\), \(\{a,c,d\}\), \(\{b,c,d\}\), \(\{a,b,c,d\}\)}
\end{enumerate}

%
\an{ssubset3}
\begin{enumerate}
\item
\(32\)개
\item
\(64\)개
\end{enumerate}

%
\par\bigskip\noindent\textbf{정리 \ref{ssubset4})}\:\:\(2^k\)\par\medskip\noindent

%
\ann{ssubset5}{\(16\)개}

%
\an{ssubset7}
\begin{enumerate*}[itemjoin={,\:}]
\item \(8\)개
\item \(8\)개
\item \(4\)개
\item \(4\)개
\end{enumerate*}

%
\par\bigskip\noindent\textbf{정리 \ref{ssubset8})}\:\:\(2^{k-m}\)\par\medskip\noindent

%
\an{operations3}
\begin{enumerate}
\item
\(
A\cup B=\{1,2,3,5,7,9\}\\
A\cap B=\{3,5,7\}\\
A-B=\{1,9\}\\
B-A=\{2\}
\)
\item
\(
A\cup B=\{1,2,3,4,6,12\}\\
A\cap B=\{1,2,3,6\}\\
A-B=\varnothing\\
B-A=\{4,12\}
\)
\end{enumerate}

%
\an{operations4}
\begin{enumerate}
\item
\(\{2,4,5,6,7,8,9,10\}\)
\item
\(\{1,2,4,5,6,7,8,9,10\}\)
\end{enumerate}

%
\ann{operations6}{\(A\), \(C\)}

%
\an{operations9}
\(
A^c=\{6,7\}\\
B^c=\{2,3,5,6\}\\
A\cap B^c=\{2,3,5\}
\)

%%
%\an{operations10}
%\(
%A\cup B=\{2,3,4,5,6,7,8,10\}\\
%(A\cup B)^c=\{1,9\}\\
%A\cap B=\{2\}\\
%(A\cap B)^c=\{1,3,4,5,6,7,8,9,10\}
%\)

%
\ann{operations11}{\four}

%
\ann{properties3}{생략}

%
\ann{properties4}{생략}

%
\ann{properties5}{생략}
성립하지 않는다, 성립하지 않는다.

%%
%\ann{properties6}{생략}

%
\ann{properties7}{생략}

%
\ann{cardinal5}3

%
\ann{cardinal6}{17}

\end{multicols*}


%%
\section*{요약}
\addcontentsline{toc}{chapter}{\protect\numberline{*}요약}
\begin{enumerate}[label=\arabic*.,itemsep=15pt]
\item
집합과 원소
\begin{itemize}
\item
원소나열법\::\:\(A=\{1,3,5,7,9,\cdots\}\)
\item
조건제시법\::\:\(A=\{x\ba x\text{는 홀수}\}=\{2k-1\ba\text{k는 자연수}\}\)
\end{itemize}
\item
부분집합\\
\(A=\{2,5\}\)이면\quad
\(\underline{2\in A,\:\:5\in A},\quad
\underline{\emptyset\subset A,\:\:\{2\}\subset A,\:\:\{3\}\subset A,\:\:\{2,5\}\subset A}\).
\item
부분집합의 개수
\[n(A)=k\text{이면, \(A\)의 부분집합의 개수는 \fbox{\(2^k\)}개}\]
\item
집합의 연산
\begin{figure}[h]
\centering
        \begin{subfigure}{0.2\textwidth}
                \includegraphics[width=\linewidth]{summary_4-1}
                \caption*{\(A\cup B\)}
        \end{subfigure}%
\quad
        \begin{subfigure}{0.2\textwidth}
                \includegraphics[width=\linewidth]{summary_4-2}
                \caption*{\(A\cap B\)}
        \end{subfigure}%
\quad
        \begin{subfigure}{0.2\textwidth}
                \includegraphics[width=\linewidth]{summary_4-3}
                \caption*{\(A-B\)}
        \end{subfigure}%
\quad
        \begin{subfigure}{0.2\textwidth}
                \includegraphics[width=\linewidth]{summary_4-4}
                \caption*{\(A^c\)}
        \end{subfigure}
\end{figure}
\item
집합의 연산법칙
\begin{enumerate}
\item
\(A\cup B=B\cup A,\quad A\cap B=B\cap A\)
\tabto{.6\textwidth}[교환법칙]
\item
\((A\cup B)\cup C=A\cup(B\cup C)\)
\tabto{.6\textwidth}[결합법칙]\\
\((A\cap B)\cap C=A\cap(B\cap C)\)
\item
\(A\cup(B\cap C)=(A\cup B)\cap(A\cup C)\)
\tabto{.6\textwidth}[분배법칙]\\
\(A\cap(B\cup C)=(A\cap B)\cup(A\cap C)\)
\item
\(A-B=A\cap B^c\)
\tabto{.6\textwidth}[차집합의 성질]
\item
\((A\cup B)^c=A^c\cap B^c\)
\tabto{.6\textwidth}[드 모르간의 법칙]\\
\((A\cap B)^c=A^c\cup B^c\)
\end{enumerate}
\item
유한집합의 원소의 개수
\[n(A\cup B)=n(A)+n(B)-n(A\cap B)\]
\end{enumerate}
\end{document}