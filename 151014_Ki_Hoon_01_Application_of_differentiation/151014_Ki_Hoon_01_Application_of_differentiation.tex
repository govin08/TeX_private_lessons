\documentclass{article}
\usepackage{amsmath,amssymb,amsthm,kotex,mdframed,paralist}

\newcounter{num}
\newcommand{\defi}[1]
{\bigskip\noindent\refstepcounter{num}\textbf{정의 \arabic{num}) #1}\par}
\newcommand{\theo}[1]
{\bigskip\noindent\refstepcounter{num}\textbf{정리 \arabic{num}) #1}\par}
\newcommand{\rema}[1]
{\bigskip\noindent\refstepcounter{num}\textbf{참고 \arabic{num}) #1}\par}
\newcommand{\ctex}[1]
{\bigskip\noindent\refstepcounter{num}\textbf{반례 \arabic{num}) #1}\par}

\newcommand{\bb}{
\begin{mdframed}[skipabove=10pt,skipbelow=10pt, leftmargin=70pt, rightmargin=70pt]}
\newcommand{\eb}{\end{mdframed}}

\renewcommand{\proofname}{증명)}

%%%
\begin{document}

\title{기훈 : 01 도함수의 활용}
\author{}
\date{\today}
\maketitle
%\tableofcontents
\newpage

%%
\section{증가함수와 도함수}

%
\defi{증가함수}\label{increasing}
임의의 \(x_1,x_2\in I\)에 대해
\[x_1<x_2\Rightarrow f(x_1)<f(x_2)\]
이면, \(f(x)\)가 구간 \(I\)에서 \textbf{증가}한다고 말한다.
%\begin{align*}
%&f(x)\text{가 구간 }I\text{에서 \textbf{증가}한다.}\\
%\iff&\text{임의의 }x_1,x_2\in I\text{에 대해, }x_1<x_2\text{이면 }f(x_1)<f(x_2)\text{이다.}
%\end{align*}

%
\theo{}\label{increasing1}
\(f(x)\)가 열린 구간 \((a,b)\)에서 미분가능하고, \(f(x)\)가 구간 \((a,b)\)에서 \(f'(x)>0\)이면 \(f(x)\)는 \((a,b)\)에서 증가함수이다.
\bb
\[f'>0\:\:\Rightarrow\:\:\text{증가}\]
\eb

\begin{proof}
\(x_1,x_2\in(a,b)\), \(x_1<x_2\)를 가정하자.
평균값의 정리에 의해
\[
f'(c)=\frac{f(x_2)-f(x_1)}{x_2-x_1}
\]
인 \(c\)가 존재한다(\(x_1<c<x_2\)).
\(c\in(a,b)\)이므로 \(f'(c)>0\)이고, 따라서 \(f(x_2)-f(x_1)>0\)이다.

즉 \(f(x)\)는 \((a,b)\)에서 증가한다.
\end{proof}

%
\ctex{}
\bb
\[f'\ge0\:\:\not\Rightarrow\:\:\text{증가}\]
\eb

\(f(x)\)가 상수함수이면 \(f'(x)\ge0\)이지만 증가함수는 아니다.

\newpage

%
\theo{}\label{increasing2}
\(f(x)\)가 \((a,b)\)에서 미분가능하고 \((a,b)\)에서 증가하면 \((a,b)\)에서 \(f'(x)\ge0\)이다.
\bb
\[\text{증가}\:\:\Rightarrow\:\: f'\ge0\]
\eb

\begin{proof}
\(x\in(a,b)\)를 가정하고 \(f'(x)\)를 살펴보면
\[f'(x)=\lim_{t\to x}\frac{f(t)-f(x)}{t-x}\]
에서 \(f(x)\)가 증가함수이므로
\[
\begin{cases}
t>x\text{이면 }f(t)>f(x)\\
t<x\text{이면 }f(t)<f(x)
\end{cases}
\]
이다.
따라서
\[\frac{f(t)-f(x)}{t-x}>0\]
그러므로 \(f'(x)\ge0\).
\end{proof}

%
\ctex{}
\bb
\[\text{증가}\:\:\not\Rightarrow\:\: f'>0\]
\eb
\(f(x)=x^3\)이면 \(f(x)\)는 증가함수이지만 \(f'(0)\not>0\)이다.

\newpage
%%
\section{극소와 도함수}

%
\defi{극소}\label{critical}
\[t\in(p,q)\Rightarrow f(a)\le f(t)\]
를 만족하는 실수 \(p\), \(q\)가 존재하면(\(p<a<q\)),
\(f(x)\)가 \(x=a\)에서 \textbf{극소}이다.

%
\theo{}\label{critical1}
미분가능한 함수 \(f(x)\)에 대해 \(f'(a)=0\)이고, \(x=a\) 좌우에서 \(f'(x)\)의 부호가 음에서 양으로 바뀌면 \(f(x)\)는 \(x=a\)에서 극소이다.

\begin{proof}
가정에 의해
\[
\begin{cases}
x<a\text{ 이면 }f'(x)<0\\
x>a\text{ 이면 }f'(x)>0
\end{cases}
\]
이다.
조금 더 정확히 쓰면
\[
\begin{cases}
p<x<a\text{ 이면 }f'(x)<0\\
a<x<q\text{ 이면 }f'(x)>0
\end{cases}
\]
인 실수 \(p\), \(q\)가 존재한다(\(p<a<q\)).

\(t\in(p,a)\)라고 하자(\(p<t<a\)).
평균값의 정리에 의해
\[f'(c)=\frac{f(t)-f(a)}{t-a}\]
인 \(c\)가 존재한다(\(t<c<a\)).
\(f'(c)<0\)이므로 \(f(t)-f(a)>0\)이다.
즉 \(f(t)>f(a)\)이다.
정리하면,
\[t\in(p,a)\text{ 이면, }f(a)<f(t)\text{ 이다.}\]
마찬가지 논리에 의해서
\[t\in(a,q)\text{ 이면, }f(a)<f(t)\text{ 이다.}\]
이 둘을 종합하면
\[t\in(p,q)\text{ 이면, }f(a)\le f(t)\text{ 이다.}\]
다시 말해, \(f(x)\)는 \(x=a\)에서 극소이다.
\end{proof}

%
\theo{}\label{critical2}
이계도함수를 가지는 함수 \(f(x)\)가 \(f'(a)=0\)이고 \(f''(a)>0\)이면 \(f(x)\)는 \(x=a\)에서 극소이다.
\bb
\[f'(a)=0,\:\:f''(a)>0\:\:\Rightarrow\:\:극소\]
\eb
\begin{proof}
(교과과정 외)
\(f''(a)\)의 좌극한을 살펴보면
\[0<f''(a)=\lim_{h\to0+}\frac{f'(a-h)-f'(a)}{-h}.\]
\(h\)가 충분히 작으면 (어떤 양수 \(\delta_1>0\)가 존재하여 \(0<h<\delta_1\)이면)
\[0<\frac{f'(a-h)-f'(a)}{-h}\]
이다.
따라서 \(f'(a-h)<0\).
즉 \(x=a\) 왼쪽에서 \(f'(x)<0\)이다.
마찬가지의 논리에 의해서 \(x=a\) 오른쪽에서 \(f'(x)>0\)이다.
그러면 정리 \ref{critical1}의 가정을 만족하므로 \(f(x)\)는 \(x=a\)에서 극소이다.
\end{proof}

\newpage

%%
\section{오목성과 볼록성(concavity)}

%
\defi{아래로 볼록}\label{concavity}
구간 \(I\)의 임의의 두 실수 \(x,y\in I\)(\(x<y\))에 대해 \(A=(x,f(x))\), \(B=(y,f(y))\)라고 하자.
임의의 \(x_3\in(x,y)\)에 대해 \(C=(x_3,f(x_3))\)가 선분 \(AB\)의 아래에 있으면 \(f(x)\)를 \(I\)에서 \textbf{아래로 볼록}이라고 한다.

%
\rema{동등한 정의들(교과과정 외)}\label{remark}
다음 다섯 가지 문장들은 모두 \(f(x)\)가 구간 \(I\)에서 아래로 볼록이기 위한 필요충분조건이다.

\begin{enumerate}[(1)]
\item
%\(x\), \(y\) 사이의 임의의 내분점 \(x_3\)는
%\(x_3=tx+(1-t)y\)로 표현될 수 있고(\(0<t<1\)), \(C\)도 비슷하게 표현될 수 있으므로
%위 정의는 다음과 같이 써질 수도 있다.
%
%``
구간 \(I\)의 임의의 두 실수 \(x,y\in I\)(\(x<y\))에 대해 \(0<t<1\)일 때
\[f(tx_1+(1-t)y)<tf(x_1)+(1-t)f(y)\]
가 성립한다.
%하면 \(f(x)\)는 \(I\)에서 아래로 볼록이다.
%''
\item
%같은 의미에서 다음과 같이 표현될 수 있다.
%
%``
구간 \(I\)의 임의의 두 실수 \(x,y\in I\)(\(x<y\))에 대해 \(\alpha+\beta=1\), \(\alpha,\beta>0\)일 때
\[f(\alpha x_1+\beta y)<\alpha f(x_1)+\beta f(y)\]
가 성립한다.
%하면 \(f(x)\)는 \(I\)에서 아래로 볼록이다.
%''
\item
%``
구간 \(I\)의 임의의 세 실수 \(x_1,x_2,x_3\)(\(x_1<x_2<x_3\))에 대해
\[
\frac{f(x_3)-f(x_2)}{x_3-x_2}<\frac{f(x_2)-f(x_1)}{x_2-x_1}
\]
가 성립한다.
%''
\item
%``
구간 \(I\)의 임의의 세 실수 \(x_1,x_2,x_3\)(\(x_1<x_2<x_3\))에 대해
\[
\frac{f(x_3)-f(x_2)}{x_3-x_2}<\frac{f(x_3)-f(x_1)}{x_3-x_1}
\]
가 성립한다.
%''
\item
%``
구간 \(I\)의 임의의 세 실수 \(x_1,x_2,x_3\)(\(x_1<x_2<x_3\))에 대해
\[
\frac{f(x_3)-f(x_1)}{x_3-x_1}<\frac{f(x_2)-f(x_1)}{x_2-x_1}
\]
가 성립한다.
%''
\end{enumerate}

\begin{proof}
(1)과 (2)가 정의 \ref{concavity}와 동치인 것은 당연하다.

또 \(\alpha=\frac{x_3-x_2}{x_3-x_1}\), \(\beta=\frac{x_2-x_1}{x_3-x_1}\), \(x=x_1\), \(y=x_3\)를 대입하면
\begin{align*}
(2)
&\iff f\left(\frac{(x_3-x_2)x_1}{x_3-x_1}+\frac{(x_2-x_1)x_3}{x_3-x_1}\right)
<\frac{x_3-x_2}{x_3-x_1}f(x_1)+\frac{x_2-x_1}{x_3-x_1}f(x_3)\\
&\iff (x_3-x_1)f(x_2)<(x_3-x_2)f(x_1)+(x_2-x_1)f(x_3)\tag{\(\ast\)}\\
&\iff (x_3-x_1)(f(x_2)-f(x_1))<(x_2-x_1)(f(x_3)-f(x_1))\\
&\iff (5).
\end{align*}

또
\begin{align*}
(\ast)
&\iff (x_3-x_2)(f(x_3)-f(x_1))<(x_3-x_1)(f(x_3)-f(x_2))\\
&\iff (4)
\end{align*}
이다.
따라서 \((2)\Leftrightarrow(4)\Leftrightarrow(5)\)이다.

한편 \((4)\wedge(5)\Rightarrow(3)\)이고, 따라서 \((2)\Rightarrow(3)\)이다.

또 (3)을 가정하면
\[\frac{f(x_2)-f(x_1)}{x_2-x_1}\times(x_2-x_1)
+
\frac{f(x_3)-f(x_2)}{x_3-x_2}\times(x_3-x_2)
=
\frac{f(x_3)-f(x_1)}{x_3-x_1}\times(x_3-x_1)\]
에서
\[\frac{f(x_3)-f(x_2)}{x_3-x_2}\times(x_2-x_1)
+
\frac{f(x_3)-f(x_2)}{x_3-x_2}\times(x_3-x_2)
>
\frac{f(x_3)-f(x_1)}{x_3-x_1}\times(x_3-x_1)\]
이고 따라서 (4)가 성립한다.

그러므로 (1) \(\sim\) (5)은 모두 동치이다.
\end{proof}

\newpage

\theo{}\label{concavity1}
\(f(x)\)가 구간 \(I\)에서 \(f''(x)>0\)이면 \(f(x)\)는 \(I\)에서 아래로 볼록이다.

\bb
\[f''>0\:\:\Rightarrow\:\:아래로볼록\]
\eb

\begin{proof}
\(x_1,x_2,x_3\in I\), \(x_1<x_2<x_3\)라고 하자.
평균값의 정리에 의해 
\[\frac{f(x_2)-f(x_1)}{x_2-x_1}=f'(c),\quad\frac{f(x_3)-f(x_2)}{x_3-x_2}=f'(d)\]
인 \(c\), \(d\)가 존재한다(\(x_1<c<x_2<d<x_3\)).
그런데 \(I\)에서 \(f''(x)>0\)이므로 \(f'(x)\)는 증가함수이고, 따라서 \(f'(c)<f'(d)\)이다.
그러므로
\[\frac{f(x_2)-f(x_1)}{x_2-x_1}=f'(c)<f'(d)=\frac{f(x_3)-f(x_2)}{x_3-x_2}\]
이다.
참고 \ref{remark}에 의해 \(f(x)\)는 \(I\)에서 아래로 볼록이다.
\end{proof}

%
\ctex{}
\bb
\[f''\ge0\:\:\not\Rightarrow\:\:아래로볼록\]
\eb
\(f(x)\)가 일차함수이거나 상수함수이면 \(f''=0\)이어서 \(f''\ge0\)이 성립하지만 아래로 볼록은 아니다.

\newpage

%
\theo{}\label{concavity2}
\(f(x)\)가 구간 \(I\)에서 아래로 볼록이고 이계도함수가 존재하면 \(f(x)\)는 구간 \(I\)에서 \(f''(x)\ge0\)이다.
\bb
\[
아래로볼록\:\:\Rightarrow\:\:f''\ge0
\]
\eb
\begin{proof}
\(x_2<x_3\)를 가정하고 \(x_1<x_2<x_3<x_4\)인 \(x_1\)과 \(x_4\)를 생각하자.
\(f(x)\)가 아래로 볼록이므로
\[
\frac{f(x_2)-f(x_1)}{x_2-x_1}
<
\frac{f(x_3)-f(x_2)}{x_3-x_2}
<
\frac{f(x_4)-f(x_4)}{x_4-x_3}
\]
이다.
\(x_1\to x_2\)와 \(x_4\to x_3\)을 하면
\[
\lim_{x_1\to x_2-}\frac{f(x_2)-f(x_1)}{x_2-x_1}
\le
\frac{f(x_3)-f(x_2)}{x_3-x_2}
\le
\lim_{x_4\to x_3+}\frac{f(x_4)-f(x_4)}{x_4-x_3}
\]
이고, 따라서 \(f'(x_2)\le f'(x_3)\) 이다.
즉 \(f'(x)\)는 (거의) 증가함수이다.
따라서 정리 \ref{increasing2}에서 사용한 논리를 똑같이 적용하면 \(f''(x)\ge0\)이다.
\end{proof}

%
\ctex{}
\bb
\[
아래로볼록\:\:\not\Rightarrow\:\:f''>0
\]
\eb
\(f(x)=x^4\)은 실수 전체에서 아래로 볼록이지만 \(f''(0)\not>0\)이다.

\newpage
%%
\section{변곡점}

%
\defi{변곡점}
\(f(x)\)가 \(x=a\)를 기준으로 오목성(볼록성)이 바뀌면 점 \((a,f(a))\)를 \textbf{변곡점}이라고 한다.

%
\theo{}
\(f''(x)\)의 부호가 \(x=a\)를 기준으로 바뀌면 점 \((a,f(a))\)는 변곡점이다.

\begin{proof}
\(x<a\)에서는 \(f''(x)>0\)이었다가 \(x>a\)에서는 \(f''(x)<0\)이다. (혹은 그 반대이다.)
정리 \ref{concavity1}에 의해 \(x<a\)에서는 아래로 볼록이었다가 \(x<a\)에서는 위로 볼록이 된다.
따라서 점 \((a,f(a))\)는 변곡점이다.
\end{proof}
\end{document}