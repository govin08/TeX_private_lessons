\documentclass[a4paper]{oblivoir}
\usepackage{amsmath,amssymb,kotex,mdframed,paralist,tabu}
\usepackage{amsmath,amssymb,kotex,paralist,graphicx}
\usepackage{mdframed}
\usepackage{../kswrapfig}
\usepackage{fapapersize}
\usefapapersize{210mm,297mm,20mm,*,20mm,*}
%\pagestyle{empty}
\usepackage{multicol}
\setlength{\columnsep}{30pt}
\setlength{\columnseprule}{1pt}
%\def\columnseprulecolor{\color{blue}}

%%% 객관식 선지

\usepackage{tabto,pifont}
\TabPositions{0.2\textwidth,0.4\textwidth,0.6\textwidth,0.8\textwidth}

\newcommand\one{\ding{172}}
\newcommand\two{\ding{173}}
\newcommand\three{\ding{174}}
\newcommand\four{\ding{175}}
\newcommand\five{\ding{176}}

\newcommand\taba[5]{\par\bigskip\noindent
\one\:{\ensuremath{#1}}
\tab\two\:\:{\ensuremath{#2}}
\tab\three\:\:{\ensuremath{#3}}
\tab\four\:\:{\ensuremath{#4}}
\tab\five\:\:{\ensuremath{#5}}}

\newcommand\tabb[5]{\par\bigskip\noindent
\one\:{\ensuremath{#1}}
\tabto{0.16\textwidth}\two\:\:{\ensuremath{#2}}
\tabto{0.33\textwidth}\three\:\:{\ensuremath{#3}}\medskip\par\noindent
\four\:\:{\ensuremath{#4}}.
\tabto{0.16\textwidth}\five\:\:{\ensuremath{#5}}}

\newcommand\tabc[5]{\par\bigskip\noindent
\one\:{\ensuremath{#1}}
\tabto{0.25\textwidth}\two\:\:{\ensuremath{#2}}\medskip\par\noindent
\three\:\:{\ensuremath{#3}}
\tabto{0.25\textwidth}\four\:\:{\ensuremath{#4}}\medskip\par\noindent
\five\:\:{\ensuremath{#5}}}

\newcommand\tabd[5]{\par\bigskip\noindent
\one\:{#1}\medskip\par\noindent
\two\:\:{#2}\medskip\par\noindent
\three\:\:{#3}\medskip\par\noindent
\four\:\:{#4}\medskip\par\noindent
\five\:\:{#5}}

%%% Counters
\newcounter{num}

%%% Commands
\newcommand{\prob}[1]
{\vs\par\noindent\refstepcounter{num}\textbf{문제 \arabic{num})}\label{#1}\par\noindent}

\newcommand\vs[1]{\vspace{70pt}}

\newcommand\inc[1]{\begin{center}\includegraphics[width=0.95\columnwidth]{#1}\end{center}}

\newcommand\pb[1]{\ensuremath{\fbox{\phantom{#1}}}}

\newcommand\ba{\ensuremath{\:|\:}}

\newcommand\an[2]{\par\bigskip\noindent\textbf{문제 \ref{#1})} #2\\}

\newcommand\ans[1]{\begin{flushright}\textbf{답 : }#1\end{flushright}}

\renewcommand{\arraystretch}{1.5}

%%% Meta Commands
\let\oldsection\section
\renewcommand\section{\clearpage\oldsection}
\let\emph\textsf

\begin{document}
\begin{center}
\LARGE하람, 미니테스트 1
\end{center}
\begin{center}
날짜 : \today
,\qquad
점수 : \pb{20} / \pb{20}
\end{center}

%
\prob
지연이는 점심식사를 위해 분식집 혹은 냉면집에 가려고 한다.
분식집에서는 떡볶이, 김밥, 순대, 튀김 중 하나를 선택해 먹을 수 있으며, 냉면집에서는 물냉면과 비빔냉면 중 하나를 선택해 먹을 수 있다.
미연이가 먹을 수 있는 점심 메뉴의 개수를 구하여라.
\vs

%
\prob
다연이는 저녁을 먹은 후 카페에 갈 예정이다.
저녁 메뉴로는 파스타와 쌀국수, 김치찌개, 짬뽕 중에 하나를, 카페에서 먹을 음료로는 카페라떼와 아이스 아메리카노, 카라멜 마키아또 중에 하나를 고를 수 있다.
다연이가 고를 수 있는 저녁 메뉴와 음료의 조합으로 가능한 모든 경우의 수를 구하여라.
\vs

%
\prob
서로 다른 두 개의 주사위를 동시에 던질 때, 나오는 두 눈의 수의 합이 3이거나 8인 경우의 수를 구하시오.
\vs

%
\prob
서로 다른 두 개의 주사위를 동시에 던질 때, 나오는 눈의 수의 합이 4의 배수가 되는 경우의 수를 구하여라.
\vs

%
\prob
12의 약수를 모두 구하여라.
\vs

%
\prob
(1) 12의 약수의 개수를 구하여라. (2) 12의 약수들의 합을 구하여라.
\vs

%
\prob
72의 약수를 모두 구하여라.
\vs

%
\prob
(1) 72의 약수의 개수를 구하여라. (2) 72의 약수들의 합을 구하여라.
\vs

\newpage
\setcounter{num}{0}

%
\ans
6

%
\ans
12

%
\ans
7

%
\ans
9

%
\ans
1, 2, 3, 4, 6, 12

%
\ans
(1) 6개, (2) 28

%
\ans
1, 2, 3, 4, 6, 8, 9, 12, 18, 24, 36, 72

%
\ans
(1)12개 , (2) 195


\end{document}