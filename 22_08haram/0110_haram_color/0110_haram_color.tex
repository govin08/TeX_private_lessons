\documentclass[a4paper]{oblivoir}
\usepackage{amsmath,amssymb,kotex,paralist,graphicx}
\usepackage{mdframed}
\usepackage{../kswrapfig}
\usepackage{fapapersize}
\usefapapersize{210mm,297mm,20mm,*,20mm,*}
%\pagestyle{empty}
\usepackage{multicol}
\setlength{\columnsep}{30pt}
\setlength{\columnseprule}{1pt}
%\def\columnseprulecolor{\color{blue}}

%%% 객관식 선지

\usepackage{tabto,pifont}
\TabPositions{0.2\textwidth,0.4\textwidth,0.6\textwidth,0.8\textwidth}

\newcommand\one{\ding{172}}
\newcommand\two{\ding{173}}
\newcommand\three{\ding{174}}
\newcommand\four{\ding{175}}
\newcommand\five{\ding{176}}

\newcommand\taba[5]{\par\bigskip\noindent
\one\:{\ensuremath{#1}}
\tab\two\:\:{\ensuremath{#2}}
\tab\three\:\:{\ensuremath{#3}}
\tab\four\:\:{\ensuremath{#4}}
\tab\five\:\:{\ensuremath{#5}}}

\newcommand\tabb[5]{\par\bigskip\noindent
\one\:{\ensuremath{#1}}
\tabto{0.16\textwidth}\two\:\:{\ensuremath{#2}}
\tabto{0.33\textwidth}\three\:\:{\ensuremath{#3}}\medskip\par\noindent
\four\:\:{\ensuremath{#4}}.
\tabto{0.16\textwidth}\five\:\:{\ensuremath{#5}}}

\newcommand\tabc[5]{\par\bigskip\noindent
\one\:{\ensuremath{#1}}
\tabto{0.25\textwidth}\two\:\:{\ensuremath{#2}}\medskip\par\noindent
\three\:\:{\ensuremath{#3}}
\tabto{0.25\textwidth}\four\:\:{\ensuremath{#4}}\medskip\par\noindent
\five\:\:{\ensuremath{#5}}}

\newcommand\tabd[5]{\par\bigskip\noindent
\one\:{#1}\medskip\par\noindent
\two\:\:{#2}\medskip\par\noindent
\three\:\:{#3}\medskip\par\noindent
\four\:\:{#4}\medskip\par\noindent
\five\:\:{#5}}

%%% Counters
\newcounter{num}

%%% Commands
\newcommand{\prob}[1]
{\vs\par\noindent\refstepcounter{num}\textbf{문제 \arabic{num})}\label{#1}\par\noindent}

\newcommand\vs[1]{\vspace{70pt}}

\newcommand\inc[1]{\begin{center}\includegraphics[width=0.95\columnwidth]{#1}\end{center}}

\newcommand\pb[1]{\ensuremath{\fbox{\phantom{#1}}}}

\newcommand\ba{\ensuremath{\:|\:}}

\newcommand\an[2]{\par\bigskip\noindent\textbf{문제 \ref{#1})} #2\\}

\newcommand\ans[1]{\begin{flushright}\textbf{답 : }#1\end{flushright}}

\renewcommand{\arraystretch}{1.5}

%%% Meta Commands
\let\oldsection\section
\renewcommand\section{\clearpage\oldsection}
\let\emph\textsf

\begin{document}

\begin{center}
미니테스트 3 관련 문제들
\end{center}
%%
\section{경우의 수}
%
\begin{minipage}{.65\textwidth}
\begin{Exercise}
오른쪽 그림의 2개의 영역을 서로 다른 4가지 색 \(a\), \(b\), \(c\), \(d\)로 칠하려고 한다.
같은 색을 중복해서 사용해도 좋으나 인접한 영역은 서로 다른 색으로 칠할 때, 칠하는 방법의 수를 구하시오.
\end{Exercise}
\end{minipage}
\quad
\begin{minipage}{.25\textwidth}
\includegraphics[width=.5\textwidth]{1}
\end{minipage}

\bigskip\noindent
\includegraphics[width=.2\textwidth]{1}\quad
\includegraphics[width=.2\textwidth]{1}\quad
\includegraphics[width=.2\textwidth]{1}\quad
\includegraphics[width=.2\textwidth]{1}\\[5pt]
\includegraphics[width=.2\textwidth]{1}\quad
\includegraphics[width=.2\textwidth]{1}\quad
\includegraphics[width=.2\textwidth]{1}\quad
\includegraphics[width=.2\textwidth]{1}\\[5pt]
\includegraphics[width=.2\textwidth]{1}\quad
\includegraphics[width=.2\textwidth]{1}\quad
\includegraphics[width=.2\textwidth]{1}\quad
\includegraphics[width=.2\textwidth]{1}\\[5pt]
\includegraphics[width=.2\textwidth]{1}\quad
\includegraphics[width=.2\textwidth]{1}\quad
\includegraphics[width=.2\textwidth]{1}\quad
\includegraphics[width=.2\textwidth]{1}

\ans{12}

\noindent
%
\begin{minipage}{.65\textwidth}
\begin{Exercise}
오른쪽 그림의 3개의 영역을 서로 다른 4가지 색 \(a\), \(b\), \(c\), \(d\)로 칠하려고 한다.
같은 색을 중복해서 사용해도 좋으나 인접한 영역은 서로 다른 색으로 칠할 때, 칠하는 방법의 수를 구하시오.
\end{Exercise}
\end{minipage}
\quad
\begin{minipage}{.25\textwidth}
\includegraphics[width=.5\textwidth]{2}
\end{minipage}

\bigskip\noindent
\includegraphics[width=.2\textwidth]{2}\quad
\includegraphics[width=.2\textwidth]{2}\quad
\includegraphics[width=.2\textwidth]{2}\quad
\includegraphics[width=.2\textwidth]{2}\\[5pt]
\includegraphics[width=.2\textwidth]{2}\quad
\includegraphics[width=.2\textwidth]{2}\quad
\includegraphics[width=.2\textwidth]{2}\quad
\includegraphics[width=.2\textwidth]{2}\\[5pt]
\includegraphics[width=.2\textwidth]{2}\quad
\includegraphics[width=.2\textwidth]{2}\quad
\includegraphics[width=.2\textwidth]{2}\quad
\includegraphics[width=.2\textwidth]{2}\\[5pt]
\includegraphics[width=.2\textwidth]{2}\quad
\includegraphics[width=.2\textwidth]{2}\quad
\includegraphics[width=.2\textwidth]{2}\quad
\includegraphics[width=.2\textwidth]{2}\\[5pt]
\includegraphics[width=.2\textwidth]{2}\quad
\includegraphics[width=.2\textwidth]{2}\quad
\includegraphics[width=.2\textwidth]{2}\quad
\includegraphics[width=.2\textwidth]{2}\\[5pt]
\includegraphics[width=.2\textwidth]{2}\quad
\includegraphics[width=.2\textwidth]{2}\quad
\includegraphics[width=.2\textwidth]{2}\quad
\includegraphics[width=.2\textwidth]{2}\\[5pt]
\includegraphics[width=.2\textwidth]{2}\quad
\includegraphics[width=.2\textwidth]{2}\quad
\includegraphics[width=.2\textwidth]{2}\quad
\includegraphics[width=.2\textwidth]{2}\\[5pt]
\includegraphics[width=.2\textwidth]{2}\quad
\includegraphics[width=.2\textwidth]{2}\quad
\includegraphics[width=.2\textwidth]{2}\quad
\includegraphics[width=.2\textwidth]{2}\\[5pt]
\includegraphics[width=.2\textwidth]{2}\quad
\includegraphics[width=.2\textwidth]{2}\quad
\includegraphics[width=.2\textwidth]{2}\quad
\includegraphics[width=.2\textwidth]{2}\\[5pt]
\includegraphics[width=.2\textwidth]{2}\quad
\includegraphics[width=.2\textwidth]{2}\quad
\includegraphics[width=.2\textwidth]{2}\quad
\includegraphics[width=.2\textwidth]{2}\\[5pt]
\includegraphics[width=.2\textwidth]{2}\quad
\includegraphics[width=.2\textwidth]{2}\quad
\includegraphics[width=.2\textwidth]{2}\quad
\includegraphics[width=.2\textwidth]{2}\\[5pt]
\includegraphics[width=.2\textwidth]{2}\quad
\includegraphics[width=.2\textwidth]{2}\quad
\includegraphics[width=.2\textwidth]{2}\quad
\includegraphics[width=.2\textwidth]{2}\\[5pt]
\includegraphics[width=.2\textwidth]{2}\quad
\includegraphics[width=.2\textwidth]{2}\quad
\includegraphics[width=.2\textwidth]{2}\quad
\includegraphics[width=.2\textwidth]{2}\\[5pt]

\ans{36}

%
\begin{minipage}{.65\textwidth}
\begin{Exercise}
오른쪽 그림의 4개의 영역을 서로 다른 4가지 색 \(a\), \(b\), \(c\), \(d\)로 칠하려고 한다.
같은 색을 중복해서 사용해도 좋으나 인접한 영역은 서로 다른 색으로 칠할 때, 칠하는 방법의 수를 구하시오.
\end{Exercise}
\end{minipage}
\quad
\begin{minipage}{.25\textwidth}
\includegraphics[width=.5\textwidth]{3}
\end{minipage}

\ans{108}

%
\begin{minipage}{.65\textwidth}
\begin{Exercise}
오른쪽 그림의 3개의 영역을 서로 다른 4가지 색 \(a\), \(b\), \(c\), \(d\)로 칠하려고 한다.
같은 색을 중복해서 사용해도 좋으나 인접한 영역은 서로 다른 색으로 칠할 때, 칠하는 방법의 수를 구하시오.
\end{Exercise}
\end{minipage}
\quad
\begin{minipage}{.25\textwidth}
\includegraphics[width=.5\textwidth]{4}
\end{minipage}

\bigskip\noindent
\includegraphics[width=.1\textwidth]{4}\quad
\includegraphics[width=.1\textwidth]{4}\quad
\includegraphics[width=.1\textwidth]{4}\quad
\includegraphics[width=.1\textwidth]{4}\quad
\includegraphics[width=.1\textwidth]{4}\quad
\includegraphics[width=.1\textwidth]{4}\quad
\includegraphics[width=.1\textwidth]{4}\quad
\includegraphics[width=.1\textwidth]{4}\\[5pt]
\includegraphics[width=.1\textwidth]{4}\quad
\includegraphics[width=.1\textwidth]{4}\quad
\includegraphics[width=.1\textwidth]{4}\quad
\includegraphics[width=.1\textwidth]{4}\quad
\includegraphics[width=.1\textwidth]{4}\quad
\includegraphics[width=.1\textwidth]{4}\quad
\includegraphics[width=.1\textwidth]{4}\quad
\includegraphics[width=.1\textwidth]{4}\\[5pt]
\includegraphics[width=.1\textwidth]{4}\quad
\includegraphics[width=.1\textwidth]{4}\quad
\includegraphics[width=.1\textwidth]{4}\quad
\includegraphics[width=.1\textwidth]{4}\quad
\includegraphics[width=.1\textwidth]{4}\quad
\includegraphics[width=.1\textwidth]{4}\quad
\includegraphics[width=.1\textwidth]{4}\quad
\includegraphics[width=.1\textwidth]{4}\\[5pt]
\includegraphics[width=.1\textwidth]{4}\quad
\includegraphics[width=.1\textwidth]{4}\quad
\includegraphics[width=.1\textwidth]{4}\quad
\includegraphics[width=.1\textwidth]{4}\quad
\includegraphics[width=.1\textwidth]{4}\quad
\includegraphics[width=.1\textwidth]{4}\quad
\includegraphics[width=.1\textwidth]{4}\quad
\includegraphics[width=.1\textwidth]{4}\\[5pt]
\includegraphics[width=.1\textwidth]{4}\quad
\includegraphics[width=.1\textwidth]{4}\quad
\includegraphics[width=.1\textwidth]{4}\quad
\includegraphics[width=.1\textwidth]{4}\quad
\includegraphics[width=.1\textwidth]{4}\quad
\includegraphics[width=.1\textwidth]{4}\quad
\includegraphics[width=.1\textwidth]{4}\quad
\includegraphics[width=.1\textwidth]{4}\\[5pt]


\ans{24}

\newpage
%
\begin{minipage}{.65\textwidth}
\begin{Exercise}
오른쪽 그림의 4개의 영역을 서로 다른 4가지 색 \(a\), \(b\), \(c\), \(d\)로 칠하려고 한다.
같은 색을 중복해서 사용해도 좋으나 인접한 영역은 서로 다른 색으로 칠할 때, 칠하는 방법의 수를 구하시오.
\end{Exercise}
\end{minipage}
\quad
\begin{minipage}{.25\textwidth}
\includegraphics[width=.5\textwidth]{5}
\end{minipage}


\bigskip\noindent
\includegraphics[width=.1\textwidth]{5}\quad
\includegraphics[width=.1\textwidth]{5}\quad
\includegraphics[width=.1\textwidth]{5}\quad
\includegraphics[width=.1\textwidth]{5}\quad
\includegraphics[width=.1\textwidth]{5}\quad
\includegraphics[width=.1\textwidth]{5}\quad
\includegraphics[width=.1\textwidth]{5}\quad
\includegraphics[width=.1\textwidth]{5}\\[5pt]
\includegraphics[width=.1\textwidth]{5}\quad
\includegraphics[width=.1\textwidth]{5}\quad
\includegraphics[width=.1\textwidth]{5}\quad
\includegraphics[width=.1\textwidth]{5}\quad
\includegraphics[width=.1\textwidth]{5}\quad
\includegraphics[width=.1\textwidth]{5}\quad
\includegraphics[width=.1\textwidth]{5}\quad
\includegraphics[width=.1\textwidth]{5}\\[5pt]
\includegraphics[width=.1\textwidth]{5}\quad
\includegraphics[width=.1\textwidth]{5}\quad
\includegraphics[width=.1\textwidth]{5}\quad
\includegraphics[width=.1\textwidth]{5}\quad
\includegraphics[width=.1\textwidth]{5}\quad
\includegraphics[width=.1\textwidth]{5}\quad
\includegraphics[width=.1\textwidth]{5}\quad
\includegraphics[width=.1\textwidth]{5}\\[5pt]
\includegraphics[width=.1\textwidth]{5}\quad
\includegraphics[width=.1\textwidth]{5}\quad
\includegraphics[width=.1\textwidth]{5}\quad
\includegraphics[width=.1\textwidth]{5}\quad
\includegraphics[width=.1\textwidth]{5}\quad
\includegraphics[width=.1\textwidth]{5}\quad
\includegraphics[width=.1\textwidth]{5}\quad
\includegraphics[width=.1\textwidth]{5}\\[5pt]
\includegraphics[width=.1\textwidth]{5}\quad
\includegraphics[width=.1\textwidth]{5}\quad
\includegraphics[width=.1\textwidth]{5}\quad
\includegraphics[width=.1\textwidth]{5}\quad
\includegraphics[width=.1\textwidth]{5}\quad
\includegraphics[width=.1\textwidth]{5}\quad
\includegraphics[width=.1\textwidth]{5}\quad
\includegraphics[width=.1\textwidth]{5}\\[5pt]
\includegraphics[width=.1\textwidth]{5}\quad
\includegraphics[width=.1\textwidth]{5}\quad
\includegraphics[width=.1\textwidth]{5}\quad
\includegraphics[width=.1\textwidth]{5}\quad
\includegraphics[width=.1\textwidth]{5}\quad
\includegraphics[width=.1\textwidth]{5}\quad
\includegraphics[width=.1\textwidth]{5}\quad
\includegraphics[width=.1\textwidth]{5}\\[5pt]
\includegraphics[width=.1\textwidth]{5}\quad
\includegraphics[width=.1\textwidth]{5}\quad
\includegraphics[width=.1\textwidth]{5}\quad
\includegraphics[width=.1\textwidth]{5}\quad
\includegraphics[width=.1\textwidth]{5}\quad
\includegraphics[width=.1\textwidth]{5}\quad
\includegraphics[width=.1\textwidth]{5}\quad
\includegraphics[width=.1\textwidth]{5}\\[5pt]
\includegraphics[width=.1\textwidth]{5}\quad
\includegraphics[width=.1\textwidth]{5}\quad
\includegraphics[width=.1\textwidth]{5}\quad
\includegraphics[width=.1\textwidth]{5}\quad
\includegraphics[width=.1\textwidth]{5}\quad
\includegraphics[width=.1\textwidth]{5}\quad
\includegraphics[width=.1\textwidth]{5}\quad
\includegraphics[width=.1\textwidth]{5}\\[5pt]
\includegraphics[width=.1\textwidth]{5}\quad
\includegraphics[width=.1\textwidth]{5}\quad
\includegraphics[width=.1\textwidth]{5}\quad
\includegraphics[width=.1\textwidth]{5}\quad
\includegraphics[width=.1\textwidth]{5}\quad
\includegraphics[width=.1\textwidth]{5}\quad
\includegraphics[width=.1\textwidth]{5}\quad
\includegraphics[width=.1\textwidth]{5}\\[5pt]
\includegraphics[width=.1\textwidth]{5}\quad
\includegraphics[width=.1\textwidth]{5}\quad
\includegraphics[width=.1\textwidth]{5}\quad
\includegraphics[width=.1\textwidth]{5}\quad
\includegraphics[width=.1\textwidth]{5}\quad
\includegraphics[width=.1\textwidth]{5}\quad
\includegraphics[width=.1\textwidth]{5}\quad
\includegraphics[width=.1\textwidth]{5}\quad


\ans{48}

%
\begin{minipage}{.65\textwidth}
\begin{Exercise}
오른쪽 그림의 4개의 영역을 서로 다른 4가지 색 \(a\), \(b\), \(c\), \(d\)로 칠하려고 한다.
같은 색을 중복해서 사용해도 좋으나 인접한 영역은 서로 다른 색으로 칠할 때, 칠하는 방법의 수를 구하시오.
\end{Exercise}
\end{minipage}
\quad
\begin{minipage}{.25\textwidth}
\includegraphics[width=.5\textwidth]{6}
\end{minipage}

\ans{48}

%
\begin{minipage}{.65\textwidth}
\begin{Exercise}
오른쪽 그림과 같은 다섯 영역 \(A\), \(B\), \(C\), \(D\), \(E\)에 각각 빨강, 파랑, 노랑, 보라, 연두 중 어느 한 색을 칠하려고 한다.
같은 색을 여러 번 사용할 수 있지만 이웃하는 영역에는 서로 다른 색을 칠한다고 할 때, 색을 칠하는 경우의 수를 구하시오.
\end{Exercise}
\end{minipage}
\quad
\begin{minipage}{.25\textwidth}
\includegraphics[width=.5\textwidth]{7}
\end{minipage}

\ans{720}

%
\begin{minipage}{.65\textwidth}
\begin{Exercise}
오른쪽 그림은 4개의 행정구역을 나타내는 지도이다. 이 지도의 \(A\), \(B\), \(C\), \(D\) 4개의 행정구역을 서로 다른 4가지 색으로 색칠하려고한다.
같은 색을 중복하여 사용해도 좋으나 인접한 영역은 서로 다른 색으로 칠할 때, 칠하는 경우의 수를 구하시오.
\end{Exercise}
\end{minipage}
\quad
\begin{minipage}{.25\textwidth}
\includegraphics[width=.5\textwidth]{8}
\end{minipage}

\ans{48}

%%
\section{순열}

\bigskip\noindent
%
\begin{minipage}{.65\textwidth}
\textbf{문제1)}
오른쪽 그림의 2개의 영역을 서로 다른 4가지 색 \(a\), \(b\), \(c\), \(d\)로 칠하려고 한다.
같은 색을 중복해서 사용해도 좋으나 인접한 영역은 서로 다른 색으로 칠할 때, 칠하는 방법의 수를 구하시오.
\end{minipage}
\quad
\begin{minipage}{.25\textwidth}
\includegraphics[width=.5\textwidth]{1}
\end{minipage}

\bigskip\noindent
%
\begin{minipage}{.65\textwidth}
\textbf{문제1*)}
오른쪽 그림의 2개의 영역을 서로 다른 4가지 색 \(a\), \(b\), \(c\), \(d\)로 칠하려고 한다.
같은 색을 중복해서 사용할 수 없을 때, 칠하는 방법의 수를 구하시오.
\end{minipage}
\quad
\begin{minipage}{.25\textwidth}
\includegraphics[width=.5\textwidth]{1}
\end{minipage}

\bigskip\noindent
\includegraphics[width=.2\textwidth]{1}\quad
\includegraphics[width=.2\textwidth]{1}\quad
\includegraphics[width=.2\textwidth]{1}\quad
\includegraphics[width=.2\textwidth]{1}\\[5pt]
\includegraphics[width=.2\textwidth]{1}\quad
\includegraphics[width=.2\textwidth]{1}\quad
\includegraphics[width=.2\textwidth]{1}\quad
\includegraphics[width=.2\textwidth]{1}\\[5pt]
\includegraphics[width=.2\textwidth]{1}\quad
\includegraphics[width=.2\textwidth]{1}\quad
\includegraphics[width=.2\textwidth]{1}\quad
\includegraphics[width=.2\textwidth]{1}

\bigskip\noindent
%
\begin{minipage}{.65\textwidth}
\textbf{문제2)}
오른쪽 그림의 3개의 영역을 서로 다른 4가지 색 \(a\), \(b\), \(c\), \(d\)로 칠하려고 한다.
같은 색을 중복해서 사용해도 좋으나 인접한 영역은 서로 다른 색으로 칠할 때, 칠하는 방법의 수를 구하시오.
\end{minipage}
\quad
\begin{minipage}{.25\textwidth}
\includegraphics[width=.5\textwidth]{2}
\end{minipage}

\bigskip\noindent
%
\begin{minipage}{.65\textwidth}
\textbf{문제2*)}
오른쪽 그림의 3개의 영역을 서로 다른 4가지 색 \(a\), \(b\), \(c\), \(d\)로 칠하려고 한다.
같은 색을 중복해서 사용할 수 없을 때, 칠하는 방법의 수를 구하시오.
\end{minipage}
\quad
\begin{minipage}{.25\textwidth}
\includegraphics[width=.5\textwidth]{2}
\end{minipage}

\bigskip\noindent
\includegraphics[width=.2\textwidth]{2}\quad
\includegraphics[width=.2\textwidth]{2}\quad
\includegraphics[width=.2\textwidth]{2}\quad
\includegraphics[width=.2\textwidth]{2}\\[5pt]
\includegraphics[width=.2\textwidth]{2}\quad
\includegraphics[width=.2\textwidth]{2}\quad
\includegraphics[width=.2\textwidth]{2}\quad
\includegraphics[width=.2\textwidth]{2}\\[5pt]
\includegraphics[width=.2\textwidth]{2}\quad
\includegraphics[width=.2\textwidth]{2}\quad
\includegraphics[width=.2\textwidth]{2}\quad
\includegraphics[width=.2\textwidth]{2}\\[5pt]
\includegraphics[width=.2\textwidth]{2}\quad
\includegraphics[width=.2\textwidth]{2}\quad
\includegraphics[width=.2\textwidth]{2}\quad
\includegraphics[width=.2\textwidth]{2}\\[5pt]
\includegraphics[width=.2\textwidth]{2}\quad
\includegraphics[width=.2\textwidth]{2}\quad
\includegraphics[width=.2\textwidth]{2}\quad
\includegraphics[width=.2\textwidth]{2}\\[5pt]
\includegraphics[width=.2\textwidth]{2}\quad
\includegraphics[width=.2\textwidth]{2}\quad
\includegraphics[width=.2\textwidth]{2}\quad
\includegraphics[width=.2\textwidth]{2}\\[5pt]
\includegraphics[width=.2\textwidth]{2}\quad
\includegraphics[width=.2\textwidth]{2}\quad
\includegraphics[width=.2\textwidth]{2}\quad
\includegraphics[width=.2\textwidth]{2}
\newpage
\shipoutAnswer

\bigskip\noindent \textbf{답 1*)} 12

\bigskip\noindent \textbf{답 2*)} 24


\end{document}