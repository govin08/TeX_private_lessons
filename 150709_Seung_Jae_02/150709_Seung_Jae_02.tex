\documentclass{oblivoir}
\usepackage{amsmath,amssymb,amsthm,kotex,mdframed,paralist,chngcntr}
\usepackage{kswrapfig}

\newcounter{num}
%\newcommand{\defi}[1]
%{\bigskip\noindent\refstepcounter{num}\textbf{정의 \arabic{num}) #1}\par}
%\newcommand{\theo}[1]
%{\bigskip\noindent\refstepcounter{num}\textbf{정리 \arabic{num}) #1}\par}
%\newcommand{\exam}[1]
%{\bigskip\noindent\refstepcounter{num}\textbf{예시 \arabic{num}) #1}\par}
\newcommand{\prob}[1]
{\bigskip\noindent\refstepcounter{num}\textbf{문제 \arabic{num}) #1}\par}
%\newcommand{\howo}[1]
%{\bigskip\noindent\refstepcounter{num}\textbf{숙제 \arabic{num}) #1}\par\bigskip}

\newcommand{\ans}{{\raggedleft\textbf{답 : (\qquad\qquad\qquad\qquad\qquad\qquad)}
\par}}

\renewcommand{\proofname}{증명)}
\counterwithout{subsection}{section}


%%%
\begin{document}
\large

\title{승재 02 - 최고수준 수학}
\author{}
\date{\today}
\maketitle
%\tableofcontents

%
\prob{p82, \#02-1}
어떤 정사각형의 가로를 20\% 늘리고 세로를 30\% 줄여서 새로운 직사각형을 만들었습니다.
정사각형의 넓이에 대한 새로 만든 직사각형의 넓이의 비율을 백분율로 나타내시오.

84\%

\prob{p82, \#02-2}
어떤 정사각형의 가로를 10\%, 세로를 15\% 줄여서 새로운 직사각형을 만들었습니다.
정사각형의 넓이에 대한 새로 만든 직사각형의 넓이의 비율을 백분율로 나타내시오.

76.5\%

\prob{p82, \#02-3}
어떤 삼각형의 밑변의 길이를 25\% 늘리고 높이를 30\% 줄여서 새로운 삼각형을 만들었습니다.
원래 삼각형의 넓이에 대한 새로 만든 삼각형의 넓이의 비율을 백분율로 나타내시오.

87.5\%

\prob{p82, \#02-4}
어떤 원의 반지름의 길이를 30\% 줄여서 새로운 원을 만들었습니다.
원래 원의 넓이에 대한 새로 만든 원의 넓이의 비율을 백분율로 나타내시오.

49\%


\end{document}