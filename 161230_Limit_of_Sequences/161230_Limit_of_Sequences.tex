\documentclass{oblivoir}
\usepackage{amsmath,amssymb,amsthm,kotex,paralist,kswrapfig}

\usepackage[skipabove=10pt,skipbelow=10pt]{mdframed}

\usepackage{tabto,pifont}
\TabPositions{0.2\textwidth,0.4\textwidth,0.6\textwidth,0.8\textwidth}
\newcommand\tabb[5]{\par\bigskip\noindent
\ding{172}\:{\ensuremath{#1}}
\tab\ding{173}\:\:{\ensuremath{#2}}
\tab\ding{174}\:\:{\ensuremath{#3}}
\tab\ding{175}\:\:{\ensuremath{#4}}
\tab\ding{176}\:\:{\ensuremath{#5}}}

\usepackage{enumitem}
\setlist[enumerate]{label=(\arabic*)}

\newcounter{num}
\newcommand{\defi}[1]
{\bigskip\noindent\refstepcounter{num}\textbf{정의 \arabic{num}) #1}\par\noindent}
\newcommand{\theo}[1]
{\bigskip\noindent\refstepcounter{num}\textbf{정리 \arabic{num}) #1}\par\noindent}
\newcommand{\exam}[1]
{\bigskip\noindent\refstepcounter{num}\textbf{예시 \arabic{num}) #1}\par\noindent}
\newcommand{\prob}[1]
{\bigskip\noindent\refstepcounter{num}\textbf{문제 \arabic{num}) #1}\par\noindent}
\newcommand{\proo}
{\bigskip\textsf{증명)}\par}

\newcommand{\ans}{
{\par
\raggedleft\textbf{답 : (\qquad\qquad\qquad\qquad\qquad\qquad)}
\par}\bigskip\bigskip}
\newcommand\an[1]{\par\bigskip\noindent\textbf{문제 #1)}\\}

\newcommand{\pb}[1]%\Phantom + fBox
{\fbox{\phantom{\ensuremath{#1}}}}

\newcommand{\procedure}[1]{\begin{mdframed}\vspace{#1\textheight}\end{mdframed}}

\let\oldsection\section
\let\emph\textsf
\renewcommand\section{\clearpage\oldsection}
%%%%
\begin{document}

\title{미적분 1 : 01 수열의 극한}
\author{}
\date{\today}
\maketitle
\tableofcontents
\newpage

%%
\section{수열의 수렴과 발산}

\begin{mdframed}[innertopmargin=-5pt]
%
\defi{수열의 수렴}
수열 \(\{a_n\}\)에서 \(n\)이 커질 때, \(a_n\)의 값이 일정한 값 \(\alpha\)에 가까워지면

\bigskip
\begin{center}
수열 \(\{a_n\}\)은 \(\alpha\)에 \emph{수렴}한다
\end{center}
\noindent라고 말하고 기호로
\[\lim_{n\to\infty}a_n=\alpha\]
로 나타낸다.
이때 \(\alpha\)를 수열 \(\{a_n\}\)의 \emph{극한값}이라고 부른다.
\end{mdframed}

%
\exam{}
\(a_n=\frac1n\)이면 이 수열은
\[1,\quad\frac12,\quad\frac13,\quad\frac14,\quad\frac15,\quad\frac16,\quad\cdots\]
와 같이 나타나고, 점점 감소하여 \(0\)으로 가까워지는 수열이다.
따라서
\begin{center}
수열 \(\{a_n\}\)은  \(0\)에 수렴한다.
\end{center}
이것을 기호로 나타내면
\[\lim_{n\to\infty}a_n=0\]
이다.

%
\exam{}
\(b_n=1-\left(\frac12\right)^n\)이면 이 수열은
\[1-\frac12,\quad1-\frac14,\quad1-\frac18,\quad1-\frac1{16},\quad1-\frac1{32},\quad1-\frac1{64},\quad\cdots\]
와 같이 나타나고, 점점 증가하여 \(1\)로 가까워지는 수열이다.
따라서
\begin{center}
수열 \(\{b_n\}\)은  \(1\)에 수렴한다.
\end{center}
이것을 기호로 나타내면
\[\lim_{n\to\infty}b_n=1\]
이다.

\clearpage
%
\prob{}
\begin{enumerate}
\item
\(a_n=2+(\frac13)^n\)이면 이 수열은
\[2+\frac13,\quad2+\frac19,\quad\pb{2+\frac1{27}}\:,\quad2+\frac1{81},\quad\cdots\]
와 같이 나타나고, 점점 \pb{감소}하여 \(2\)로 가까워지는 수열이다.
따라서
\begin{center}
수열 \(\{a_n\}\)은  \(2\)에 수렴한다.
\end{center}
이것을 기호로 나타내면
\[\lim_{n\to\infty}a_n=\pb{2}\]
이다.
\item
\(b_n=\frac1{2n-1}\)이면 이 수열은
\[1,\quad\frac13,\quad\frac15,\quad\frac17,\quad\pb{\frac19},\quad\cdots\]
와 같이 나타나고, 점점 감소하여 \(0\)으로 가까워지는 수열이다.
따라서
\begin{center}
수열 \(\{b_n\}\)은  \(0\)에 \pb{수렴}한다.
\end{center}
이것을 기호로 나타내면
\[\lim_{\pb{........}}b_n=0\]
이다.
\item
\(a_n=\frac n{n+1}\)이면 이 수열은
\[\frac12,\quad\frac23,\quad\frac34,\quad\pb{\frac45}\:,\quad\frac56,\quad\cdots\]
와 같이 나타나고, 점점 증가하여 \(\pb{1}\)로 가까워지는 수열이다.
따라서
\begin{center}
수열 \(\{c_n\}\)은 \pb2에 수렴한다.
\end{center}
이것을 기호로 나타내면
\[\lim_{n\to\infty}c_n=\pb{1}\]
이다.
\end{enumerate}

\begin{mdframed}[innertopmargin=-5pt]
%
\defi{수열의 발산}
수열 \(\{a_n\}\)이 수렴하지 않으면

\bigskip
\begin{center}
수열 \(\{a_n\}\)은 \emph{발산}한다.
\end{center}
\noindent라고 말한다.
이때
\begin{enumerate}
\item
\(n\)이 커질 때, \(a_n\)의 값이 한없이 커지면
\begin{center}
수열 \(\{a_n\}\)은 \emph{양의 무한대로 발산}한다.
\end{center}
라고 말하고 기호로
\[\lim_{n\to\infty}a_n=\infty\]
로 나타낸다.
\item
\(n\)이 커질 때, \(a_n\)의 값이 한없이 작아지면
\begin{center}
수열 \(\{a_n\}\)은 \emph{음의 무한대로 발산}한다.
\end{center}
라고 말하고 기호로
\[\lim_{n\to\infty}a_n=-\infty\]
로 나타낸다.
\item
\(\{a_n\}\)이 수렴하지도 않고, 양의 무한대나 음의 무한대로 발산하지도 않으면
\begin{center}
수열 \(\{a_n\}\)은 \emph{진동}한다.
\end{center}
라고 말한다.
\end{enumerate}
\end{mdframed}

%
\exam{}
\(a_n=2n+1\)이면 이 수열은
\[1,\quad3,\quad5,\quad7,\quad9,\quad11,\quad\cdots\]
와 같이 나타나고, 한없이 증가하는 수열이다.
따라서
\begin{center}
수열 \(\{a_n\}\)은  양의 무한대로 발산한다.
\end{center}
이것을 기호로 나타내면
\[\lim_{n\to\infty}a_n=\infty\]
이다.

%
\exam{}
\(b_n=-2^n\)이면 이 수열은
\[-2,\quad-4,\quad-8,\quad-16,\quad-32,\quad-64,\quad\cdots\]
와 같이 나타나고, 한없이 감소하는 수열이다.
따라서
\begin{center}
수열 \(\{b_n\}\)은  음의 무한대로 발산한다.
\end{center}
이것을 기호로 나타내면
\[\lim_{n\to\infty}b_n=-\infty\]
이다.

%
\exam{}
\(c_n=(-1)^n\)이면 이 수열은
\[-1,\quad1,\quad-1,\quad1,\quad-1,\quad1,\quad\cdots\]
와 같이 나타나므로 수렴하지도 않고, 양의 무한대나 음의 무한대로 발산하지도 않는다.
따라서
\begin{center}
수열 \(\{c_n\}\)은 진동한다.
\end{center}

%
\exam{}
\(d_n=(-1)^n\cdot n\)이면 이 수열은
\[-1,\quad2,\quad-3,\quad4,\quad-5,\quad6,\quad\cdots\]
와 같이 나타나므로 수렴하지도 않고, 양의 무한대나 음의 무한대로 발산하지도 않는다.
따라서
\begin{center}
수열 \(\{d_n\}\)은 진동한다.
\end{center}

%
\prob{}
\begin{enumerate}
\item
\(a_n=-3n+5\)이면 이 수열은
\[2,\quad\pb{-1},\quad-4,\quad-7,\quad-10,\quad-13,\quad\cdots\]
와 같이 나타나고, 한없이 \pb{감소}하는 수열이다.
따라서
\begin{center}
수열 \(\{a_n\}\)은 \pb{양의 무한대}로 \pb{발산}한다.
\end{center}
이것을 기호로 나타내면
\[\lim_{n\to\infty}a_n=\pb{-\infty}\]
이다.
\item
\(b_n=2^n-1\)이면 이 수열은
\[1,\quad3,\quad7,\quad15,\quad\pb{31},\quad63,\quad\cdots\]
와 같이 나타나고, 한없이 \pb{증가}하는 수열이다.
따라서
\begin{center}
수열 \(\{b_n\}\)은 \pb{음의 무한대}로 \pb{발산}한다.
\end{center}
이것을 기호로 나타내면
\[\lim_{n\to\infty}b_n=\pb{\infty}\]
이다.
\item
\(c_n=(-2)^{n-1}\)이면 이 수열은
\[1,\quad-2,\quad4,\quad-8,\quad16,\quad-32,\quad\cdots\]
와 같이 나타나므로 수렴하지도 않고, 양의 무한대나 음의 무한대로 발산하지도 않는다.
따라서
\begin{center}
수열 \(\{c_n\}\)은 \pb{진동}한다.
\end{center}
\end{enumerate}

%%
\section{극한값의 계산}

%
\exam{}
\(a_n=3\)이면 이 수열은
\[3,\quad3,\quad3,\quad3,\quad3,\quad3,\quad\cdots\]
와 같이 \(3\)으로 일정한 수열이다.
따라서 \(3\)에 수렴한다고 볼 수있다.
\[\lim_{n\to\infty}a_n=3.\]

\begin{mdframed}
상수 \(c\)에 대하여 \(a_n=c\)이면
\[\lim_{n\to\infty}a_n=c.\]
\end{mdframed}

%
\exam{}
\(a_n=2+\frac2n\)이면 이 수열은 \(2\)에 가까워지는 수열이므로
\[\lim_{n\to\infty}a_n=2\]
이다.
한편 \[\lim_{n\to\infty}2=2,\quad\lim_{n\to\infty}\frac2n=0\]
이므로
\[
\lim_{n\to\infty}a_n
=\lim_{n\to\infty}\left(2+\frac2n\right)
=\lim_{n\to\infty}2+\lim_{n\to\infty}\frac2n
=2+0=2
\]
로 계산할 수도 있을 것이다.

\begin{mdframed}
두 수열 \(\{a_n\}\)과 \(\{b_n\}\)이 수렴하면
\[\lim_{n\to\infty}(a_n+b_n)=\lim_{n\to\infty}a_n+\lim_{n\to\infty}b_n.\]
\end{mdframed}

\bigskip
이 성질은 덧셈 말고도 뺄셈, 곱셈, 나눗셈에서도 똑같이 성립한다.

\begin{mdframed}[innertopmargin=-5pt]
%
\theo{수열의 기본 성질}
수열 \(\{a_n\}\), \(\{b_n\}\)이 수렴할 때,
\begin{enumerate}
\item
\(\displaystyle\lim_{n\to\infty}(a_n+b_n)=\lim_{n\to\infty}a_n+\lim_{n\to\infty}b_n\)
\item
\(\displaystyle\lim_{n\to\infty}(a_n-b_n)=\lim_{n\to\infty}a_n-\lim_{n\to\infty}b_n\)
\item
\(\displaystyle\lim_{n\to\infty}a_nb_n=\lim_{n\to\infty}a_n\cdot\lim_{n\to\infty}b_n\)
\item
\(\displaystyle\lim_{n\to\infty}\frac{a_n}{b_n}
=\frac{\displaystyle\lim_{n\to\infty}a_n}{\displaystyle\lim_{n\to\infty}b_n}
\qquad\left(단,\:\:\displaystyle\lim_{n\to\infty}b_n\neq0\right)\)
\end{enumerate}
\end{mdframed}

또한, \(k\)가 상수이면
\begin{mdframed}
\begin{enumerate}
\item[(5)]
\(\displaystyle\lim_{n\to\infty}ka_n=k\lim_{n\to\infty}a_n\)
\end{enumerate}
\end{mdframed}

%
\exam{}
\(\displaystyle\lim_{n\to\infty}\frac1n=0\)을 이용하여 다음 계산들을 해보자.
\begin{enumerate}
\item
\(\displaystyle\lim_{n\to\infty}\frac3n
\stackrel{(5)}=3\lim_{n\to\infty}\frac1n=3\times0=0\)
\item
\(\displaystyle\lim_{n\to\infty}\left(2+\frac1n\right)
\stackrel{(1)}=
\lim_{n\to\infty}2+\lim_{n\to\infty}\frac1n=2+0=2\)
\item
\(\displaystyle\lim_{n\to\infty}\left(3-\frac2n\right)
\stackrel{(2)}=
\lim_{n\to\infty}3-\lim_{n\to\infty}\frac2n
=\lim_{n\to\infty}3-2\lim_{n\to\infty}\frac1n
=3-2\times0=3\)
\item
\(\displaystyle\lim_{n\to\infty}\frac1{n^2}=\lim_{n\to\infty}\left(\frac1n\cdot\frac1n\right)
\stackrel{(3)}=\lim_{n\to\infty}\frac1n\cdot\lim_{n\to\infty}\frac1n=0\cdot0=0\)
\item
\(\displaystyle\lim_{n\to\infty}\frac{n+1}{2n-3}
=\lim_{n\to\infty}\frac{1+\frac1n}{2-\frac3n}
\stackrel{(4)}=
\frac{\displaystyle\lim_{n\to\infty}1+\frac1n}{\displaystyle\lim_{n\to\infty}2-\frac3n}
=
\frac{\displaystyle\lim_{n\to\infty}1+\lim_{n\to\infty}\frac1n}
{\displaystyle\lim_{n\to\infty}2-\lim_{n\to\infty}\frac3n}
\\
=\frac{1+0}{2-0}=\frac12
\)
\end{enumerate}

\clearpage
%
\prob{}
\(\displaystyle\lim_{n\to\infty}\frac1n=0\)을 이용하여 다음을 구하여라.
\begin{enumerate}
\item
\(\displaystyle\lim_{n\to\infty}\left(-\frac2n\right)\)
\item
\(\displaystyle\lim_{n\to\infty}\left(\frac1n-4\right)\)
\item
\(\displaystyle\lim_{n\to\infty}\left(5+\frac2n\right)\)
\item
\(\displaystyle\lim_{n\to\infty}\frac1n\left(1+\frac1n\right)\)
\item
\(\displaystyle\lim_{n\to\infty}\frac{3n-1}{n+2}\)
\end{enumerate}
\procedure{0.5}
{\par
\raggedleft\textbf{답 : (1)\qquad\qquad, (2) \qquad\qquad, (3) \qquad\qquad, (4)\qquad\qquad, (5)\qquad\qquad}
\par}\bigskip\bigskip

%%
\section{수열의 극한값과 대소관계}
두 수열 \(\{a_n\}\), \(\{b_n\}\)에 대하여
\begin{mdframed}
\[a_n\le b_n\quad\Longrightarrow\quad\lim_{n\to\infty}a_n\le\lim_{n\to\infty}b_n\]
\end{mdframed}
가 성립한다.
하지만,
\begin{mdframed}
\[a_n<b_n\quad\Longrightarrow\quad\lim_{n\to\infty}a_n<\lim_{n\to\infty}b_n\]
\end{mdframed}
가 성립하지는 않는다.

예를 들어, \(a_n=1-\frac1n\), \(b_n=1+\frac1n\)이면 \(a_n<b_n\)이지만,
\[\lim_{n\to\infty}a_n=\lim_{n\to\infty}b_n\]
이다.
대신
\begin{mdframed}
\[a_n< b_n\quad\Longrightarrow\quad\lim_{n\to\infty}a_n\le\lim_{n\to\infty}b_n\]
\end{mdframed}
는 성립한다.

%
\exam{}
수열 \(\{a_n\}\)이 모든 자연수 \(n\)에 대해
\[1-\frac2n<a_n<1+\frac1n\]
을 만족시킬 때, \(\displaystyle\lim_{n\to\infty}a_n\)의 값을 구하여라.
\begin{mdframed}
\(1-\frac2n<a_n\)이므로
\[\lim_{n\to\infty}\left(1-\frac2n\right)\le\lim_{n\to\infty}a_n\]
\[1\le\lim_{n\to\infty}a_n\tag{1}\]
이다.
또, \(a_n<1+\frac1n\)이므로
\[\lim_{n\to\infty}a_n\le\lim_{n\to\infty}\left(1+\frac1n\right)\]
\end{mdframed}
\begin{mdframed}
\[\lim_{n\to\infty}a_n\le1\tag{2}\]
이다.
(1), (2)에서
\[1\le\lim_{n\to\infty}a_n\le1\]
이고, 따라서
\[\lim_{n\to\infty}a_n=1\]
\end{mdframed}

%
\prob{}
수열 \(\{a_n\}\)이 모든 자연수 \(n\)에 대해
\[\frac{3n-4}{n-1}<a_n<\frac{3n+2}{n-1}\]
을 만족시킬 때, \(\displaystyle\lim_{n\to\infty}a_n\)의 값을 구하여라.
\procedure{0.5}
\ans

%%
\section{등비수열의 극한}
%
\exam{}
다음 수열들의 수렴, 발산을 조사하여라.\\
(1) \(a_n=2^n\),		\tabto{0.33\textwidth}
(2) \(b_n=(-\frac13)^n\),	\tabto{0.66\textwidth}
(3) \(c_n=1^n\)
\begin{mdframed}
\begin{enumerate}
\item
\(a_n=2^n\)이면 이 수열은
\[2,\quad4,\quad8,\quad16,\quad32,\quad64,\quad\cdots\]
와 같이 나타난다.
따라서
\[\lim_{n\to\infty}2^n=\infty\]
이다.
\item
\(b_n=\left(-\frac13\right)^n\)이면 이 수열은
\[-\frac13,\quad\frac19,\quad-\frac1{27},\quad\frac1{81},\quad\cdots\]
와 같이 나타난다.
따라서
\[\lim_{n\to\infty}\left(-\frac13\right)^n=0\]
이다.
\item
\(c_n=1^n\)이면 \(c_n=1\)이다.
따라서
\[\lim_{n\to\infty}1^n=1\]
이다.
\end{enumerate}
\end{mdframed}
{\par
\raggedleft\textbf{답 : (1) 양의 무한대로 발산, (2) \(0\)으로 수렴,  (3) \(1\)로 수렴}
\par}\bigskip\bigskip

\clearpage
%
\prob{}
다음 수열들의 수렴, 발산을 조사하여라.\\
(1) \(a_n=(-3)^n\),		\tabto{0.33\textwidth}
(2) \(b_n=\left(\frac12\right)^n\),	\tabto{0.66\textwidth}
(3) \(c_n=(-1)^n\)
\procedure{0.7}
{\par
\raggedleft\textbf{답 : (1)\qquad\qquad\qquad, (2) \qquad\qquad\qquad, (3) \qquad\qquad\qquad}
\par}\bigskip\bigskip

\clearpage
위의 결과들을 종합하면 다음 결과를 얻을 수 있다.
\begin{mdframed}[innertopmargin=-5pt]
%
\theo{등비수열의 수렴과 발산}
등비수열 \(\{r^n\}\)에 대하여
\begin{enumerate}
\item
\(r>1\)이면 \(\displaystyle\lim_{n\to\infty}r^n=\infty\)
\tabto{0.5\textwidth} \(\Longrightarrow\)\:\:양의 무한대로 발산
\item
\(r=1\)이면 \(\displaystyle\lim_{n\to\infty}r^n=1\)
\tabto{0.5\textwidth} \(\Longrightarrow\)\:\:\(1\)에 수렴
\item
\(-1<r<1\)이면 \(\displaystyle\lim_{n\to\infty}r^n=0\)
\tabto{0.5\textwidth} \(\Longrightarrow\)\:\:\(0\)에 수렴
\item
\(r\le-1\)이면 \(\displaystyle\lim_{n\to\infty}r^n\)는 진동
\tabto{0.5\textwidth} \(\Longrightarrow\)\:\:발산(진동)
\end{enumerate}
\end{mdframed}

%%
\section*{답}

%
\an{4}
\begin{enumerate}[itemsep=0pt]
\item
\(2+\frac1{27}\), 증가, 2
\item
\(\frac19\), 수렴, \(n\to\infty\)
\item
\(\frac45\), \(1\), \(1\), \(1\)
\end{enumerate}

%
\an{10}
\begin{enumerate}[topsep=0pt]
\item
\(-1\), 감소, 음의 무한대, 발산, \(-\infty\)
\item
\(31\), 증가, 양의 무한대, 발산, \(\infty\)
\item
진동
\end{enumerate}

%
\an{15}
\begin{enumerate}[topsep=0pt]
\item
\(0\)
\item
\(-4\)
\item
\(5\)
\item
\(0\)
\item
\(3\)
\end{enumerate}

%
\an{17}
\(3\)

%
\an{19}
\begin{enumerate}[topsep=0pt]
\item
진동
\item
\(0\)으로 수렴
\item
진동
\end{enumerate}
\end{document}