\documentclass{oblivoir}
\usepackage{amsmath,amssymb,amsfonts,amsthm,mathrsfs,kotex,mdframed,paralist}

\newcounter{num}
\newcommand{\prob}[1]
{\bigskip\noindent\refstepcounter{num}\textbf{문제 \arabic{num}) #1 참조}\par}
\newcommand{\probb}
{\bigskip\noindent\refstepcounter{num}\textbf{문제 \arabic{num})}\par}

\renewcommand{\figurename}{그림}
\renewcommand{\proofname}{증명)}
\newcommand{\sol}{\par\bigskip\noindent{\bfseries풀이)}\par}

\newcommand{\uu}{\ensuremath{\boldsymbol{u}}}
\newcommand{\xx}{\ensuremath{\boldsymbol{x}}}
\newcommand{\vv}{\ensuremath{\boldsymbol{v}}}
\newcommand{\ww}{\ensuremath{\boldsymbol{w}}}
\newcommand{\zz}{\ensuremath{\boldsymbol{0}}}

\newcommand{\sleb}{\begin{array}{r@{\;}r@{\;=\;}c}}
\newcommand{\slec}{\begin{array}{r@{\;}r@{\;}r@{\;=\;}c}}
\newcommand{\sled}{\begin{array}{r@{\;}r@{\;}r@{\;}r@{\;=\;}c}}

%%%
\begin{document}

\title{동락 03 - Chapter 1 문제들}
\author{}
\date{\today}
\maketitle
\newpage

\prob{(1,2) \#3}
Describe the intersection of the three planes \(u+v+w+z=3\), \(u+v=2\), \(v+w=1\) (all in four-dimensional space).
Is it a line or a point or an empty set?
What is the intersection if the fourth plane \(w=-1\) is included?
Find a fourth equation that leaves us with no solution.

\prob{(1,2) \#3}
Describe the intersection of the three planes \(u+2v+3w+4z=10\), \(2u+3w=5\), \(u-2v-4z=-5\) (all in four-dimensional space).
Is it a line or a point or an empty set?
What is the intersection if the fourth plane \(w=4\) is included?
Find a fourth equation that leaves us with no solution.

\prob{(1,2) \#3}
Describe the intersection of the three planes \(u+v+w=1\), \(u+v+z=2\), \(u+w+z=3\) (all in four-dimensional space).
Is it a line or a point or an empty set?
What is the intersection if the fourth plane \(v+w+z=4\) is included?
Find a fourth equation that leaves us with no solution.

\prob{(1,2) \#11}
Describe the column picture for the system
\[
\begin{array}{r@{\;}r@{\;}r@{\;=\;}c}
u+&&w&b_1\\
3u+&2v+&w&b_2\\
-u+&v-&2w&b_3
\end{array}
\]

Show that the three columns on the left lie in the same plane by expressing the third column as a combination of the first two.
What all the solutions \((u,v,w)\) if \(b\) is the zero vector \((0,0,0)\)?

\prob{(1,2) \#8}
For which number \(a\) does elimination break down (a) permanently, and (b) temporarily?
\[
\begin{array}{r@{\;}r@{\;=\;}c}
ax-&5y&-10\\
x+&2y&5
\end{array}
\]

\prob{(1,3) \#25}
Solve the system and find the pivots when
\[
\begin{array}{r@{\;}r@{\;}r@{\;}r@{\;=\;}c}
-3u+	&2v\phantom{+}	&	&	&0\\
u- 	&3v+	&2w\phantom{+}	&	&0\\
	&v- 	&3w+	&2z	&0\\
	&	&w- 	&3z	&-31
\end{array}
\]
You may carry the right-hand side as a fifth column (and omit writing \(u\), \(v\), \(w\), \(z\) until the solution at the end).

\prob{(1,3) \#30}
For which three numbers \(a\) will elimination fail to give three pivots?
\[
\begin{array}{r@{\;}r@{\;}r@{\;=\;}c}
ax+	&3y+	&4z	&b_1\\
ax+	&ay+	&z	&b_2\\
ax+	&ay+	&az	&b_3
\end{array}
\]

\prob{(1,3) \#30}
For which three numbers \(a\) will elimination fail to give three pivots?
\[
\begin{array}{r@{\;}r@{\;}r@{\;=\;}c}
ax+	&ay+	&az	&b_1\\
ax+	&ay+	&2z	&b_2\\
ax+	&3y+	&1z	&b_3
\end{array}
\]

\prob{(1,3) \#30}
For which three numbers \(a\) will elimination fail to give three pivots?
\[
\begin{array}{r@{\;}r@{\;}r@{\;=\;}c}
ax+	&2y+	&z	&b_1\\
2ax+	&ay+	&4z	&b_2\\
4ax+	&2ay+&3az	&b_3
\end{array}
\]

\prob{(1,3) \#32}
Use elimination to solve 
\[
\slec
2u+	&v+	&3w&1\\
2u+	&6v+	&8w&3\\
6u+	&8v+	&18w&5
\end{array}
\]

\(답 : (\frac3{10},\frac25,0)\)

\prob{(1,3) \#32}
Use elimination to solve 
\[
\slec
	&2v+	&w&-8\\
u-	&2v-	&3w&0\\
-u+	&v+	&2w&3
\end{array}
\]

\(답 : (-4,-5,2)\)

\prob{(1,3), \# 32}
Use elimination to solve 
\[
\slec
u\phantom{+}&+	&2w	&3\\
4u+	&v+	&3w	&4\\
3u+ 	&v+	&w	&5
\end{array}
\]

\(답 : 해가 없다.\)

\prob{(1,3), \# 32}
Use elimination to solve 
\[
\slec
u+	&v+	&w&0\\
u-	&2v+	&2w&4\\
	&v-	&2w&2
\end{array}
\]

\(답 : (4,-2,-2)\)

\prob{(1,3), \# 32}
Use elimination to solve 
\[
\slec
2u+	&4v+	&w&1\\
u+	&2v+	&2w&0\\
 	&	&3w&-1
\end{array}
\]

답 : 해가 무한히 많다.

\prob{(1,3), \# 32}
Use elimination to solve 
\[
\slec
u-	&2v+	&3w&1\\
-2u-	&4v+	&5w&-4\\
3u+ 	&5v-	&6w&6
\end{array}
\]

\(답 : (1,3,2)\)

\prob{(1,3), \# 32}
Use elimination to solve 
\[
\slec
	&v-	&2w&-8\\
-u+	&\phantom{+2v}	&3w&11\\
2u- 	&3v	&&2
\end{array}
\]

\(답 : (-2,-2,3)\)

\prob{(1,3), \# 32}
Use elimination to solve 
\[
\slec
u+	&3v+	&2w	&3\\
3u-	&2v+	&w	&4\\
5u- 	&7v\phantom{+}	&&5
\end{array}
\]

답 : 해가 무한히 많다.

\prob{(1,4)}
피보나치 수열 \(\{x_n\}\)은 다음과 같이 정의된다.
\[x_0=0,\quad x_1=1,\quad x_n=x_{n-1}+x_{n-2},\quad n=2,3,\cdots\]

(1)
\[X_n=
\begin{bmatrix}
x_n\\x_{n-1}
\end{bmatrix}
,\quad A=
\begin{bmatrix}
1&1\\1&0
\end{bmatrix}
\]
를 이용하여 피보나치 수열의 점화식을 간단히 하시오.

(2)
\[X_n=A^{n-1}X_1,\quad n=1,2,\cdots\]
이 성립함을 확인하시오.

\prob{(1,4)}
\[
B=
\begin{bmatrix}
1&-2\\3&2
\end{bmatrix}
\]
일 때,
\(AB=BA\)인 이차 정사각행렬 \(A\)를 모두 구하시오.

\prob{(1,4)}
다음 조건을 만족하는 이차 정사각행렬 \(A\)를 모두 구하시오.\\
(1)
\(
A^2=
\begin{bmatrix}
x&0\\0&y
\end{bmatrix}
\) (\(x,y>0\))\\
(2)
\(
A^2=
\begin{bmatrix}
x&x\\x&x
\end{bmatrix}
\) (\(x>0\))

\prob{(1,4)}
대각행렬은 \(a_{ij}=0(i\neq j)\)인 행렬 \(A=[a_{ij}]\)를 말한다.
또 위삼각행렬은 \(a_{ij}=0(i>j)\)인 행렬 \(A=[a_{ij}]\)를 말한다.
\(A\), \(B\)는 모두 정사각행렬일 때, 다음 명제의 참/거짓을 판별하시오.\\
(1) \(B\)가 대각행렬이면 \(AB\)도 대각행렬이다.\\
(2) \(AB\)가 대각행렬이면 \(A\) 혹은 \(B\)가 대각행렬이다.\\
(3) \(A\)와 \(B\)가 모두 위삼각 행렬이면 \(AB\)도 위삼각행렬이다.\\
(4) \(AB\)가 위삼각행렬이면 \(A\) 혹은 \(B\)가 위삼각행렬이다.\\
(5) \(A\)의 열들이 일차독립이면 \(AB\)의 열들도 일차독립이다.

\prob{(1,4)}\label{mul. el. matrix on the right}
각각의 \(B\)와 \(M\)에 대해 \(B=AM\)이 성립할 때, \(A\)를 구하시오.

(1)
\[
B=
\begin{bmatrix}
1&4&2\\
2&0&1\\
-2&1&3
\end{bmatrix}
,\quad
M=
\begin{bmatrix}
1&0&0\\
0&1&0\\
2&0&1
\end{bmatrix}
\]

(2)
\[
B=
\begin{bmatrix}
4&0\\
1&-2\\
0&6
\end{bmatrix}
,\quad
M=
\begin{bmatrix}
1&-2\\
0&1
\end{bmatrix}
\]

(3)
\[
B=
\begin{bmatrix}
1&0&2\\
2&-4&7
\end{bmatrix}
,\quad
M=
\begin{bmatrix}
1&0&0\\
0&0&1\\
0&1&0
\end{bmatrix}
\]

(4)
\[
B=
\begin{bmatrix}
2&3&1&0\\
1&0&3&1\\
1&1&-1&7
\end{bmatrix}
,\quad
M=
\begin{bmatrix}
1&0&0&0\\
0&1&0&0\\
0&0&-\frac12&0\\
0&0&0&1\\
\end{bmatrix}
\]

\prob{(1,5)}
정사각행렬 \(A\)를 \(A=LU\)꼴로 분해하시오.

(1)
\[A=
\begin{bmatrix}
1&3\\
4&-2
\end{bmatrix}
\]

(2)
\[A=
\begin{bmatrix}
3&0\\
-1&2
\end{bmatrix}
\]

(3)
\[A=
\begin{bmatrix}
1&-3&-2\\
3&2&0\\
-1&4&2
\end{bmatrix}
\]

(4)
\[A=
\begin{bmatrix}
2&0&4\\
0&4&-8\\
3&-1&2
\end{bmatrix}
\]

(5)
\[A=
\begin{bmatrix}
-1&2&-3&0\\
0&-2&1&-1\\
2&4&0&3\\
0&0&0&5
\end{bmatrix}
\]

(6)
\[A=
\begin{bmatrix}
2&0&1&1\\
-8&1&0&3\\
6&3&1&10\\
4&-2&-2&0
\end{bmatrix}
\]

\prob{(1,5)}
정사각행렬이 아닌 다음 행렬 \(A\)를 \(A=LU\)꼴로 분해하시오.

(1)
\[A=
\begin{bmatrix}
1&-1&-2&0\\
2&0&3&-2\\
0&1&1&-1
\end{bmatrix}
\]

(2)
\[A=
\begin{bmatrix}
0&-1&3&0&3\\
0&1&1&2&0\\
0&2&-1&0&-4
\end{bmatrix}
\]

(3)
\[A=
\begin{bmatrix}
1&0&1\\
2&2&0\\
3&4&-2\\
1&-4&1
\end{bmatrix}
\]

\prob{(1,5)}
다음 행렬 \(A\)에 대해 \(PA=LU\)를 만족하는 \(P\), \(L\), \(U\)를 구하시오.

(1)
\[A=
\begin{bmatrix}
0&2\\
2&-3
\end{bmatrix}
\]

(2)
\[A=
\begin{bmatrix}
0&2&1\\
-1&-3&2\\
1&1&-1
\end{bmatrix}
\]

(3)
\[A=
\begin{bmatrix}
1&-2&-3\\
-1&2&5\\
2&0&1
\end{bmatrix}
\]

(4)
\[A=
\begin{bmatrix}
0&1&3&4\\
0&2&1&-1\\
-1&4&1&5
\end{bmatrix}
\]

(5)
\[A=
\begin{bmatrix}
0&0&1&1\\
1&2&-1&0\\
1&-1&3&2\\
0&3&1&-2
\end{bmatrix}
\]

\prob{(1,5)}
다음 명제의 참/거짓을 판별하시오.

\(L=[l_{ij}]_{n\times n}\)이 아래삼각(=lower triangular)이고 \(i=j\)이면 \(l_{ij}\neq0\)일 때 \(L\)이 가역행렬이며 \(L^{-1}\)도 아래삼각이다.

\prob{(1,6)}
다음 행렬들 중 가역행렬인 것을 고르시오.

(1)
\[A=
\begin{bmatrix}
2&-6&6\\
-5&13&1\\
-2&4&10
\end{bmatrix}
\]

(2)
\[A=
\begin{bmatrix}
4&4&4\\
3&4&2\\
-6&1&7
\end{bmatrix}
\]

(3)
\[A=
\begin{bmatrix}
5&2&7\\
2&1&0\\
2&9&3
\end{bmatrix}
\]

(4)
\[A=
\begin{bmatrix}
2&1&5\\
0&3&9\\
4&-4&-8
\end{bmatrix}
\]

\prob{(1,6)}
다음 가역행렬들의 역행렬을 구하시오.

(1)
\[A=
\begin{bmatrix}
2&4\\
3&4
\end{bmatrix}
\]

(2)
\[A=
\begin{bmatrix}
-1&1\\
1&0
\end{bmatrix}
\]

(3)
\[A=
\begin{bmatrix}
3&4\\
2&1
\end{bmatrix}
\]

(4)
\[A=
\begin{bmatrix}
0&3\\
-2&6
\end{bmatrix}
\]

(5)
\[A=
\begin{bmatrix}
2&0&0\\
0&-1&0\\
0&0&3
\end{bmatrix}
\]

(6)
\[A=
\begin{bmatrix}
5&0&0&0\\
0&-6&0&0\\
0&0&3&0\\
0&0&0&-4
\end{bmatrix}
\]

\prob{(1,6)}
다음 가역행렬들의 역행렬을 구하시오.

(1)
\[A=
\begin{bmatrix}
1&2&1\\
1&0&1\\
0&1&-1
\end{bmatrix}
\]

(2)
\[A=
\begin{bmatrix}
-4&-5&3\\
3&3&-2\\
-1&-1&1
\end{bmatrix}
\]

(3)
\[A=
\begin{bmatrix}
1&0&1\\
1&1&2\\
3&4&-2
\end{bmatrix}
\]

(4)
\[A=
\begin{bmatrix}
1&2&1\\
-1&-1&1\\
0&1&3
\end{bmatrix}
\]

\prob{(1,6), \#11}
대칭행렬(symmetric matrix)은 \(A^T=A\)인 행렬을 말한다.
교대대칭행렬(skew-symmetric matrix)은 \(A^T=-A\)인 행렬을 말한다.
이때, 다음 명제들의 참/거짓을 판별하시오\\
(1) \(A\), \(B\)가 대칭행렬이면 \(AB\)도 대칭행렬이다.\\
(2) \(A\), \(B\)가 대칭행렬이면 \(A+B\)도 대칭행렬이다.\\
(3) \(A\), \(B\)가 교대대칭행렬이면 \(AB\)도 대칭행렬이다.\\
(4) \(A\), \(B\)가 교대대칭행렬이면 \(A+B\)도 대칭행렬이다.\\
(5) \(A\), \(B\)가 대칭행렬이면 \(AB=BA\)이다.\\
(6) \(A=[a_{ij}]\)가 대칭행렬이면 \(a_{ii}=a_{jj}\)이다.(대각 성분들이 일정한 값을 가진다.)\\
(7) \(A=[a_{ij}]\)가 교대대칭행렬이면 \(a_{ii}=a_{jj}\)이다.\\
(8) \(AA^T=\boldsymbol0\)이면 \(A=\boldsymbol0\)이다.\\
(9) \(A\)가 대칭행렬이면 \(A^k\)도 대칭행렬이다.(\(k\)는 자연수)\\
(10) \(A\)가 대칭행렬이고, \(A^{-1}\)이 존재하면 \(A^{-1}\)도 대칭행렬이다.\\
(11) \(A\)가 교대대칭행렬이고, \(A^{-1}\)이 존재하면 \(A^{-1}\)도 교대대칭행렬이다.\\
(12) \(A=[a_{ij}]_{n\times n}\)이고 \(a_{ik}=a_{ij}+a_{jk}\)이면 \(A\)는 교대대칭행렬이다.

\prob{(1,6) \#11}
\(A\)가 정사각행렬이고 \(A=[a_{ij}]_{n\times n}\)일 때 다음 물음에 답하시오.\\
(1) \(B=[b_{ij}]_{n\times n}\)이고 \(b_{ij}=a_{ij}+a_{ji}\)일 때 \(B\)는 무슨 행렬인가?\\
(2) \(B=[b_{ij}]_{n\times n}\)이고 \(b_{ij}=a_{ij}-a_{ji}\)일 때 \(B\)는 무슨 행렬인가?\\
(각각의 경우에 \(B\)는 대칭행렬이거나 교대대칭행렬이다.)

\end{document}