\documentclass{article}
\usepackage{amsmath,amssymb,amsthm,kotex,mdframed,paralist,chngcntr}

\newcounter{num}
\newcommand{\defi}[1]
{\bigskip\noindent\refstepcounter{num}\textbf{정의 \arabic{num}) #1}\par}
\newcommand{\theo}[1]
{\bigskip\noindent\refstepcounter{num}\textbf{정리 \arabic{num}) #1}\par}
\newcommand{\exam}[1]
{\bigskip\noindent\refstepcounter{num}\textbf{예시 \arabic{num}) #1}\par}
\newcommand{\prob}[1]
{\bigskip\noindent\refstepcounter{num}\textbf{문제 \arabic{num}) #1}\par}


\renewcommand{\proofname}{증명)}
\newcommand{\ov}[1]{\ensuremath{\overline{#1}}}
\counterwithout{subsection}{section}


%%%
\begin{document}

\title{준수 : 03 사원수}
\author{}
\date{\today}
\maketitle
\tableofcontents
\newpage

%%
\subsection{군(群, Group)}
%
\defi{군(群, Group)}\label{group}
다음 조건들을 만족시키면 집합 \(G\)는 연산 \(\circ\)에 대한 \textbf{군}이다.
\begin{enumerate}
\item
\(a\in G\), \(b\in G\)이면 \(a\circ b\in G\)이다.
\item
\(a\in G\), \(b\in G\), \(c\in G\)이면 \((a\circ b)\circ c=a\circ(b\circ c)\)이다.
\item
임의의 \(a\in G\)에 대해 \(a\circ e=e\circ a=a\)를 만족시키는 \(e\in G\)가 존재한다.
\item
임의의 \(a\in G\)에 대해 \(a\circ x=x\circ a=e\)를 만족시키는 \(x\in G\)가 존재한다.
\end{enumerate}

%
\defi{가환군(可換群, Abelian Group)}\label{abelian group}
군 \(G\)가 다음 추가조건을 만족시키면 \textbf{가환군}이라고 부른다.
\begin{enumerate}
\item[5. ]
\(a\in G\), \(b\in G\)이면 \(a\circ b=b\circ a\)이다.
\end{enumerate}

%
\defi{}
정의 \ref{group}, \ref{abelian group}에서 1번 조건은, `\(G\)가 연산 \(\circ\)에 대해 닫혀있다.'라고도 말할 수 있다.
2번 조건은 \textbf{결합법칙(associative law)}이라고 부른다.
3번 조건을 만족시키는 원소 \(e\)는 \textbf{항등원}이라고 부른다.
4번 조건을 만족시키는 원소 \(x\)는 \(a\)의 \textbf{역원}이라고 부르며, \(a^{-1}\)이라고 쓴다.
5번 조건은 \textbf{교환법칙(commutative law)}이라고 부른다.

%
\exam{}
(1)
자연수의 집합 \(\mathbb N\)은 덧셈에 대해 닫혀있다.
즉 \(a\in\mathbb N\)이고 \(b\in\mathbb N\)이면 \(a+b\in\mathbb N\)이다.
또 결합법칙이 성립한다.
즉 \(a+(b+c)=(a+b)+c\)이다.
하지만 항등원이 존재하지 않는다.
즉 임의의 \(a\in\mathbb N\)에 대해 \(a+e=e+a=a\)를 만족시키는 \(e\)의 값은 \(0\)인데 \(0\not\in\mathbb N\)이기 때문이다.

따라서 \(\mathbb N\)은 덧셈에 대한 군이 아니다.
\medskip

(2)
정수의 집합 \(\mathbb Z\)는 덧셈에 대해 닫혀있고 결합법칙이 성립한다.
또 항등원이 존재한다.
\(0\in\mathbb Z\)이기 때문이다.
또한 임의의 정수 \(a\)에 대해서 \(a+x=0\)을 만족시키는 \(x\)는 \(-a\)이며, \(-a\in\mathbb Z\)이다.

따라서 \(\mathbb Z\)는 덧셈에 대한 군이다.
또한 \(\mathbb Z\)는 덧셈에 대한 교환법칙을 만족시키므로 가환군이라고 볼 수 있다.
\medskip

(3)
마찬가지로 유리수의 집합 \(\mathbb Q\), 실수의 집합 \(\mathbb R\), 복소수의 집합 \(\mathbb C\)도 덧셈에 대한 가환군이라는 것을 쉽게 보일 수 있다.
\medskip

(4)
정수의 집합 \(\mathbb Z\)는 곱셈에 대해 닫혀있다.
즉 \(ab\in\mathbb Z\)이다.
또 결합법칙이 성립한다.
즉 \(a(bc)=(ab)c\)이다.
또한 항등원이 존재한다.
즉 임의의 \(a\in\mathbb Z\)에 대해, \(ae=ea=a\)를 만족하는 \(e\)의 값은 \(1\)인데 \(1\in\mathbb Z\)이다.
하지만 모든 \(\mathbb Z\)의 원소에 대해 역원이 존재하지는 않는다.
예를 들어 \(2\in\mathbb Z\)에 대한 역원 \(x\)는 \(2x=1\)을 만족시키는 값으로서 \(x=\frac12\)인데 \(\frac12\not\in\mathbb Z\)이기 때문이다.
또한  곱셈에 대한 \(0\)의 역원도 존재하지 않는다.

따라서 \(\mathbb Z\)는 곱셈에 대한 군이 아니다.
\medskip

(5)
유리수의 집합에서 0을 뺀 집합인 \(\mathbb Q-\{0\}\)는 곱셈에 대해 닫혀있고, 결합법칙이 성립하며, 항등원이 존재한다.
이 때의 항등원은 \(1\)이다.
또 임의의 \(a=\frac pq\in\mathbb Q-\{0\}\)에 대해 \(p\neq0\)이고 \(q\neq0\)이므로 \(x\)를 \(x=\frac qp\)로 잡으면
\[ax=1\]
이 성립하고 이때 \(\frac qp\in\mathbb Q-\{0\}\)이다.

따라서 \(\mathbb Q-\{0\}\)은 곱셈에 대한 군이며, 교환법칙도 만족하므로 가환군이다.
\medskip


(6)
마찬가지로 \(\mathbb R-\{0\}\), \(\mathbb C-\{0\}\)도 곱셈에 대한 가환군이라는 것을 쉽게 밝힐 수 있다.

%%
%\exam{}
%실수의 집합 \(\mathbb R\)에 대해 연산 \(\circ\)를
%\[a\circ b=a+b+ab\]
%로 정의하자.
%그러면 \(a\circ b\in\mathbb R\)이고 \((a\circ b)\circ c=a\circ(b\circ c)\)이며, \(a\circ b=b\circ a\)이다.
%따라서 \(\mathbb R\)은 \(\circ\)에 대해 닫혀있고 결합법칙과 교환법칙이 성립한다.
%
%항등원이 존재하려면 \(a\)의 값에 상관없이
%\[a\circ e=a\]
%를 만족시키는 \(e\)가 존재해야 하고 또 \(e\)가 실수여야 한다.
%즉
%\[a+e+ae=a\]
%이고
%\[e(1+a)=0\]
%이다.
%이 식이 \(a\)의 값에 상관없이 성립해야 하므로 \(e=0\)이며 \(0\)은 실수이다.
%따라서 항등원이 존재한다.
%
%마지막 조건을 따지기 위해서는 임의의 실수 \(a\)에 대해서 \(a\circ x=e\)를 만족시키는 \(x\)가 존재하는 지 살펴야 한다.
%식을 풀면
%\[a+x+ax=0\]
%이다.
%좀 더 정리하면
%\[x=-\frac{a}{a+1}\]
%이다.

%
\exam{}
집합 \(A\)를 정의역과 공역이 모두 \(\{1,2,3\}\)인 일대일 대응 함수들의 집합이라고 하자.
그러면 \(A\)는 `함수의 합성'이라는 연산에 대해 군을 이룬다.
즉
\begin{enumerate}
\item
\(f\in A\), \(g\in A\)이면 \(f\circ g\in A\)이다.
\item
\(f\in A\), \(g\in A\), \(h\in A\)이면 \((f\circ g)\circ h=f\circ(g\circ h)\)이다.
\item
임의의 \(f\in A\)에 대해 \(f\circ I=I\circ f=f\)이다.
\item
임의의 \(f\in A\)에 대해 \(f\circ f^{-1}=f^{-1}\circ f=I\)를 만족시키는 \(f^{-1}\in A\)가 존재한다.
\end{enumerate}

하지만 일반적으로 \(f\circ g\neq g\circ f\)이므로 가환군은 아니다.

\subsection{환(環, Ring)과 체(體, Field)}

%
\defi{환(環, Ring)}\label{ring}
다음 조건들을 만족시키면 집합 \(R\)은 두 연산 \(+\), \(\cdot\)에 대해 \textbf{환}이다.
\begin{enumerate}
\item
\(a\in R\), \(b\in R\)이면 \(a+ b\in R\)이다.
\item
\(a\in R\), \(b\in R\), \(c\in R\)이면 \((a+ b)+ c=a+(b+ c)\)이다.
\item
임의의 \(a\in R\)에 대해 \(a+ 0=0+ a=a\)를 만족시키는 \(0\in R\)가 존재한다.
\item
임의의 \(a\in R\)에 대해 \(a+ x=0\)를 만족시키는 \(x\in R\)가 존재한다.
\item
\(a\in R\), \(b\in R\)이면 \(a+ b=b+ a\)이다.
\item
\(a\in R\), \(b\in R\)이면 \(a\cdot b\in R\)이다.
\item
\(a\in R\), \(b\in R\), \(c\in R\)이면 \((a\cdot b)\cdot c=a\cdot(b\cdot c)\)이다.
\item
\(a\in R\), \(b\in R\), \(c\in R\)이면  \((a+b)\cdot c=a\cdot c+b\cdot c\)이고 \(a\cdot(b+c)=a\cdot b+a\cdot c\)이다.
\end{enumerate}

정의 \ref{ring}의 1--5번 조건에 의해 \(R\)은 첫 번째 연산인 `\(+\)에 대해 가환군이다.
이때의 항등원은 \(0\)이라고 쓴다.
4번 조건의 첫번째 연산 `\(+\)에 대한 역원 \(x\)는 정의 \ref{group}에서와는 달리 \(-a\)라고 표기한다.
8번 조건은 \textbf{분배법칙}이라고 부른다.
가끔 \textbf{좌분배법칙}과 \textbf{우분배법칙}으로 나누어서 부르기도 한다.
\(a\cdot b\)는 간단히 \(ab\)라고 표기하기도 한다.

%
\defi{나눗셈환(Division Ring)}
환 \(R\)이 다음 추가 조건을 만족시키면 \textbf{나눗셈환}이라고 부른다.
\begin{enumerate}
\item[9.]
임의의 \(a\in R\)에 대해 \(a\cdot1=1\cdot a=a\)를 만족시키는 \(1\in R\)가 존재한다.
\item[10.]
임의의 \(a(\neq0)\in R\)에 대해 \(ax=xa=1\)를 만족시키는 \(x\in R\)가 존재한다.
\end{enumerate}

\(R\)의 두 번째 연산 \(\cdot\)에 대한 항등원은 \(1\)로 표기한다.
마지막 조건을 만족시키는 \(x\), 즉 \(a\)의 두 번째 연산 \(\cdot\)에 대한 역원은 \(a^{-1}\)이라고 표기한다.

%
\exam{}
정수들의 집합 \(\mathbb Z\)는 환이지만, 나눗셈환은 아니다.
짝수인 정수들의 집합은 환이지만 나눗셈환이 아니며, 심지어 곱셈에 대한 항등원도 존재하지 않는다.

%
\defi{가환환(可換環, Commutative Ring)}
환 \(R\)이 다음 추가조건을 만족시키면 \textbf{가환환}이라고 부른다.
\begin{enumerate}
\item[9*. ]
\(a\in R\), \(b\in R\)이면 \(a\cdot b=b\cdot a\)이다.
\end{enumerate}

%
\exam{}
사원수의 집합 \(\mathbb H\)는 나눗셈환이지만 가환환은 아니다.
이차정사각행렬들의 집합 또한 가환환이 아니며 또한 나눗셈환도 아니다.

%
\defi{체(體, Field)}
나눗셈환이면서 동시에 가환환인 것을 \textbf{체}라고 부른다.

%
\exam{}
유리수의 집합 \(\mathbb Q\), 실수의 집합 \(\mathbb R\), 복소수의 집합 \(\mathbb C\) 등은 환이면서 곱셈의 교환법칙이 성립하고 곱셈의 항등원과 역원이 존재하므로 체이다.

%%
\subsection{\(\mathbb C\)는 체이다.}
\theo{}
\(\mathbb C\)를 복소수의 집합이라고 하자.
즉 \(\mathbb C=\{a+bi\:|\:a\in\mathbb R,b\in\mathbb R\}\)이다.
\(\mathbb C\)에 두 연산 \(+\)와 \(\cdot\)을
\begin{equation}
(a+bi)+(c+di)=(a+c)+(b+d)i,
\end{equation}
\begin{equation}
(a+bi)(c+di)=(ac-bd)+(ad+bc)i
\end{equation}
로 정의하자.

그러면 \(\mathbb C\)는 체이다.
(실수의 집합인 \(\mathbb R\)이 체라는 것을 가정하자.)

\begin{proof}
체가 되기 위한 조건을 하나 하나 따져보면 된다.

1. \(z_1=a+bi\in\mathbb C\), \(z_2=c+di\in\mathbb C\)라고 하면, (1)에 의해 \((a+bi)+(c+di)=(a+c)+(b+d)i\)이다.
\(\mathbb R\)이 환이므로 \(\mathbb R\)은 덧셈에 대해 닫혀있다.
따라서 \(a+c\in\mathbb R\), \(b+d\in\mathbb R\)이다.
그러므로 \(z_1+z_2\in\mathbb C\)이다.

3. \(0+0i\)를 간단히 \(0\)이라고 쓰자.
그러면, 임의의 \(z=a+bi\in\mathbb C\)에 대해 \(z+0=(a+bi)+(0+0i)=(a+0)+(b+0)i\)이다.
\(0\)이 실수의 덧셈에 대한 항등원이므로 \(z+0=a+bi=z\)이다.
마찬가지로 \(0+z=z\)이다.
따라서 덧셈에 대한 항등원 \(0\)이 존재한다.

4. 임의의 \(z=a+bi\in\mathbb C\)에 대해 \(x=(-a)+(-b)i\)라고 하면 \(z+x=(a+bi)+[(-a)+(-b)i]=[a+(-a)]+[b+(-b)]i=0+0i=0\)이다.

6. 1번과 마찬가지로 하면 된다.

9. 임의의 \(z=a+bi\)에 대해 \(z\cdot1=(a+bi)(1+0i)=(a\cdot1-b\cdot0)+(a\cdot0+b\cdot1)i=a+bi=z\)이다.
마찬가지로 \(1\cdot z=z\)이다.
따라서 곱셈에 대한 항등원 \(1\)이 존재한다.

10. 임의의 \(z=a+bi\neq0\)에 대해 \(x=\frac{a-bi}{a^2+b^2}\)라고 하면 \(zx=(a+bi)\frac{a-bi}{a^2+b^2}=1\)이다.

나머지 것들은 다음과 같이 증명된다.
(실수가 체라는 사실이 중요한 역할을 한다는 것을 관찰하여라.)

\(z_1=a+bi\), \(z_2=c+di\), \(z_3=e+fi\), \(a\), \(b\), \(c\), \(d\), \(e\), \(f\)는 실수라고 가정하자.

2.
\begin{align*}
(z_1+z_2)+z_3
&=[(a+bi)+(c+di)]+(e+fi)\\
&=[(a+c)+(b+d)i]+(e+fi)\\
&=[(a+c)+e]+[(b+d)+f]i\\
&=[a+(c+e)]+[b+(d+f)]i\\
&=a+bi+[(c+e)+(d+f)i]\\
&=a+bi+[(c+di)+(e+fi)]\\
&=z_1+(z_2+z_3).
\end{align*}

5.
\begin{align*}
z_1+z_2
&=(a+bi)+(c+di)\\
&=(a+c)+(b+d)i\\
&=(c+a)+(d+b)i\\
&=(c+di)+(a+bi)\\
&=z_2+z_1.
\end{align*}

7.
\begin{align*}
(z_1z_2)z_3
&=[(a+bi)(c+di)](e+fi)\\
&=[(ac-bd)+(ad+bc)i](e+fi)\\
&=[(ac-bd)e-(ad+bc)f]+[(ac-bd)f+(ad+bc)e]i\\
&=(ace-bde-adf-bcf)+(acf-bdf+ade+bce)i
\end{align*}
\begin{align*}
z_1(z_2z_3)
&=(a+bi)[(c+di)(e+fi)]\\
&=(a+bi)[(ce-df)+(cf+de)i]\\
&=[a(ce-df)-b(cf+de)]+[a(cf+de)+b(ce-df)]i\\
&=(ace-adf-bcf-bde)+(acf+ade+bce-bdf)i
\end{align*}
따라서 좌변과 우변이 같다.

8.
\begin{align*}
z_1(z_2+z_3)
&=(a+bi)[(c+di)+(e+fi)]\\
&=(a+bi)[(c+e)+(d+f)i]\\
&=[a(c+e)-b(d+f)]+[a(d+f)+b(c+e)]i\\
&=(ac+ae-bd-bf)+(ad+af+bc+be)i\\
&=[(ac-bd)+(ae-bf)]+[(ad+bc)+(af+be)]i\\
&=[(ac-bd)+(ad+bc)i]+[(ae-bf)+(af+be)i]\\
&=(a+bi)(c+di)+(a+bi)(e+fi)\\
&=z_1z_2+z_1z_3.
\end{align*}
(곱셈에 대한 교환법칙이 성립하므로 두 번째 등식은 증명할 필요가 없다.)

9*.
\begin{align*}
z_1z_2
&=(a+bi)(c+di)\\
&=(ac-bd)+(ad+bc)i\\
&=(ca-db)+(cb+da)i\\
&=(c+di)(a+bi)\\
&=z_2z_1.
\end{align*}

체가 되기 위한 조건들을 모두 만족시키므로 \(\mathbb C\)는 체이다.

\end{proof}

%%
\subsection{\(\mathbb H\)는 나눗셈환이다.}

%
\defi{사원수(Quaternion)}
실수 \(a\), \(b\), \(c\), \(d\)에 대해
\begin{equation}
a+bi+cj+dk
\end{equation}
처럼 표현되는 숫자를 \textbf{사원수(Quaternion)}라고 부른다.

집합 \(\mathbb H\)를 사원수들의 집합이라고 하자.
즉
\[\mathbb H=\{a+bi+cj+dk\:|\:a,b,c,d\in\mathbb R\}\]
로 정의하자.

\(\mathbb H\)에 두 연산 \(+\)와 \(\cdot\)을
\begin{multline}
(a_1+b_1i+c_1j+d_1k)+(a_2+b_2i+c_2j+d_2k)\\=(a_1+a_2)+(b_1+b_2)i+(c_1+c_2)j+(d_1+d_2)k
\end{multline}
\begin{multline}
(a_1+b_1i+c_1j+d_1k)
(a_2+b_2i+c_2j+d_2k)
\\=(a_1a_2-b_1b_2-c_1c_2-d_1d_2)
+(a_1b_2+b_1a_2+c_1d_2-d_1c_2)i
\\+(a_1c_2-b_1d_2+c_1a_2+d_1b_2)j
+(a_1d_2+b_1c_2-c_1b_2+d_1a_2)k
\end{multline}
로 정의하자.

그러면 \(\mathbb H\)는 나눗셈환이다.

\begin{proof}
1번 조건과 6번 조건은 정의로부터 당연하다.
2번부터 5번까지의 조건도 거의 당연하다.
\(\mathbb C\)가 체라는 사실을 증명할 때와 비슷하게 하면 될 것이다.
9번 조건은 (5)번 식에 대입하면 쉽게 얻어진다.

마지막으로 \(h_1,h_2,h_3\in\mathbb H\)일 때,
\begin{enumerate}\setcounter{enumi}{6}
\item
\((h_1h_2)h_3=h_1(h_2h_3)\)
\item
\((h_1+h_2)h_3=h_1h_3+h_2h_3\), \(h_1(h_2+h_3)=h_1h_2+h_1h_3\)
\end{enumerate}
이고 임의의 \(h\in \mathbb H\)에 대해,
\begin{enumerate}\setcounter{enumi}{9}
\item
\(hx=1\)을 만족시키는 \(x\in H\)가 존재한다
\end{enumerate}
는 것을 증명해야 한다.

\(\mathbb C\)가 체라는 것을 증명했던 방법과 똑같이 직접적인 증명방법으로 접근하면, 가능은 하겠지만 굉장히 복잡하게 될 것이다.
가령 7번의 결합법칙을 증명하려면, 세 개의 사원수를 곱했을 때 나오는 64개의 항을 정리해야 할 것이다.
그러니 다른 방법으로 사원수를 표기하여 계산을 간단히 하자.

사원수를
\begin{equation}
a+bi+cj+dk=(x,y)\quad(\text{단, }x=a+bi,y=c+di)
\end{equation}
와 같이 나타내자.
즉 복소수의 순서쌍으로 나타내자.
예를 들어
\begin{align*}
1+2i+3j+4k	&=(1+2i, 3+4i)\\
1+3j		&=(1,3)\\
-2+4k		&=(-2,4i)\\
k		&=(0,i)\\
-2j		&=(0,-2)\\
0		&=(0,0)
\end{align*}
등으로 나타내자.

그리고 두 복소수 순서쌍 사이의 덧셈을
\begin{equation}
(x_1,y_1)+(x_2,y_2)=(x_1+x_2,y_1+y_2)
\end{equation}
로 정의하면(\(x_1=a_1+b_1i\), \(y_1=c_1+d_1i\), \(x_2=a_2+b_2i\), \(y_2=c_2+d_2i\))
%\begin{multline}
%(a_1+b_1i,c_1+d_1i)+(a_2+b_2i,c_2+d_2i)
%\\=\big((a_1+a_2)+(b_1+b_2)i,(c_1+c_2)+(d_1+d_2)i\big)
%\end{multline}
%으로 정의하면
%\begin{align*}
%(x_1+x_2,y_1+y_2)
%&=\big((a_1+a_2)+(b_1+b_2)i,(c_1+c_2)+(d_1+d_2)i\big)\\
%&=(a_1+a_2)+(b_1+b_2)i+(c_1+c_2)j+(d_1+d_2)k
%\end{align*}
원래 사원수의 덧셈인 (4)번 식과 정확히 일치한다.

또한 두 복소수 순서쌍 사이의 곱셈을

\begin{equation}
(x_1,y_1)(x_2,y_2)=(x_1x_2-y_2\overline{y_1},\overline{x_1}y_2+y_1x_2)
\end{equation}
로 정의하면,
\begin{align*}
&(x_1x_2-y_2\overline{y_1},\overline{x_1}y_2+y_1x_2)\\
=&\Big((a_1+b_1i)(a_2+b_2i)-(c_2+d_2i)(c_1-d_1i),\\
&\qquad\qquad\qquad(a_1-b_1i)(c_2+d_2i)+(c_1+d_1i)(a_2+b_2i)\Big)\\
=&\Big((a_1a_2-b_1b_2-c_1c_2-d_1d_2)+(a_1b_2+b_1a_2+c_1d_2-d_1c_2)i,\\
&\qquad\qquad\qquad(a_1c_2-b_1d_2+c_1a_2+d_1b_2)+(a_1d_2+b_1c_2-c_1b_2+d_1a_2)i\Big)\\
=&(a_1a_2-b_1b_2-c_1c_2-d_1d_2)+(a_1b_2+b_1a_2+c_1d_2-d_1c_2)i\\
+&(a_1c_2-b_1d_2+c_1a_2+d_1b_2)j+(a_1d_2+b_1c_2-c_1b_2+d_1a_2)k
\end{align*}
이 되어 원래 사원수의 곱셈인 (5)번 식과 일치한다.

따라서 사원수를 (3)번 식과 (4)번 식, (5)번식으로 덧셈과 곱셈을 생각하지 말고, 사원수를 (6)번 식과 (7)번 식, (8)번 식을 통해 덧셈과 곱셈을 생각해도 아무 문제가 없다.

그러므로

\begin{align*}
(h_1h_2)h_3
&=\Big((x_1,y_1)(x_2,y_2)\Big)(x_3,y_3)\\
&=\Big(x_1x_2-y_2\ov{y_1},\ov{x_1}y_2+y_1x_2\Big)(x_3,y_3)\\
&=\Big((x_1x_2-y_2\ov{y_1})x_3-y_3\ov{(\ov{x_1}y_2+y_1x_2)},\quad
\ov{(x_1x_2-y_2\ov{y_1})}y_3+(\ov{x_1}y_2+y_1x_2)x_3\Big)\\
&=\Big((x_1x_2-y_2\ov{y_1})x_3-y_3(x_1\ov{y_2}+\ov{y_1}\ov{x_2}),\quad
(\ov{x_1}\ov{x_2}-\ov{y_2}y_1)y_3+(\ov{x_1}y_2+y_1x_2)x_3\Big)\\
&=\Big(x_1x_2x_3-\ov{y_1}y_2x_3-x_1\ov{y_2}y_3-\ov{y_1}\ov{x_2}y_3,\quad
\ov{x_1}\ov{x_2}y_3-y_1\ov{y_2}y_3+\ov{x_1}y_2x_3+y_1x_2x_3\Big)\\
&=\Big(x_1x_2x_3-x_1\ov{y_2}y_3-\ov{y_1}\ov{x_2}y_3-\ov{y_1}y_2x_3,\quad
\ov{x_1}\ov{x_2}y_3+\ov{x_1}y_2x_3+y_1x_2x_3-y_1\ov{y_2}y_3\Big)\\
&=\Big(x_1(x_2x_3-\ov{y_2}y_3)-(\ov{x_2}y_3-y_2x_3)\ov{y_1},\quad
\ov{x_1}(\ov{x_2}y_3+y_2x_3)+y_1(x_2x_3-\ov{y_2}y_3)\Big)\\
&=(x_1,y_1)\Big(x_2x_3-\ov{y_2}y_3,\ov{x_2}y_3+y_2x_3\Big)\\
&=h_1(h_2h_3).
\end{align*}

이기 때문에 7번 조건이 성립한다.
8번 조건도 마찬가지의 방법으로 증명할 수 있을 것이다.

10번 조건의 경우, 임의로 사원수 \((x_1,y_1)\neq(0,0)\)가 주어졌을 때, \((x_1,y_1)(x_2,y_2)=1\)을 풀면
\begin{gather*}
(x_1x_2-y_2\ov{y_1},\ov{x_1}y_2+y_1x_2)=(1,0)
\\
\begin{cases}
x_1x_2-\ov{y_1}y_2=1\\
y_1x_2+\ov{x_1}y_2=0
\end{cases}
\\
\begin{cases}
x_1\ov{x_1}x_2-\ov{x_1}\ov{y_1}y_2=\ov{x_1}\\
y_1\ov{y_1}x_2+\ov{x_1}\ov{y_1}y_2=0
\end{cases}
,\qquad
\begin{cases}
x_1y_1x_2-y_1\ov{y_1}y_2=y_1\\
x_1y_1x_2+x_1\ov{x_1}y_2=0
\end{cases}
\\
x_2=\frac{\ov{x_1}}{x_1\ov{x_1}+y_1\ov{y_1}}
,\qquad
y_2=-\frac{y_1}{x_1\ov{x_1}+y_1\ov{y_1}}
\\
x_2=\frac{\ov{x_1}}{{a_1}^2+{b_1}^2+{c_1}^2+{d_1}^2}
,\qquad
y_2=-\frac{y_1}{{a_1}^2+{b_1}^2+{c_1}^2+{d_1}^2}
\end{gather*}
이므로 역원이 항상 존재한다.

\(\mathbb H\)가 나눗셈환이 되기 위한 모든 조건을 만족시키므로 \(\mathbb H\)는 나눗셈 환이다.
\end{proof}
\end{document}