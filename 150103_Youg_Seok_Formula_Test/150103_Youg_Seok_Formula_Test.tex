\documentclass{article}
\usepackage{amsmath,amssymb,amsthm,kotex,paralist,mathrsfs,centernot,marvosym}
%\newcounter{num}[section]
%\newcommand{\defi}[1]
%{\bigskip\noindent\refstepcounter{num}\textbf{정의 \arabic{section}. \arabic{num}) #1}\par}
%\newcommand{\theo}[1]
%{\bigskip\noindent\refstepcounter{num}\textbf{정리 \arabic{section}. \arabic{num}) #1}\par}
%\newcommand{\axio}[1]
%{\bigskip\noindent\refstepcounter{num}\textbf{공리 \arabic{section}. \arabic{num}) #1}\par}
%
%\newcommand{\notiff}{%
%  \mathrel{{\ooalign{\hidewidth$\not\phantom{"}$\hidewidth\cr$\iff$}}}}
%\newcommand{\LHS}{\text{LHS}}
%\newcommand{\RHS}{\text{RHS}}
%\newcommand{\irange}{\ensuremath{1\le i\le n}}
%\newcommand{\jrange}{\ensuremath{1\le j\le n}}
%\newcommand{\bb}[2]{\ensuremath{(^{#1}_{#2})}}
%\newcommand{\cc}[2]{\ensuremath{_{#1}C_{#2}}}
%
%\renewcommand{\figurename}{그림.}
%\renewcommand{\proofname}{증명.}
%\renewcommand{\contentsname}{목차}
%\renewcommand\emph{\textbf}

%%%
\begin{document}

\title{인수분해 공식 테스트}
\author{}
\date{\today}
\maketitle
%\tableofcontents

%%

%
1.
실수 \(a\), \(b\), \(x\)에 대해 다음을 전개하여라.
\begin{enumerate}[(1)]
\item
\((a+b)^2=\)
\item
\((a-b)^2=\)
\item
\((a+b)(a-b)=\)
\item
\((x+1)^2=\)
\item
\((x+3)^2=\)
\item
\((x-2)^2=\)
\item
\((x-4)^=\)
\item
\((x+3)(x-3)=\)
\item
\((x-\frac12)(x+\frac12)=\)
\item
\((x+5)(x-5)=\)
\item
\((a+1)(b+1)=\)
\end{enumerate}
\end{document}