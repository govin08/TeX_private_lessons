\documentclass[a4paper]{oblivoir}
\usepackage{amsmath,amssymb,kotex,kswrapfig,mdframed,paralist}
\usepackage{fapapersize}
\usefapapersize{210mm,297mm,20mm,*,20mm,*}

\usepackage{tabto,pifont}
\TabPositions{0.2\textwidth,0.4\textwidth,0.6\textwidth,0.8\textwidth}
\newcommand\tabb[5]{\par\noindent
\ding{172}\:{\ensuremath{#1}}
\tab\ding{173}\:\:{\ensuremath{#2}}
\tab\ding{174}\:\:{\ensuremath{#3}}
\tab\ding{175}\:\:{\ensuremath{#4}}
\tab\ding{176}\:\:{\ensuremath{#5}}}

\usepackage{graphicx}

\pagestyle{empty}

%%% Counters
\newcounter{num}

%%% Commands
\newcommand\prob[1]
{\vs\bigskip\bigskip\par\noindent\stepcounter{num} \textbf{문제 \thenum) #1}\par\noindent}

\newcommand\pb[1]{\ensuremath{\fbox{\phantom{#1}}}}

\newcommand\ba{\ensuremath{\:|\:}}

\newcommand\vs[1]{\vspace{25pt}}

\newcommand\an[1]{\bigskip\par\noindent\textbf{문제 #1)}\par\noindent}

%%% Meta Commands
\let\oldsection\section
\renewcommand\section{\clearpage\oldsection}

\let\emph\textsf

\begin{document}
\begin{center}
\LARGE종현, 추가과제 04
\end{center}
\begin{flushright}
날짜 : 2017년 \(\pb3\)월 \(\pb{10}\)일 \(\pb{월}\)요일
,\qquad
제한시간 : \pb{17년}분
,\qquad
점수 : \pb{20} / \pb{20}
\end{flushright}

%
\prob{}
모평균이 \(36\), 모분산이 \(2\)인 모집단에서 크기가 \(18\)인 표본을 임의추출할 때, 표본평균 \(\overline X\)에 대하여 \(E(\overline X)\sigma(\overline X)\)는?
\tabb48{12}{16}{20}

%
\prob{}
모집단의 확률변수 \(X\)의 확률질량함수가
\[P(X=k)=\frac k{16}\quad(k=1,3,5,7)\]
이다.
이 모집단에서 크기가 9인 표본을 임의추출할 때, 표본평균 \(\overline X\)에 대하여 \(\sigma(12\overline X)\)의 값은?
\tabb{5\sqrt2}{\sqrt{55}}{2\sqrt{15}}{\sqrt{65}}{\sqrt{70}}

%
\prob{}
1, 2, 3의 숫자가 각각 하나씩 적힌 카드가 3장, 2장, 3장씩 들어있는 상자에서 크기가 \(n\)인 표본을 임의추출할 때, 카드에 적힌 숫자의 평균 \(\overline X\)의 분산이 \(\frac1{20}\)이다.
이때 \(n\)의 값은?
\tabb{12}{15}{18}{21}{24}

%
\prob{}
\begin{minipage}{0.6\textwidth}
어느 학교 학생들의 몸무게는 평균 65kg, 표준펀차 5kg인 정규분포를 따른다고 한다.
이 학교 학생들 중에서 25명을 임의추출할 때, 오른쪽 표준정규분포표를 이용하여 몸무게의 평균이 64kg 이상 66kg 이하일 확률을 구하면?
\end{minipage}
\begin{minipage}{0.3\textwidth}
\centering
\begin{tabular}{c|c}
\hline
\(z\)	&\(P(0\le Z\le z)\)\\
\hline
1.0		&0.34\\
1.5		&0.43\\
2.0		&0.48\\
\hline
\end{tabular}
\end{minipage}

\bigskip
\tabb{0.68}{0.72}{0.76}{0.80}{0.84}



%
\prob{}
\begin{minipage}{0.6\textwidth}
정규분포 \(N(250,18^2)\)을 따르는 모집단에서 크기가 \(n\)인 표본을 임의추출할 때, 표본평균 \(\overline X\)에 대하여 \(P(\overline X\ge253)=0.068\)이다.
이때 오른쪽 표준정규분포표를 이용하여 \(n\)의 값을 구하면?
\end{minipage}
\begin{minipage}{0.3\textwidth}
\centering
\begin{tabular}{c|c}
\hline
\(z\)	&\(P(0\le Z\le z)\)\\
\hline
0.5		&0.1915\\
1.0		&0.3413\\
1.5		&0.4332\\
\hline
\end{tabular}
\end{minipage}

\bigskip
\tabb{36}{49}{64}{81}{100}

%
\prob{}
정규분포 \(N(0,9^2)\)을 따르는 모집단에서 크기가 324인 표본을 임의추출할 때, 표본평균 \(\overline X\)에 대하여\\ \(P(\overline X\ge k)\ge0.983\)이 성립하도록 하는 실수 \(k\)의 최댓값은?
(단, \(P(0\le Z\le2.12)=0.483\))
\tabb{-1.06}{-1}{-0.975}{-0.895}{-0.852}

\end{document}