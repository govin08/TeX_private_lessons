\documentclass[a4paper]{oblivoir}
\usepackage{amsmath,amssymb,kotex,kswrapfig,mdframed,paralist}
\usepackage{fapapersize}
\usefapapersize{210mm,297mm,20mm,*,20mm,*}

\usepackage{tabto,pifont}
\TabPositions{0.2\textwidth,0.4\textwidth,0.6\textwidth,0.8\textwidth}
\newcommand\tabb[5]{\par\noindent
\ding{172}\:{\ensuremath{#1}}
\tab\ding{173}\:\:{\ensuremath{#2}}
\tab\ding{174}\:\:{\ensuremath{#3}}
\tab\ding{175}\:\:{\ensuremath{#4}}
\tab\ding{176}\:\:{\ensuremath{#5}}}

\usepackage{graphicx}

%\pagestyle{empty}

%%% Counters
\newcounter{num}

%%% Commands
\newcommand\prob[1]
{\vs\bigskip\bigskip\par\noindent\stepcounter{num} \textbf{문제 \thenum) #1}\par\noindent}

\newcommand\pb[1]{\ensuremath{\fbox{\phantom{#1}}}}

\newcommand\ba{\ensuremath{\:|\:}}

\newcommand\vs[1]{\vspace{40pt}}

\newcommand\an[1]{\bigskip\par\noindent\textbf{문제 #1)}\par\noindent}

%%% Meta Commands
\let\oldsection\section
\renewcommand\section{\clearpage\oldsection}

\let\emph\textsf

\begin{document}
\begin{center}
\LARGE종현, 추가과제 03
\end{center}
\begin{flushright}
날짜 : 2017년 \(\pb3\)월 \(\pb{10}\)일 \(\pb{월}\)요일
,\qquad
제한시간 : \pb{17년}분
,\qquad
점수 : \pb{20} / \pb{20}
\end{flushright}

%
\prob{}
\(\displaystyle\lim_{n\to\infty}\left(1+\frac2x\right)^x\)
의 값은?
\tabb{\frac1e}{\frac1{\sqrt e}}{\sqrt e}{e}{e^2}

%
\prob{}
\(\displaystyle\lim_{x\to0}\frac{\ln(1+x)}{2x}\)의 값은?
\tabb{\frac14}{\frac12}{1}{\sqrt e}{e^2}

%
\prob{}
두 함수 \(f(x)=(x^2+2x)e^{x+1}\), \(g(x)\)에 대하여 함수 \(f(x)\)의 도함수가 \(f'(x)=g(x)e^{x+1}\)일 때, \(g(1)\)의 값은?
\tabb34567

%
\prob{}
곡선 \(y=x^2\ln x\) 위의 점 \((e,e^2)\)에서의 접선의 기울기는?
\tabb{2e}{3e}{2e^2}{3e^2}{2e^3}

%
\prob{}
\(\displaystyle\lim_{x\to2}\left(\frac x2\right)^{\frac1{2-x}}\)의 값은?
\tabb{e}{\sqrt e}1{\frac1{\sqrt e}}{\frac1e}

%
\prob{}
함수 \(y=\ln 2x+3\) 의 그래프 위의 점 \(\left(\frac e2,4\right)\)에서의 접선의 기울기를 구하시오.
\tabb{2e}e1{\frac1e}{\frac2e}

%
\prob{}
\(\displaystyle\lim_{x\to0}xf(x)=6\)일 때, \(\displaystyle\lim_{x\to0}f(x)(e^{2x}-1)\)의 값을 구하시오.
\tabb6{12}{18}{24}{36}
\end{document}