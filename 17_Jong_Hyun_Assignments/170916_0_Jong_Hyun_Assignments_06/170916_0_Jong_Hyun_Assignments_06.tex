\documentclass[a4paper]{oblivoir}
\usepackage{amsmath,amssymb,kotex,kswrapfig,mdframed,paralist,tabu}
\usepackage{fapapersize}
\usefapapersize{210mm,297mm,40mm,*,20mm,*}

\usepackage{tabto,pifont}
\TabPositions{0.2\textwidth,0.4\textwidth,0.6\textwidth,0.8\textwidth}
\newcommand\tabb[5]{\par\noindent
\ding{172}\:{\ensuremath{\displaystyle#1}}
\tab\ding{173}\:\:{\ensuremath{\displaystyle#2}}
\tab\ding{174}\:\:{\ensuremath{\displaystyle#3}}
\tab\ding{175}\:\:{\ensuremath{\displaystyle#4}}
\tab\ding{176}\:\:{\ensuremath{\displaystyle#5}}}

\usepackage{graphicx}

\pagestyle{empty}

%%% Counters
\newcounter{num}
\counterwithout{subsection}{section}

%%% Commands
\newcommand\prob[1]
{\vs\bigskip\bigskip\par\noindent\stepcounter{num} \textbf{문제 \thenum) #1}\par\noindent}

\newcommand\pb[1]{\ensuremath{\fbox{\phantom{#1}}}}

\newcommand\ba{\ensuremath{\:|\:}}

\newcommand\vs[1]{\vspace{25pt}}

\newcommand\an[1]{\bigskip\par\noindent\textbf{문제 #1)}\par\noindent}

%%% Meta Commands
\let\oldsection\section
\renewcommand\section{\clearpage\oldsection}

\let\emph\textsf

\begin{document}
\begin{center}
\LARGE종현, 추가과제 06
\end{center}
\begin{flushright}
날짜 : 2017년 \(\pb3\)월 \(\pb{10}\)일 \(\pb{월}\)요일
,\qquad
제한시간 : \pb{17년}분
,\qquad
점수 : \pb{20} / \pb{20}
\end{flushright}

%%
\subsection{확률분포}
\vspace{-30pt}
% 확률분포 1
\prob{}
2, 4, 6, 8, 10의 숫자가 각각 하나씩 적혀잇는 5장의 카드 중에서 임의로 2장의 카드를 동시에 뽑을 때, 뽑힌 2장의 카드에 적혀 있는 두 수의 차를 확률변수 \(X\)라고 하자.
\(P(X\ge6)\)의 값은?
\par\bigskip
\tabb{\frac3{10}}{\frac25}{\frac12}{\frac35}{\frac7{10}}

% 확률분포 2
\prob{}
확률변수 \(X\)가 가질 수 있는 값이 1, 2, 4, 8이고,
\[P(X=2k)=kP(X=k)\qquad(k=1,2,4)\]
일 때, \(P(X^2-6X+8\le0)\)의 값은?
\par\bigskip
\tabb{\frac14}{\frac13}{\frac5{12}}{\frac12}{\frac7{12}}

% 확률분포 3
\prob{}
확률변수 \(X\)의 확률분포를 표로 나타내면 다음과 같다.
\par\bigskip
\begin{tabu}to0.8\textwidth{X[2,c]|X[c]X[c]X[c]X[c]|X[c]}
\toprule
\(X\)		&$-3$				&$-1$				&$1$				&$3$			&합계\\\hline
\(P(X=x)\)	&\(\frac a{10}\)	&\(\frac1{10}\)	&\(\frac b{10}\)	&\(\frac15\)	&\(1\)\\
\bottomrule
\end{tabu}
\par\bigskip
\(P(X^2=1)=\frac35\)일 때, 확률변수 \(X\)의 평균은?
(단, \(a\), \(b\)는 상수이다.)
\par\bigskip
\tabb{\frac7{20}}{\frac25}{\frac9{20}}{\frac12}{\frac{11}{20}}

% 확률분포 4
\prob{}
확률변수 \(X\)의 확률분포를 표로 나타내면 다음과 같다.
\par\bigskip
\begin{tabu}to0.8\textwidth{X[2,c]|X[c]X[c]X[c]X[c]|X[c]}
\toprule
\(X\)		&$-1$	&$0$			&$1$			&$2$	&합계\\\hline
\(P(X=x)\)	&\(a\)	&\(\frac25\)	&\(\frac15\)	&\(b\)	&\(1\)\\
\bottomrule
\end{tabu}
\par\bigskip
\(E(X)=\frac25\)일 때, \(V(X)\)의 값은?
(단, \(a\), \(b\)는 상수이다.)
\par\bigskip
\tabb{\frac{21}{25}}{\frac{23}{25}}{\frac{24}{25}}{\frac{26}{25}}{\frac{27}{25}}

\newpage
% 확률분포 5
\prob{}
확률변수 \(X\)의 평균과 분산이 각각 \(7\), \(3\)일 때, 확률변수 \(2X+1\)의 평균과 분산은 각각 \(a\), \(b\)이다.
\(a+b\)의 값은?
\par\bigskip
\tabb{24}{27}{30}{33}{36}

% 확률분포 6
\prob{}
확률변수 \(X\)가 이항분포 \(B\left(18,\frac13\right)\)을 따를 때, \(E(X^2)\)의 값은?
\par\bigskip
\tabb{40}{44}{48}{52}{56}

% 확률분포 7
\prob{}
동전 2개를 동시에 던지는 시행을 \(16\)번 반복할 때 동전 \(2\)개가 모두 앞면이 나오는 횟수를 확률변수 \(X\)라고 하고, 주사위 한 개를 \(n\)번 던질 때 \(3\)의 배수의 눈이 나오는 횟수를 확률변수 \(Y\)라고 하자.
\(V(X)<V(Y)\)가 성립하도록 하는 자연수 \(n\)의 최솟값은?
\par\bigskip
\tabb{13}{14}{15}{16}{17}

%%
\vspace{20pt}
\subsection{정규분포}
\vspace{-30pt}

% 정규분포 1
\prob{}
연속확률변수 \(X\)가 갖는 값의 범위가 \(0\le X\le 2\)이고, 확률변수 \(X\)의 확률밀도함수 \(f(x)\)가
\[f(x)=-\frac12x+1\quad(0\le x\le2)\]
일 때, \(P\left(1\le X\le2\right)\)의 값은?
\par\bigskip
\tabb{\frac1{16}}{\frac18}{\frac14}{\frac13}{\frac12}

% 정규분포 2
\prob{}
확률변수 \(X\)가 정규분포 \(N(0, 1^2)\)을 따르고,\\
\(P(-2\le X\le 2)=a\), \(P(1\le X\le2)=b\)이다.
다음 중 \(P(-1\le X\le1)\)의 값은?
(단, \(a\), \(b\)는 상수이다.)
\par\bigskip
\tabb{\frac{a+b-1}2}{\frac{1-a-b}2}{a-2b}{a+b}{a+2b-1}

\newpage
% 정규분포 3
\prob{}
\begin{minipage}{0.6\textwidth}
어느 지역의 대학생들이 하루동안 SNS를 이용하는 시간은 평균이 \(67\)분이고 표준편차가 \(15\)분인 정규분포를 따른다고 한다.
이 지역의 대학생들 중에서 임의로 선택한 한 대학생의 하루동안 SNS를 이용하는 시간이 \(52\)분 이상일 확률을 오른쪽 표준정규분포표를 이용하여 구한 것은?
\end{minipage}
\begin{minipage}{0.3\textwidth}
\centering
\begin{tabular}{c|c}
\hline
\(z\)	&\(P(0\le Z\le z)\)\\
\hline
0.5		&0.1915\\
1.0		&0.3413\\
1.5		&0.4332\\
2.0		&0.4772\\
\hline
\end{tabular}
\end{minipage}
\par\bigskip
\tabb{0.5328}{0.6247}{0.6915}{0.7745}{0.8413}

% 정규분포 4
\prob{}
확률변수 \(X\)가 이항분포 \(B\left(150,p\right)\)을 따르고 \\
\(E(X)=60\)일 때, \(P(57\le X\le 63)=k\)이다.
\(100k\)의 값은?
(단, \(Z\)가 표준정규분포를 따르는 확률변수일 때, \(P(0\le Z\le0.5)=0.19\)로 계산한다.)
\par\bigskip
\tabb{26}{32}{38}{44}{50}

% 정규분포 5
\prob{} 
\begin{minipage}{0.6\textwidth}
자유투 성공률이 \(75\%\)인 어느 농구 선수가 \(48\)번의 자유투를 던질 때, \(39\)번 이상 성공할 확률을 오른쪽 표준정규분포표를 이용하여 구한 것은?
\end{minipage}
\begin{minipage}{0.3\textwidth}
\centering
\begin{tabular}{c|c}
\hline
\(z\)	&\(P(0\le Z\le z)\)\\
\hline
0.5		&0.1915\\
1.0		&0.3413\\
1.5		&0.4332\\
2.0		&0.4772\\
\hline
\end{tabular}
\end{minipage}
\par\bigskip
\tabb{0.0228}{0.0456}{0.0668}{0.1587}{0.2166}

\vspace{20pt}
%%
\subsection{통계적 추정}
\vspace{-30pt}

% 통계적 추정 1
\prob{}
모평균이 \(102\), 모표준편차가 \(26\)인 모집단에서 크기가 \(n\)인 표본을 임의추출할 때, 표본평균 \(\overline X\)의 표준편차가 \(2\) 이하가 되도록 하는 자연수 \(n\)의 최솟값은?
\par\bigskip
\tabb{81}{100}{121}{144}{169}

% 통계적 추정 2
\prob{}
\begin{minipage}{0.6\textwidth}
어느 도시의 1인당 하루 물 사용량은 평균이 300 L이고 표준편차가 40 L인 정규분포를 따른다고 한다.
이 도시에서 임의로 \(100\)명을 추출하였을 때, 표본 \(100\)명의 1인당 하루 평균 물 사용량의 차가 8 L 이상일 확률을 오른쪽 표준정규분포표를 이용하여 구한 것은?
\end{minipage}
\begin{minipage}{0.3\textwidth}
\centering
\begin{tabular}{c|c}
\hline
\(z\)	&\(P(0\le Z\le z)\)\\
\hline
0.5		&0.1915\\
1.0		&0.3413\\
1.5		&0.4332\\
2.0		&0.4772\\
\hline
\end{tabular}
\end{minipage}
\par\bigskip
\tabb{0.0085}{0.0228}{0.0456}{0.0772}{0.0915}

% 통계적 추정 3
\prob{}
어느 농장에서 재배하여 판매하는 한라봉 한 개의 무게는 표준편차 25 g인 정규분포를 따른다고 한다.
이 농장에서 재배하여 판매하는 한라봉 중에서 \(100\)개를 임의추출하여 무게를 조사하였더니 그 평균이 280 g이었다고 할 때, 이 농장에서 재배하여 판매하는 전체 한라봉의 평균 무게 \(m\)에 대한 신뢰도 \(95\%\)의 신뢰구간은?
(단, \(Z\)가 표준정규분포를 따르는 확률변수일 때, \(P(0\le Z\le1.96)=0.4750\)으로 계산한다.)
\par\bigskip
\par\noindent
\ding{172}\:{\ensuremath{277.1\le m\le 282.9}}
\tab\ding{173}\:\:{\ensuremath{276.9\le m\le 283.1}}\\
\tab\ding{174}\:\:{\ensuremath{275.1\le m\le 284.9}}
\tab\ding{175}\:\:{\ensuremath{274.8\le m\le 285.2}}\\
\tab\ding{176}\:\:{\ensuremath{273.4\le m\le 286.6}}
\end{document}