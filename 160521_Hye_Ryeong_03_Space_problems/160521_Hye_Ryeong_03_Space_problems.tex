\documentclass{oblivoir}
\usepackage{amsmath,amssymb,amsthm,kotex,mdframed,paralist,kswrapfig}

\newcounter{num}
\newcommand{\prob}[1]
{\newpage\bigskip\noindent\refstepcounter{num}\textbf{문제 \arabic{num})#1}\par}

\newcommand{\ans}{
{\par\medskip\begin{mdframed}
\textbf{풀이 : }
\vspace{0.6\textheight}
\end{mdframed}\par
\raggedleft\textbf{답 : (\qquad\qquad\qquad\qquad\qquad\qquad)}
\par}\bigskip\bigskip}

\newcommand\ov[2]{\ensuremath{\overline{#1#2}}}

%%%
\begin{document}
\setcounter{page}{0}
\title{혜령 03 - 기벡[쎈], 5단원 보충문제}
\author{}
\date{\today}
\maketitle
%\tableofcontents

\prob{\#607}
\kswrapfig[Pos=r]{01}{
오른쪽 그림과 같이 \(\overline{AB}=2\), \(\overline{BC}=1\)인 직사각형 ABCD를 대각선 AC를 접는 선으로 하여 두 면이 서로 수직이 되도록 접었다.
이때,\par (1) \(\overline{BD}\)의 길이를 구하시오.\par
(2) \(D\)에서 \(\overline{BC}\)가지의 거리를 구하시오.
}
\ans

\prob{\#610-1}
\kswrapfig[Pos=r]{02}{
오른쪽 그림에서 두 평면 \(\alpha\), \(\beta\)는 서로 수직이고 두 직선 \(l\), \(m\)은 각각 \(\alpha\), \(\beta\) 위에 있다.
\(l\), \(m\)은 \(\alpha\), \(\beta\)의 교선 \(XY\) 위의 한 점에서 만나고, 교선 \(XY\)와 각각 \(30^\circ\), \(45^\circ\)의 각을 이룬다.
두 직선 \(l\), \(m\)이 이루는 각의 크기를 \(\theta\)라 할 때, \(\cos\theta\)의 값을 구하여라.}
\ans

\prob{\#610-2}
\kswrapfig[Pos=r]{03}{
오른쪽 그림에서 두 평면 \(\alpha\), \(\beta\)는 서로 수직이고 두 직선 \(l\), \(m\)은 각각 \(\alpha\), \(\beta\) 위에 있다.
\(l\), \(m\)은 \(\alpha\), \(\beta\)의 교선 \(XY\) 위의 한 점에서 만나고, 교선 \(XY\)와 각각 \(45^\circ\)의 각을 이룬다.
두 직선 \(l\), \(m\)이 이루는 각의 크기를 \(\theta\)라 할 때, \(\cos\theta\)의 값을 구하여라.}
\ans

\prob{\#623}
\kswrapfig[Pos=r]{04}{
오른쪽 그림과 같이 한 모서리의 길이가 6인 정사면체가 있다. 모서리 \(CD\)의 중점을 \(E\), \(AE\)의 중점을 \(F\)이라고 할 때, \(CF\)이 면 \(BCD\)와 이루는 각의 크기 \(\theta\)에 대해 \(\cos\theta\)의 값을 구하여라.
}
\ans

\prob{627}
\kswrapfig[Pos=r]{05}{
오른쪽 그림과 같이 \(\overline{FG}=a\), \(\overline{GH}=b\), \(\overline{DH}=c\)인 직육면체에서 대각선 \(BH\)가 직선 \({FG}\), \({GH}\), \({DH}\)와 이루는 각의 크기를 각각 \(\alpha\), \(\beta\), \(\gamma\)라고 하자.
이때, \(\cos^2\alpha+\cos^2\beta+\cos^2\gamma\)의 값을 구하여라.}
\ans

\prob{\#636}
\kswrapfig[Pos=r]{06}{
오른쪽 그림과 같은 정육면체에서 \(I\)와 \(J\)는 각각 \(\overline{AB}\), \(\overline{GH}\)의 중점이다. 평면 \(CIEJ\)와 평면 \(EFGH\)가 이루는 각의 크기를 \(\theta\)라고 할 때, \(\cos\theta\)의 값을 구하시오.
}

\ans

\prob{번외-1}
\kswrapfig[Pos=r]{07}{
오른쪽 그림과 같이 \(\overline{AB}=3\), \(\overline{BC}=2\), \(\overline{CD}=4\)이고 \(AB\perp BC\), \(BC\perp CD\), \(AB\perp CD\)일 때, \(\overline{AD}\)의 길이를 구하시오.
}

\par\medskip\begin{mdframed}
\textbf{풀이 : }
\vspace{0.5\textheight}
\end{mdframed}\par
\raggedleft\textbf{답 : (\qquad\qquad\qquad\qquad\qquad\qquad)}
\par

\raggedright

\prob{번외-2}
\kswrapfig[Pos=r]{08}{
오른쪽 그림과 같이 \(\overline{AB}=3\), \(\overline{BC}=2\), \(\overline{CD}=4\)이고 \(AB\perp BC\), \(BC\perp CD\), \(AB\)와 \(CD\)가 이루는 각이 \(60^\circ\)일 때, \(\overline{AD}\)의 길이를 구하시오.
}

\par\medskip\begin{mdframed}
\textbf{풀이 : }
\vspace{0.5\textheight}
\end{mdframed}\par
\raggedleft\textbf{답 : (\qquad\qquad\qquad\qquad\qquad\qquad)}
\par

\end{document}