\documentclass{oblivoir}
\usepackage{amsmath,amssymb,amsthm,kotex,mdframed,paralist,tabto,pifont,kswrapfig}

\counterwithout{subsection}{section}
\newcounter{num}
\newcommand{\prob}
{\par\bigskip\noindent\textbf{\thenum. }\refstepcounter{num}}

\newcommand{\ans}{
{\par\medskip\begin{mdframed}
\textbf{풀이 : }
\vspace{0.3\textheight}
\end{mdframed}\par
\raggedleft\textbf{답 : (\qquad\qquad\qquad\qquad\qquad\qquad)}
\par}\bigskip\bigskip}

\newcommand\ov[2]{\ensuremath{\overline{#1#2}}}

\newcommand\ve[2]{\ensuremath{\overrightarrow{#1#2}}}

\newcommand{\pa}{\mathbin{\!/\mkern-5mu/\!}}

\TabPositions{0.2\textwidth,0.4\textwidth,0.6\textwidth,0.8\textwidth}
\newcommand\tabb[5]{\par\noindent
\ding{172}{#1}
\tab\ding{173}{#2}
\tab\ding{174}{#3}
\tab\ding{175}{#4}
\tab\ding{176}{#5}}

\newcounter{pnum}
\newcommand\pn{\stepcounter{pnum}\textbf{\thepnum}}

%%%
\begin{document}
\title{준영 01 - 코시-슈바르츠 부등식}
\author{}
\date{\today}
\maketitle

\setcounter{num}{1}

\begin{mdframed}
\textbf{코시-슈바르츠 부등식}\par
실수 \(a\), \(b\), \(x\), \(y\)에 대해 다음과 같은 식이 성립한다
\[(a^2+b^2)(x^2+y^2)\ge(ax+by)^2.\]
(단 등호는 \(a:b=x:y\)일 때 성립한다.)
\end{mdframed}
\medskip

증명 : 
\vspace{0.3\textheight}
\begin{mdframed}
\textbf{코시-슈바르츠 부등식 2}\par
실수 \(a\), \(b\), \(c\), \(x\), \(y\), \(z\)에 대해 다음과 같은 식이 성립한다
\[(a^2+b^2+c^2)(x^2+y^2+z^2)\ge(ax+by+cz)^2.\]
(단 등호는 \(a:b:c=x:y:z\)일 때 성립한다.)
\end{mdframed}
\medskip

(증명생략)

\newpage
%
\prob
\(3x+4y=13\)일 때, \(x^2+y^2\)의 최솟값을 구하여라.
\ans

%
\prob
\(x^2+y^2+z^2=14\)일 때, \(3x+2y+z\)의 값의 범위를 구하여라.
\ans

\newpage
%
\prob
\(5x+12y=13\)일 때, \(x^2+y^2\)의 최솟값을 구하여라.
\ans

%
\prob
\(x^2+y^2+z^2=56\)일 때, \(x+3y+2z\)의 값의 범위를 구하여라.
\ans

\newpage
%
\prob
실수 \(x\), \(y\)에 대하여 \(x^2+y^2=4\)일 때, \(4x+3y\)의 값의 범위를 구하여라.
\ans

%
\prob
실수 \(a\), \(b\), \(x\), \(y\)에 대해 \(a^2+b^2=4\), \(x^2+y^2=5\)일 때, \(ax+by\)의 최댓값을 구하여라.
\ans

\newpage
%
\prob
실수 \(a\), \(b\), \(c\)에 대해 \(a^2+b^2+c^2=2\)일 때, \(a+2b+3c\)의 최솟값을 구하여라.
\ans

%
\prob
두 양수 \(a\), \(b\)가 \((a^2+1)(b^2+9)=36\)을 만족할 때, \(ab\)의 최댓값은?
\ans

\newpage
%
\prob
\(a^2+b^2=100\)을 만족하는 두 실수 \(a\), \(b\)에 대하여 \((a+3b)^2\)의 값이 최대가 될 때, \(a\)값을 구하여라.
{\par\medskip\begin{mdframed}
\textbf{풀이 : }
\vspace{0.2\textheight}
\end{mdframed}\par
\raggedleft\textbf{답 : (\qquad\qquad\qquad\qquad\qquad\qquad)}
\par}\bigskip\bigskip

%
\prob
\kswrapfig[Pos=r,Width=3cm]{01}{
오른쪽 그림과 반지름의 길이가 \(2\)인 원에 내접하는 직사각형의 둘레의 길이의 최댓값을 \(A\), 넓이의 최댓값을 \(B\)라고 할 때, \(A^2+B\)의 값을 구하여라.}
\ans

\end{document}