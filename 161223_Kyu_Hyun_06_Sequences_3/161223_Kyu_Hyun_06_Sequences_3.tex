\documentclass{oblivoir}
\usepackage{amsmath,amssymb,amsthm,kotex,paralist,kswrapfig}

\usepackage[skipabove=10pt]{mdframed}

\usepackage{tabto,pifont}
\TabPositions{0.2\textwidth,0.4\textwidth,0.6\textwidth,0.8\textwidth}
\newcommand\tabb[5]{\par\bigskip\noindent
\ding{172}\:{\ensuremath{#1}}
\tab\ding{173}\:\:{\ensuremath{#2}}
\tab\ding{174}\:\:{\ensuremath{#3}}
\tab\ding{175}\:\:{\ensuremath{#4}}
\tab\ding{176}\:\:{\ensuremath{#5}}}

\usepackage{enumitem}
\setlist[enumerate]{label=(\arabic*)}

\newcounter{num}
\newcommand{\defi}[1]
{\bigskip\noindent\refstepcounter{num}\textbf{정의 \arabic{num}) #1}\par\noindent}
\newcommand{\theo}[1]
{\bigskip\noindent\refstepcounter{num}\textbf{정리 \arabic{num}) #1}\par\noindent}
\newcommand{\exam}[1]
{\bigskip\noindent\refstepcounter{num}\textbf{예시 \arabic{num}) #1}\par\noindent}
\newcommand{\prob}[1]
{\bigskip\noindent\refstepcounter{num}\textbf{문제 \arabic{num}) #1}\par\noindent}
\newcommand{\proo}
{\bigskip\textsf{증명)}\par}

\newcommand{\ans}{
{\par
\raggedleft\textbf{답 : (\qquad\qquad\qquad\qquad\qquad\qquad)}
\par}\bigskip\bigskip}
\newcommand\an[1]{\par\bigskip\noindent\textbf{문제 #1)}\\}

\newcommand{\pb}[1]%\Phantom + fBox
{\fbox{\phantom{\ensuremath{#1}}}}

\newcommand\summ[4]{\ensuremath{\displaystyle\sum_{#1=#2}^{#3}#4}}

\let\oldsection\section
\renewcommand\section{\clearpage\oldsection}
%%%%
\begin{document}

\title{규현 : 06 수열(3)}
\author{}
\date{\today}
\maketitle
\tableofcontents
\newpage

%%

%%
\section{합의 기호 \(\Sigma\)}
\begin{mdframed}[innertopmargin=-5pt]
%
\defi{합의 기호 \(\Sigma\)}
수열 \(\{a_n\}\)의 첫째항부터 제 \(n\)항까지의 합 \(a_1+a_2+\cdots+a_n\)은 합의 기호 \(\Sigma\)를 통해 다음과 같이 나타낸다.

\[a_1+a_2+\cdots+a_n=\summ k1n{a_k}\]
이때 \(\summ k1n{a_k}\)는 \(a_k\)의 \(k\)에 \(1\), \(2\), \(3\), \(\cdots\), \(n\)을 차례로 대입하여 얻은 항 \(a_1\), \(a_2\), \(a_3\), \(\cdots\), \(a_n\)의 합을 뜻한다.
\end{mdframed}

%
\exam{}
\begin{enumerate}
\item
\summ k1{10}{(2k-1)}은 \(2k-1\)의 \(k\)에 \(1\), \(2\), \(3\), \(\cdots\), \(n\)을 차례로 대입하여 얻은 항의 합이므로
\[\summ k1{10}{(2k-1)}=1+3+5+\cdots+19\]
\item
\(2+4+8+\cdots+2^{10}\)은 수열의 제 \(i\)항 \(2^i\)의 \(i\)에 \(1\), \(2\), \(3\), \(\cdots\), \(10\)을 차례로 대입하여 얻은 항을 모두 더한 것이므로 기호 \(\Sigma\)를 사용하여 나타내면,
\[2+4+8+\cdots+2^{10}=\summ i1{10}2^i\]
\end{enumerate}

%
\prob{}
다음을 합의 기호 \(\Sigma\)를 사용하지 않은 합의 꼴로 나타내어라.
\begin{enumerate}
\item
\(\summ k1{10}{2k}=2+4+6+8+\cdots+20\)
\item
\(\summ k17{3^k}=\)
\item
\(\summ k38{\sqrt k}=\)
\item
\(\summ m15{\frac1{2m+1}}=\)
\end{enumerate}

%
\prob{}
다음을 합의 기호 \(\Sigma\)를 사용하여 나타내시오.
\begin{enumerate}
\item
\(4+7+10+\cdots+31=\summ k1{10}(3k+1)\)
\item
\(2+2^2+2^3+\cdots+2^8=\)
\item
\(1+\frac12+\frac13+\frac14+\cdots+\frac1{15}=\)
\item
\(1^2+2^2+3^2+4^2+\cdots+n^2=\)
\item
\(\frac1{1\cdot2}+\frac1{2\cdot3}+\frac1{3\cdot4}+\frac1{4\cdot5}
+\cdots+\frac1{15\cdot16}=\)
\end{enumerate}

%
\prob{}\label{property_example}
다음을 계산하시오.
\begin{enumerate}
\item
\(\summ k1{10}(4k+2)
=6+10+14+18+\cdots+42
=\frac{10(6+42)}2=240\)
\item
\(\summ n1{10}n=\)
\item
\(\summ j1{10}2^j=\)
\end{enumerate}

%%
\section{수열의 성질}

%
\prob{}
다음 계산을 해보자.

\par\smallskip\noindent
(1)
\(\summ k13{2^k}=\)
\tabto{0.5\textwidth}
(2)
\(\summ k13{k}=\)
\par\smallskip\noindent
(3)
\(\summ k13{(2^k+k)}=\)
\tabto{0.5\textwidth}
(4)
\(\summ k13{(2^k-k)}=\)
\par\smallskip\noindent
(5)
\(\summ k13{2^k\times k}=\)
\tabto{0.5\textwidth}
(6)
\(\summ k13{\frac{2^k}k}=\)
\par\smallskip\noindent
(7)
\(\summ k13{(3\cdot2^k)}=\)

%
\prob{}
빈칸에 \(=\) 또는 \(\neq\)를 넣으시오.
\begin{align*}
\summ k13{(2^k+k)}		\quad&\pb{=}\quad\summ k13{2^k}+\summ k13{k}\\
\summ k13{(2^k-k)}		\quad&\pb{=}\quad\summ k13{2^k}-\summ k13{k}\\
\summ k13{2^k\times k}	\quad&\pb{=}\quad\summ k13{2^k}\times\summ k13{k}\\
\summ k13{\frac{2^k}k}	\quad&\pb{=}\quad\frac{\summ k13{2^k}}{\summ k13{k}}\\
%\summ k13{\frac{2^k}k}	&\neq\left(\summ k13{2^k}\right)/\left(\summ k13{k}\right)\\
\summ k13{3\cdot2^k}	\quad&\pb{=}\quad3\summ k13{2^k}
\end{align*}

\begin{mdframed}[innertopmargin=-5pt]
%
\theo{수열의 기본성질}\label{sequence_property}
수열 \(\{a_n\}\)과 \(\{b_n\}\), 실수 \(c\)에 대해 다음이 성립한다.
\begin{enumerate}[label=(\emph{\alph*})]
\item
\(\summ k1n{(a_k+b_k)}=\summ k1n{a_k}+\summ k1n{b_k}\)
\item
\(\summ k1n{(a_k-b_k)}=\summ k1n{a_k}-\summ k1n{b_k}\)
\item
\(\summ k1n{ca_k}=c\summ k1n{a_k}\)
\item
\(\summ k1nc=cn\)
\end{enumerate}
\end{mdframed}

\bigskip
또 다음은 성립하지 않는다.
\begin{mdframed}
\begin{enumerate}[label=(\emph{\alph*})]
\item
\(\summ k1n{a_kb_k}\neq\summ k1n{a_k}\times\summ k1n{b_k}\)
\item
\(\summ k1n{\frac{a_k}{b_k}}\neq\frac{\summ k1n{a_k}}{\summ k1n{b_k}}\)
\end{enumerate}
\end{mdframed}



%%
\section{자연수의 거듭제곱의 합}
%
\begin{mdframed}[innertopmargin=-5pt]
\theo{}\label{sequence_formula}
\begin{enumerate}[label=(\emph{\alph*})]
\item
\(\summ k1nk=\frac{n(n+1)}2\)
\item
\(\summ k1n{k^2}=\frac{n(n+1)(2n+1)}6\)
\item
\(\summ k1n{k^3}=\left\{\frac{n(n+1)}2\right\}^2\)
\end{enumerate}
\end{mdframed}
\proo
\begin{enumerate}
\item
등차수열의 합 공식을 이용하면
\[\summ k1nk=1+2+3+\cdots+n=\frac{n(a+l)}2=\frac{n(n+1)}2\]
\item
식 \((k+1)^3-k^3=3k^2+3k+1\)에 \(k\)대신 \(1\), \(2\), \(\cdots\), \(n\)을 차례로 대입하고
\begin{align*}
2^3-1^3&=3\cdot1^2+3\cdot1+1\\
3^3-2^3&=3\cdot2^2+3\cdot2+1\\
4^3-3^3&=3\cdot3^2+3\cdot3+1\\
&\vdots\\
(n+1)^3-n^3&=3\cdot n^2+3\cdot n+1
\end{align*}
이것들을 모두 더하면,
\begin{align*}
(n+1)^3-1^3=&3(1^2+2^2+3^2+\cdots+n^2)+3(1+2+3+\cdots+n)\\
&+(1+1+1+\cdots+1)\\
n^3+3n^2+3n=&3\summ k1n{k^2}+3\summ k1n{k}+\summ k1n1\\
n^3+3n^2+3n=&3\summ k1n{k^2}+3\cdot\frac{n(n+1)}2+n
\end{align*}
이다.
이것을 정리하면
\[\summ k1n{k^2}=\frac{n(n+1)(2n+1)}6\]
이 된다.
\item
(생략, (2)와 같은 방법을 적용하면 된다.)
\end{enumerate}

%
\prob{}
다음을 구하여라.
\begin{enumerate}
\item
\(1+2+3+\cdots+10=\frac{10\times11}2=55\)
\item
\(1^2+2^2+3^2+\cdots+10^2=\)
\item
\(1^3+2^3+3^3+\cdots+10^3=\)
\end{enumerate}

%
\prob{}
\begin{enumerate}
\item
\(2^2+4^2+6^2+\cdots+14^2=\summ k17{(2k)^2}=\summ k174k^2=4\summ k17k^2\\\)
\(=4\times\frac{7\times8\times15}6=560\)
\item
\(1+3+5+7+9+\cdots+19=\)
\item
\(2^3+4^3+6^3+\cdots+12^3=\)
\end{enumerate}


%%
\section*{답}
\begin{minipage}{0.49\textwidth}
%
\an{3}
\begin{enumerate}[topsep=0pt]
\item[(2)]
\(3+3^2+3^3+\cdots+3^7\)
\item[(3)]
\(\sqrt1+\sqrt2+\sqrt3+\cdots+\sqrt8\)
\item[(4)]
\(\frac13+\frac15+\frac17+\frac19+\frac1{11}\)
\end{enumerate}

%
\an{4}
\begin{enumerate}[topsep=0pt]
\item[(2)]
\summ k18{2^k}
\item[(3)]
\summ k1{15}{\frac1k}
\item[(4)]
\summ k1n{k^2}
\item[(5)]
\summ k1{15}{\frac1{k(k+1)}}
\end{enumerate}

%
\an{5}
\begin{enumerate}[topsep=0pt]
\item[(2)]
\(55\)
\item[(3)]
\(2046\)
\end{enumerate}
\end{minipage}
%%%
\begin{minipage}{0.49\textwidth}

%
\an{6}
\begin{enumerate}[topsep=0pt]
\item
\(14\)
\item
\(6\)
\item
\(20\)
\item
\(8\)
\item
\(34\)
\item
\(\frac{20}3\)
\item
\(42\)
\end{enumerate}

%
\an{7}
\(=\), \(=\), \(=\), \(\neq\). \(\neq\)

%
\an{10}
\begin{enumerate}
\item[(2)]
\(385\)
\item[(3)]
\(3025\)
\end{enumerate}

%
\an{11}
\begin{enumerate}
\item[(2)]
\(100\)
\item[(3)]
\(3528\)
\end{enumerate}

\end{minipage}
\end{document}