\documentclass{oblivoir}
\usepackage{amsmath,amssymb,amsfonts,amsthm,mathrsfs,kotex,mdframed,paralist}

\newcounter{num}
\newcommand{\defi}[1]
{\bigskip\noindent\refstepcounter{num}\textbf{정의 \arabic{num}) #1}\par}
\newcommand{\theo}[1]
{\bigskip\noindent\refstepcounter{num}\textbf{정리 \arabic{num}) #1}\par}
\newcommand{\exam}[1]
{\bigskip\noindent\refstepcounter{num}\textbf{예시 \arabic{num}) #1}\par}
\newcommand{\rema}[1]
{\bigskip\noindent\refstepcounter{num}\textbf{참고 \arabic{num}) #1}\par}
\newcommand{\prob}[1]
{\bigskip\noindent\refstepcounter{num}\textbf{문제 \arabic{num}) #1}\par}

\renewcommand{\figurename}{그림}
\renewcommand{\proofname}{증명)}
\newcommand{\sol}{\par\bigskip\noindent{\bfseries풀이)}\par}

\newcommand{\uu}{\ensuremath{\boldsymbol{u}}}
\newcommand{\xx}{\ensuremath{\boldsymbol{x}}}
\newcommand{\vv}{\ensuremath{\boldsymbol{v}}}
\newcommand{\ww}{\ensuremath{\boldsymbol{w}}}
\newcommand{\zz}{\ensuremath{\boldsymbol{0}}}

%%%
\begin{document}

\title{동락 02 - Chapter 2, Vector Space}
\author{}
\date{\today}
\maketitle
\tableofcontents

\newpage

\setcounter{chapter}{01}

%%%
\chapter{벡터공간}

%%
\section{벡터공간과 부분공간}

%
\defi{실수 집합 \(\mathbb R\)}
\(\mathbb R\)은 실수의 집합이다.

%
\exam{}
\(1\in\mathbb R\)이고, \(\frac12\in\mathbb R\)이며, \(\sqrt2\in\mathbb R\)이다. 하지만 \(1+i\not\in\mathbb R\)이다.

%
\defi{\(n\)차원 유클리드 공간 \(\mathbb R^n\)}
(1)
\(\mathbb R^2\)은 실수들의 순서쌍(=ordered pair)을 원소로 가지는 집합이다.
즉
\[\mathbb R^2=\{(a,b)\mid a,b\in\mathbb R\}\]
이다.
따라서 \(\mathbb R^2\)은 \(xy\)-평면과 동일시될 수 있다.
비슷한 의미에서 \(\mathbb R\)은 수직선과 동일시될 수 있다.
이러한 \(\mathbb R^2\)를 \textbf{2차원 유클리드 공간} 이라고 부른다.

(2)
\(\mathbb R^3\)는 실수들의 ordered triple을 원소로 가지는 집합이다.
즉
\[\mathbb R^3=\{(a,b,c)\mid a,b,c\in\mathbb R\}\]
따라서 \(\mathbb R^3\)은 삼차원의 공간과 동일시될 수 있다.
이러한 \(\mathbb R^3\)은 \textbf{3차원 유클리드 공간} 이라고 부른다.

(3)
일반적으로 \(\mathbb R^4\)는 실수들의 n-tuple을 원소로 가지는 집합이다.
즉
\[\mathbb R^n=\{(a_1,a_2,\cdots,a_n)\mid a_1,a_2,\cdots,a_n\in\mathbb R\}\]
이다.
이러한 \(\mathbb R^n\)을 \textbf{n차원 유클리드 공간}이라고 부른다.

%
\rema{}
\(\mathbb R^n\)의 원소는 \textbf{벡터}라고 불린다.
이는 <<기하와 벡터>>에서 점 \(A=(1,2)\)가 그 방향벡터 \(\overrightarrow{OA}=(1,2)\)와 동일시되는 것과 비슷한 맥락이다.

%
\defi{벡터의 상등}
두 벡터 \(u=(a_1,\cdots,a_n)\), \(v=(b_1,\cdots,b_n)\)에 대해서, \(a_1=b_1\), \(\cdots\), \(a_n=b_n\)이면 `두 벡터가 같다'고 말하고 \(u=v\)라고 쓴다.

%
\defi{벡터의 덧셈과 실수배}
두 벡터 \(u=(a_1,\cdots,a_n)\), \(v=(b_1,\cdots,b_n)\)와 실수 \(k\)에 대해서

\begin{enumerate}[(1)]\tightlist
\item
\(u+v=(a_1+b_1,\cdots,a_n+b_n)\)이다.%(벡터의 덧셈)
\item
\(ku=(ka_1,\cdots,ka_n)\)이다.%(벡터의 실수배)
\end{enumerate}

%
\rema{}\label{closure}

모든 \(i\)에 대해 \(a_i+b_i\in\mathbb R\)이므로 \(u+v\in\mathbb R^n\)이다.
모든 \(i\)에 대해 \(ka_i\in\mathbb R\)이므로 \(ku\in\mathbb R^n\)이다.
즉 두 벡터를 더해도 여전히 벡터이고 또 어떤 벡터에 실수배를 해도 여전히 벡터이다.
이것을 `\(\mathbb R^n\)은 덧셈과 실수배에 대해 닫혀있다.'라고도 말한다.

%
\theo{}\label{euclid vector space}
벡터의 덧셈과 실수배에 대해 다음과 같은 성질들이 성립한다 ;
\(\boldsymbol{u}=(a_1,\cdots,a_n)\), \(\boldsymbol{v}=(b_1,\cdots,b_n)\), \(\boldsymbol{w}=(c_1,\cdots,c_n)\)이고 \(k\), \(l\)이 실수이면

\begin{enumerate}[(1)]\tightlist
\item
\(\boldsymbol{u}+\boldsymbol{v}\in\mathbb R^n\)이다.
\item
\(\boldsymbol{u}+\boldsymbol{v}=\boldsymbol{v}+\boldsymbol{u}\)이다.% (덧셈에 대한 교환법칙)
\item
\((\boldsymbol{u}+\boldsymbol{v})+w=\boldsymbol{u}+(\boldsymbol{v}+\boldsymbol{w})\)이다.% (덧셈에 대한 결합법칙)
\item
\(\boldsymbol{u}+\mathbf0=\mathbf0+\boldsymbol{u}=\boldsymbol{u}\)를 만족시키는 벡터 \(\mathbf0\)이 존재한다.% (덧셈에 대한 항등원)
\item
임의의 \(\boldsymbol{u}\)에 대해  \(\boldsymbol{u}+\xx=\xx+\boldsymbol{u}=\mathbf0\)를 만족시키는 벡터 \(\xx\)이 존재한다.% (덧셈에 대한 역원)
\item
\(k\boldsymbol{u}\in\mathbb R^n\)이다.
\item
\(1\cdot \boldsymbol{u}=\boldsymbol{u}\)이다.
\item
\((kl)\boldsymbol{u}=k(l\boldsymbol{u})\)이다.
\item
\(k(\boldsymbol{u}+\boldsymbol{v})=k\boldsymbol{u}+k\boldsymbol{v}\)이다.
\item
\((k+l)\boldsymbol{u}=k\boldsymbol{u}+l\boldsymbol{u}\)이다.
\end{enumerate}

\begin{proof}
(1), (6)은 참고 \ref{closure}과 정확히 같다.
(2)와 (3)은 실수의 덧셈에 대한 교환법칙 \((*)\)과 결합법칙\((**)\)으로부터 얻을 수 있다;
\[\boldsymbol{u}+\boldsymbol{v}=(a_1+b_1,\cdots,a_n+b_n)\stackrel{(*)}{=}(b_1+a_1,\cdots,b_n+a_n)=\boldsymbol{v}+\boldsymbol{u}.\]

\begin{align*}
(\boldsymbol{u}+\boldsymbol{v})+\boldsymbol{w}
&=(a_1+b_1,\cdots,a_n+b_n)+(c_1,\cdots,c_n)\\
&=((a_1+b_1)+c_1,\cdots,(a_n+b_n)+c_n)\\
&\stackrel{(**)}{=}(a_1+(b_1+c_1),\cdots,a_n+(b_n+c_n)\\
&=(a_1,\cdots,a_n)+(b_1+c_1,\cdots,b_n+c_n)\\
&=\boldsymbol{u}+(\boldsymbol{v}+\boldsymbol{w}).
\end{align*}

(4) \(0=(0,\cdots,0)\)이라고 하면

\begin{align*}
\boldsymbol{u}+\mathbf0
&=(a_1,\cdots,a_n)+(0,\cdots,0)\\
&=(a_1+0,\cdots,a_n+0)=(a_1,\cdots,a_n)\\
&=\boldsymbol{u}\\
&=(a_1,\cdots,a_n)=(0+a_1,\cdots,0+a_n)\\
&=(0,\cdots,0)+(a_1,\cdots,a_n)\\
&=\mathbf0+\boldsymbol{u}.
\end{align*}

(5) \(\boldsymbol x=(-a_1,\cdots,a_n)\)이라고 하면

\begin{align*}
\boldsymbol{u}+\boldsymbol x
&=(a_1,\cdots,a_n)+(-a_1,\cdots,-a_n)\\
&=(a_1+(-a_1),\cdots,a_n+(-a_n))=(0,\cdots,0)\\
&=\mathbf0\\
&=(0,\cdots,0)=((-a_1)+a_1,\cdots,(-a_n)+a_n)\\
&=(-a_1,\cdots,-a_n)+(a_1,\cdots,a_n)\\
&=\boldsymbol x+\boldsymbol{u}.
\end{align*}

(7)
\(1\cdot \boldsymbol{u}=1\cdot(a_1,\cdots,a_n)=(1\cdot a_1,\cdots,1\cdot a_n)=(a_1,\cdots,a_n)=\boldsymbol{u}\).

(8)은 실수의 곱셈에 대한 결합법칙으로\((\star)\)부터 성립한다;

\begin{align*}
(kl)\boldsymbol{u}
&=(kl)(a_1,\cdots,a_n)=((kl)a_1,\cdots,(kl)a_n)\\
&\stackrel{(\star)}{=}(k(la_1),\cdots,k(la_n))=k(la_1,\cdots,la_n)\\
&=k(l\boldsymbol{u}).
\end{align*}


(9), (10)은 실수의 분배법칙으로\((\star\star)\)부터 성립한다;

\begin{align*}
k(\boldsymbol{u}+\boldsymbol{v})
&=k(a_1+b_n,\cdots,a_n+b_n)=(k(a_1+b_1),\cdots,k(a_n+b_n)\\
&\stackrel{(\star\star)}{=}(ka_1+kb_1,\cdots,ka_n+kb_n)\\
&=(ka_1,\cdots,ka_n)+(kb_1,\cdots,kb_n)\\
&=k(a_1,\cdots,a_n)+k(b_1,\cdots,b_n)\\
&=k\boldsymbol{u}+k\boldsymbol{v}.\\
(k+l)\boldsymbol{u}
&=((k+l)a_1,\cdots,(k+l)a_n)\\
&\stackrel{(\star\star)}{=}(ka_1+la_1,\cdots,ka_n+la_n)\\
&=(ka_1,\cdots,ka_n)+(la_1,\cdots,la_n)\\
&=k(a_1,\cdots,a_n)+l(a_1,\cdots,a_n)\\
&=k\boldsymbol{u}+l\boldsymbol{u}.
\end{align*}

%
\rema{}
정리 \ref{euclid vector space}의 (4)를 만족시키는 벡터 \(\boldsymbol0\)을 \textbf{영벡터}라고 부른다.
이것은 덧셈에 대한 항등원이라고 볼 수 있다.
한 가지 중요한 사실은, 영벡터가 유일하다는 것이다; 만약 \(\boldsymbol0\)과 \(\boldsymbol0'\)이 모두 영벡터이면, 영벡터의 성질에 의해 \(\boldsymbol0=\boldsymbol0+\boldsymbol0'=\boldsymbol0'\)이기 때문이다. (첫 번째 등호는 \(\boldsymbol0'\)이 영벡터라는 사실을, 두 번째 등호는 \(\boldsymbol0\)이 영벡터라는 성질을 사용했다.)

정리 \ref{euclid vector space}의 (5)를 만족시키는 벡터 \(\xx\)는 \(\boldsymbol{u}\)의 덧셈에 대한 역원이라고 볼 수 있다.
이러한 \(\xx\)도 유일하다 ; 만약 \(\xx\)와 \(\xx'\) 모두 \(\uu\)의 덧셈에 대한 역원이면 \(\xx=\xx+\boldsymbol0=\xx+(\uu+\xx')=(\xx+\uu)+\xx'=\boldsymbol0+\xx'=\xx'\)이기 때문이다.

정리 \ref{euclid vector space}의 (5)의 증명에서 알 수 있듯, \(\xx=(-1)\uu\)이다.
따라서 \(\xx\)를 \(\xx=-\boldsymbol{u}\)라고도 쓴다.

\defi{벡터공간}\label{vector space}
공집합이 아닌 어떤 집합 \(V\)에 대해 \(V\)에서 덧셈과 실수배가 정의되어 있고 다음 열 가지 성질을 만족하면 \(V\)를 벡터공간이라고 부른다;

\(\boldsymbol{u}\), \(\boldsymbol{v}\), \(\boldsymbol{w}\)가 모두 \(V\)의 원소이고 \(k\), \(l\)이 실수이면,

\begin{enumerate}[(1)]\tightlist
\item
\(\boldsymbol{u}+\boldsymbol{v}\in V\)이다.
\item
\(\boldsymbol{u}+\boldsymbol{v}=\boldsymbol{v}+\boldsymbol{u}\)이다.% (덧셈에 대한 교환법칙)
\item
\((\boldsymbol{u}+\boldsymbol{v})+\boldsymbol{w}=\boldsymbol{u}+(\boldsymbol{v}+\boldsymbol{w})\)이다.% (덧셈에 대한 결합법칙)
\item
\(\boldsymbol{u}+0=0+\boldsymbol{u}=\boldsymbol{u}\)를 만족시키는 벡터 \(0\)이 존재한다.% (덧셈에 대한 항등원)
\item
임의의 \(\boldsymbol{u}\)에 대해  \(\boldsymbol{u}+x=x+\boldsymbol{u}=0\)를 만족시키는 벡터 \(x\)이 존재한다.% (덧셈에 대한 역원)
\item
\(k\boldsymbol{u}\in V\)이다.
\item
\(1\cdot \boldsymbol{u}=\boldsymbol{u}\)이다.
\item
\((kl)\boldsymbol{u}=k(l\boldsymbol{u})\)이다.
\item
\(k(\boldsymbol{u}+\boldsymbol{v})=k\boldsymbol{u}+k\boldsymbol{v}\)이다.
\item
\((k+l)\boldsymbol{u}=k\boldsymbol{u}+l\boldsymbol{u}\)이다.
\end{enumerate}
\end{proof}

%
\exam{}
(1)
집합
\[\mathbb R=\{(a_1,a_2,a_3,\cdots)\mid a_i\in\mathbb R\}\]
을 생각하자.
이것은 유한차원의 유클리드공간을 일반화한 것이라고 볼 수 있다.
또한 \(\mathbb R^\infty\)의 원소는 \((a_1,a_2,a_3,\cdots)\)꼴로서, 하나의 수열과 같다.
즉 \(\mathbb R^\infty\)는 수열들의 집합이다.
덧셈과 실수배를 
\begin{gather*}
(a_1,a_2,a_3,\cdots)+(b_1,b_2,b_3,\cdots)=(a_1+b_1,a_2+b_2,a_3+b_3,\cdots)\\
k(a_1,a_2,a_3,\cdots)=(ka_1,ka_2,ka_3,\cdots)
\end{gather*}
와 같이 정의하면 \(\mathbb R^\infty\)는 벡터공간을 이룬다.

(2)
집합 \(\mathfrak M\)을
\[\mathfrak M=\left\{
\begin{bmatrix}
a&b\\
c&d\\
e&f\\
\end{bmatrix}
\Bigg| a,b,c,d,e,f\in\mathbb R\right\}\]
로 정의하면 통상적인 행렬의 덧셈과 실수배에 대해서 \(\mathfrak M\)은 벡터공간을 이룬다.

(3)
집합 \(\mathscr F\)를
\[\mathscr F=\{f:[0,1]\to\mathbb R\}\mid f\text{는 함수}\}\]
라고 하자.
두 함수 \(f,g\in\mathscr F\)의 합인 \(f+g:[0,1]\to R\)는
\[(f+g)(x)=f(x)+g(x)\]
를 만족시키는 함수이고 
두 함수 의 곱인 \(f\cdot g:[0,1]\to R\)는
\[(f\cdot g)(x)=f(x)\cdot g(x)\]
를 만족시키는 함수라고 하자.
그러면 \(\mathscr F\)는 벡터공간을 이룬다.

(4)
집합 \(\mathcal P_3\)를 2차 이하의 다항식들의 집합
\[\mathcal P_3=\{a+bx+cx^2\mid a,b,c\in\mathbb R\}\]
이라고 하자.
통상적인 다항식의 덧셈과 실수배에 대해서 \(\mathcal P_3\)는 벡터공간을 이룬다.

\defi{부분공간}
벡터공간 \(V\)의 부분집합 \(W\)에 대해 \(W\neq\emptyset\)이고, \(W\)가 그 자체로 벡터공간을 이루면 \(W\)를 \(V\)의 \textbf{부분공간}이라고 부른다.

이때, 기호로 `\(W<V\)'라고 나타내기도 한다.

\rema{}\label{basic conditions}
\(W\)가 벡터공간 \(V\)의 부분집합이고 \(W\neq\emptyset\)이라고 하자.
또, \uu, \vv, \ww가 \(W\)의 원소이고 \(k\), \(l\)이 실수라고 하자.
그러면 \uu, \vv, \ww가 \(V\)의 원소이고 따라서,

\begin{enumerate}[(1)]\tightlist
\item[(2)]
\(\boldsymbol{u}+\boldsymbol{v}=\boldsymbol{v}+\boldsymbol{u}\)
\item[(3)]
\((\boldsymbol{u}+\boldsymbol{v})+\boldsymbol{w}=\boldsymbol{u}+(\boldsymbol{v}+\boldsymbol{w})\)
\item[(7)]
\(1\cdot \boldsymbol{u}=\boldsymbol{u}\)
\item[(8)]
\((kl)\boldsymbol{u}=k(l\boldsymbol{u})\)
\item[(9)]
\(k(\boldsymbol{u}+\boldsymbol{v})=k\boldsymbol{u}+k\boldsymbol{v}\)
\item[(10)]
\((k+l)\boldsymbol{u}=k\boldsymbol{u}+l\boldsymbol{u}\)
\end{enumerate}
이다.

%
\theo{}\label{subspacethm1}
벡터공간 \(V\)의 부분집합 \(W\)에 대해
\begin{enumerate}[(a)]\tightlist
\item
\(W\neq\emptyset\)
\item
\(W\) 임의의 원소 \(\uu\)와 \(\vv\)에 대해 \(\uu+\vv\in W\)
\item
\(W\) 임의의 원소 \(\uu\)와 임의의 실수 \(k\)에 대해 \(k\uu\in W\)
\end{enumerate}
이면 \(W\)는 \(V\)의 부분공간이다.

\begin{proof}
참고 \ref{basic conditions}에 의해 정의 \ref{vector space}의 (1), (4), (5), (6)번만 증명하면 된다.
(1), (6)은 정확히 (b), (c)과 같다.
첫 번째 가정(\(W\neq\emptyset\))에 의해 \(W\)의 원소 \(\uu\)가 적어도 하나 존재한다.
그러면 (c)에 의해 \(0\cdot\uu\in W\)인데
\[0\cdot\uu=(1+(-1))\uu=\uu+(-\uu)=\boldsymbol0\]
이므로 \(\boldsymbol0\in W\)이다.
따라서 (4)번이 증명되었다.
마지막으로 \(-\uu=(-1)\uu\in W\)이므로 (5)번도 증명되었다.
\end{proof}

%
\theo{}
벡터공간 \(V\)의 부분집합 \(W\)에 대해
\begin{enumerate}[(a)]\tightlist
\item
\(W\neq\emptyset\)
\item
\(W\) 임의의 원소 \(\uu\), \(\vv\)와 임의의 실수 \(k\), \(l\)에 대해 \(k\uu+l\vv\in W\)
\end{enumerate}
이면 \(W\)는 \(V\)의 부분공간이다.

\begin{proof}
(b)에 \(k=l=1\)을 대입하면 정확히 정리 \ref{subspacethm1}의 (b)를 얻을 수 있다.
또 (b)에 \(l=0\)을 대입하면 정확히 정리 \ref{subspacethm1}의 (c)를 얻을 수 있다.
그러면 정리 \ref{subspacethm1}의 가정들이 모두 성립하므로 \(W\)는 \(V\)의 부분공간이다.
\end{proof}

%
\defi{}
실수 \(c_1\), \(c_2\), \(\cdots\), \(c_n\)와 벡터 \(\uu_1\), \(\uu_2\), \(\cdots\), \(\uu_n\)에 대하여 벡터
\[c_1\uu_1+c_2\uu_2+\cdots+c_n\uu_n=\sum_{i=1}^nc_i\uu_i\]
를 벡터 \(\uu_1\), \(\uu_2\), \(\cdots\), \(\uu_n\)의 \textbf{일차결합}이라고 부른다.

%
\prob{}
집합
\[\mathcal P_3=\{a+bx+cx^2\mid a,b,c\in\mathbb R\}\]
의 부분집합
\[\{a+bx+cx^2\mid a+b+c=0\}\]
이 \(\mathcal P_3\)의 부분공간임을 증명하시오.

%
\prob{}
다음 명제의 참/거짓을 판별하시오.(참인 경우 증명하고, 거짓인 경우 반례를 제시하시오.)\\
(1)
\[\left\{
\begin{bmatrix}
a&b\\c&0
\end{bmatrix}
\Bigg|a,b,c\in\mathbb R
\right\}\]
는 통상적인 행렬의 덧셈과 실수배에 대하여 벡터공간을 이룬다.\\
(2)
\[\left\{
\begin{bmatrix}
a&b\\c&d
\end{bmatrix}
\Bigg|ad-bc\neq0
\right\}\]
는 통상적인 행렬의 덧셈과 실수배에 대하여 벡터공간을 이룬다.\\
(3)
\(V\), \(W\)가 벡터공간이면 \(V\cap W\)도 벡터공간을 이룬다.\\
(4)
\(V\), \(W\)가 벡터공간이면 \(V\cup W\)도 벡터공간을 이룬다.\\

%%
\section{Solving \(Ax=b\) and \(Ax=b\)}

%%
\section{일차독립, 기저, 차원}

%
\defi{}
벡터 \(\uu_1\), \(\uu_2\), \(\cdots\), \(\uu_n\)에 대하여
\[c_1\uu_1+c_2\uu_2+\cdots+c_n\uu_n=\zz\]
를 만족하는 \(c_i\)들이 존재하고 이 때 모든 \(c_i\)가 동시에 0은 아닐 때(\(\sim(c_1=c_2=\cdots=c_n=0)\)), 이 벡터들을 \textbf{일차 종속}이라고 말한다.

한편, 
\[c_1\uu_1+c_2\uu_2+\cdots+c_n\uu_n=\zz\]
를 만족하고 모든 \(c_i\)가 동시에 0이 되지는 않는 \(c_i\)들이 존재하지 않으면 이 벡터들을 \textbf{일차 독립}이라고 말한다.

%
\rema{}\label{alternative definition of l'indep}
다시 말해, \(\uu_1\), \(\uu_2\), \(\cdots\), \(\uu_n\)들이 일차독립이기 위한 필요충분조건은
\[``c_1\uu_1+c_2\uu_2+\cdots+c_n\uu_n=\zz\text{\quad 이면 \quad} c_1=c_2=\cdots=c_n=0''\]
인 것이다.

%
\theo{}
\(\uu_1\), \(\uu_2\), \(\cdots\), \(\uu_n\)들이 일차종속이면 어느 한 벡터가 나머지 벡터들의 일차결합으로 표현될 수 있다.
즉, \(\uu_j=\sum_{i\neq j}^nk_i\uu_i\)인 \(j\)가 존재한다.
또한 이 명제의 역도 성립한다.

\begin{proof}
\(\uu_1\), \(\uu_2\), \(\cdots\), \(\uu_n\)들이 일차종속이면
\[c_1\uu_1+c_2\uu_2+\cdots+c_n\uu_n=\zz\]
를 만족하는 \(c_i\)들이 존재하고 이 때 모든 \(c_i\)가 동시에 0은 아니다.
즉, \(c_j\neq0\)인 \(j\)가 적어도 하나 존재한다.
식을 정리해 \(c_j\uu_j\)에 관한 식으로 만들면
\[c_j\uu_j=-c_1\uu_1-\cdots-c_{j-1}\uu_{j-1}-c_{j+1}\uu_{j+1}-\cdots-c_n\uu_n=\sum_{i\neq j}^n(-c_i)\uu_i\]
이고 이를 \(c_j(\neq0)\)로 나누면
\[\uu_j=-\frac{c_1}{c_j}\uu_1-\cdots-\frac{c_{j-1}}{c_j}\uu_{j-1}-\frac{c_{j+1}}{c_j}\uu_{j+1}-\cdots-\frac{c_n}{c_j}\uu_n=\sum_{i\neq j}^n\left(-\frac{c_i}{c_j}\right)\uu_i\]
이다.
따라서 \(\uu_j\)가 나머지 벡터들의 일차결합으로 나타났다.

또, 만약 \(\uu_j=\sum_{i\neq j}^nk_i\uu_i\)인 \(j\)가 존재한다면, 즉
\[\uu_j=k_1\uu_1+\cdots+k_{j-1}\uu_{j-1}+k_{j+1}\uu_{j+1}+\cdots+k_n\uu_n\]
이라면,
\[\uu_1+\cdots+k_{j-1}\uu_{j-1}-\uu_j+k_{j+1}\uu_{j+1}+\cdots+k_n\uu_n=\zz\]
가 성립하고 따라서 일차독립이다.
\end{proof}

%
\exam{}
(1)
\((1,2,-1)\), \((3,6,-3)\), \((3,9,3)\), \((2,5,0)\)는 일차 종속이다.
왜냐하면
\[3(1,2,-1)-(3,6,-3)+0(3,9,3)+0(2,5,0)=(0,0,0)\]
이기 때문이다.
(\(c_1=3\), \(c_2=-1\), \(c_3=0\), \(c_4=0\))

혹은, \((1,2,-1)\)이 \((3,6,-3)\), \((3,9,3)\), \((2,5,0)\)들의 일차결합으로 표현될 수 있기 때문이다 ;
\[(1,2,-1)=\frac13\cdot(3,6,-3)+0\cdot(3,9,3)+0\cdot(2,5,0)\]

(2) 
\((3,0,0)\), \((4,1,0)\), \((2,5,2)\)는 일차 독립이다.
왜나햐면
\[c_1(3,0,0)+c_2(4,1,0)+c_3(2,5,2)=(0,0,0)\]
이 성립하려면
\[
\begin{cases}
3c_1+4c_2+2c_3=0\\
c_2+5c_3=0\\
c_3=0
\end{cases}\]
이 성립해야 하는데 그러면 \(c_1=c_2=c_3=0\)이 되어
\(c_i\)들이 동시에 0이 되지는 않는 \(c_1\), \(c_2\), \(c_3\)가 존재하지 않기 때문이다.

혹은, 참고 \ref{alternative definition of l'indep}에 의해
\[c_1(3,0,0)+c_2(4,1,0)+c_3(2,5,2)=(0,0,0)\]
를 가정하면 \(c_1=c_2=c_3=0\)가 얻어지기 때문이다.

%
\prob{}
다음 벡터들(\(\in\mathbb R^3\))이 일차결합인지 일차종속인지 판단하시오.
\begin{enumerate}[(1)]
\item
\((1,-3,5)\), \((2,2,4)\), \((4,-4,14)\)
\item
\((1,7,7)\), \((2,7,7)\), \((3,7,7)\)
\item
\((0,0,-1)\), \((1,0,4)\)
\item
\((9,9,0)\), \((2,0,-1)\), \((3,5,-4)\), \((12,12,-1)\)
\end{enumerate}

%
\prob{}
다음 벡터들(\(\in\mathcal P_3\))이 일차결합인지 일차종속인지 판단하시오.
\begin{enumerate}[(1)]
\item
\(3-x+9x^2\), \(5-6x+3x^2\), \(1+1x-5x^2\)
\item
\(-x^2\), \(1+4x^2\)
\item
\(2+x+7x^2\), \(3-x+2x^2\), \(4-3x^2\)
\item
\(8+3x+3x^2\), \(x+2x^2\), \(2+2x+2x^2\), \(8-2x+5x^2\)
\end{enumerate}


%
\prob{}
다음 벡터들(\(\in\mathscr F\))이 일차결합인지 일차종속인지 판단하시오.
이때,
\[\mathscr F=\{f:\mathbb R\to\mathbb R\mid f\text{는 함수}\}\]
\begin{enumerate}[(1)]
\item
\(f(x)=x\), \(g(x)=\frac1x\)
\item
\(f(x)=e^x\), \(g(x)=\ln x\)
\item
\(f(x)=\sin x\), \(g(x)=\cos x\)
\item
\(f(x)=\sin x\), \(g(x)=\cos x\), \(h(x)=\sin(x+\frac{\pi}4)\)
\item
\(f(x)=e^x\), \(g(x)=\frac{e^x+e^{-x}}2\), \(h(x)=\frac{e^x-e^{-x}}2\)
\item
\(f(x)=0\), \(g(x)=x\), \(h(x)=x^2\)
\end{enumerate}

%
\prob{}
다음 명제의 참/거짓을 판별하시오.\\
(1) \uu, \vv, \ww가 일차 독립일 때, \uu, \uu+\vv, \uu+\ww도 일차 독립이다.\\
(2) \uu, \vv, \ww가 일차 독립일 때, \uu+\vv, \uu+\ww, \vv+\ww도 일차 독립이다.\\
(3) \(\uu=\zz\) 이면 \uu, \vv, \ww가 일차종속이다.\\
(4) \uu, \vv, \ww가 일차 독립일 때, \uu, \vv도 일차독립이다.

%
\defi{}
\(\ww_1\), \(\ww_2\), \(\cdots\), \(\ww_n\)들이 \(V\)의 원소라고 가정하자.
임의의 \(v\in V\)에 대해 \(v\)가 \(\ww_i\)들의 일차결합으로 표현될 수 있으면 `\textbf{\(\ww_1\), \(\ww_2\), \(\cdots\), \(\ww_n\)들이 \(V\)를 생성(span)한다}'고 말한다.

\end{document}