\documentclass[a4paper]{oblivoir}
\usepackage{amsmath,amssymb,kotex,kswrapfig,mdframed,paralist}
\usepackage{fapapersize}
\usefapapersize{210mm,297mm,20mm,*,20mm,*}

\usepackage{tabto,pifont}
\TabPositions{0.2\textwidth,0.4\textwidth,0.6\textwidth,0.8\textwidth}
\newcommand\tabb[5]{\par\noindent
\ding{172}\:{\ensuremath{#1}}
\tab\ding{173}\:\:{\ensuremath{#2}}
\tab\ding{174}\:\:{\ensuremath{#3}}
\tab\ding{175}\:\:{\ensuremath{#4}}
\tab\ding{176}\:\:{\ensuremath{#5}}}

\pagestyle{empty}

%%% Counters
\newcounter{num}

%%% Commands
\newcommand\defi[1]
{\bigskip\par\noindent\stepcounter{num} \textbf{정의 \thenum) #1}\par\noindent}
\newcommand\theo[1]
{\bigskip\par\noindent\stepcounter{num} \textbf{정리 \thenum) #1}\par\noindent}
\newcommand\exam[1]
{\bigskip\par\noindent\stepcounter{num} \textbf{예시 \thenum) #1}\par\noindent}
\newcommand\prob[1]
{\bigskip\par\noindent\stepcounter{num} \textbf{문제 \thenum) #1}\par\noindent}

\newcommand\pb[1]{\ensuremath{\fbox{\phantom{#1}}}}

\newcommand\ba{\ensuremath{\:|\:}}

\newcommand\vs[1]{\vspace{50pt}}

\newcommand\an[1]{\bigskip\par\noindent\textbf{문제 #1)}\par\noindent}

%%% Meta Commands
\let\oldsection\section
\renewcommand\section{\clearpage\oldsection}

\let\emph\textsf

\begin{document}
\begin{center}
\LARGE준영, 미니테스트 04
\end{center}
\begin{flushright}
날짜 : 2017년 \(\pb3\)월 \(\pb{10}\)일 \(\pb{월}\)요일
,\qquad
제한시간 : \pb{17년}분
,\qquad
점수 : \pb{20} / \pb{20}
\end{flushright}

%고1
\prob{}
두 수열 \(\{a_n\}\), \(\{b_n\}\)에 대하여 옳은 것만을 <보기>에서 있는 대로 고른 것은?
\begin{mdframed}[frametitle=<보기>]
\begin{enumerate}[ㄱ.]
\item
두 수열 \(\{a_n-b_n\}\), \(\{a_nb_n\}\)이 각각 수렴하면 수열 \(\{{a_n}^2b_n-a_n{b_n}^2\}\)은 수렴한다.
\item
두 수열 \(\{a_n+b_n\}\), \(\{a_n\}\)이 각각 수렴하면 수열 \(\left\{\frac{b_n}{a_n}\right\}\)은 수렴한다.
\item
두 수열 \(\{a_n\}\), \(\{b_n\}\)이 각각 수렴하고, 수열 \(\{a_n-b_n\}\)이 음의 값에 수렴하면 모든 자연수 \(n\)에 대하여 \(a\le b_n\)이다.
\end{enumerate}
\end{mdframed}
\vs

%고2
\prob{}
두 수열 \(\{a_n\}\), \(\{b_n\}\)이 다음 조건을 만족시킨다.
\begin{mdframed}
\begin{enumerate}[(가)]
\item
\(\displaystyle\lim_{n\to\infty}\frac{a_n}{2n+1}=10\)
\item
모든 자연수 \(n\)에 대하여 \(3n-1<2a_n-nb_n<3n+5\)
\end{enumerate}
\end{mdframed}
\(\displaystyle\lim_{n\to\infty}b_n\)의 값을 구하시오.
\vs

%고3
\prob{}
일 년 농사의 벼 베기를 끝낸 날 밤에 형제가 서로의 어려움을 걱정해 자신의 볏단을 몰래 가져다주다 도중에 만나 서로 얼싸안고 울었다는 의좋은 형제 이야기가 있다.
이 이야기에서 형제가 함께 수확한 쌀이 1200(kg)이고, 이를 두 형제가 적당히 나누어 가졌다고 하자.
다음날 밤 형은 전날 자신이 가지고 있는 쌀의 20\%를 동생에게 몰래 가져다 놓고, 동생도 역시 전날 자신이 가지고 있는 쌀의 10\%를 형에게 몰래 가져다 놓았다.
\(n\)일 밤 동안 이러한 일이 반복된다고 할 때, 형이 가지게 되는 쌀은 \(a_n\)(kg)이다.
\(\displaystyle\lim_{n\to\infty}a_n\)의 값은?
(단, 수확한 쌀의 전체의 양은 줄지 않는다.)
\tabb{320}{340}{360}{380}{400}
\end{document}