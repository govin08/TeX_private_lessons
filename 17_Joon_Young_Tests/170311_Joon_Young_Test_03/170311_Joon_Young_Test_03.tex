\documentclass[a4paper]{oblivoir}
\usepackage{amsmath,amssymb,kotex,kswrapfig,mdframed,paralist}
\usepackage{fapapersize}
\usefapapersize{210mm,297mm,20mm,*,20mm,*}

\usepackage{tabto,pifont}
\TabPositions{0.2\textwidth,0.4\textwidth,0.6\textwidth,0.8\textwidth}
\newcommand\tabb[5]{\par\noindent
\ding{172}\:{\ensuremath{#1}}
\tab\ding{173}\:\:{\ensuremath{#2}}
\tab\ding{174}\:\:{\ensuremath{#3}}
\tab\ding{175}\:\:{\ensuremath{#4}}
\tab\ding{176}\:\:{\ensuremath{#5}}}

\pagestyle{empty}

%%% Counters
\newcounter{num}

%%% Commands
\newcommand\defi[1]
{\bigskip\par\noindent\stepcounter{num} \textbf{정의 \thenum) #1}\par\noindent}
\newcommand\theo[1]
{\bigskip\par\noindent\stepcounter{num} \textbf{정리 \thenum) #1}\par\noindent}
\newcommand\exam[1]
{\bigskip\par\noindent\stepcounter{num} \textbf{예시 \thenum) #1}\par\noindent}
\newcommand\prob[1]
{\bigskip\par\noindent\stepcounter{num} \textbf{문제 \thenum) #1}\par\noindent}

\newcommand\pb[1]{\ensuremath{\fbox{\phantom{#1}}}}

\newcommand\ba{\ensuremath{\:|\:}}

\newcommand\vs[1]{\vspace{50pt}}

\newcommand\an[1]{\bigskip\par\noindent\textbf{문제 #1)}\par\noindent}

%%% Meta Commands
\let\oldsection\section
\renewcommand\section{\clearpage\oldsection}

\let\emph\textsf

\begin{document}
\begin{center}
\LARGE준영, 미니테스트 03
\end{center}
\begin{flushright}
날짜 : 2017년 \(\pb3\)월 \(\pb{10}\)일 \(\pb{월}\)요일
,\qquad
제한시간 : \pb{17년}분
,\qquad
점수 : \pb{20} / \pb{20}
\end{flushright}

%형1
\prob{}
<보기>의 수열 중 수렴하는 것만을 있는 대로 고른 것은?
\begin{mdframed}[frametitle=<보기>]
ㄱ. \(\left\{\frac{n+1}{\sqrt{3n+4}}\right\}\)
\tabto{0.33\textwidth}
ㄴ. \(\frac{2^{2n+1}-5^{n+1}}{2^{2n}-5^{n-1}}\)
\tabto{0.66\textwidth}
ㄷ. \(\left\{\left(\frac{3\sqrt2}{2\pi}\right)^n\right\}\)
\end{mdframed}
\tabb{\text{ㄱ}}{\text{ㄷ}}{\text{ㄱ, ㄴ}}{\text{ㄴ, ㄷ}}{\text{ㄱ, ㄴ, ㄷ}}
\vs

%형2
\prob{}
수열 \(\{a_n\}\)에 대하여 \(\displaystyle\lim_{n\to\infty}(n+1)^2a_n=2\)일 때, \(\displaystyle\lim_{n\to\infty}(4n^2+3n)a_n\)의 값은?
\tabb6789{10}
\vs

%형4
\prob{}
\(\displaystyle\sum_{k=1}^na_k=(n+2)^2\)일 때, \(\displaystyle\lim_{n\to\infty}\frac{a_1n^2}{{a_{2n}}^2}\)의 값은?

\bigskip
\tabb{\frac32}{\frac74}{2}{\frac94}{\frac52}
\vs

%형5
\prob{}
두 수열 \(\{a_n\}\), \(\{b_n\}\)에 대하여 옳은 것만을 <보기>에서 있는 대로 고른 것은?
\begin{mdframed}[frametitle=<보기>]
\begin{enumerate}[ㄱ.]
\item
\(\displaystyle\lim_{n\to\infty}(3a_n+b_n)=0\)이고, \(\displaystyle\lim_{n\to\infty}a_n=1\)이면 \(\displaystyle\lim_{n\to\infty}b_n=-3\)이다.
\item[ㄴ.]
수열 \(\{a_{2n+1}\}\)이 수렴하면 수열 \(\{a_n\}\)은 수렴한다.
\item[ㄷ.]
\(\displaystyle\lim_{n\to\infty}a_nb_n=0\)이면 \(\displaystyle\lim_{n\to\infty}a_n=0\) 또는 \(\displaystyle\lim_{n\to\infty}b_n=0\)이다.
\end{enumerate}
\end{mdframed}
\tabb{\text{ㄱ}}{\text{ㄴ}}{\text{ㄷ}}{\text{ㄱ, ㄴ}}{\text{ㄱ, ㄷ}}
\vs

%형6
\prob{}
수열 \(\{a_n\}\)이 모든 자연수 \(n\)에 대하여 \(-1\le a_n\le 1\)을 만족시킬 때, \(\displaystyle\lim_{n\to\infty}\frac{a_n-n^2}{\sqrt{2n^4+1}}\)의 값은?
\tabb{-\frac{\sqrt2}2}{-\frac12}{0}{\frac12}{\frac{\sqrt2}2}

%%형7
%\prob{}
%모든 항이 자연수인 수열 \(\{a_n\}\)에 대하여 \(a_n\)은 \(n\)자리의 수라 할 때, \(\displaystyle\lim_{n\to\infty}\frac{a_n}{11^{n-1}}\)의 값을 구하시오.
%\vs
%
%%형8
%\prob{}
%두 함수 \(f(x)=\sqrt{2x-1}\), \(g(x)=\sqrt x\)에 대하여 그림과 같이 직선 \(x=n\)이 곡선 \(y=f(x)\)와 만나는 점을 \(P\), 곡선 \(g(x)\)와 만나는 점을 \(Q\), \(x\)축과 만나는 점을 \(A\)라고 할 때,
%삼각형 \(OAP\)의 넓이를 \(S_P(n)\), 삼각형 \(OAQ\)의 넓이를 \(S_Q(n)\)이라 하자.\\
%\(\displaystyle\lim_{n\to\infty}\frac{S_P(n)-S_Q(n)}{S_P(n)+S_Q(n)}=p+q\sqrt2\)가 성립할 때, 정수 \(p\), \(q\)에 대하여 \(p^2+q^2\)의 값을 구하시오.
%(단, \(n\)은 \(2\) 이상의 자연수이다.)
%\vs
%
%%서1
%\prob{}
%등비수열 \(\{a_n\}\)의 공비가 \(3\)일 때,
%\(\displaystyle\lim_{n\to\infty}\frac{a_n}{3^{n+1}+2^{n+1}}=2\)가 성립한다.
%등비수열 \(\{a_n\}\)의 첫째항을 구하시오.
%\vs
%
%%서2
%\prob{}
%\(\displaystyle\lim_{n\to\infty}\frac{\sqrt{n+3}-\sqrt n}{\sqrt{n+1}-\sqrt n}\)의 값을 구하시오.
%\vs
%
%%서3
%\prob{}
%수열 \(\left\{\left(\frac12x(x+1)\right)^n\right\}\)이 수렴하기 위한 \(x\)의 값의 범위를 구하시오.
%\vs
%
%%서4
%\prob{}
%수열 \(\left\{\frac{3r^n}{1+2r^n}\right\}\)의 수렴, 발산을 양수 \(r\)의 값의 범위에 따라 구하시오.
\end{document}