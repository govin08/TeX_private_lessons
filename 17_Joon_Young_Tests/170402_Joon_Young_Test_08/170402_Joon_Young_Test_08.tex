\documentclass[a4paper]{oblivoir}
\usepackage{amsmath,amssymb,kotex,kswrapfig,mdframed,paralist}
\usepackage{fapapersize}
\usefapapersize{210mm,297mm,20mm,*,20mm,*}

\usepackage{tabto,pifont}
\TabPositions{0.2\textwidth,0.4\textwidth,0.6\textwidth,0.8\textwidth}
\newcommand\tabb[5]{\par\noindent
\ding{172}\:{\ensuremath{#1}}
\tab\ding{173}\:\:{\ensuremath{#2}}
\tab\ding{174}\:\:{\ensuremath{#3}}
\tab\ding{175}\:\:{\ensuremath{#4}}
\tab\ding{176}\:\:{\ensuremath{#5}}}

\pagestyle{empty}

%%% Counters
\newcounter{num}

%%% Commands
\newcommand\defi[1]
{\bigskip\par\noindent\stepcounter{num} \textbf{정의 \thenum) #1}\par\noindent}
\newcommand\theo[1]
{\bigskip\par\noindent\stepcounter{num} \textbf{정리 \thenum) #1}\par\noindent}
\newcommand\exam[1]
{\bigskip\par\noindent\stepcounter{num} \textbf{예시 \thenum) #1}\par\noindent}
\newcommand\prob[1]
{\bigskip\par\noindent\stepcounter{num} \textbf{문제 \thenum) #1}\par\noindent}

\newcommand\pb[1]{\ensuremath{\fbox{\phantom{#1}}}}

\newcommand\ba{\ensuremath{\:|\:}}

\newcommand\vs[1]{\vspace{25pt}}

\newcommand\an[1]{\bigskip\par\noindent\textbf{문제 #1)}\par\noindent}

%%% Meta Commands
\let\oldsection\section
\renewcommand\section{\clearpage\oldsection}

\let\emph\textsf

\begin{document}
\begin{center}
\LARGE준영, 미니테스트 08
\end{center}
\begin{flushright}
날짜 : 2017년 \(\pb3\)월 \(\pb{10}\)일 \(\pb{월}\)요일
,\qquad
제한시간 : \pb{17년}분
,\qquad
점수 : \pb{20} / \pb{20}
\end{flushright}

%
\prob{}
다음 <보기> 중 옳은 것만을 있는 대로 고른 것은?
\begin{mdframed}[frametitle=<보기>]
ㄱ. \(\displaystyle\lim_{n\to\infty}\frac{6n^2}{3n^3-n}=0\)
\tabto{0.5\textwidth}
ㄴ. \(\displaystyle\lim_{n\to\infty}\frac{4n+3}n=7\)
\tabto{0.5\textwidth}\par\noindent
ㄷ. \(\displaystyle\lim_{n\to\infty}\left(-\frac12\right)^n=1\)
\tabto{0.5\textwidth}
ㄹ. \(\displaystyle\lim_{n\to\infty}\frac{2^n-4^n}{3^n+5^n}=0\)
\tabto{0.5\textwidth}\par\noindent
ㅁ. \(\displaystyle\lim_{n\to\infty}(7-3n)=\infty\)
\end{mdframed}
\tabb{\text{ㄱ, ㄹ}}{\text{ㄴ, ㄷ}}{\text{ㄴ, ㄹ}}{\text{ㄹ, ㅁ}}{\text{ㄱ, ㄷ, ㅁ}}
\vs

%
\prob{}
\(\displaystyle\lim_{n\to\infty}\{\log_2(2n-1)+\log_2(8n+1)-2\log_2(n+1)\}\)의 값은?
\tabb1248{16}
\vs

%
\prob{}
\(f(x)=\displaystyle\lim_{n\to\infty}\frac{x^{n+2}-6x+2}{x^n+1}\)에 대하여 \(f\left(-\frac12\right)+f(4)\)의 값을 구하여라.
\vs

%
\prob{}
수열 \(\{a_n\}\)에 대하여 \(\displaystyle\lim_{n\to\infty}\frac{2a_n+1}{5a_n-3}=-1\)일 때, \(\displaystyle\lim_{n\to\infty}\frac{a_n+1}{a_n-1}\)의 값은?
\tabb{-10}{-5}05{10}
\vs

%
\prob{}
함수 \(f(x)=\displaystyle\lim_{n\to\infty}\frac{x^2-x^{2n+1}}{2+x^{2n}}\)에 대하여 다음 <보기> 중 옳은 것만을 있는 대로 고른 것은?
\begin{mdframed}[frametitle=<보기>]
\begin{enumerate}
\item[ㄱ.]
\(|x|>1\)이면 \(f(x)=-x\)이다.
\item[ㄴ.]
\(|x|=1\)이면 \(f(x)=0\)이다.
\item[ㄷ.]
\(|x|<1\)이면 \(f(x)=\frac12x^2\)이다.
\end{enumerate}
\end{mdframed}
\tabb{\text{ㄱ}}{\text{ㄴ}}{\text{ㄷ}}{\text{ㄱ, ㄷ}}{\text{ㄱ, ㄴ, ㄷ}}
\vs

%
\prob{}
수열 \(\{a_n\}\)에 대하여 \(\displaystyle\lim_{n\to\infty}\frac{na_n-3}{2a_n+1}=1\)일 떄, \(\displaystyle\lim_{n\to\infty}na_n\)의 값을 구하여라.

\end{document}