\documentclass[a4paper]{oblivoir}
\usepackage{amsmath,amssymb,kotex,kswrapfig,mdframed,paralist}
\usepackage{fapapersize}
\usefapapersize{210mm,297mm,20mm,*,20mm,*}

\usepackage{tabto,pifont}
\TabPositions{0.2\textwidth,0.4\textwidth,0.6\textwidth,0.8\textwidth}
\newcommand\tabb[5]{\par\noindent
\ding{172}\:{\ensuremath{#1}}
\tab\ding{173}\:\:{\ensuremath{#2}}
\tab\ding{174}\:\:{\ensuremath{#3}}
\tab\ding{175}\:\:{\ensuremath{#4}}
\tab\ding{176}\:\:{\ensuremath{#5}}}

\usepackage{graphicx}

\pagestyle{empty}

%%% Counters
\newcounter{num}

%%% Commands
\newcommand\defi[1]
{\bigskip\par\noindent\stepcounter{num} \textbf{정의 \thenum) #1}\par\noindent}
\newcommand\theo[1]
{\bigskip\par\noindent\stepcounter{num} \textbf{정리 \thenum) #1}\par\noindent}
\newcommand\exam[1]
{\bigskip\par\noindent\stepcounter{num} \textbf{예시 \thenum) #1}\par\noindent}
\newcommand\prob[1]
{\bigskip\par\noindent\stepcounter{num} \textbf{문제 \thenum) #1}\par\noindent}

\newcommand\pb[1]{\ensuremath{\fbox{\phantom{#1}}}}

\newcommand\ba{\ensuremath{\:|\:}}

\newcommand\vs[1]{\vspace{60pt}}

\newcommand\an[1]{\bigskip\par\noindent\textbf{문제 #1)}\par\noindent}

%%% Meta Commands
\let\oldsection\section
\renewcommand\section{\clearpage\oldsection}

\let\emph\textsf

\begin{document}
\begin{center}
\LARGE준영, 미니테스트 10
\end{center}
\begin{flushright}
날짜 : 2017년 \(\pb3\)월 \(\pb{10}\)일 \(\pb{월}\)요일
,\qquad
제한시간 : \pb{17년}분
,\qquad
점수 : \pb{20} / \pb{20}
\end{flushright}

%
\prob{}
다음 급수의 합을 구하여라.
\begin{enumerate}[(1)]
\item
\(\displaystyle1+\frac13+\frac19+\frac1{27}+\cdots\)
\item
\(\displaystyle\frac1{2\cdot5}+\frac1{5\cdot8}+\frac1{8\cdot11}+\frac1{11\cdot14}+\cdots\)
\end{enumerate}
\vs

%
\prob{}
급수 \(\displaystyle\sum_{n=2}^\infty\left(\log_n{10}-\log_{n+1}10\right)\)의 합은?
\vspace{10pt}
\tabb{2\log2}0{\displaystyle\frac1{\log2}}{\log2}1
\vs

%
\prob{}
급수 \(\displaystyle1-\frac12+\frac12-\frac13+\frac13-\cdots\)의 합은?
\vspace{10pt}
\tabb01234
\vs

%
\prob{}
다음 급수의 합을 구하여라.
\begin{enumerate}[(1)]
\item
\(\displaystyle\sum_{n=1}^\infty\left(\frac13\right)^{n-1}\)
\item
\(\displaystyle\sum_{n-1}^\infty\frac{2^n+3^n}{5^n}\)
\item
\(\displaystyle\sum_{n-1}^\infty\left\{\frac6{5^n}+\frac2{(-3)^n}\right\}\)
\end{enumerate}
\vs

\end{document}