\documentclass[a4paper]{oblivoir}
\usepackage{amsmath,amssymb,kotex,kswrapfig,mdframed,paralist}
\usepackage{fapapersize}
\usefapapersize{210mm,297mm,20mm,*,20mm,*}

\usepackage{tabto,pifont}
\TabPositions{0.2\textwidth,0.4\textwidth,0.6\textwidth,0.8\textwidth}
\newcommand\tabb[5]{\par\noindent
\ding{172}\:{\ensuremath{#1}}
\tab\ding{173}\:\:{\ensuremath{#2}}
\tab\ding{174}\:\:{\ensuremath{#3}}
\tab\ding{175}\:\:{\ensuremath{#4}}
\tab\ding{176}\:\:{\ensuremath{#5}}}

\pagestyle{empty}

%%% Counters
\newcounter{num}

%%% Commands
\newcommand\defi[1]
{\bigskip\par\noindent\stepcounter{num} \textbf{정의 \thenum) #1}\par\noindent}
\newcommand\theo[1]
{\bigskip\par\noindent\stepcounter{num} \textbf{정리 \thenum) #1}\par\noindent}
\newcommand\exam[1]
{\bigskip\par\noindent\stepcounter{num} \textbf{예시 \thenum) #1}\par\noindent}
\newcommand\prob[1]
{\bigskip\par\noindent\stepcounter{num} \textbf{문제 \thenum) #1}\par\noindent}

\newcommand\pb[1]{\ensuremath{\fbox{\phantom{#1}}}}

\newcommand\ba{\ensuremath{\:|\:}}

\newcommand\vs[1]{\vspace{20pt}}

\newcommand\an[1]{\bigskip\par\noindent\textbf{문제 #1)}\par\noindent}

%%% Meta Commands
\let\oldsection\section
\renewcommand\section{\clearpage\oldsection}

\let\emph\textsf

\begin{document}
\begin{center}
\LARGE준영, 미니테스트 02
\end{center}
\begin{flushright}
날짜 : 2017년 \(\pb3\)월 \(\pb{10}\)일 \(\pb{월}\)요일
,\qquad
제한시간 : \pb{17년}분
,\qquad
점수 : \pb{20} / \pb{20}
\end{flushright}

%유5a
\prob{01-5}
다음 극한값을 구하시오.\\
(1) \(\displaystyle\lim_{n\to\infty}\frac{7^{n-1}-5^n}{7^n+5^{n+1}}\)
\tabto{0.5\textwidth}
(2) \(\displaystyle\lim_{n\to\infty}\frac{3^{2n+1}-2^{3n}}{3^{2n}-2^{3n-1}}\)
\vs

%형1
\prob{}
<보기>의 수열 중 수렴하는 것만을 있는 대로 고른 것은?
\begin{mdframed}[frametitle=<보기>]
ㄱ. \(\left\{\frac{n^4-5}{3n^4}\right\}\)
\tabto{0.33\textwidth}
ㄴ. \(\left\{\frac1{2^n}(2^{n+2}-1)\right\}\)
\tabto{0.66\textwidth}
ㄷ. \(\left\{\left(\frac{2\sqrt{13}}{3\pi}\right)^n\right\}\)
\end{mdframed}
\tabb{\text{ㄱ}}{\text{ㄷ}}{\text{ㄱ, ㄴ}}{\text{ㄱ, ㄷ}}{\text{ㄱ, ㄴ, ㄷ}}
\vs

%형2
\prob{}
수열 \(\{a_n\}\)에 대하여 \(\displaystyle\lim_{n\to\infty}(n+2)^2a_n=3\)일 때, \(\displaystyle\lim_{n\to\infty}(2n^2-3n)a_n\)의 값은?
\tabb6789{10}
\vs

%형3
\prob{}
\(\displaystyle\lim_{n\to\infty}\frac{\sqrt{5n+1}}{n\left(\sqrt{3n+2}-\sqrt{3n-1}\right)}\)의 값은?
\tabb{\frac{5\sqrt5}9}{\frac{2\sqrt5}3}{\frac{7\sqrt5}9}{\frac{2\sqrt{15}}3}{\frac{7\sqrt{15}}9}
\vs

%형4
\prob{}
\(\displaystyle\sum_{k=1}^na_k=(n+1)^2\)일 때, \(\displaystyle\lim_{n\to\infty}\frac{a_1n^2}{{a_{2n}}^2}\)의 값은?

\bigskip
\tabb{\frac18}{\frac14}{\frac38}{\frac12}{\frac58}
\vs

%형5
\prob{}
두 수열 \(\{a_n\}\), \(\{b_n\}\)에 대하여 옳은 것만을 <보기>에서 있는 대로 고른 것은?
\begin{mdframed}[frametitle=<보기>]
\begin{enumerate}[ㄱ.]
\item
모든 자연수 \(n\)에 대하여 \(a_n<b_n\)이고, \(\displaystyle\lim_{n\to\infty}a_n=\infty\)이면 \(\displaystyle\lim_{n\to\infty}b_n=\infty\)이다.
\item[ㄴ.]
수열 \(\{a_{2n+1}\}\)이 수렴하면 수열 \(\{a_n\}\)은 수렴한다.
\item[ㄷ.]
\(\displaystyle\lim_{n\to\infty}a_nb_n=0\)이면 \(\displaystyle\lim_{n\to\infty}a_n=0\) 또는 \(\displaystyle\lim_{n\to\infty}b_n=0\)이다.
\end{enumerate}
\end{mdframed}
\tabb{\text{ㄱ}}{\text{ㄴ}}{\text{ㄷ}}{\text{ㄱ, ㄴ}}{\text{ㄱ, ㄷ}}
\vs

%%형6
%\prob{}
%수열 \(\{a_n\}\)이 모든 자연수 \(n\)에 대하여 \(-1\le a_n\le 1\)을 만족시킬 때, \(\displaystyle\lim_{n\to\infty}\frac{a_n-n^2}{\sqrt{2n^4+1}}\)의 값은?
%\tabb{-\frac{\sqrt2}2}{-\frac12}{0}{\frac12}{\frac{\sqrt2}2}
\end{document}