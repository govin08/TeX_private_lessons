\documentclass[a4paper]{oblivoir}
\usepackage{amsmath,amssymb,kotex,kswrapfig,mdframed,paralist}
\usepackage{fapapersize}
\usefapapersize{210mm,297mm,20mm,*,20mm,*}

\usepackage{tabto,pifont}
\TabPositions{0.2\textwidth,0.4\textwidth,0.6\textwidth,0.8\textwidth}
\newcommand\tabb[5]{\par\noindent
\ding{172}\:{\ensuremath{#1}}
\tab\ding{173}\:\:{\ensuremath{#2}}
\tab\ding{174}\:\:{\ensuremath{#3}}
\tab\ding{175}\:\:{\ensuremath{#4}}
\tab\ding{176}\:\:{\ensuremath{#5}}}

\pagestyle{empty}

%%% Counters
\newcounter{num}

%%% Commands
\newcommand\defi[1]
{\bigskip\par\noindent\stepcounter{num} \textbf{정의 \thenum) #1}\par\noindent}
\newcommand\theo[1]
{\bigskip\par\noindent\stepcounter{num} \textbf{정리 \thenum) #1}\par\noindent}
\newcommand\exam[1]
{\bigskip\par\noindent\stepcounter{num} \textbf{예시 \thenum) #1}\par\noindent}
\newcommand\prob[1]
{\bigskip\par\noindent\stepcounter{num} \textbf{문제 \thenum) #1}\par\noindent}

\newcommand\pb[1]{\ensuremath{\fbox{\phantom{#1}}}}

\newcommand\ba{\ensuremath{\:|\:}}

\newcommand\vs[1]{\vspace{30pt}}

\newcommand\an[1]{\bigskip\par\noindent\textbf{문제 #1)}\par\noindent}

%%% Meta Commands
\let\oldsection\section
\renewcommand\section{\clearpage\oldsection}

\let\emph\textsf

\begin{document}
\begin{center}
\LARGE준영, 미니테스트 07
\end{center}
\begin{flushright}
날짜 : 2017년 \(\pb3\)월 \(\pb{10}\)일 \(\pb{월}\)요일
,\qquad
제한시간 : \pb{17년}분
,\qquad
점수 : \pb{20} / \pb{20}
\end{flushright}

%
\prob{}
자연수 \(n\)에 대하여 이차함수 \(f(x)=3x^2\)이 그래프 위의 점 \(P(n.f(n))\), \(Q(n+1,f(n+1))\) 사이의 거리를 \(a_n\)이라 할 때, \(\displaystyle\lim_{n\to\infty}\frac{a_n}n\)의 값을 구하여라.
\vs

%
\prob{}
자연수 \(n\)에 대하여 \(\sqrt{9n^2+5n+1}\)의 소수 부분을 \(a_n\)이라 할 때, \(\displaystyle\lim_{n\to\infty}a_n\)의 값을 구하여라.
\vs

%
\prob{}
다음 <보기>의 수열 중 수렴하는 것만을 있는 대로 고른 것은?
\begin{mdframed}[frametitle=<보기>]
ㄱ. \(\{3+(-1)^n\}\)
\tabto{0.25\textwidth}
ㄴ. \(\{(\log2+\log5)^n\}\)
\tabto{0.5\textwidth}
ㄷ. \(\displaystyle\left\{\left(\frac23\right)^{1-n}\right\}\)
\tabto{0.75\textwidth}
ㄹ. \(\displaystyle\left\{2^{-n}+3^{-n}\right\}\)
\end{mdframed}
\tabb{\text{ㄱ, ㄴ}}{\text{ㄱ, ㄷ}}{\text{ㄴ, ㄹ}}{\text{ㄱ, ㄴ, ㄹ}}{\text{ㄱ, ㄷ, ㄹ}}
\vs

%
\prob{}
\(\displaystyle\lim_{n\to\infty}\frac{3^n}{1+3+3^2+\cdots+3^n}\)의 값은?
\vs

%
\prob{}
\(\displaystyle\lim_{n\to\infty}\frac{r^{2n}}{1+r^{2n}}\)의 값이 \(|r|>1\)이면 \(a\), \(|r|=1\)이면 \(b\), \(|r|<1\)이면 \(c\)일 때, \(a+b-c\)의 값을 구하여라.
\vs

%
\prob{}
\(a_1=1\), \(a_2=3\), \(2a_{n+2}-3a_{n+1}+a_n=0\) (\(n=1,2,3,\cdots\))으로 정의된 수열 \(\{a_n\}\)에서 \(\displaystyle\lim_{n\to\infty}a_n\)의 값은?
\tabb54321
\vs

%
\prob{}
모든 자연수 \(n\)에 대하여 이차방정식
\[x^2-2\sqrt{a_n}x+3(a_{n+1}+2)=0\]
이 중근을 가질 때, \(\displaystyle\lim_{n\to\infty}a_n\)의 값을 구하여라.

\end{document}