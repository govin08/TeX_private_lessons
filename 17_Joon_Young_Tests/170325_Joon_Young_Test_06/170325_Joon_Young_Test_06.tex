\documentclass[a4paper]{oblivoir}
\usepackage{amsmath,amssymb,kotex,kswrapfig,mdframed,paralist}
\usepackage{fapapersize}
\usefapapersize{210mm,297mm,20mm,*,20mm,*}

\usepackage{tabto,pifont}
\TabPositions{0.2\textwidth,0.4\textwidth,0.6\textwidth,0.8\textwidth}
\newcommand\tabb[5]{\par\noindent
\ding{172}\:{\ensuremath{#1}}
\tab\ding{173}\:\:{\ensuremath{#2}}
\tab\ding{174}\:\:{\ensuremath{#3}}
\tab\ding{175}\:\:{\ensuremath{#4}}
\tab\ding{176}\:\:{\ensuremath{#5}}}

\pagestyle{empty}

%%% Counters
\newcounter{num}

%%% Commands
\newcommand\defi[1]
{\bigskip\par\noindent\stepcounter{num} \textbf{정의 \thenum) #1}\par\noindent}
\newcommand\theo[1]
{\bigskip\par\noindent\stepcounter{num} \textbf{정리 \thenum) #1}\par\noindent}
\newcommand\exam[1]
{\bigskip\par\noindent\stepcounter{num} \textbf{예시 \thenum) #1}\par\noindent}
\newcommand\prob[1]
{\bigskip\par\noindent\stepcounter{num} \textbf{문제 \thenum) #1}\par\noindent}

\newcommand\pb[1]{\ensuremath{\fbox{\phantom{#1}}}}

\newcommand\ba{\ensuremath{\:|\:}}

\newcommand\vs[1]{\vspace{40pt}}

\newcommand\an[1]{\bigskip\par\noindent\textbf{문제 #1)}\par\noindent}

%%% Meta Commands
\let\oldsection\section
\renewcommand\section{\clearpage\oldsection}

\let\emph\textsf

\begin{document}
\begin{center}
\LARGE준영, 미니테스트 06
\end{center}
\begin{flushright}
날짜 : 2017년 \(\pb3\)월 \(\pb{10}\)일 \(\pb{월}\)요일
,\qquad
제한시간 : \pb{17년}분
,\qquad
점수 : \pb{20} / \pb{20}
\end{flushright}

%
\prob{}
수열 \(\{a_n\}\)이 모든 자연수 \(n\)에 대하여
\[3n^2-n<(n^2+1)a_n<3n^2+n\]
을 만족시킬 때, \(\displaystyle\lim_{n\to\infty}a_n\)의 값은?
\tabb{-3}{-1}013
\vs

%
\prob{}
자연수 \(n\)에 대하여 \(\sqrt{9n^2+5n+1}\)의 소수 부분을 \(a_n\)이라 할 때, \(\displaystyle\lim_{n\to\infty}a_n\)의 값을 구하여라.
\vs

%
\prob{}
다음 <보기>의 수열 중 수렴하는 것만을 있는 대로 고른 것은?
\begin{mdframed}[frametitle=<보기>]
ㄱ. \(\{(-3)^n\}\)
\tabto{0.25\textwidth}
ㄴ. \(\{(\log4-\log5)^n\}\)
\tabto{0.5\textwidth}
ㄷ. \(\displaystyle\left\{\left(-\frac34\right)^n\right\}\)
\tabto{0.75\textwidth}
ㄹ. \(\displaystyle\left\{\frac{5^n}{4^{n+1}}\right\}\)
\end{mdframed}
\tabb{\text{ㄱ, ㄴ}}{\text{ㄴ, ㄷ}}{\text{ㄷ, ㄹ}}{\text{ㄱ, ㄴ, ㄷ}}{\text{ㄴ, ㄷ, ㄹ}}
\vs

%
\prob{}
\(\displaystyle\lim_{n\to\infty}\frac{2^{n+1}+3^{n-1}}{\sqrt{9^n+2^{2n}}}\)의 값은?
\tabb{\frac13}{\frac23}1{\frac43}{\frac53}
\vs

%
\prob{}
다음 중 수열 \(\displaystyle\left\{\frac{r^{2n+1}-1}{r^{2n}+r^2}\right\}\)의 극한값이 될 수 없는 것은? (단 \(r\neq0\))\par\medskip
\tabb{-1}{\frac12}{\frac32}2{\frac52}
\vs

%
\prob{}
\(a_1=2\), \(2a_{n+1}=a_n+3\) (\(n=1,2,3,\cdots\))으로 정의된 수열 \(\{a_n\}\)에서 \(\displaystyle\lim_{n\to\infty}a_n\)의 값은?
\tabb{\frac32}{\frac53}23{\frac72}
\vs


\end{document}