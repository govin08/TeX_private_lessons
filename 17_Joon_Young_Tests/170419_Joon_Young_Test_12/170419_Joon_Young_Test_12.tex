\documentclass[a4paper]{oblivoir}
\usepackage{amsmath,amssymb,kotex,kswrapfig,mdframed,paralist}
\usepackage{fapapersize}
\usefapapersize{210mm,297mm,20mm,*,20mm,*}

\usepackage{tabto,pifont}
\TabPositions{0.2\textwidth,0.4\textwidth,0.6\textwidth,0.8\textwidth}
\newcommand\tabb[5]{\par\noindent
\ding{172}\:{\ensuremath{#1}}
\tab\ding{173}\:\:{\ensuremath{#2}}
\tab\ding{174}\:\:{\ensuremath{#3}}
\tab\ding{175}\:\:{\ensuremath{#4}}
\tab\ding{176}\:\:{\ensuremath{#5}}}

\usepackage{graphicx}

\pagestyle{empty}

%%% Counters
\newcounter{num}

%%% Commands
\newcommand\prob[1]
{\bigskip\par\noindent\stepcounter{num} \textbf{문제 \thenum) #1}\par\noindent}

\newcommand\pb[1]{\ensuremath{\fbox{\phantom{#1}}}}

\newcommand\ba{\ensuremath{\:|\:}}

\newcommand\vs[1]{\vspace{25pt}}

\newcommand\an[1]{\bigskip\par\noindent\textbf{문제 #1)}\par\noindent}

%%% Meta Commands
\let\oldsection\section
\renewcommand\section{\clearpage\oldsection}

\let\emph\textsf

\begin{document}
\begin{center}
\LARGE준영, 미니테스트 12
\end{center}
\begin{flushright}
날짜 : 2017년 \(\pb3\)월 \(\pb{10}\)일 \(\pb{월}\)요일
,\qquad
제한시간 : \pb{17년}분
,\qquad
점수 : \pb{20} / \pb{20}
\end{flushright}

%
\prob{}
두 함수 \(f(x)\), \(g(x)\)가 다음 조건을 모두 만족시킬 때,
\(\displaystyle\lim_{x\to0}\frac{x+f(x)g(x)}{x^2-f(x)}\)
의 값을 구하여라.
\begin{mdframed}
\begin{enumerate}
\item[(가)]
\(f(x)\{g(x)+2\}=x\{g(x)-2\}\)
\item[(나)]
\(\displaystyle\lim_{x\to0}g(x)=1\)
\end{enumerate}
\end{mdframed}

%
\prob{}
최고차항의 계수가 \(1\)인 두 삼차함수 \(f(x)\), \(g(x)\)가 다음 조건을 만족시킨다.
\begin{mdframed}
\begin{enumerate}
\item[(가)]
\(g(x)=1\)
\item[(나)]
\(\displaystyle\lim_{x\to n}\frac{f(x)}{g(x)}=(n-1)(n-2)\) \((n=1,2,3,4)\)
\end{enumerate}
\end{mdframed}
\(g(5)\)의 값은?
\tabb468{10}{12}
\end{document}