\documentclass[a4paper]{oblivoir}
\usepackage{amsmath,amssymb,kotex,kswrapfig,mdframed,paralist}
\usepackage{fapapersize}
\usefapapersize{210mm,297mm,20mm,*,20mm,*}

\usepackage{tabto,pifont}
\TabPositions{0.2\textwidth,0.4\textwidth,0.6\textwidth,0.8\textwidth}
\newcommand\tabb[5]{\par\noindent
\ding{172}\:{\ensuremath{#1}}
\tab\ding{173}\:\:{\ensuremath{#2}}
\tab\ding{174}\:\:{\ensuremath{#3}}
\tab\ding{175}\:\:{\ensuremath{#4}}
\tab\ding{176}\:\:{\ensuremath{#5}}}

\usepackage{graphicx}

%\pagestyle{empty}

%%% Counters
\newcounter{num}

%%% Commands
\newcommand\prob[1]
{\vs\bigskip\bigskip\par\noindent\stepcounter{num} \textbf{문제 \thenum) #1}\par\noindent}

\newcommand\pb[1]{\ensuremath{\fbox{\phantom{#1}}}}

\newcommand\ba{\ensuremath{\:|\:}}

\newcommand\vs[1]{\vspace{40pt}}

\newcommand\an[1]{\bigskip\par\noindent\textbf{문제 #1)}\par\noindent}

%%% Meta Commands
\let\oldsection\section
\renewcommand\section{\clearpage\oldsection}

\let\emph\textsf

\begin{document}
\begin{center}
\LARGE준영, 미니테스트 17
\end{center}
\begin{flushright}
날짜 : 2017년 \(\pb3\)월 \(\pb{10}\)일 \(\pb{월}\)요일
,\qquad
제한시간 : \pb{17년}분
,\qquad
점수 : \pb{20} / \pb{20}
\end{flushright}

%
\prob{}
곡선 \(y=x^3-2x^2+1\) 위의 점 \((-1,-2)\)에서의 접선의 방정식을 구하여라.

%
\prob{}
곡선 \(y=x^3+ax+b\) 위의 점 \((-1,-2)\)에서의 접선의 방정식이 \(y=2x+3\)일 때, 상수 \(a\), \(b\)의 값을 구하여라.

%
\prob{}
곡선 \(y=x^3-x+1\) 위의 점 \((1,1)\)을 지나고 이 점에서의 접선에 수직인 직선의 방정식을 구하여라.

%
\prob{}
곡선 \(y=x^2\)의 접선이 \(x\)축의 양의 방향과 \(45^\circ\)의 각을 이룰 때, 그 접선의 방정식을 구하여라.

%
\prob{}
직선 \(2x-y+3=0\)에 평행하고 곡선 \(y=-x^2+1\)에 접하는 직선의 방정식을 구하여라.

%
\prob{}
직선 \(x-8y+3=0\)에 수직이고 곡선 \(y=x^3-11x+2\)에 접하는 직선의 방정식을 구하여라.

%
\prob{}
다음 주어진 점에서 곡선에 그은 접선의 방정식을 구하여라.
\begin{enumerate}[(1)]
\item
\(y=-x^2+2x+3,\quad(2,4)\)
\item
\(y=x^3-2x,\quad(0,2)\)
\end{enumerate}

%
\prob{}
점 \((1,-6)\)에서 곡선 \(y=x^3-2\)에 그은 접선이 점 \((k,30)\)을 지날 때, \(k\)의 값을 구하여라.

%
\prob{}
다음을 구하여라.
\begin{enumerate}[(1)]
\item
곡선 \(f(x)=x^3+ax+b\)가 점 \((0,1)\)에서 \(y=x+1\)에 접할 때, 상수 \(a\), \(b\)의 값
\item
\(y=x^3-ax+2\)가 직선 \(y=2x\)에 접할 때, 상수 \(a\)의 값
\end{enumerate}

%
\prob{}
두 곡선 \(f(x)=x^2+ax+b\), \(g(x)=-x^3+c\)가 \((1,-2)\)에서 접하도록 하는 상수 \(a\), \(b\), \(c\)의 값을 각각 구하여라.
또, 이때의 공통접선을 구하여라.

%
\prob{}
두 곡선 \(y=x^3\), \(y=ax^2+bx\)가 점 \((1,1)\)에서 만나고 이 점에서 두 곡선에 그은 접선이 서로 수직일 때, 상수 \(a\), \(b\)의 값을 구하여라.

%
\prob{}
다음 함수에 대하여 주어진 구간에서 롤의 정리를 만족시키는 상수 \(c\)의 값을 구하여라.
\begin{enumerate}[(1)]
\item
\(f(x)=4x-x^2,\quad[1,3]\)
\item
\(f(x)=(x+1)^2(x-2),\quad[-1,2]\)
\end{enumerate}

%
\prob{}
다음 함수에 대하여 주어진 구간에서 평균값 정리를 만족시키는 상수 \(c\)의 값을 구하여라.
\begin{enumerate}[(1)]
\item
\(f(x)=3x^2+2x+1,\quad[-1,1]\)
\item
\(f(x)=x^3+2x,\quad[0,3]\)
\end{enumerate}

\end{document}