\documentclass{article}
\usepackage{amsmath,amssymb,amsthm,kotex,mdframed,paralist}
\newcounter{num}
\newcommand{\defi}[1]
{\bigskip\noindent\refstepcounter{num}\textbf{정의 \arabic{num}) #1}\par}
\newcommand{\theo}[1]
{\bigskip\noindent\refstepcounter{num}\textbf{정리 \arabic{num}) #1}\par}
\newcommand{\exam}[1]
{\bigskip\noindent\refstepcounter{num}\textbf{예시 \arabic{num}) #1}\par}
\newcommand{\prob}[1]
{\bigskip\noindent\refstepcounter{num}\textbf{연습문제 \arabic{num}) #1}\par}

\newcommand{\pr}
{\begin{mdframed}[skipabove=10pt,skipbelow=10pt]
증명)
\vspace{0.5\textheight}
\par
\end{mdframed}
\par
}
\newcommand{\sol}
{\begin{mdframed}[skipabove=10pt]
풀이)
\vspace{0.5\textheight}
\par
\end{mdframed}
\par
}

\renewcommand{\proofname}{증명)}
\newcommand{\mo}[1]{\ensuremath{\:(\text{mod}\:\:#1)}}

%%%
\begin{document}

\title{현빈 : 15 수의 합동(3)}
\author{}
\date{\today}
\maketitle
%\tableofcontents
%
%\newpage

%%
\setcounter{section}{5}
\setcounter{num}{30}

%
\section{중국인의 나머지 정리(The Chinese Remainder Theorem)}

%
\exam{}
다음과 같은 문제를 풀어보자.

\begin{quote}
\begin{mdframed}
5으로 나누었을 때의 나머지가 1이고, 6으로 나누었을 때의 나머지가 2이고, 7로 나누었을 때의 나머지가 3인 숫자를 구하여라.
\end{mdframed}
\end{quote}

이 문제는 다음 연립 일차 합동식을 푸는 것과 같다.
\begin{gather}
x\equiv1\mo{5}\\
x\equiv2\mo{6}\\
x\equiv3\mo{7}
\end{gather}

(1)에 의해 \(x=5t+1\)이다.
이를 (2)에 넣으면
\[5t+1\equiv2\mo{6}\]
이고 1을 이항하면
\[5t\equiv1\mo{6}\]
이다.
따라서 \(t\)는 \(5\)의 역원인 \(5\)이다(mod 6).
즉
\[t\equiv5\mo{6}\]
혹은
\[t=6u+5\]
이다.
따라서
\[x=5(6u+5)+1=30u+26\]
이다.
이것을 다시 (3)에 넣으면
\[30u+26\equiv3\mo{7}\]
이고 \(30\equiv2\mo{7}\)이므로
\[2u\equiv30u\equiv3-26\equiv-23\equiv5\mo{7}\]
이다.
양변에 \(2^{-1}(\equiv4)\)를 곱하면
\[u\equiv20\equiv6\mo{7}\]
따라서
\[u=7v+6\]
이고
\[x=30(7v+6)+26=210v+206\]
이다.
즉
\[x\equiv206\mo{210}\]
인 모든 정수이다.

\bigskip
이와 같은 문제를 푸는 일반적인 해법은 5세기 남북조 시대의 중국 수학서 ≪손자산경(孫子算經)≫에 최초로 등장하였다.
따라서 ``중국인의 나머지 정리''라는 이름으로 불린다.

%
\theo{중국인의 나머지 정리(The Chinese Remainder Theorem)}
\(m_1\), \(m_2\), \(m_3\), \(\cdots\), \(m_r\)이 자연수이고 임의의 두 쌍의 \(m_i\), \(m_j\)가 서로소이다.
(즉 \(i\neq j\)이면 \((m_i,m_j)=1\)이다.)
\(M=m_1\cdots m_r\)이라고 하자.
그러면 다음 연립 합동 방정식의 해는 mod \(M\)에 대해 유일하게 존재한다.
\begin{gather*}
x\equiv a_1\mo{m_1}\\
x\equiv a_2\mo{m_2}\\
\vdots\\
x\equiv a_r\mo{m_r}
\end{gather*}

\begin{proof}
먼저 이 연립 합동 방정식의 해가 존재함을 밝히고, 그 다음에 그 해가 mod \(M\)에 대해 유일하다는 것을 밝히면 된다.

\smallskip\smallskip
(존재성) :\\
임의의 \(k\)에 대해(\(1\le k\le r\))
\[M_k=M/m_k\]
라고 하자.
가정에 의해
\[(M_k,m_k)=1\]
이다.
따라서 mod \(m_k\)에서 \(M_k\)의 역원 \(y_k\)가 존재한다.
즉
\[M_ky_k\equiv1\mo{m_k}\]
이다.
그러면
\[x=a_1M_1y_1+\cdots+a_rM_ry_r\]
이 연립 합동 방정식을 만족한다.

\smallskip\smallskip
(유일성) : \\
만약 \(x\), \(x'\)이 모두 연립 합동 방정식의 근이라면,
\begin{gather*}
m_1\mid x-x'\\
m_2\mid x-x'\\
\vdots\\
m_r\mid x-x'
\end{gather*}
이다.
따라서
\[M\mid x-x'\]
이고,
\[x\equiv x'\mo{M}\]
\end{proof}

\exam{}
연립 합동 방정식
\begin{gather*}
x\equiv 1\mo{3}\\
x\equiv 2\mo{5}\\
x\equiv 3\mo{7}\\
\end{gather*}
을 생각하자.
\(M=3\cdot5\cdot7=105\), \(M_1=105/3=35\), \(M_2=105/5=21\), \(M_3=105/7=15\)이다.
또 \(35y_1\equiv1\mo{3}\)를 풀면 \(y_1\equiv2\mo{3}\)이고, 마찬가지로 \(y_2\equiv1\mo{5}\), \(y_3\equiv 1\mo{7}\)이다.
따라서
\[x\equiv1\cdot35\cdot2+2\cdot21\cdot1+3\cdot15\cdot1\equiv151\equiv52\mo{105}\]
이다.

\prob{}
다음 연립 합동 방정식을 풀어라.
\begin{enumerate}[(1)]
\item
\begin{gather*}
x\equiv4\mo{11}\\
x\equiv3\mo{17}
\end{gather*}
\item
\begin{gather*}
x\equiv1\mo{2}\\
x\equiv2\mo{3}\\
x\equiv3\mo{5}
\end{gather*}
\item
\begin{gather*}
x\equiv0\mo{2}\\
x\equiv0\mo{3}\\
x\equiv1\mo{5}\\
x\equiv6\mo{7}
\end{gather*}
\end{enumerate}
\end{document}