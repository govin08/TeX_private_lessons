\documentclass{article}
\usepackage{amsmath,amssymb,amsthm,kotex,mdframed,paralist,chngcntr}

\newcounter{num}
\newcommand{\defi}[1]
{\bigskip\noindent\refstepcounter{num}\textbf{정의 \arabic{num}) #1}\par}
\newcommand{\theo}[1]
{\bigskip\noindent\refstepcounter{num}\textbf{정리 \arabic{num}) #1}\par}
\newcommand{\exam}[1]
{\bigskip\noindent\refstepcounter{num}\textbf{예시 \arabic{num}) #1}\par}
\newcommand{\prob}[1]
{\bigskip\noindent\refstepcounter{num}\textbf{문제 \arabic{num}) #1}\par}
\newcommand{\howo}[1]
{\bigskip\noindent\refstepcounter{num}\textbf{숙제 \arabic{num}) #1}\par}


\renewcommand{\proofname}{증명)}
\newcommand{\mo}[1]{\ensuremath{\:(\text{mod}\:\:#1)}}
\counterwithout{subsection}{section}


%%%
\begin{document}

\title{영석 : 05 이차부등식}
\author{}
\date{\today}
\maketitle
\tableofcontents
\newpage

%%
\subsection{연립 일차 부등식}

%
\prob{}
\begin{gather}
\begin{cases}
x>2\\
x<3
\end{cases}\\
\begin{cases}
x\le2\\
x\ge3
\end{cases}\\
\begin{cases}
x\ge2\\
x>3
\end{cases}\\
\begin{cases}
x\le2\\
x<3
\end{cases}\\
\begin{cases}
x\ge2\\
x<3
\end{cases}
\end{gather}

%
\prob{}
\setcounter{equation}{0}
\begin{gather}
\begin{cases}
x>-2\\
x<5
\end{cases}\\
\begin{cases}
x\le2\\
x\ge5
\end{cases}\\
\begin{cases}
x>2\\
x>1\\
x\le3
\end{cases}
\end{gather}

%
\howo{}
\setcounter{equation}{0}
\begin{gather}
\begin{cases}
x>1\\
x<3
\end{cases}\\
\begin{cases}
x\le-1\\
x>3
\end{cases}\\
\begin{cases}
x>1\\
x\ge3\\
x<5
\end{cases}
\end{gather}

%%
\subsection{절댓값이 포함된 부등식}

%
\defi{절댓값}
실수 \(a\)에 대해 \(|a|\)를
\[
|a|=\left\{
\begin{array}{cc}
a&(a>0)\\
-a&(a<0)
\end{array}
\right.
\]
로 정의하자.
읽을 때에는 `\textbf{절댓값} \(\mathbf a\)'라고 읽는다.

%
\exam{}
\(1\)은 \(1\ge0\)이므로 \(|1|=1\)이다.

\(0\)도 \(0\ge0\)이므로 \(|0|=0\)이다.

\(-1\)은 \(-1<0\)이므로 \(|-1|=-(-1)=1\)이다.

마찬가지의 논리에 의해
\begin{gather*}
|12|=12\\
|100|=100\\
|\frac12|=\frac12\\
|\sqrt 2|=\sqrt 2\\
|-5|=5\\
|-512|=512\\
|-\pi|=\pi\\
|-\frac4{2\sqrt 5}=\frac4{2\sqrt5}
\end{gather*}
등이 성립한다.

%
\prob{}
\begin{gather}
|-3|=\\
|2.4|=\\
\left|\frac{19}2\right|=\\
|\sqrt2-1|=\\
|\sqrt3-3|=\\
\left|\frac{\sqrt5-2}2\right|=\\
|-3|+|4|-|-2|=\\
\frac{|2|-|-4|}{|-3|}=
\end{gather}

%
\howo{}
\begin{gather}
\setcounter{equation}{0}
|2|=\\
|-2.14|=\\
\left|-\frac{21}5\right|=\\
|\sqrt2-\sqrt3|=\\
|\sqrt{50}-7|=\\
\left|\frac{\sqrt7-9}3\right|=\\
-|-2|+|-3|-|1|=\\
\frac{|3|+|-4|}{|-7|}=
\end{gather}

%
\prob{}
(1)
\(a>3\)일 때, \(|a-3|=\)

(2)
\(a<-2\)일 때, \(|a+2|=\)

%
\prob{}
(1)
\(a\ge-1\)일 때, \(|a+1|=\)

(2)
\(a\le 4\)일 때, \(|a-4|=\)

(3)
\(a>1\)일 때, \(|a+2|=\)

(4)
\(a<-4\)일 때, \(|a|=\)

%
\howo{}
(1)
\(a\ge-2\)일 때, \(|a+2|=\)

(2)
\(a\le 3\)일 때, \(|a-3|=\)

(3)
\(a>5\)일 때, \(|a+1|=\)

(4)
\(a<-10\)일 때, \(|a+5|=\)

%
\prob{}
(1)
\(-3<a<2\)일 때
\[|a+3|-|a-2|\]
를 간단히하시오.

(2)
\(a>100\)일 때
\[|a+10|+|a-10|\]
을 간단히하시오.

%
\prob{}
(1)
\(2\le a\le 4\)일 때, \(|a-2|-|a-4|=\)

(2)
\(-4< a\le -2\)일 때, \(-|a+2|+2|a+4|=\)

(3)
\(a<-5\)일 때, \(|a+2|+|a-3|=\)

%
\howo{}
(1)
\(3<a<5\)일 때, \(|a-3|+|a-5|=\)

(2)
\(a>100\)일 때, \(|a+15|-2|a-25|=\)

%
\exam{}
(1)
\(|a|\)는 수직선 상에서 `원점(0)에서부터 \(a\)를 나타내는 점까지의 거리'와 같음을 알 수 있다.
예를 들어 \(|2|\)은 \(0\)에서 \(1\)까지의 거리인 \(2\)이고 \(|-3|\)은 \(0\)에서 \(-3\)까지의 거리인 \(3\)이다.
또 \(|0|\)은 \(0\)에서 \(0\)까지의 거리인 \(0\)이다.
그렇기 때문에 부등식
\[|x|<1\]
은 원점에서부터의 거리가 \(1\)보다 작은 \(x\)의 범위를 나타낸다고 볼 수 있다.
따라서
\[-1<x<1\]
이다.

(2) 마찬가지로 부등식
\[|x|>1\]
은 원점으로부터의 거리가 \(1\)보다 큰 \(x\)의 범위를 나타낸다고 볼 수 있다.
따라서
\[x<-1\text{ 또는 }x>1\]
이다.

%
\theo{}
일반적으로 양의 실수 \(a\)(\(a>0\))에 대해
\[|x|<a\]
의 해는
\[-a<x<a\]
이다.
또,
\[|x|>a\]
의 해는
\[x<-a\text{ 또는 }x>a\]
이다.

%또
%\(<\)이 \(\le\)로, \(>\)이 \(\ge\)로 바뀌어도 비슷한 결과가 성립한다.

%
\prob{}
\begin{gather}
\setcounter{equation}{0}
|x|<2\\
|x|\le\frac15\\
|x|>1.2\\
|x|\ge 5
\end{gather}

%
\howo{}
\begin{gather}
\setcounter{equation}{0}
|x|>10\\
|x|<3.5\\
|x|\le3\\
|x|\ge 2
\end{gather}

%
\prob{}
\begin{gather}
\setcounter{equation}{0}
|-x|>3\\
|2x|<4\\
|-3x|\le-6\\
\left|\frac12x\right|\ge3\\
|x|<|2x|+3
\end{gather}

%
\prob{}
\begin{gather}
\setcounter{equation}{0}
|x+3x|<8\\
2|x|\ge6\\
-3|-x|\le-9\\
\left|\frac13x\right|\ge1
\end{gather}

%
\howo{}
\begin{gather}
\setcounter{equation}{0}
|-x|<2\\
|3x|>9\\
|-2x|\ge-8\\
\left|\frac23x\right|\le4\\
|4x|>|2x|+2
\end{gather}

%
\prob{}
\begin{gather}
\setcounter{equation}{0}
|x-3|\le5\\
|x-\frac12|<\frac12\\
|2x-4|\ge6\\
|x+5|>1
\end{gather}

%
\prob{}
\begin{gather}
\setcounter{equation}{0}
|x-1|\le\sqrt2\\
|x-\frac32|<\frac12\\
|4-2x|\ge3\\
|1-\frac x2|>1
\end{gather}

%
\howo{}
\begin{gather}
\setcounter{equation}{0}
|x-1|<3\\
|x-\sqrt2|<\sqrt2\\
|3x-9|\ge15\\
|x+2|\le4
\end{gather}

%%
\subsection{이차부등식}

%
\defi{}
이항했을 때
\begin{gather*}
ax^2+bx+c>0
ax^2+bx+c<0
ax^2+bx+c\ge0
ax^2+bx+c\le0
\end{gather*}
꼴로 표시되는 부등식을 \textbf{이차부등식}이라고 한다.

%
\exam{}
(1) \(x^2+2x+3>0\)은 이차부등식이다.
(2) \(-3x+5<0\)은 이차부등식이 아니다.
(3) \((x-3)^2\ge2(x-1)^2\)은 이차부등식이다.
(4) \(\frac12x+3<x-7\)은 이차부등식이 아니다.

%
\prob{}
다음 중 이차부등식을 고르시오.
\begin{gather}
2x+4>0\\
-2x^2+5<0\\
(x-4)^2<7\\
(x+3)^2+(x-1)^2<2(x-4)^2
\end{gather}

%
\howo{}
다음 중 이차부등식을 고르시오.
\begin{gather}
x^2<0\\
\frac1x>0\\
x^3+7x<7\\
2x^2+(x-1)^2<3(x-2)^2
\end{gather}

이차부등식을 푸는 방법은 인수분해를 사용한 방법과 그래프를 사용한 방법이 있으나 여기서는 인수분해를 사용한 방법을 소개하겠다.

두 실수 \(A\), \(B\)에 대해 \(AB<0\)이기 위해서는 \(A\), \(B\)의 부호가 서로 달라야 한다.
즉
\[
\begin{cases}
A>0\\
B<0
\end{cases}
\text{ 이거나 }
\begin{cases}
A<0\\
B>0
\end{cases}
\]
이어야 한다.

한편 \(AB>0\)이기 위해서는 \(A\), \(B\)의 부호가 서로 같아야 한다.
즉
\[
\begin{cases}
A>0\\
B>0
\end{cases}
\text{ 이거나 }
\begin{cases}
A<0\\
B<0
\end{cases}
\]
이어야 한다.

이를 토대로 이차부등식을 풀 수 있다.

%
\exam{}
이차부등식
\[x^2-4x+3<0\]
을 풀어보자.
인수분해하면
\[(x-1)(x-3)<0\]
이 된다.
따라서 \(x-1\)과 \(x-3\)의 부호는 서로 다르다.
즉
\[
\begin{cases}
x-1>0\\
x-3<0
\end{cases}
\text{ 이거나 }
\begin{cases}
x-1<0\\
x-3>0
\end{cases}
\]
이어야 한다.
조금 더 정리하면
\[
\begin{cases}
x>1\\
x<3
\end{cases}
\text{ 이거나 }
\begin{cases}
x<1\\
x>3
\end{cases}
\]
이어야 한다.
앞의 것은 정리하면\(1<x<3\)이 되고, 뒤의 것은 발생할 수 없다.
따라서 답은
\[1<x<3\]
이 된다.

\exam{}
이차부등식
\[x^2-4x+3>0\]
을 풀어보자.
인수분해하면
\[(x-1)(x-3)>0\]
이 된다.
따라서 \(x-1\)과 \(x-3\)의 부호는 서로 같다.
즉
\[
\begin{cases}
x-1>0\\
x-3>0
\end{cases}
\text{ 이거나 }
\begin{cases}
x-1<0\\
x-3<0
\end{cases}
\]
이어야 한다.
조금 더 정리하면
\[
\begin{cases}
x>1\\
x>3
\end{cases}
\text{ 이거나 }
\begin{cases}
x<1\\
x<3
\end{cases}
\]
이어야 한다.
더 정리하면
\[
x>3
\text{ 이거나 }
x<1
\]
이 된다.

같은 방법으로
\begin{gather*}
x^2-4x+3\ge0\\
x^2-4x+3\le0
\end{gather*}
도 풀 수 있다.

%
\theo{}
두 실수 \(\alpha\), \(\beta\)에 대해 \(\alpha<\beta\)가 성립할 때,\\
(1) \((x-\alpha)(x-\beta)<0\)의 해는 \(\alpha<x<\beta\)이다.\\
(2) \((x-\alpha)(x-\beta)\le0\)의 해는 \(\alpha\le x\le\beta\)이다.\\
(3) \((x-\alpha)(x-\beta)>0\)의 해는 \(x<\alpha\) 혹은 \(x>\beta\)이다.\\
(4) \((x-\alpha)(x-\beta)\ge0\)의 해는 \(x\le\alpha\) 혹은 \(x\ge\beta\)이다.

%
\prob{}
다음 이차 부등식을 풀어라.
\setcounter{equation}{0}
\begin{gather}
x^2+5x+6>0\\
x^2+3x+4\ge0\\
x^2-2x-3<0\\
-2x^2-6x+8\le0
\end{gather}

%
\prob{}
다음 이차 부등식을 풀어라.
\setcounter{equation}{0}
\begin{gather}
-x^2+x+2>0\\
2x^2+x-1\ge0\\
3x^2-x-4<0\\
x^2-6x+8\le0\\
x(x-2)<0
\end{gather}

%
\howo{}
다음 이차 부등식을 풀어라.
\setcounter{equation}{0}
\begin{gather}
-x^2+2x>0\\
x^2+3x-4\ge0\\
2x^2-2x-4<0\\
x^2-2x-8\le0
\end{gather}

%
\prob{}
다음 이차 연립 부등식을 풀어라.
\setcounter{equation}{0}
\begin{gather}
\begin{cases}
x^2+5x+6>0\\
x^2+3x+4\ge0
\end{cases}\\
\begin{cases}
x^2-2x-3<0\\
-2x^2-6x+8\le0
\end{cases}
\end{gather}

%
\howo{}
다음 이차 연립 부등식을 풀어라.
\setcounter{equation}{0}
\begin{gather}
\begin{cases}
x^2-3x-4<0\\
x^2+2x-8>0
\end{cases}\\
\begin{cases}
x^2-4x-5\le0\\
x^2+x-6\ge0
\end{cases}
\end{gather}
\end{document}