\documentclass{oblivoir}
%%%Default packages
\usepackage{amsmath,amssymb,amsthm,kotex,tabu,graphicx,pifont}
\usepackage{../kswrapfig}

\usepackage{gensymb} %\degree

%%%More packages
%\usepackage{caption,subcaption}
%\usepackage[perpage]{footmisc}
%
\usepackage[skipabove=10pt,innertopmargin=10pt,nobreak=true]{mdframed}

\usepackage[inline]{enumitem}
\setlist[enumerate,1]{label=(\arabic*)}
\setlist[enumerate,2]{label=(\alph*)}

\usepackage{multicol}
\setlength{\columnsep}{30pt}
\setlength{\columnseprule}{1pt}
%
%\usepackage{forest}
%\usetikzlibrary{shapes.geometric,arrows.meta,calc}
%
%%%defi theo exam prob rema proo
%이 환경들 아래에 문단을 쓸 경우 살짝 들여쓰기가 되므로 \hspace{-.7em}가 필요할 수 있다.

\newcounter{num}
\newcommand{\defi}[1]
{\noindent\refstepcounter{num}\textbf{정의 \arabic{num})} #1\par\noindent}
\newcommand{\theo}[1]
{\noindent\refstepcounter{num}\textbf{정리 \arabic{num})} #1\par\noindent}
\newcommand{\revi}[1]
{\noindent\refstepcounter{num}\textbf{복습 \arabic{num})} #1\par\noindent}
\newcommand{\exam}[1]
{\bigskip\bigskip\noindent\refstepcounter{num}\textbf{예시 \arabic{num})} #1\par\noindent}
\newcommand{\prob}[1]
{\bigskip\bigskip\noindent\refstepcounter{num}\textbf{문제 \arabic{num})} #1\par\noindent}
\newcommand{\rema}[1]
{\bigskip\bigskip\noindent\refstepcounter{num}\textbf{참고 \arabic{num})} #1\par\noindent}
\newcommand{\proo}
{\bigskip\noindent\textsf{증명)}}

\newenvironment{talign}
 {\let\displaystyle\textstyle\align}
 {\endalign}
\newenvironment{talign*}
 {\let\displaystyle\textstyle\csname align*\endcsname}
 {\endalign}
%
%%%Commands

\newcommand{\procedure}[1]{\begin{mdframed}\vspace{#1\textheight}\end{mdframed}}

\newcommand\an[1]{\par\bigskip\noindent\textbf{문제 \ref{#1})}\par\noindent}

\newcommand\ann[2]{\par\bigskip\noindent\textbf{문제 \ref{#1})}\:\:#2\par\medskip\noindent}

\newcommand\ans[1]{\begin{flushright}\textbf{답 : }#1\end{flushright}}

\newcommand\anssec[1]{\bigskip\bigskip\noindent{\large\bfseries#1}}

\newcommand{\pb}[1]%\Phantom + fBox
{\fbox{\phantom{\ensuremath{#1}}}}

\newcommand\ba{\,|\,}

\newcommand\ovv[1]{\ensuremath{\overline{#1}}}
\newcommand\ov[2]{\ensuremath{\overline{#1#2}}}
%
%%%% Settings
%\let\oldsection\section
%
%\renewcommand\section{\clearpage\oldsection}
%
%\let\emph\textsf
%
%\renewcommand{\arraystretch}{1.5}
%
%%%% Footnotes
%\makeatletter
%\def\@fnsymbol#1{\ensuremath{\ifcase#1\or
%*\or **\or ***\or
%\star\or\star\star\or\star\star\star\or
%\dagger\or\dagger\dagger\or\dagger\dagger\dagger
%\else\@ctrerr\fi}}
%
%\renewcommand{\thefootnote}{\fnsymbol{footnote}}
%\makeatother
%
%\makeatletter
%\AtBeginEnvironment{mdframed}{%
%\def\@fnsymbol#1{\ensuremath{\ifcase#1\or
%*\or **\or ***\or
%\star\or\star\star\or\star\star\star\or
%\dagger\or\dagger\dagger\or\dagger\dagger\dagger
%\else\@ctrerr\fi}}%
%}   
%\renewcommand\thempfootnote{\fnsymbol{mpfootnote}}
%\makeatother
%
%%% 객관식 선지
\newcommand\one{\ding{172}}
\newcommand\two{\ding{173}}
\newcommand\three{\ding{174}}
\newcommand\four{\ding{175}}
\newcommand\five{\ding{176}}
\usepackage{tabto,pifont}
%\TabPositions{0.2\textwidth,0.4\textwidth,0.6\textwidth,0.8\textwidth}

\newcommand\taba[5]{\par\noindent
\one\:{#1}
\tabto{0.2\textwidth}\two\:\:{#2}
\tabto{0.4\textwidth}\three\:\:{#3}
\tabto{0.6\textwidth}\four\:\:{#4}
\tabto{0.8\textwidth}\five\:\:{#5}}

\newcommand\tabb[5]{\par\noindent
\one\:{#1}
\tabto{0.33\textwidth}\two\:\:{#2}
\tabto{0.67\textwidth}\three\:\:{#3}\medskip\par\noindent
\four\:\:{#4}
\tabto{0.33\textwidth}\five\:\:{#5}}

\newcommand\tabc[5]{\par\noindent
\one\:{#1}
\tabto{0.5\textwidth}\two\:\:{#2}\medskip\par\noindent
\three\:\:{#3}
\tabto{0.5\textwidth}\four\:\:{#4}\medskip\par\noindent
\five\:\:{#5}}

\newcommand\tabd[5]{\par\noindent
\one\:{#1}\medskip\par\noindent
\two\:\:{#2}\medskip\par\noindent
\three\:\:{#3}\medskip\par\noindent
\four\:\:{#4}\medskip\par\noindent
\five\:\:{#5}}
%
%%%% fonts
%
%\usepackage{fontspec, xunicode, xltxtra}
%\setmainfont[]{은 바탕}
%\setsansfont[]{은 돋움}
%\setmonofont[]{은 바탕}
%\XeTeXlinebreaklocale "ko"
%%%%
\begin{document}

\title{수학(상) : 01 다항식의 연산}
\author{}
\date{\today}
\maketitle
\tableofcontents
\newpage

%%
\section{다항식}

%
\exam{\(x^2-3x+4\)}
\begin{enumerate}[(1)]
\item
\(x\)에 대한 \emph{다항식} \(x^2-3x+4\)은 세 개의 \emph항 \(x^2\), \(-3x\), \(4\)로 이루어져 있다.
\item
\(x^2\)의 \emph{차수}는 \(2\)이고 \emph{이차항}이라고 부른다.
이차항의 \emph{계수}는 \(1\)이다.
\item
\(-3x\)의 차수는 \(1\)이고 \emph{일차항}이라고 부른다.
일차항의 계수는 \(-3\)이다.
\item
\(4\)의 차수는 \(0\)이고 \emph{상수항}이라고 부른다.
\item
\emph{최고차항}의 차수가 \(2\)이므로 이 다항식 \(x^2+3x+4\)는 \emph{이차식}이다.
\item
이 다항식은 \emph{내림차순}으로 정리되어 있다.
이것을 \emph{오름차순}으로 정리하면 \(4+3x+x^2\)이다.
\end{enumerate}

%
\exam{\(x^3+2x^2y-x-2y\)}
\begin{enumerate}[(1)]
\item
\(x\), \(y\)에 대한 다항식 \(x^3+2x^2y-x-2y\)은 네 개의 항으로 이루어져 있다.
\item
\(x^3\)의 차수는 \(x\)에 대하여 3차이고, 계수는 1이다.
\item
\(2x^2y\)의 차수는 \(x\)에 대하여 2차, \(y\)에 대하여 1차이며 계수는 2이다.
\item
이 다항식을 \(x\)에 대해 내림차순으로 정리하면 \(x^3+2yx^2-x-2y\)이고, \(y\)에 대해 내림차순으로 정리하면 \((2x^2-2)y+(x^3-x)\)이다.
\item
이 다항식은 \(x\)에 대하여 삼차식, \(y\)에 대하여 일차식이다.
\end{enumerate}

\newpage
%
\prob{다항식 \(x^3-6x+4\)에 대한 다음 설명 중 틀린 것을 고르시오.}
\vspace{-20pt}
\tabd
{\text{세 개의 항으로 이루어져 있다.}}
{\text{일차항의 계수는 \(6\)이다.}}
{\text{상수항은 \(4\)이다.}}
{\text{3차 다항식이다.}}
{\text{오름차순으로 정리하면 \(4-6x+x^3\)이다.}}

%
\prob{다음 다항식에 대한 설명 중 틀린 것을 고르시오.}
\vspace{-20pt}
\tabd
{\(\sqrt{x+1}\), \(\frac1{x^2}+3\)은 다항식이 아니다.}
{\(x^2+y^2+z^2-xy-yz-zx\)는 \(x\), \(y\), \(z\)에 대한 다항식이다.}
{\(x^2+3xy+2y^2\)은 \(x\), \(y\)에 대한 이차식이다.}
{\(x^4-4x^2y^2+3y^4\)에서 \(x^2y^2\)의 계수는 \(-4\)이다.}
{\(2x^2-3xy+3y^2-2x+4y-3\)을 \(x\)에 대한 내림차순으로 정리하면\\ \(2x^2-(3y-2)x+3y^2-4y-3\)이다.}

%%
\section{다항식의 덧셈, 뺄셈}

다항식 \(A\), \(B\), \(C\)에 대하여 다음 법칙들이 성립한다.
\begin{mdframed}
%
\defi{}
\begin{enumerate}
\item
\emph{교환법칙} : \(A+B=B+A\)			\tabto{0.6\textwidth}\(AB=BA\)
\item
\emph{결합법칙} : \((A+B)+C=A+(B+C)\)	\tabto{0.6\textwidth}\((AB)C=A(BC)\)
\item
\emph{분배법칙} : \(A(B+C)=AB+AC\)		
\end{enumerate}
\end{mdframed}

%
\exam{}
\vspace{-25pt}
\begin{align*}
(1)&\:\:
(3x+2y)+(4x-3y)
=(3x+4x)+(2y-3y)
=(3+4)x+(2-3)y=7x-y\\
(2)&\:\:
3(x^2-x+1)+2(-x^2+2x-3)
=(3x^2-3x+3)+(-2x^2+4x-6)\\
&=(3x^2-2x^2)+(-3x+4x)+(3-6)
=(3-2)x^2+(-3+4)x+(-3)\\
&=x^2+x-3.
\end{align*}

%
\prob{}
\(A=3x^2+3xy-5y^2\), \(B=x^2-xy-3y^2\)일 때, 다음 물음에 답하여라.
\begin{enumerate}[(1)]
\item
\(2A-B\)를 계산하여라.
\item
\(A-2X=B\)를 만족시키는 다항식 \(X\)를 구하여라.
\end{enumerate}

%%
\section{다항식의 곱셈}

%
\begin{mdframed}
\theo{곱셈공식(1)}
\begin{enumerate}[(1)]
\item
\((a+b)^2=a^2+2ab+b^2\)
\item
\((a-b)^2=a^2-2ab+b^2\)
\item
\((a+b)(a-b)=a^2-b^2\)
\item
\((x+a)(x+b)=x^2+(a+b)x+ab\)
\item
\((ax+b)(bx+d)=acx^2+(ad+bc)x+bd\)
\end{enumerate}
\end{mdframed}

%
\prob{다음 식을 전개하여라.}
\begin{enumerate}[(1)]
\item
\((2x+1)^2=\)
\item
\((3x-1)^2=\)
\item
\((x+3)(x+4)=\)
\item
\((x+2)(3x+2)=\)
\end{enumerate}

%
\prob{}
\begin{enumerate}[(1)]
\item
\(x+y=5\), \(xy=6\)일 때 \(x^2+y^2\)의 값을 구하여라.
\item
\(x-y=2\), \(x^2+y^2=34\)일 때, \(xy\)의 값을 구하여라.
\end{enumerate}

%
\prob{\(x+\frac1x=4\)일 때, 다음 식의 값을 구하여라(단, \(x>1\)).}
\begin{enumerate}[(1)]
\item
\(x^2+\frac1{x^2}=\)
\item
\(x-\frac1x=\)
\end{enumerate}

\newpage
%
\begin{mdframed}
\theo{곱셈공식(2)}
\begin{enumerate}[(1)]
\setcounter{enumi}{5}
\item
\((a+b)^3=a^3+3a^2b+3ab^2+b^3\)
\item
\((a-b)^3=a^3-3a^2b+3ab^2-b^3\)
\item
\((a+b)(a^2-ab+b^2)=a^3+b^3\)
\item
\((a-b)(a^2+ab+b^2)=a^3-b^3\)
\item
\((a+b+c)^2=a^2+b^2+c^2+2ab+2bc+2ca\)
\item
\((x+a)(x+b)(x+c)=x^3+(a+b+c)x^2+(ab+bc+ca)x+abc\)
\end{enumerate}
\end{mdframed}

%
\exam{(6), (8), (10)의 식을 유도해보면}
\begin{enumerate}[(1)]
\item[(6)]
\((a+b)^3=(a+b)(a+b)^2=(a+b)(a^2+2ab+b^2)\\
=a(a^2+2ab+b^2)+b(a^2+2ab+b^2) =(a^3+2a^2b+2ab^2)+(a^2b+2ab^2+b^3)\\
=a^3+3a^2b+3ab^2+b^3\)
\item[(8)]
\((a+b)(a^2-ab+b^2)=a(a^2-ab+b^2)+b(a^2-ab+b^2)\\
=(a^3-a^2b+ab^2)+(a^2b-ab^2+b^3)=a^3+b^3\)
\item[(10)]
\((a+b+c)^2=\{(a+b)+c\}^2=(a+b)^2+2(a+b)c+c^2\\
=(a^2+2ab+b^2)+(2ac+2bc)+c^2=a^2+b^2+c^2+2ab+2bc+2ca\)
\end{enumerate}

%
\prob{(7), (9), (11)의 식을 유도하여라.}
\procedure{0.2}

%
\prob{다음 식을 전개하여라.}
\begin{enumerate}[(1)]
\item
\((x+2)^3=\)
\item
\((x-1)^3=\)
\item
\((x+2y)(x^2-2xy+4y^2)=\)
\item
\((3a-1)(9a^2+3a+1)=\)
\item
\((a+2b-c)^2=\)
\item
\((x+2)(x-4)(x+5)=\)
\end{enumerate}

%
\prob{}
\(a+b=3\), \(ab=-2\)일 때, \(a^3+b^3\)의 값을 구하여라.

%
\prob{}
\(a-b=1\), \(ab=4\)일 때, \(a^3-b^3\)의 값을 구하여라.

%
\prob{}
\(a+b+c=9\), \(ab+bc+ca=8\)일 때, \(a^2+b^2+c^2\)의 값을 구하여라.

\newpage
%
\begin{mdframed}
\theo{곱셈공식(3)}
\begin{enumerate}[(1)]
\setcounter{enumi}{11}
\item
\((a+b+c)(a^2+b^2+c^2-ab-bc-ca)=a^3+b^3+c^3-3abc\)
\item
\((a^2+ab+b^2)(a^2-ab+b^2)=a^4+a^2b^2+b^4\)
\end{enumerate}
\end{mdframed}

%
\exam{(13)의 식을 유도해보면}
\begin{enumerate}[(1)]
\setcounter{enumi}{9}
\item
\((a^2+ab+b^2)(a^2-ab+b^2)=\{(a^2+b^2)+ab\}\{(a^2+b^2)-ab\}\\
=(a^2+b^2)^2-(ab)^2=(a^4+2a^2b^2+b^4)-a^2b^2=a^4+a^2b^2+b^4\)
\end{enumerate}

%
\prob{(12)의 식을 유도하여라.}
\procedure{0.2}

%
\prob{다음 식을 전개하여라.}
\begin{enumerate}[(1)]
\item
\((2a+b-c)(4a^2+b^2+c^2-2ab+bc+2ca)=\)
\item
\((x-y-1)(x^2+y^2+xy+x-y+1)=\)
\item
\((4x^2+2xy+y^2)(4x^2-2xy+y^2)=\)
\end{enumerate}

%
\prob{}
\(a+b+c=3\), \(a^2+b^2+c^2=9\), \(abc=-4\)일 때, \(a^3+b^3+c^3\)의 값을 구하여라.

%%
\section{다항식의 나눗셈}

%%
\subsection{정수의 나눗셈}

\exam{}
\(32\)을 \(5\)로 나누면 몫은 \(6\)이고 나머지는 \(2\)이다.
\begin{table}[h!]
\centering
\begin{tabular}{cc@{}c}
&&6\\
\cline{2-3}
5	&\multicolumn{1}{|c}{3}	&2\\
	&3							&0\\
\hline
	&							&2
\end{tabular}
\end{table}

이것을 \[32=5\times6+2\]로 표현할 수 있다.
하지만
\[32=5\times5+7\]
이라고 해서 몫이 \(5\)이고 나머지가 \(7\)이라고 말하지는 않는다.
또한,
\[32=5\times7+(-3)\]
라고 해서 몫이 \(7\)이고 나머지가 \(-3\)이라고 말하지는 않는다.
%\[28=5\times q+r\]
%으로 표현되었을 때, \(r\)의 범위가 \(0\le r<5\)인 경우에만 \(r\)을 나머지라고 부른다.

\begin{mdframed}
%
\defi{}
\(a\)가 정수이고, \(b\)가 자연수일 때,
\[a=bq+r\qquad(0\le r<b)\]
가 성립하면, \(a\)를 \(b\)로 나누었을 떄의 \emph몫은 \(q\), \emph{나머지}는 \(r\)이다.
\end{mdframed}

\clearpage
%
\exam{}
\begin{enumerate}[(1)]
\item
\(32\)\를 \(4\)\으로 나누면(\(28=4\times8+0\)) 몫이 \(8\)이고 나머지가 \(0\)이다.
이때 \(32\)\은 \(4\)\으로 \emph{나누어떨어진다}고 한다.
또한 \(4\)\는 \(32\)의 약수, \(32\)\은 \(4\)의 배수이다.
\item
\(-32\)\은 \(5\)\로 나누면(\(-32=5\times(-7)+3\)) 몫이 \(-7\)이고 나머지가 \(3\)이다.
\item
\(2\)로 나누어떨어지는 정수를 \emph{짝수}, \(2\)로 나누었을 때 나머지가 \(1\)인 정수를 \emph{홀수}라고 한다.
따라서 \(5\)는 홀수, \(0\)은 짝수, \(-4\)는 짝수이다.
\end{enumerate}

%
\exam{주어진 \(a\), \(b\)에 대하여, \(a\)를 \(b\)로 나누었을 때의 몫 \(q\)와 나머지 \(r\)을 각각 구하여라.}
\begin{enumerate}[(1)]
\item
\(a=50,\quad b=4\)
\item
\(a=50,\quad b=5\)
\item
\(a=50,\quad b=12\)
\item
\(a=0,\quad b=4\)
\item
\(a=-14,\quad b=5\)
\item
\(a=-14,\quad b=2\)
\end{enumerate}

\clearpage
%%
\subsection{다항식의 나눗셈}

정수와 마찬가지로 다항식도 나눌 수 있다.

%
\exam{}
다항식 \(x^3+2x^2+5x-3\)을 \(x^2-2x-1\)으로 나누면
\begin{equation*}
\begin{array}{c@{\:}c@{\:}cc@{\:}c@{\:}c@{\:}c}
&&&x&+4\\
\cline{4-7}
x^2&-2x&-1	&\multicolumn{1}{|c}{x^3}&+2x^2&+5x&-3\\
&&							&x^3	&-2x^2	&-x\\
\cline{4-7}
&&							&			&4x^2	&+6x		&-3\\
&&							&			&4x^2	&-8x	&-4\\
\cline{4-7}
&&							&			&			&14x	&+1
\end{array}
\end{equation*}
몫이 \(x+4\)이고 나머지가 \(14x+1\)이다.

이것을
\[x^3+2x^2+5x-3=(x^2-2x-1)(x+4)+14x+1\]
로 표현한다.
한편
\[x^3+2x^2+5x-3=(x^2-2x-1)(x+3)+x^2+12x\]
도 성립한다.
하지만 몫이 \(x+3\)이고 나머지가 \(x^2+12x\)이라고 말하지는 않는다.

\begin{mdframed}
%
\defi{}
\(A\), \(B\)가 다항식일 때,
\[A=BQ+R\qquad(R\text{의 차수}<B\text{의 차수})\]
가 성립하면, \(A\)를 \(B\)로 나누었을 떄의 \emph몫은 \(Q\), \emph{나머지}는 \(R\)이다.
\end{mdframed}

\clearpage
%
\exam{}
\begin{enumerate}[(1)]
\item
\(x^3+11=(x+2)(x^2-2x+4)+3\)이므로 \(x^3+11\)을 \(x+2\)로 나누었을 때의 몫은 \(x^2-2x+4\), 나머지는 \(3\)이다.
\item
\(x^4+4x^2+16=(x^2+2x+4)(x^2-2x+4)\)이므로 \(x^4+4x^2+16\)를 \(x^2+2x+4\)로 나누었을 때의 몫은 \(x^2-2x+4\)이고 나머지는 \(0\)이다.
이때, \(x^4+4x^2+16\)는 \(x^2+2x+4\)로 \emph{나누어떨어진다}고 말한다.
\end{enumerate}

%
\prob{다음 나눗셈의 몫과 나머지를 구하여라.}
\begin{enumerate}[(1)]\label{poly_div_prob}
\item
\((3x^3-2x^2-5x+1)\div(x-2)\)
\item
\((2x^3-5x^2+4x-1)\div(x^2-2x-2)\)
\end{enumerate}

%
\prob{}
다항식 \(A\)는 \(x^3+1\)으로 나누어떨어지고 이 때의 몫이 \(x+1\)이다.
다항식 \(A\)를 구하여라.

\clearpage
%%
\subsection{조립제법}
나누는 수 \(B\)가 일차식일 때, 다음의 \emph{조립제법}을 사용할 수 있다.

%
\exam{}
\(3x^3-2x^2-5x+1\)를 \(x-2\)로 나누는 과정은
\begin{equation*}
\begin{array}{ccccc}
\multicolumn{1}{c|}{2}	&3	&-2	&-5	&1\\
\multicolumn{1}{c|}{}		&	&6	&8	&6\\
\cline{2-5}
							&3	&4	&3	&\multicolumn1{|c}{7}\\
\cline{5-5}
\end{array}
\end{equation*}
로 간단히 표시할 수 있다.
이때, 몫은 \(3x^2+4x+3\)이고, 나머지는 \(7\)이다.
이것은 문제 \ref{poly_div_prob})의 (1)에서 구한 답과 일치한다.

%
\prob{조립제법을 사용하여 다음 나눗셈의 몫과 나머지를 구하여라.}
\begin{enumerate}[(1)]
\item
\((x^3-2x^2-5x+3)\div(x+2)\)
\item
\((2x^3+3x^2-6x+1)\div(x-\frac12)\)
\end{enumerate}

%
\prob{}
위의 결과를 이용하여 \(2x^3+3x^2-6x+1\)를 \(2x-1\)로 나눈 몫과 나머지를 구하여라.

%%
\section*{답}

\begin{minipage}{0.49\textwidth}
%
\an{7}
(1) \(5x^2+7xy-7y^2\)\\
(2) \(x^2+2xy-y^2\)

%
\an{9}
(1) \(4x^2+4x+1\)\\
(2) \(9x^2-6x+1\)\\
(3) \(x^2+7x+12\)\\
(4) \(3x^2+8x+4\)

%
\an{10}
(1) \(13\)\\
(2) \(15\)

%
\an{11}
(1) \(14\)\\
(2) \(2\sqrt3\)

%
\an{15}
(1) \(x^3+6x^2+12x+8\)\\
(2) \(x^3-3x^2+3x-1\)\\
(3) \(x^3+8y^3\)\\
(4) \(27a^3-1\)\\
(5) \(a^2+4b^2+c^2+4ab-4bc-2ca\)\\
(6) \(x^3+3x^2-18x-40\)

%
\an{16}
\(45\)

%
\an{17}
\(13\)

\end{minipage}
\begin{minipage}{0.49\textwidth}

%
\an{18}
\(65\)

%
\an{22}
(1) \(8a^3+b^3+b^3+6abc\)\\
(2) \(x^3-y^3-1+xy+x-y\)\\
(3) \(16x^4+4x^2y^2+y^4\)

%
\an{23}
\(15\)

%
\an{27}
(1) \(q=12,\quad r=2\)\\
(2) \(q=10,\quad r=0\)\\
(3) \(q=4,\quad r=2\)\\
(4) \(q=0,\quad r=0\)\\
(5) \(q=-3,\quad r=1\)\\
(6) \(q=-7,\quad r=0\)

%
\an{31}
(1) 몫 : \(3x^2+4x+3\), 나머지 : \(7\)\\
(2) 몫 : \(2x-1\), 나머지 : \(6x-3\)

%
\an{32}
\(A=x^4+x^3+x+1\)

%
\an{34}
(1) 몫 : \(x^2-4x+3\), 나머지 : \(-3\)\\
(2) 몫 : \(2x^2+4x-4\), 나머지 : \(-1\)

%
\an{35}
몫 : \(x^2+2x-2\), 나머지 : \(-1\)

\end{minipage}
\end{document}