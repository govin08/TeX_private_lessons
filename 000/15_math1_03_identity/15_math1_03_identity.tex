\documentclass{oblivoir}
\usepackage{amsmath,amssymb,kotex,paralist,graphicx}
\usepackage{mdframed}
\usepackage{../kswrapfig}
\usepackage{fapapersize}
\usefapapersize{210mm,297mm,20mm,*,20mm,*}
%\pagestyle{empty}
\usepackage{multicol}
\setlength{\columnsep}{30pt}
\setlength{\columnseprule}{1pt}
%\def\columnseprulecolor{\color{blue}}

%%% 객관식 선지

\usepackage{tabto,pifont}
\TabPositions{0.2\textwidth,0.4\textwidth,0.6\textwidth,0.8\textwidth}

\newcommand\one{\ding{172}}
\newcommand\two{\ding{173}}
\newcommand\three{\ding{174}}
\newcommand\four{\ding{175}}
\newcommand\five{\ding{176}}

\newcommand\taba[5]{\par\bigskip\noindent
\one\:{\ensuremath{#1}}
\tab\two\:\:{\ensuremath{#2}}
\tab\three\:\:{\ensuremath{#3}}
\tab\four\:\:{\ensuremath{#4}}
\tab\five\:\:{\ensuremath{#5}}}

\newcommand\tabb[5]{\par\bigskip\noindent
\one\:{\ensuremath{#1}}
\tabto{0.16\textwidth}\two\:\:{\ensuremath{#2}}
\tabto{0.33\textwidth}\three\:\:{\ensuremath{#3}}\medskip\par\noindent
\four\:\:{\ensuremath{#4}}.
\tabto{0.16\textwidth}\five\:\:{\ensuremath{#5}}}

\newcommand\tabc[5]{\par\bigskip\noindent
\one\:{\ensuremath{#1}}
\tabto{0.25\textwidth}\two\:\:{\ensuremath{#2}}\medskip\par\noindent
\three\:\:{\ensuremath{#3}}
\tabto{0.25\textwidth}\four\:\:{\ensuremath{#4}}\medskip\par\noindent
\five\:\:{\ensuremath{#5}}}

\newcommand\tabd[5]{\par\bigskip\noindent
\one\:{#1}\medskip\par\noindent
\two\:\:{#2}\medskip\par\noindent
\three\:\:{#3}\medskip\par\noindent
\four\:\:{#4}\medskip\par\noindent
\five\:\:{#5}}

%%% Counters
\newcounter{num}

%%% Commands
\newcommand{\prob}[1]
{\vs\par\noindent\refstepcounter{num}\textbf{문제 \arabic{num})}\label{#1}\par\noindent}

\newcommand\vs[1]{\vspace{70pt}}

\newcommand\inc[1]{\begin{center}\includegraphics[width=0.95\columnwidth]{#1}\end{center}}

\newcommand\pb[1]{\ensuremath{\fbox{\phantom{#1}}}}

\newcommand\ba{\ensuremath{\:|\:}}

\newcommand\an[2]{\par\bigskip\noindent\textbf{문제 \ref{#1})} #2\\}

\newcommand\ans[1]{\begin{flushright}\textbf{답 : }#1\end{flushright}}

\renewcommand{\arraystretch}{1.5}

%%% Meta Commands
\let\oldsection\section
\renewcommand\section{\clearpage\oldsection}
\let\emph\textsf

%%%%
\begin{document}

\title{수학(상) : 03 항등식과 나머지정리}
\author{}
\date{\today}
\maketitle
\tableofcontents
\newpage

%%
\section{항등식과 미정계수법}

%%
\subsection{항등식}
\begin{mdframed}
%
\defi{항등식}
주어진 등식의 문자에 어떤 값을 대입해도 항상 성립하는 등식을 \emph{항등식}이라고 한다.
\end{mdframed}

%
\exam{}
\begin{enumerate}[(1)]
\item
등식 \(x^2-2x-3=0\)에 \(x=3\)을 대입하면 \(0=0\)이 되어 성립한다.
하지만 \(x=4\)을 대입하면 \(5\neq0\)이 되어 성립하지 않는다.
따라서 이 등식은 항등식이 아니다.
\item
등식 \(x^2-2x-3=(x-3)(x+1)\)에는 \(x=3\)을 대입하면 \(0=0\)이 되고 \(x=4\)를 대입하면 \(5=5\)가 되어 성립한다.
그밖에 \(x\)에 어떤 값을 대입하더라도 항상 성립한다.
따라서 항등식이 맞다.
실제로 좌변을 잘 정리하면 우변이 되므로, 항등식인 것이 당연하다.
\item
\((x-3y)^2=x^2-6xy+9y^2\)에서 좌변을 전개하면 우변이 된다.
따라서 \(x\)와 \(y\)에 각각 어떤 값을 대입하더라도 이 등식은 항상 성립하며, 항등식이 맞다.
\item
\(x^3+2x^2+5x-3\)를 \(x^2-2x-1\)로 나누어 생기는 몫과 나머지로 만든 
\[x^3+2x^2+5x-3=(x^2-2x-1)(x+4)+14x+1\]
에서도, 우변을 전개하면 좌변이 된다.
따라서 항등식이다.
\end{enumerate}

\clearpage
%
\exam{}
\begin{enumerate}[(1)]
\item
등식 \(ax+b=0\)이 항등식이면 \(a=0\), \(b=0\)임을 설명하여라.
\item
등식 \(ax+b=a'x+b'\)이 항등식이면 \(a=a'\), \(b=b'\)임을 설명하여라.
\end{enumerate}

\begin{mdframed}
\begin{enumerate}[(1)]
\item
등식 \(ax+b=0\)에서 \(x\)에 어떤 값을 대입하더라도 항상 성립해야 한다.\\
%\(ax+b=0\)이 \(x\)에 대한 항등식이므로 \(x\)에 어떤 값을 대입하더라도 항상 성립해야 한다.
\(x=0\)을 대입하면 \(a\cdot0+b=0\), 즉 \(b=0\)이다.\\
\(x=1\)을 대입하면 \(a\cdot1+0=0\), 즉 \(a=0\)이다.
\item
주어진 식의 우변을 좌변으로 이항하면 \[(a-a')x+(b-b')=0\]이 되는데, 이 식이 항등식이 되려면 (1)에 의해 \(a-a'=0\), \(b-b'=0\)이어야 한다.
따라서 \(a=a'\)이고 \(b=b'\)이다.
\end{enumerate}
\end{mdframed}

%
\exam{}
등식 \((m-1)x+(n+2)=3x+7\)이 항등식이 되려면 \(m-1=3\), \(n+2=7\)이어야 한다.
따라서 \(m=4\), \(n=5\)이다.

\clearpage
%
\prob{}
\begin{enumerate}[(1)]
\item
등식 \(ax^2+bx+c=0\)가 항등식이면 \(a=0\), \(b=0\), \(c=0\)임을 설명하여라.
\item
등식 \(ax^2+bx+c=a'x^2+b'x+c'\)이 항등식이면 \(a=a'\), \(b=b'\), \(c=c'\)임을 설명하여라.
\end{enumerate}

\procedure{0.3}

%
\prob{다음 등식이 \(x\)에 대한 항등식일 때, \(a+b+c\)의 값을 구하여라.}
\begin{enumerate}[(1)]
\item
\((a+2)x^2+(b-2)x+3-c=0\)
\item
\(1-2x+ax^2=5x^2+bx+c\)
\end{enumerate}

\clearpage
%
\prob{}
\begin{enumerate}[(1)]
\item
등식 \(ax+by+c=0\)가 항등식이면 \(a=0\), \(b=0\), \(c=0\)임을 설명하여라.
\item
등식 \(ax+by+c=a'x+b'y+c'\)이 항등식이면 \(a=a'\), \(b=b'\), \(c=c'\)임을 설명하여라.
\end{enumerate}

\procedure{0.3}

%%
%\prob{다음 등식이 \(x\), \(y\)에 대한 항등식일 때, \(a+b+c\)의 값을 구하여라.}
%\[(a+b)x+(b+c)y+(c+a)=3x+6y-5\]
\bigskip\bigskip
이상에서 다음의 항등식의 성질을 얻을 수 있다.
\begin{mdframed}
%
\theo{항등식의 성질}
\begin{enumerate}[(a)]
\item
\(ax+b=0\)이 항등식이면 \(a=0\), \(b=0\)이다.
\item
\(ax+b=a'x+b'\)이 항등식이면 \(a=a'\), \(b=b'\)이다.
\item
\(ax^2+bx+c=0\)이 항등식이면 \(a=0\), \(b=0\), \(c=0\)이다.
\item
\(ax^2+bx+c=a'x^2+b'x+c'\)이 항등식이면 \(a=a'\), \(b=b'\), \(c=c'\)이다.
\item
\(ax+by+c=0\)이 항등식이면 \(a=0\), \(b=0\), \(c=0\)이다.
\item
\(ax+by+c=a'x+b'y+c'\)이 항등식이면 \(a=a'\), \(b=b'\), \(c=c'\)이다.
\end{enumerate}
\end{mdframed}

\clearpage
%%
\subsection{미정계수법}

%
\exam{다음 등식이 \(x\)에 대한 항등식일 때, 상수 \(a\), \(b\), \(c\)의 값을 구하여라.}
\[a(x-1)^2+b(x-1)+c=2x^2-3x+4\]
\begin{mdframed}[frametitle=<풀이1>]
좌변을 전개하여 정리하면
\begin{align*}
a(x-1)^2+b(x-1)+c
&=ax^2-2ax+a+bx-b+c\\
&=ax^2+(-2a+b)x+(a-b+c)
\end{align*}
이므로 주어진 항등식은
\[ax^2+(-2a+b)x+(a-b+c)=2x^2-3x+4\]
가 되고, 따라서 \(a=2\), \(-2a+b=-3\), \(a-b+c=4\)이다.
이 식을 연립하여 풀면 \(a=2\), \(b=1\), \(c=3\)이 된다.
\end{mdframed}

\begin{mdframed}[frametitle=<풀이2>]
주어진 식이 항등식이므로 \(x\)에 \(x=1\), \(x=0\), \(x=2\)를 넣어도 성립해야 한다.
\begin{align*}
x=1	\:\:;	&\:\:c=3\\
x=0	\:\:;	&\:\:a-b+c=4\\
x=2\:\:;	&\:\:a+b+c=6
\end{align*}
이 식들을 연립하면 \(a=2\), \(b=1\), \(c=3\)이다.
\end{mdframed}

이처럼 항등식에서 아직 정해지지 않은 계수(=미정계수)의 값을 정하는 방법을 \emph{미정계수법}이라고 한다.
<풀이 1>에서처럼 양변의 계수를 비교하여 미정계수를 구하는 방법을 \emph{계수비교법},
<풀이 2>에서처럼 \(x\)에 여러 값들을 대입해 미정계수를 구하는 방법을 \emph{수치대입법}이라고 한다.

%
\prob{계수비교법을 이용하여 다음 항등식에서 \(a+b+c\)\을 구하여라.}
\[x^3+ax^2-36=(x+c)(x^2+bx-12)\]

%
\prob{수치대입법을 이용하여 다음 항등식에서 \(a^2-b^2\)의 값을 구하여라.}
\[(x-1)^4=x^4-4x^3+ax^2+bx+1\]

%
\prob{다음 등식이 \(x\)에 대한 항등식일 때, 상수 \(a\), \(b\), \(c\)의 값을 구하여라.}
\begin{enumerate}[(1)]
\item
\(2x^2-2=(x+1)(ax+b)\)
\item
\(a(x-1)+b(x-2)=2x-3\)
\end{enumerate}

%%
\section{나머지정리와 인수정리}
\(3x^3-2x^2-5x+1\)를 \(x-2\)로 나누었을 때 몫과 나머지를 구하는 과정에는 다음의 두 방법이 있었다.
\begin{equation*}
\begin{array}{c@{\:}cc@{\:}c@{\:}c@{\:}c}
&&3x^2&+4x&+3\\
\cline{3-6}
x&-2	&\multicolumn{1}{|c}{3x^3}	&-2x^2&-5x&+1\\
&							&3x^3		&-6x^2\\
\cline{3-6}
&							&			&4x^2	&-5x	&+1\\
&							&			&4x^2	&-8x	&\\
\cline{3-6}
&							&			&		&3x		&+1\\
&							&			&		&3x 	&-6\\
\cline{3-6}
&							&			&		&	 	&7
\end{array}
\qquad\qquad
\begin{array}{ccccc}
\multicolumn{1}{c|}{2}	&3	&-2	&-5	&1\\
\multicolumn{1}{c|}{}		&	&6	&8	&6\\
\cline{2-5}
							&3	&4	&3	&\multicolumn1{|c}{7}\\
\cline{5-5}
\end{array}
\end{equation*}

하지만 몫은 제외하고, 나머지만 구하려고 한다면 훨씬 쉽게 구하는 방법이 있다.

\begin{mdframed}
%
\theo{나머지정리}
다항식 \(f(x)\)를 \(x-\alpha\)로 나누었을 때의 나머지는 \(f(\alpha)\)이다.
\end{mdframed}

%
\proo{}
\(f(x)\)를 \(x-\alpha\)로 나누었을 때의 몫을 \(Q(x)\), 나머지를 \(R\)이라고 하면
\[f(x)=(x-\alpha)Q(x)+R\]
이다.
이 식은 항등식이므로 \(x=\alpha\)를 대입해도 성립한다.
따라서 \(f(\alpha)=R\)이다.
\qed

\clearpage
%
\exam{}
\begin{enumerate}[(1)]
\item
위의 예에서 나머지정리를 쓰면, \(f(x)=3x^3-2x^2-5x+1\)을 \(x-2\)로 나누었을 때의 나머지는
\[R=f(2)=3\cdot2^3-2\cdot2^2-5\cdot2+1=7\]
이다.
\item
\(x^3-2x^2+3x+4\)를 \(x+3\)으로 나누었을 때의 나머지는
\((-3)^3-2(-3)^2+3(-3)+4=-50\)
이다.
\end{enumerate}


%
\prob{\(2x^3-2x^2+1\)을 다음 일차식으로 나누었을 때의 나머지를 구하여라.}
\par\noindent
(1) \(x-2\)
\tabto{0.5\textwidth}
(2) \(x+\frac12\)

%
\prob{}
\(x^3-x^2+ax+4\)을 \(x+2\)로 나누었을 때의 나머지가 \(2\)일 때, 상수 \(a\)의 값을 구하여라.

%
\exam{}
다항식 \(f(x)\)를 일차식 \(ax+b\)로 나누었을 때의 나머지가 \(f\left(-\frac ba\right)\)임을 보여라.
\begin{mdframed}
\(f(x)\)를 \(ax+b\)로 나누었을 때의 몫을 \(Q(x)\), 나머지를 \(R\)이라고 하면
\[f(x)=(ax+b)Q(x)+R\]
이다.
이 식은 항등식이므로 \(x=-\frac ba\)를 대입해도 성립한다.
따라서 \(f\left(-\frac ba\right)=R\)이다.
\qed
\end{mdframed}

%
\prob{\(3x^2+x+2\)를 다음 일차식으로 나누었을 때의 나머지를 구하여라.}
\par\noindent
(1) \(2x+1\)
\tabto{0.5\textwidth}
(2) \(3x-2\)

\clearpage
%
\exam{}
다항식 \(f(x)\)를 \(x-1\)로 나누었을 때의 나머지가 \(3\)이고, \(x+2\)로 나누었을 때의 나머지가 \(-3\)이다.
이때, \(f(x)\)를 \((x-1)(x+2)\)로 나누었을 때의 나머지를 구하여라.
\begin{mdframed}
\(f(x)\)를 \((x-1)(x+2)\)로 나누었을 때의 몫을 \(Q(x)\), 나머지를 \(R(x)\)이라고 하자.
나누는 식 \((x-1)(x+2)\)가 이차식이므로, \(R(x)\)는 일차 이하의 다항식 \(ax+b\)이다.
그러면
\[f(x)=(x-1)(x+2)Q(x)+ax+b\]
이다.
문제의 조건에서 \(f(1)=3\)이므로
\[a+b=3\]
또, \(f(-2)=-3\)이므로
\[-2a+b=-3\]
이다.
두 식을 빼면 \(3a=6\), \(a=2\)이다.
따라서 \(b=1\)이다.
그러므로 \(R(x)=2x+1\)이다.
\qed
\end{mdframed}

%
\prob{}
다항식 \(f(x)\)를 \(x+1\)로 나누었을 때의 나머지가 \(4\)이고, \(x-3\)으로 나누었을 때의 나머지가 \(8\)이다.
이때 \(f(x)\)를 \((x+1)(x-3)\)으로 나누었을 때의 나머지를 구하여라.

\clearpage
\(f(\alpha)\)는 \(x-\alpha\)로 나누었을 때의 나머지이다.
따라서 \(f(\alpha)=0\)이면 \(f(x)\)는 \(x-\alpha\)로 나누어떨어진다.
\begin{mdframed}
%
\theo{인수정리}
\(f(\alpha)=0\)이면 \(f(x)\)는 \(x-\alpha\)로 나누어떨어진다.
\end{mdframed}

%
\exam{}
다항식 \(f(x)=x^3+x^2+4\)에서 \(f(-2)=(-2)^3+(-2)^2+4=0\)이므로, \(f(x)\)는 \(x+2\)로 나누어 떨어진다.
조립제법을 써서 \(f(x)\)를 인수분해하면
\[f(x)=(x+2)(x^2-x+2)\]
가 된다.
이때 \(f(x)\)가 \(x+2\)를 \emph{인수로 가진다}라고 말한다.

%
\prob{다음 일차식 중에서 다항식 \(x^3+5x^2+2x-8\)의 인수인 것을 모두 찾아라.}
\begin{mdframed}[skipabove=-10pt,innertopmargin=-3pt,leftmargin=60pt,rightmargin=60pt]
\[x,\qquad x-1,\qquad x+1,\qquad x+2\]
\end{mdframed}

%
\prob{}
다항식 \(x^3+x^2+ax+a\)가 \(x-4\)로 나누어떨어지도록 상수 \(a\)의 값을 정하여라.

%%
%\prob{}
%다항식 \(x^3+ax^2+Bx+2\)가 \((x-1)(x+2)\)로 나누어떨어지도록 상수 \(a\), \(b\)의 값을 정하여라.


%%
\section*{답}

\an{5}
\begin{mdframed}
\begin{enumerate}[(1)]
\item
\(x=0\)을 대입하면 \(c=0\)이다.
\(x=1\)을 대입하면 \(a+b=0\)이다.
\(x=-1\)을 대입하면 \(a-b=0\)이다.
두 식을 연립하면 \(a=0\), \(b=0\)를 얻는다.
\item
주어진 식의 우변을 좌변으로 이항하면 \[(a-a')x^2+(b-b')x+(c-c')=0\]이다.
따라서 (1)에 의해 \(a=a'\), \(b=b'\), \(c=c'\)를 얻는다.
\end{enumerate}
\end{mdframed}

%
\an{6}
(1) \(a=-2\), \(b=2\), \(c=3\)\\
(2) \(a=5\), \(b=-2\), \(c=1\)

\an{7}
\begin{mdframed}
\begin{enumerate}[(1)]
\item
\(x=0\), \(y=0\)을 대입하면 \(c=0\)이다.
\(x=1\), \(y=0\)을 대입하면 \(a=0\)이다.
\(x=0\), \(y=1\)을 대입하면 \(b=0\)이다.
\item
주어진 식의 우변을 좌변으로 이항하면 \[(a-a')x+(b-b')y+(c-c')=0\]이다.
따라서 (1)에 의해 \(a=a'\), \(b=b'\), \(c=c'\)를 얻는다.
\end{enumerate}
\end{mdframed}

\begin{minipage}{0.49\textwidth}
%
\an{10}
\(14\)

%
\an{11}
\(20\)

%
\an{12}
(1) \(a=2\), \(b=-2\)\\
(2) \(a=1\), \(b=-6\), \(c=10\)

%
\an{15}
(1) \(9\)\\
(2) \(\frac14\)

%
\an{16}
\(-5\)

%
\an{18}
(1) \(\frac94\)\\
(2) \(4\)

%
\an{20}
\(x+5\)

%
\an{23}
\(x-1\), \(x+2\)

%
\an{24}
\(-16\)

\end{minipage}
\end{document}