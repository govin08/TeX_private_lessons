\documentclass{oblivoir}
\usepackage{amsmath,amssymb,amsthm,kotex,paralist,kswrapfig,graphicx}

\usepackage[skipabove=10pt,innertopmargin=10pt]{mdframed}

\usepackage{tabto,pifont}
\TabPositions{0.2\textwidth,0.4\textwidth,0.6\textwidth,0.8\textwidth}
\newcommand\tabb[5]{\par\bigskip\noindent\ding{172}\:{\ensuremath{#1}}
\tab\ding{173}\:\:{\ensuremath{#2}}\tab\ding{174}\:\:{\ensuremath{#3}}
\tab\ding{175}\:\:{\ensuremath{#4}}\tab\ding{176}\:\:{\ensuremath{#5}}}
\newcommand\tabfive[5]{\par\medskip\noindent\ding{172}\:\:{\ensuremath{#1}}\\
\ding{173}\:\:{\ensuremath{#2}}\\\ding{174}\:\:{\ensuremath{#3}}\\
\ding{175}\:\:{\ensuremath{#4}}\\\ding{176}\:\:{\ensuremath{#5}}}

\usepackage{enumitem}
\setlist[enumerate]{label=(\arabic*)}

\newcounter{num}
\newcommand{\defi}[1]
{\noindent\refstepcounter{num}\textbf{정의 \arabic{num}) #1}\par\noindent}
\newcommand{\theo}[1]
{\noindent\refstepcounter{num}\textbf{정리 \arabic{num}) #1}\par\noindent}
\newcommand{\exam}[1]
{\bigskip\bigskip\noindent\refstepcounter{num}\textbf{예시 \arabic{num}) #1}\par\noindent}
\newcommand{\prob}[1]
{\bigskip\bigskip\noindent\refstepcounter{num}\textbf{문제 \arabic{num}) #1}\par\noindent}
\newcommand{\proo}
{\bigskip\textsf{증명)}\par}

\newcommand{\ans}{
{\par
\raggedleft\textbf{답 : (\qquad\qquad\qquad\qquad\qquad\qquad)}
\par}\bigskip\bigskip}
\newcommand{\procedure}[1]{\begin{mdframed}\vspace{#1\textheight}\end{mdframed}}
\newcommand\an[1]{\par\bigskip\noindent\textbf{문제 #1)}\\}

\newcommand{\pb}[1]%\Phantom + fBox
{\fbox{\phantom{\ensuremath{#1}}}}
\newcommand\ba{\,|\,}

\let\oldsection\section
\renewcommand\section{\clearpage\oldsection}

\renewcommand{\arraystretch}{1.5}

\let\emph\textsf
%%%%
\begin{document}

\title{윤영 : 09 명제(2)}
\author{}
\date{\today}
\maketitle
\tableofcontents
\newpage

%%
\section{여러 가지 증명 방법}
어떤 명제를 증명하는 방법에는 여러 가지가 있다.
%
\subsection{직접 증명법}
가장 많이 쓰이는 것은 \emph{직접 증명법}으로, `가정에서 출발하여 추론을 거쳐 결론에 도달하는 방법'이다.

예전에 했었던 다음 증명에서, 
\begin{figure}[h!]
\centering
\begin{tabular}{c|p{0.8\textwidth}}
\hline
정리	&두 직선이 한 점에서 만날 때, 맞꼭지각의 크기는 서로 같다.\\\hline
가정	&	두 직선이 한 점에서 만난다.\\\hline
결론	&	맞꼭지각의 크기는 서로 같다.\\\hline
증명 &
\kswrapfig[Pos=r,Width=3cm]{opposite_angle}{
오른쪽 그림과 같이 직선 \(AC\)와 \(BD\)가 한 점 \(O\)에서 만날 때, \(\angle AOC\)는 평각이므로
\[\angle AOB+\angle BOC=180^\circ\tag{1}\]
또 \(\angle BOD\)는 평각이므로
\[\angle BOC+\angle COD=180^\circ\tag{2}\]
(1), (2)에서
\[\angle AOB+\angle BOC=\angle BOC+\angle COD\]
따라서 \(\angle AOB=\angle COD\)이다.
}\\\hline
\end{tabular}
\end{figure}
\par\noindent
먼저 `두 직선이 한 점에서 만난다'는 것을 가정하고, 여러 과정을 거쳐 `맞꼭지각의 크기가 서로 같다'라는 결론에 도달해 증명을 완성했었다.

%
\subsection{간접 증명법}
그 외에는 \emph{간접 증명법}을 사용할 수 있는데, 여기에는 \emph{대우를 이용한 증명법}과 \emph{귀류법}이 있다.

%
\subsubsection{대우를 이용한 증명법}
\[P\subset Q\iff Q^c\subset P^c\]
이므로, 주어진 명제(\(p\to q\))\가 참이면 대우(\(\sim q\to\sim p\))도 참이다.
따라서 명제 \(p\to q\)를 증명할 때, 대우인 \(\sim q\to\sim p\)가 참이라는 것을 증명해도 된다.

%
\exam{}
명제 `자연수 \(n\)에 대하여 \(n^2\)이 짝수이면 \(n\)도 짝수이다.'가 참임을 대우를 이용하여 증명하여라.
\begin{mdframed}
대우 : 자연수 \(n\)에 대하여 \(n\)이 홀수이면 \(n^2\)도 홀수이다.
\par\medskip\noindent
\(n\)이 홀수임을 가정하면 \(n\)을 \(n=2k-1\) (\(k\)는 자연수)로 나타낼 수 있으므로,
\[n^2=(2k-1)^2=4k^2-4k+1=2\cdot(2k^2-2k)+1\]
여기서 \(2k^2-2k\)\이 자연수이므로 \(n^2\)도 홀수이다.
%따라서 주어진 명제의 대우가 참이므로 주어진 명제도 참이다.
\end{mdframed}

%
\prob{}
명제 `두 자연수 \(m\), \(n\)에 대하여 \(mn\)이 짝수이면 \(m\) 또는 \(n\)이 짝수이다.'가 참임을 대우를 이용하여 증명하여라.
\procedure{0.38}

%
\subsubsection{귀류법}
\begin{align*}
P\subset Q
&\iff P-Q=\varnothing\\
&\iff P\cap Q^c=\varnothing
\end{align*}
이다.
따라서 \(p\to q\)가 참임을 보일 때, `결론을 부정하여(\(Q^c\)) 가정(\(P\))에 모순(\(\varnothing\))됨을 보이는 방법'을 사용해도 된다.

%
\exam{}
\(\sqrt2\)가 무리수임을 증명하여라.
\begin{mdframed}
이 명제를 \(p\to q\)꼴로 바꾸면
\begin{center}
\(x=\sqrt2\)이면 \(x\)는 무리수이다.
\end{center}
\noindent
이다.
결론을 부정하여 \(x\)가 무리수가 아니라고 가정하면, \(x\)는 유리수이다.
따라서 \(x\)를 \(x=\frac nm\) (\(m\), \(n\)은 서로소인 자연수)로 나타낼 수 있다.
즉 \[\sqrt2=\frac nm\]이다.
이 식의 양변을 제곱하여 정리하면
\[2m^2=n^2\]
여기서 \(n^2\)은 짝수이므로 \(n\)도 짝수이다.
그러므로 \(n=2k\) (\(k\)는 자연수)라고 하고 위 식에 대입해 정리하면
\[m^2=2k^2\]
이다.
이번에도 \(m^2\)이 짝수이고, 따라서 \(m\)도 짝수이다.
그러면 \(m\)과 \(n\)이 모두 짝수이며, 이것은 아까 \(m\), \(n\)이 서로소라는 가정에 모순이다.

따라서 \(\sqrt2\)는 무리수이다.
\end{mdframed}

\clearpage
%
\prob{}
명제 `\(\sqrt2\)가 무리수이면 \(1+\sqrt2\)가 무리수이다.'를 증명하여라.
\procedure{0.8}

%%
\section{필요조건, 충분조건}

%
\defi{필요조건, 충분조건}
명제 \(p\to q\)가 참일 때, 
\begin{mdframed}[skipabove=5pt,skipbelow=5pt,leftmargin=140pt,rightmargin=140pt]
\[p\Rightarrow q\]
\end{mdframed}
와 같이 나타낸다.
이때
\begin{mdframed}[skipabove=5pt,skipbelow=5pt,leftmargin=90pt,rightmargin=90pt]
\begin{center}
\(p\)는 \(q\)이기 위한 \emph{충분조건},\\
\(q\)는 \(p\)이기 위한 \emph{필요조건}
\end{center}
\end{mdframed}
라고 말한다.
또한 \(p\Rightarrow q\)이고 \(q\Rightarrow p\)이면, 이것을 기호로 
\begin{mdframed}[skipabove=5pt,skipbelow=5pt,leftmargin=140pt,rightmargin=140pt]
\[p\iff q\]
\end{mdframed}
와 같이 나타내고
\begin{mdframed}[skipabove=5pt,skipbelow=5pt,leftmargin=90pt,rightmargin=90pt]
\begin{center}
\(p\)는 \(q\)이기 위한 \emph{필요충분조건}
\end{center}
\end{mdframed}
라고 말한다.

%
\exam{}
두 조건 \(p\), \(q\)가 다음과 같을 때, \(p\)는 \(q\)이기 위한 무슨 조건인지 말하여라.

\vspace{-10pt}
\begin{enumerate}[itemsep=0pt]
\item
\(p:x=2\),		\tabto{0.49\textwidth}\(q:x^2-x-2=0\)
\item
\(p:3x-1<2x+3\),	\tabto{0.49\textwidth}\(q:2x-3<1\)
\item
\(p:x+2=1\),		\tabto{0.49\textwidth}\(q:x^2+2x+1=0\)
\end{enumerate}
\begin{mdframed}[skipabove=0pt]
\(p\)의 진리집합을 \(P\), \(q\)의 진리집합을 \(Q\)라고 하자.
\begin{enumerate}
\item
\(P=\{2\}\), \(Q=\{-1,2\}\)에서 \(P\subset Q\)이므로 \(p\Rightarrow q\)이다.\\
따라서 \(p\)는 \(q\)이기 위한 충분조건이다.
\item
\(P=\{x\ba x<4\}\), \(Q=\{x\ba x<2\}\)에서 \(Q\subset P\)이므로 \(q\Rightarrow p\)이다.\\
따라서 \(p\)는 \(q\)이기 위한 필요조건이다.
\item
\(P=\{-1\}\), \(Q=\{-1\}\)에서 \(P=Q\)이므로 \(p\Leftrightarrow q\)이다.\\
따라서 \(p\)는 \(q\)이기 위한 필요충분조건이다.
\end{enumerate}
\end{mdframed}
{\par\raggedleft\textbf{답 : (1) 충분조건, (2) 필요조건, (3) 필요충분조건}\par}\bigskip

%
\prob{}
두 조건 \(p\), \(q\)가 다음과 같을 때, \(p\)는 \(q\)이기 위한 무슨 조건인지 말하여라.
\begin{enumerate}
\item
\(p:x=1\)이고 \(y=1\),
\tabto{0.49\textwidth}
\(q:x+y=2\)
\item
\(p\) : `\(x\)는 \(3\)의 배수이다.',
\tabto{0.49\textwidth}
\(q\) : `\(x\)는 \(6\)의 배수이다.'
\item
\(p\) :`\(\triangle ABC\)가 이등변삼각형이다.',
\tabto{0.49\textwidth}
\(q\):`\(\triangle ABC\)의 두 변의 길이가 같다.'
\end{enumerate}
%\procedure{0.3}
{\par\raggedleft\textbf{답 :
(1)\:\:\qquad\qquad\qquad
(2)\:\:\qquad\qquad\qquad
(3)\:\:\qquad\qquad\qquad\qquad}\par}\bigskip
\vspace{0.2\textheight}

%
\prob{}
두 실수 \(x\), \(y\)에 대하여 다음 빈 칸에 `필요', `충분', `필요충분' 중 알맞은 것을 구하여라.
\begin{enumerate}
\item
\(x^2=y^2\)은 \(x=y\)이기 위한 \pb{필요충분}조건이다.
\item
\(x=2+\sqrt3\), \(y=2-\sqrt3\)은 \(x+y=4\)이기 위한 \pb{필요충분}조건이다.
\item
\(x^2-3x=0\)은 \(x=0\) 또는 \(x=3\)이기 위한 \pb{필요충분}조건이다.
\end{enumerate}

\section{절대부등식}

\begin{mdframed}[skipbelow=40pt]
%
\defi{절대부등식}
부등식 \((x-1)^2\ge0\), \(x^2+1>0\), \(|x-1|+1>0\)은 모두 \(x\)에 어떤 실수를 대입해도 항상 성립한다.
이와 같이 문자에 어떤 실수를 대입해도 항상 성립하는 부등식을 \emph{절대부등식}이라고 한다.
\end{mdframed}

%
\prob{}
다음 부등식 중 절대부등식인 것을 골라라.
\tabfive
{x+1>0}
{x^2+1>0}
{x^3+1>0}
{x^2-2x+1>0}
{x^2-2x-3>0}

\vspace{0.1\textheight}
부등식을 증명할 때, 다음과 같은 기본 성질을 자주 이용한다.
\begin{mdframed}
%
\defi{부등식의 기본 성질}
실수 \(a\), \(b\)에 대하여
\begin{enumerate}
\item
\(a>b\iff a-b>0\)
\item
\(a^2\ge0\), \(a^2+b^2\ge0\)
\item
\(a^2+b^2=0\iff a=0,\:\:b=0\)
\item
\(|a|^2=a^2\),\:\:\(|ab|=|a||b|\)
\item
\(a>0\), \(b>0\)일 때, \(a>b\iff a^2>b^2\)
\end{enumerate}
\end{mdframed}

\clearpage
%
\exam{}
\(a\), \(b\)가 실수일 때, 부등식 \(a^2+b^2\ge ab\)를 증명하여라.
\begin{mdframed}
\(a^2+b^2-ab\ge0\)임을 보이면 된다.
\begin{align*}
a^2-ab+b^2
&=a^2-ab+\frac14b^2+\frac34b^2\\
&=(a-\frac12b)^2+\frac34b^2\\
&\ge0
\end{align*}
따라서 주어진 부등식이 증명되었다.
여기서 등호가 성립하는 경우는 \(a-\frac12b=0\), \(b=0\)일 때이다.
즉 \(a=0\), \(b=0\)일 때이다.
\end{mdframed}

%
\prob{}
\(a\), \(b\)가 실수일 때, 다음 부등식을 증명하여라.
\par\noindent
(1)\:\:\(a^2-2ab+2b^2\ge0\),
\tabto{0.49\textwidth}
(2)\:\:\(a^2+4ab+6b^2\ge0\)
\procedure{0.5}

\clearpage
%
\exam{}\label{t_ineq}
\(a\), \(b\)가 실수일 때, 다음  부등식 \(|a|+|b|\ge|a+b|\)를 증명하여라.
\begin{mdframed}
\(|a|+|b|\ge0\), \(|a+b|\ge0\)이므로 \((|a|+|b|)^2\ge(|a+b|)^2\)을 증명하면 되고, 이것은 다시,  \((|a|+|b|)^2-(|a+b|)^2\ge0\)을 증명하면 된다.
\begin{align*}
(|a|+|b|)^2-(|a+b|)^2
&=(|a|^2+2|a||b|+|b|^2)-(a+b)^2\\
&=(a^2+2|ab|+b^2)-(a^2+2ab+b^2)\\
&=2(|ab|-ab)\\
&\ge0
\end{align*}
여기서 등호가 성립하는 경우는 \(|ab|-ab=0\)일 때이다.
즉 \(ab\ge0\)일 때이다.
\end{mdframed}

%
\prob{}
\(a\), \(b\)가 실수일 때, \(|a|-|b|\le|a-b|\)\를 증명하여라.
\procedure{0.45}

\clearpage
%
\exam{}\label{arithmetic_geometric}
\(a,b>0\)일 때, 부등식 \(\frac{a+b}2\ge\sqrt{ab}\)를 증명하여라.
\begin{mdframed}
\(a+b\ge2\sqrt{ab}\)를 증명하면 된다.
이때 \(a+b>0\), \(2\sqrt{ab}>0\)이므로 양변을 제곱한
\[(a+b)^2\ge\left(2\sqrt{ab}\right)^2\]
를 증명해도 된다.
\begin{align*}
(a+b)^2-\left(2\sqrt{ab}\right)^2
&=(a^2+2ab+b^2)-4ab\\
&=a^2-2ab+b^2\\
&=(a-b)^2\\
&\ge0
\end{align*}
따라서 주어진 부등식이 증명되었다.
여기서 등호가 성립하는 경우는 \(a-b=0\)일 때이다.
즉 \(a=b\)일 때이다.
\end{mdframed}

%
\prob{}\label{geometric_harmonic}
\(a,b>0\)일 때, 부등식 \(\sqrt{ab}\ge\frac{2ab}{a+b}\)를 증명하여라.
\procedure{0.4}

\clearpage
\begin{mdframed}
%
\defi{산술평균, 기하평균, 조화평균}
예제 \ref{arithmetic_geometric}와 문제 \ref{geometric_harmonic}에 나타난 세 값을 각각 \emph{산술평균}, \emph{기하평균}, \emph{조화평균}이라고 부른다.
\[산술평균=\frac{a+b}2,\quad기하평균=\sqrt{ab},\quad조화평균=\frac{2ab}{a+b}\]
\end{mdframed}

%
\exam{}
\(a=2\), \(b=8\)이라고 하자.
\begin{enumerate}
\item
산술평균은 우리가 흔히 쓰는 `평균'의 의미이다. \(2\)와 \(8\)의 중간에 위치해 있는 값인 \(\frac{2+8}2=5\)를 뜻한다.
\item
기하평균은 `곱셈을 기준으로 한 평균'이다.
\(a=2^1\)이고 \(b=2^3\)이므로, 평균을 \(2^2=4\)로 정하겠다는 뜻이다.
%위의 식으로 계산을 해봐도 \(\sqrt{2\times8}=4\)가 나온다.
\item
조화평균은 `역수를 기준으로 한 평균'이다.
\(2\)의 역수는 \(\frac12\)이고, \(8\)의 역수는 \(\frac18\)이므로, 두 수의 (산술)평균을 구하면 \(\frac{\frac12+\frac18}2=\frac5{16}\)이 된다.
여기에 다시 역수를 취한 \(\frac{16}5=3.2\)가 두 수의 조화평균이다.
\end{enumerate}
예제 \ref{arithmetic_geometric}와 문제 \ref{geometric_harmonic}에 의하면 산술평균이 가장 크고, 조화평균이 가장 작다.
\begin{mdframed}[skipabove=0pt,skipbelow=10pt,leftmargin=100pt,rightmargin=100pt]
\[\frac{a+b}2\ge\sqrt{ab}\ge\frac{2ab}{a+b}\]
\end{mdframed}
\(2\)와 \(8\)의 경우에도 \(5\ge4\ge3.2\)이다.

%
\exam{}
\(a=1\), \(b=9\)일 때, 산술평균과 기하평균, 조화평균을 구하여라.
\begin{mdframed}[skipabove=5pt]
\vspace{0.17\textheight}
\end{mdframed}
{\par\raggedleft\textbf{답 :
산술평균\(=(\quad\qquad)\),\quad
기하평균\(=(\quad\qquad)\),\quad
조화평균\(=(\quad\qquad)\)}\par}\bigskip

\clearpage
\vspace{20pt}
예시 \ref{arithmetic_geometric}를 변형한 다음 식은 자주 쓰이는 부등식이다.
\begin{mdframed}
%
\theo{산술-기하 부등식}
\(a>0\), \(b>0\)이면
\[a+b\ge2\sqrt{ab}.\]
(단 등호는 \(a=b\)일 때 성립한다.)
\end{mdframed}

%
\exam{}
\(x>0\)일 때, 산술-기하 부등식을 이용하여 \(\frac{x^2+4x+9}x\ge10\)을 증명하여라.
\begin{mdframed}
\(x>0\), \(\frac9x>0\)이므로 
\begin{align*}
\frac{x^2+4x+9}x
&=x+4+\frac9x\\
&=\left(x+\frac9x\right)+4\\
&\ge2\sqrt{x\times\frac9x}+4\\
&=2\sqrt9+4=10
\end{align*}
%주어진 식의 좌변을 통분하면 \(x+4+\frac9x\)이다.
%\(x>0\), \(\frac1x>0\)이므로 산술-기하 부등식을 사용하면
%\[
%x+\frac9x\ge2\sqrt{x\times\frac9x}=2\sqrt9=6
%\]
%따라서 \(x+4+\frac9x\ge6+4=10\)이고, 주어진 부등식이 증명되었다.
여기서 등호가 성립하는 경우는 \(x=\frac9x\)일 때이다.
즉  \(x=3\)일 때이다.
\end{mdframed}

%
\prob{}
\(a,b>0\)일 때, 다음 부등식을 증명하여라.
\par\noindent
(1)\:\:\(a+\frac1a\ge2\),
\tabto{0.49\textwidth}
(2)\:\:\(\frac ab+\frac ba\ge2\)
\procedure{0.2}

\clearpage
%
\exam{코시-슈바르츠 부등식}\label{CS}
\(a\), \(b\), \(x\), \(y\)가 실수일 때, 다음 부등식을 증명하여라
\[(a^2+b^2)(x^2+y^2)\ge(ax+by)^2\]

\begin{mdframed}[skipabove=0pt]
\((a^2+b^2)(x^2+y^2)-(ax+by)^2\ge0\)을 증명하면 된다.
\begin{align*}
&(a^2+b^2)(x^2+y^2)-(ax+by)^2\\
=&(a^2x^2+a^2y^2+b^2x^2+b^2y^2)-(a^2x^2+2abxy+b^2y^2)\\
=&a^2y^2-2abxy+b^2x^2\\
=&(ay)^2-2(ay)(bx)+(bx)^2\\
=&(ay-bx)^2\\
\ge&0
\end{align*}
따라서 주어진 부등식이 증명되었다.
여기서 등호가 성립하는 경우는 \(ay-bx=0\)일 때이다.
즉 \(a:b=x:y\)일 때이다.
\end{mdframed}

%
\prob{}
\(a\), \(b\), \(c\)가 실수일 때, 다음 부등식을 증명하여라.
\[a^2+b^2+c^2\ge ab+bc+ca\]
\begin{mdframed}[skipabove=0pt]
\vspace{0.3\textheight}
\end{mdframed}

%%
\section*{답}

%
\an{2}
대우 : 두 자연수 \(m\), \(n\)에 대하여 \(m\)과 \(n\)이 모두 홀수이면 \(mn\)이 홀수이다.
\par\medskip\noindent
\(m\)과 \(n\)이 홀수임을 가정하면 \(m\)과 \(n\)을 각각 \(m=2k-1\), \(n=2l-1\)(\(k\), \(l\)\은 자연수)로 나타낼 수 있으므로
\[mn=(2k-1)(2l-1)=4kl-2k-2l+1=2(2kl-k-l)+1\]
여기서 \(2kl-k-l\)이 자연수이므로 \(mn\)도 홀수이다.

%
\an{4}
결론을 부정해 \(1+\sqrt2\)가 유리수라고 가정하자.
두 유리수 사이의 차는 유리수이다.
따라서 \(1\)도 유리수이므로
\[(1+\sqrt2)-1=\sqrt2\]
도 유리수이다.
이것은 가정과 모순이다.

따라서 \(1+\sqrt2\)는 무리수이다.

%
\an{7}
(1)\:\:충분조건
\tabto{0.33\textwidth}
(2)\:\:필요조건
\tabto{0.66\textwidth}
(3)\:\:필요충분조건

%
\an{8}
(1)\:\:필요
\tabto{0.33\textwidth}
(2)\:\:충분
\tabto{0.66\textwidth}
(3)\:\:필요충분

%
\an{10}
\ding{173}

%
\an{13}
\begin{enumerate}
\item
\[a^2-2ab+2b^2=(a-b)^2+b^2\ge0\]
(단 등호는 \(a-b=0\), \(b=0\)일 때, 즉 \(a=0\), \(b=0\)일 때 성립)
\item
\[a^2+4ab+6b^2=(a+2b)^2+2b^2\ge0\]
(단 등호는 \(a+2b=0\), \(b=0\)일 때, 즉 \(a=0\), \(b=0\)일 때 성립)
\end{enumerate}

%
\an{15}
\begin{enumerate}[label=(\roman*)]
\item
\(|a|-|b|\ge0\)인 경우\\
\(\left(|a|-|b|\right)^2\le\left(|a-b|\right)^2\)를 증명하면 된다.
다시 말해, \(\left(|a-b|\right)^2-\left(|a|-|b|\right)^2\ge0\)을 증명하면 된다.
\begin{align*}
\left(|a-b|\right)^2-\left(|a|-|b|\right)^2
&=(a-b)^2-\left(|a|^2-2|a||b|+|b|^2\right)\\
&=(a^2-2ab+b^2)-(a^2-2|ab|+b^2)\\
&=2\left(|ab|-ab\right)\\
&\ge0
\end{align*}
\item
\(|a|-|b|<0\)인 경우\\
\(0\le|a-b|\)이므로
\[|a|-|b|<0\le|a-b|\]
\end{enumerate}
(i), (ii)로부터
\[|a|-|b|\le|a-b|\]
(단, 등호는 \(|ab|=ab\)일 때, 즉 \(ab\ge0\)일 때 성립)

\vspace{40pt}
\textbf{(다른방법)}
예시 \ref{t_ineq}의 부등식에 \(a\) 대신 \(a-b\)를 넣으면
\[|a-b|+|b|\ge|(a-b)+b|\]
이고, 이것을 정리하면
\[|a|-|b|\le|a-b|\]

\clearpage
%
\an{17}
\begin{align*}
\sqrt{ab}-\frac{2ab}{a+b}
&=\frac{(a+b)\sqrt{ab}-2ab}{a+b}\\
&=\frac{\sqrt{ab}\left(a+b-\sqrt{ab}\right)}{a+b}\\
&=\frac{\sqrt{ab}(\sqrt a-\sqrt b)^2}{a+b}\\
&\ge0
\end{align*}
(단, 등호는 \(\sqrt a=\sqrt b\)일 때, 즉 \(a=b\)일 때 성립)

\vspace{40pt}
\textbf{(다른방법)}
예시 \ref{arithmetic_geometric}에서 얻은 식의 양변에 \(\frac{2\sqrt{ab}}{a+b}\)를 곱하면
\[\sqrt{ab}\ge\frac{2ab}{a+b}\]

%
\an{20}
\(산술평균=5\), \(기하평균=3\), \(조화평균=1.8\left(=\frac95\right)\)

%
\an{23}
\begin{enumerate}
\item
\[a+\frac1a\ge2\sqrt{a\times\frac1a}=2\sqrt1=2\]
(단, 등호는 \(a=\frac1a\)일 때, 즉 \(a=1\)일 때 성립)
\item
\[\frac ab+\frac ba\ge2\sqrt{\frac ab\times\frac ba}=2\sqrt1=2\]
(단, 등호는 \(\frac ab=\frac ba\)일 때, 즉 \(a=b\)일 때 성립)
\end{enumerate}

\clearpage
%
\an{25}
\begin{align*}
&a^2+b^2+c^2-ab-bc-ca\\
=&\frac12\left[2a^2+2b^2+2c^2-2ab-2bc-2ca\right]\\
=&\frac12\left[(a^2-2ab+b^2)+(b^2-2bc+c^2)+(c^2-2ca+a^2)\right]\\
=&\frac12\left[(a-b)^2+(b-c)^2+(c-a)^2\right]\\
\ge&0
\end{align*}
(단, 등호는 \(a-b=0\), \(b-c=0\), \(c-a=0\)일 때, 즉 \(a=b=c\)일 때 성립)
\end{document}