\documentclass{oblivoir}
\usepackage{amsmath,amssymb,kotex,paralist,graphicx}
\usepackage{mdframed}
\usepackage{../kswrapfig}
\usepackage{fapapersize}
\usefapapersize{210mm,297mm,20mm,*,20mm,*}
%\pagestyle{empty}
\usepackage{multicol}
\setlength{\columnsep}{30pt}
\setlength{\columnseprule}{1pt}
%\def\columnseprulecolor{\color{blue}}

%%% 객관식 선지

\usepackage{tabto,pifont}
\TabPositions{0.2\textwidth,0.4\textwidth,0.6\textwidth,0.8\textwidth}

\newcommand\one{\ding{172}}
\newcommand\two{\ding{173}}
\newcommand\three{\ding{174}}
\newcommand\four{\ding{175}}
\newcommand\five{\ding{176}}

\newcommand\taba[5]{\par\bigskip\noindent
\one\:{\ensuremath{#1}}
\tab\two\:\:{\ensuremath{#2}}
\tab\three\:\:{\ensuremath{#3}}
\tab\four\:\:{\ensuremath{#4}}
\tab\five\:\:{\ensuremath{#5}}}

\newcommand\tabb[5]{\par\bigskip\noindent
\one\:{\ensuremath{#1}}
\tabto{0.16\textwidth}\two\:\:{\ensuremath{#2}}
\tabto{0.33\textwidth}\three\:\:{\ensuremath{#3}}\medskip\par\noindent
\four\:\:{\ensuremath{#4}}.
\tabto{0.16\textwidth}\five\:\:{\ensuremath{#5}}}

\newcommand\tabc[5]{\par\bigskip\noindent
\one\:{\ensuremath{#1}}
\tabto{0.25\textwidth}\two\:\:{\ensuremath{#2}}\medskip\par\noindent
\three\:\:{\ensuremath{#3}}
\tabto{0.25\textwidth}\four\:\:{\ensuremath{#4}}\medskip\par\noindent
\five\:\:{\ensuremath{#5}}}

\newcommand\tabd[5]{\par\bigskip\noindent
\one\:{#1}\medskip\par\noindent
\two\:\:{#2}\medskip\par\noindent
\three\:\:{#3}\medskip\par\noindent
\four\:\:{#4}\medskip\par\noindent
\five\:\:{#5}}

%%% Counters
\newcounter{num}

%%% Commands
\newcommand{\prob}[1]
{\vs\par\noindent\refstepcounter{num}\textbf{문제 \arabic{num})}\label{#1}\par\noindent}

\newcommand\vs[1]{\vspace{70pt}}

\newcommand\inc[1]{\begin{center}\includegraphics[width=0.95\columnwidth]{#1}\end{center}}

\newcommand\pb[1]{\ensuremath{\fbox{\phantom{#1}}}}

\newcommand\ba{\ensuremath{\:|\:}}

\newcommand\an[2]{\par\bigskip\noindent\textbf{문제 \ref{#1})} #2\\}

\newcommand\ans[1]{\begin{flushright}\textbf{답 : }#1\end{flushright}}

\renewcommand{\arraystretch}{1.5}

%%% Meta Commands
\let\oldsection\section
\renewcommand\section{\clearpage\oldsection}
\let\emph\textsf

%%%%
\begin{document}

\title{수학(상) : 02 인수분해}
\author{}
\date{\today}
\maketitle
\tableofcontents
\newpage

%%
\section{인수분해}

%
\exam{}
하나의 다항식을 두 개 이상의 다항식의 곱으로 나타내는 것을 인수분해라고 한다.
예를 들어, \(x^2-x-2\)를 \emph{인수분해}하면 \((x-2)(x+1)\)이다.
반면 \((x-2)(x+1)\)를 \emph{전개}하면 \(x^2-x-2\)이다.
\begin{align*}
x^2-x-2&\xrightarrow{인수분해}(x-2)(x+1)\\
(x-2)(x+1)&\xrightarrow{\phantom{인}전개\phantom{인}}x^2-x-2
\end{align*}

%\[x^3-x^2+2x-2=(x-1)(x^2+2)\]에서 좌변을 우변으로 바꾸는 것이 \emph{인수분해}이다.
%반면에 우변을 좌변으로 바꾸는 것은 \emph{전개}라고 한다.
%\((x-1)(x^2+2)\)를 \emph{전개}하면 \(x^3-x^2+2x-2\)가 된다.
%반대로 \(x^3

%
\begin{mdframed}
\theo{인수분해공식(1)}
\begin{enumerate}[(1)]
\item
\(a^2+2ab+b^2=(a+b)^2\)
\item
\(a^2-2ab+b^2=(a-b)^2\)
\item
\(a^2-b^2=(a+b)(a-b)\)
\item
\(x^2+(a+b)x+ab=(x+a)(x+b)\)
\item
\(acx^2+(ad+bc)x+bd=(ax+b)(bx+d)\)
\end{enumerate}
\end{mdframed}

%
\prob{다음 식을 인수분해하여라.}
\begin{enumerate}[(1)]
\item
\(2ab^2+6b=\)
\item
\(x^2+6x+9=\)
\item
\(25a^2-10ab+b^2=\)
\item
\(16x^2-y^2=\)
\item
\(x^2+10x+21=\)
\item
\(6a^2-13a-5=\)
\item
\(3a^2+4ab-7b^2=\)
\end{enumerate}

\newpage
%
\begin{mdframed}
\theo{인수분해 공식(2)}
\begin{enumerate}[(1)]
\setcounter{enumi}{5}
\item
\(a^3+3a^2b+3ab^2+b^3=(a+b)^3\)
\item
\(a^3-3a^2b+3ab^2-b^3=(a-b)^3\)
\item
\(a^3+b^3=(a+b)(a^2-ab+b^2)\)
\item
\(a^3-b^3=(a-b)(a^2+ab+b^2)\)
\item
\(a^2+b^2+c^2+2ab+2bc+2ca=(a+b+c)^2\)
\end{enumerate}
\end{mdframed}

%
\prob{다음 식을 인수분해하여라.}
\begin{enumerate}[(1)]
\item
\(x^3+9x^2+27x+27=\)
\item
\(x^3-6x^2+12x-8=\)
\item
\(-8a^3+36a^2b-54ab^2+27b^3=\)
\item
\(a^3+8=\)
\item
\(27a^3-64b^3=\)
\item
\(a^2+b^2+4c^2+2ab+4bc+4ca=\)
\item
\(x^2+y^2+z^2-2xy-2yz+2zx=\)
\end{enumerate}

\newpage
%
\begin{mdframed}
\theo{인수분해 공식(3)}
\begin{enumerate}[(1)]
\setcounter{enumi}{10}
\item
\(a^3+b^3+c^3-3abc=(a+b+c)(a^2+b^2+c^2-ab-bc-ca)\)
\item
\(a^4+a^2b^2+b^4=(a^2+ab+b^2)(a^2-ab+b^2)\)
\end{enumerate}
\end{mdframed}

%
\prob{다음 식을 인수분해하여라}
\begin{enumerate}[(1)]
\item
\(a^3-b^3+c^3+3abc=\)
\item
\(x^3+y^3-3xy+1=\)
\item
\(a^4+a^2+1=\)
\item
\(x^4+4x^2y^2+16y^4=\)
\end{enumerate}

\vspace{70pt}

%%
\subsection*{답}

\begin{minipage}{0.49\textwidth}
%
\an{3}
(1) \(2b(3ab+3)\)\\
(2) \((x+3)^2\)\\
(3) \((5a-b)^2\)\\
(4) \((4x+y)(4x-y)\)\\
(5) \((x+3)(x+7)\)\\
(6) \((2a-5)(3a+1)\)\\
(7) \((a-b)(3a+7b)\)

%
\an{5}
(1) \((x+3)^3\)\\
(2) \((x-2)^3\)\\
\end{minipage}
\begin{minipage}{0.49\textwidth}
(3) \((-2a+3b)^3\)\\
(4) \((a+2)(a^2-2a+4)\)\\
(5) \((3a-4b)(9a^2+12ab+16b^2\)\\
(6) \(a+b+2c)^2\)\\
(7) \((x-y+z)^2\)

%
\an{7}
(1) \((a-b+c)(a^2+b^2+c^2+ab+bc-ca)\)\\
(2) \((x+y+1)(x^2+y^2-xy-x-y+1\)\\
(3) \((a^2+a+1)(a^2-a+1)\)\\
(4) \((x^2+2xy+4y^2)(x^2-2xy+4y^2)\)
\end{minipage}

\end{document}