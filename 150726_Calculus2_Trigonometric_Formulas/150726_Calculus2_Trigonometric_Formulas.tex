\documentclass{article}
\usepackage{amsmath,amssymb,amsthm,kotex,paralist,mathrsfs}
\renewcommand\a{\ensuremath{\alpha}}
\renewcommand\b{\ensuremath{\beta}}
%\renewcommand{\c}{\ensuremath{\cos}}
%\renewcommand{\s}{\ensuremath{\sin}}
\renewcommand{\proofname}{증명}

\begin{document}

\title{미적분2 : 삼각함수의 덧셈공식과 활용}
\author{}
\date{\today}
\maketitle

\section{삼각함수의 덧셈공식}
\begin{align}
\sin(\a+\b)&=\sin\a\cos\b+\cos\a\sin\b\\
\sin(\a-\b)&=\sin\a\cos\b-\cos\a\sin\b\\
\cos(\a+\b)&=\cos\a\cos\b-\sin\a\sin\b\\
\cos(\a-\b)&=\cos\a\cos\b+\sin\a\sin\b\\
\tan(\a+\b)&=\frac{\tan\a+\tan\b}{1-\tan\a\tan\b}\\
\tan(\a-\b)&=\frac{\tan\a-\tan\b}{1+\tan\a\tan\b}
\end{align}
\begin{proof}
생략.
\end{proof}

\section{삼각함수의 배각(삼배각)공식}
\begin{align}
\sin2\a&=2\sin\a\cos\a\\
\cos2\a&=\cos^2\a-\sin^2\a=2\cos^2\a-1=1-2\sin^2\a\\
\tan2\a&=\frac{2\tan\a}{1-\tan^2\a}\\
\sin3\a&=3\sin\a-4\sin^3\a\\
\cos3\a&=4\cos^3\a-3\cos\a\\
\tan3\a&=\frac{3\tan\a-\tan^3\a}{1-\tan^\a}
\end{align}
\begin{proof}
(1), (3), (5)에 \b 대신 \(\a\) 혹은 \(2\a\)를 넣어 정리하면 얻어진다.
\end{proof}

\section{삼각함수의 반각공식}
\begin{align}
\sin^2\frac\a2&=\frac{1-\cos\a}2\\
\cos^2\frac\a2&=\frac{1+\cos\a}2\\
\tan^2\frac\a2&=\frac{1+\cos\a}{1-\cos\a}
\end{align}
\begin{proof}
(13), (14)은 (8)으로부터 당연하다.
(15)은 (13)에서 (14)을 나누면 얻어진다.
\end{proof}

\section{곱을 합으로 바꾸는 공식}
\begin{align}
\sin\a\cos\b&=\frac12\big[\sin(\a+\b)+\sin(\a-\b)\big]\\
\cos\a\sin\b&=\frac12\big[\sin(\a+\b)-\sin(\a-\b)\big]\\
\cos\a\cos\b&=\frac12\big[\cos(\a+\b)+\cos(\a-\b)\big]\\
\sin\a\sin\b&=-\frac12\big[\cos(\a+\b)-\cos(\a-\b)\big]
\end{align}
\begin{proof}
(1), (2)을 더하거나 빼고 2로 나누면 (16), (17)이 얻어진다.
(3), (4)을 더하거나 빼고 2로 나누어 잘 정리하면 (18), (19)이 얻어진다.
\end{proof}

\section{합을 곱으로 바꾸는 공식}
\begin{align}
\sin A+\sin B&=2\sin\frac{A+B}2\cos\frac{A-B}2\\
\sin A-\sin B&=2\cos\frac{A+B}2\sin\frac{A-B}2\\
\cos A+\cos B&=2\cos\frac{A+B}2\cos\frac{A-B}2\\
\sin A+\sin B&=-2\sin\frac{A+B}2\sin\frac{A-B}2
\end{align}
\begin{proof}
(16)--(19) 식에 \(\a+\b=A\), \(\a-\b=B\)로 치환해 정리한다.
다시말해, \(\a=\frac{A+B}2\), \(\b=\frac{A-B}2\)를 대입해 정리한다.
\end{proof}
\end{document}