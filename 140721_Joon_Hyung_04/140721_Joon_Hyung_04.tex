\documentclass{article}
\usepackage{amsmath,amssymb,amsthm,mdframed,kotex,paralist,kswrapfig}
\usepackage{tabto}
%\TabPositions{0.5\textwidth}
\TabPositions{0.33\textwidth,0.66\textwidth}
\newcommand\bp[1]{\begin{mdframed}[frametitle={#1},skipabove=10pt,skipbelow=20pt,innertopmargin=5pt,innerbottommargin=40pt]}
\newcommand\ep{\end{mdframed}\newpage}

\begin{document}
\title{준형04, 원의 성질---어려운 문제들}
\author{}
\date{\today}
\maketitle
%\section{이차함수의 개형}

\bp{01}
\kswrapfig[Pos=r]{problem_01}{
오른쪽 그림에서 \(\overline{AB}\)는 고정된 현이다.
임의의 현 \(CD\)를 현 \(AB\)에 수직으로 긋고, 현 \(CD\)가 현 \(AB\)에 의해서 나누어지는 두 부분의 길이의 차 \(\overline{CP}-\overline{DP}\)는 항상 일정함을 보여라.
}
\ep

\bp{02}
\kswrapfig[Pos=r]{problem_02}{
오른쪽 그림과 같이 외접원과 내접원이 존재하고, \(\overline{AB}\), \(\overline{BC}\), \(\overline{CD}\), \(\overline{DA}\)의 길이가 각각 70. 90. 80, 60인 \(\square ABCD\)의 내접원과 변 \(\overline{BC}\)의 접점일 때, \(\overline{BP}\)와 \(\overline{PC}\)의 길이의 차를 구하여라.
}
\ep

\bp{03}
\kswrapfig[Pos=r]{problem_03}{
오른쪽 그림과 같이 반지름의 길이가 각각 9cm, 4cm인 두 원 \(A\), \(B\)가 서로 외접하고 있다.
이 두 개의 원에 한 개의 직선이 각각 두 점 \(P\), \(Q\)에서 접할 때, 두 원 \(A\), \(B\)에 외접하고 직선 \(PQ\)위의 점 \(T\)에 접하는 원 \(C\)의 반지름의 길이를 구하여라.
}
\ep

\bp{04}
\kswrapfig[Pos=r]{problem_04}{
오른쪽 그림과 같은 원 \(O\)에서 \(\overline{AB}\perp\overline{PQ}\)이고 \(\overline{PH}=\overline{PK}=\overline{PL}\)이다.
점 \(R\)은 현 \(KL\)과 \(\overline{PH}\)의 교점이고 \(\overline{PK}\)의 중점 \(M\)과 점 \(H\)를 연결한 선분과 \(\overline{KL}\)의 교점을 \(N\)이라 할 때, 다음 물음에 답하여라.
\\
(1) \(\overline{PR}:\overline{PH}\)를 구하여라.\\
(2) \(\overline{KN}=3\)cm일 때, \(\overline{NR}\), \(\overline{KQ}\)의 길이를 각각 구하여라.
}
\ep

\bp{05}
\kswrapfig[Pos=r]{problem_05}{
오른쪽 그림과 같이 원 \(O\) 외부의 두 점 \(A\), \(B\)에서 원 \(O\)에 그은 네 접점을 각각 \(C\), \(D\), \(E\), \(F\)라고 하고 \(G\)와 \(H\)를 각각 \(\overline{CD}\), \(\overline{EF}\)의 중점이라고 하자.
또 \(\overline{AG}\)의 수직이등분선과 \(\overline{BH}\)의 수직이등분선의 교점을 \(I\)라고 하자.
\(\overline{IA}\)의 길이가 \(4cm\)일 때 \(\overline{IB}\)의 길이는?
}
\ep

\bp{06}
\kswrapfig[Pos=r]{problem_06}{
예각삼각형 \(ABC\)의 외접원의 중심을 \(O\), \(\overline{AB}\)와 \(\overline{CO}\)의 연장선의 교점을 \(D\), \(\overline{AC}\)와 \(\overline{BO}\)의 연장선의 교점을 \(E\)라 한다.
다음 명제의 참 거짓을 판별하시오.\\
(1) \(\overline{BD}=\overline{DE}=\overline{EC}\)이면 \(\angle A=60\textdegree\)이다.\\
(2) \(\angle A=60\textdegree\)이면 \(\overline{BD}=\overline{DE}=\overline{EC}\)이다.
}
\ep

\newpage
\section*{답}
01 : 생략\\
02 : 6\\
03 : 36/25(1.44)\\
04 : (1) 1:2, (2) \(\overline{NR}=\)1.5cm, \(\overline{KQ}=\)9cm.\\
05 : 4cm\\
06 : (1) 참, (2) 참.
\end{document}