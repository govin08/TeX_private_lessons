\documentclass{article}
\usepackage{amsmath,amssymb,amsthm,kotex,paralist,mathrsfs,centernot,marvosym}
\newcounter{num}[section]
\newcommand{\defi}[1]
{\bigskip\noindent\refstepcounter{num}\textbf{정의 \arabic{num}) #1}\par}
\newcommand{\theo}[1]
{\bigskip\noindent\refstepcounter{num}\textbf{정리 \arabic{num}) #1}\par}
\newcommand{\rema}[1]
{\bigskip\noindent\refstepcounter{num}\textbf{참고 \arabic{num}) #1}\par}

\newcommand{\notiff}{%
  \mathrel{{\ooalign{\hidewidth$\not\phantom{"}$\hidewidth\cr$\iff$}}}}
\newcommand{\LHS}{\text{LHS}}
\newcommand{\RHS}{\text{RHS}}
%\newcommand{\irange}{\ensuremath{1\le i\le n}}
%\newcommand{\jrange}{\ensuremath{1\le j\le n}}
%\newcommand{\bb}[2]{\ensuremath{(^{#1}_{#2})}}
%\newcommand{\cc}[2]{\ensuremath{_{#1}C_{#2}}}

\renewcommand{\figurename}{그림.}
\renewcommand{\proofname}{증명.}
\renewcommand{\contentsname}{목차}
\renewcommand\emph{\textbf}

%%%
\begin{document}

\title{미리-05}
\author{}
\date{\today}
\maketitle
\tableofcontents
\newpage

%
\defi{허수단위 \(i\)}
\(i^2=-1\)을 만족시키는 어떤 숫자를 \(i\)라고 표기한다.
즉 \(i\)는 이차방정식 \(x^2=-1\)의 한 근이다.
따라서 \(i\)는 실수가 아닌 수이다.
\(i\)는 \emph{허수단위}라고 불린다.

%
\rema{\label{remark}}
(1)
\(-i\) 또한 이차방정식 \(x^2=-1\)의 한 근이라는 점을 주목하자.
왜냐하면
\[(-i)^2=[(-1)\cdot i]^2=(-1)^2\cdot i^2=1\cdot(-1)=-1\]
으로 생각할 수 있기 때문이다.

(2) \(i\)를 정의함으로써, 이차방정식
\[x^2=-A\tag{a}\]
의 근도 구할 수 있다(\(A>0\)).
왜냐하면 (a)는
\[\left(\frac x{\sqrt A}\right)^2=-1\]
와 동치이고 따라서
\[\frac x{\sqrt A}=\pm i\]
이기 때문이다.
즉 \(x=\pm\sqrt A i\)이다.

물론, 아직 \(i\)에 대해서 곱셈이나 나눗셈을 정의하지 않았기 때문에 이런 식의 논리 전개는 엄밀하지 않다.
하지만 정의 \ref{operations}에서 덧셈과 곱셈을 잘 정의하고 나면 엄밀한 관점에서도 옳다는 것을 알 수 있을 것이다.

%
\defi{음수의 제곱근}
(1) 참고 \ref{remark}-(1)로부터, \(-1\)의 제곱근에는 \(\pm i\)의 두 개가 있다는 것을 알 수 있다.
3의 양의 제곱근을 \(\sqrt 3\)으로 쓰는 것과 비슷하게, \(i\)를 \(i=\sqrt{-1}\)라고 쓴다.

(2) 참고 \ref{remark}-(2)로부터, \(A>0\)일 때 \(-A\)의 제곱근에는 \(\pm\sqrt Ai\)의 두 개가 있다는 것을 알 수 있다.
마찬가지로 \(\sqrt Ai=\sqrt{-A}\)라고 쓴다.

%
\defi{복소수}
\(\mathbb R\)을 실수들의 집합이라고 하자.
즉 \(-1\in\mathbb R\), \(0\in\mathbb R\), \(\frac12\in\mathbb R\), \(\pi\in\mathbb R\), \(\cdots\).
새로운 집합 \(\mathbb C\)를
\[\mathbb C=\{a+bi\:|\:a,b\in\mathbb R\}\]
로 정의하자.
그러면
\[
z\text{가 \emph{복소수}이다. }
\iff
z\in\mathbb C
\]
이다.
즉 복소수는 \(a+bi\)꼴로 표현되는 숫자를 말한다(\(a\), \(b\)는 실수).

복소수 \(z\)가 \(z=a+bi\)로 표현될 때(\(a\), \(b\)는 실수), \(a\)를 \emph{실수부분}, \(b\)를 \emph{허수부분}이라고 부른다.
만약 \(b=0\)이면 \(z=a\)가 되어 \(z\)는 실수이다.
만약 \(b\neq0\)이면 \(z\)를 \emph{허수}라고 부른다.
만약 \(b\neq0\)이고 \(a=0\)이면 \(z=bi\)이다.
이 때 \(z\)를 \emph{순허수}라고 부른다.

%
\defi{복소수의 상등\label{equality}}
실수 \(a\), \(b\), \(c\), \(d\)에 대해
\begin{enumerate}[(1)]
\item
\(a+bi=c+di\iff a=c\:\:\&\:\:b=d\)
\item
\(a+bi=0\iff a=0\:\:\&\:\:b=0\)
이다.
\end{enumerate}

%
\defi{복소수의 연산\label{operations}}
실수들의 집합인 \(\mathbb R\)은, 덧셈과 곱셈이라는 두 가지 연산이 정의되어 있을 뿐아니라 부등호도 정의되어 있었다.
복소수들의 집합인 \(\mathbb C\)에도 덧셈과 곱셈을 정의할 수 있다.
물론 따라서 뺄셈과 나눗셈도 정의할 수 있다.
하지만 복소수에서는 부등호를 정의하지 않는다.

복소수 \(a+bi\), \(c+di\)에 대해(\(a\), \(b\), \(c\), \(d\)는 실수) 덧셈, 뺄셈, 곱셈, 나눗셈은 다음과 같이 정의한다.
\begin{enumerate}[(1)]
\item
\((a+bi)+(c+di)=(a+c)+(b+d)i\)
\item
\((a+bi)-(c+di)=(a-c)+(b-d)i\)
\item
\((a+bi)(c+di)=(ac-bd)+(ad+bc)i\)
\item
\(\displaystyle\frac{a+bi}{c+di}=\frac{ac+bd}{c^2+d^2}+\frac{bc-ad}{c^2+d^2}i\) (\(c+di\neq0\)).
\end{enumerate}

\theo{복소수의 연산의 성질}
실수에서 정의된 덧셈과 곱셈에는 \emph{결합법칙}, \emph{교환법칙}, \emph{분배법칙}이 성립했다.
복소수에서도 정의 \ref{operations}에 의해 정의된 덧셈과 곱셈에 대해 이 법칙들이 모두 성립하게 된다.

복소수 \(z_1=a+bi\), \(z_2=c+di\), \(z_3=e+fi\)(\(a\), \(b\), \(c\), \(d\), \(e\), \(f\)는 모두 실수)
\begin{enumerate}[(1)]
\item
덧셈에 대한 결합법칙 : \((z_1+z_2)+z_3=z_1+(z_2+z_3)\).
\item
덧셈에 대한 교환법칙 : \(z_1+z_2=z_2+z_1\).
\item
곱셈에 대한 결합법칙 : \((z_1z_2)z_3=z_1(z_2z_3)\).
\item
곱셈에 대한 교환법칙 : \(z_1z_2=z_2z_1\).
\item
덧셈과 곱셈에 대한 분배법칙 : \(z_1(z_2+z_3)=z_1z_2+z_1z_3\).
\end{enumerate}

\begin{proof}
(1)
\begin{align*}
(z_1+z_2)+z_3
&=[(a+bi)+(c+di)]+(e+fi)\\
&=[(a+c)+(b+d)i]+(e+fi)\\
&=[(a+c)+e]+[(b+d)+f]i\\
&=[a+(c+e)]+[b+(d+f)]i\\
&=a+bi+[(c+e)+(d+f)i]\\
&=a+bi+[(c+di)+(e+fi)]\\
&=z_1+(z_2+z_3).
\end{align*}

(2)
\begin{align*}
z_1+z_2
&=(a+bi)+(c+di)\\
&=(a+c)+(b+d)i\\
&=(c+a)+(d+b)i\\
&=(c+di)+(a+bi)\\
&=z_2+z_1.
\end{align*}

(3)
\begin{align*}
(z_1z_2)z_3
&=[(a+bi)(c+di)](e+fi)\\
&=[(ac-bd)+(ad+bc)i](e+fi)\\
&=[(ac-bd)e-(ad+bc)f]+[(ac-bd)f+(ad+bc)e]i\\
&=(ace-bde-adf-bcf)+(acf-bdf+ade+bce)i
\end{align*}
\begin{align*}
z_1(z_2z_3)
&=(a+bi)[(c+di)(e+fi)]\\
&=(a+bi)[(ce-df)+(cf+de)i]\\
&=[a(ce-df)-b(cf+de)]+[a(cf+de)+b(ce-df)]i\\
&=(ace-adf-bcf-bde)+(acf+ade+bce-bdf)i
\end{align*}
따라서 좌변과 우변이 같다.

(4)
\begin{align*}
z_1z_2
&=(a+bi)(c+di)\\
&=(ac-bd)+(ad+bc)i\\
&=(ca-db)+(cb+da)i\\
&=(c+di)(a+bi)\\
&=z_2z_1.
\end{align*}

(5)
\begin{align*}
z_1(z_2+z_3)
&=(a+bi)[(c+di)+(e+fi)]\\
&=(a+bi)[(c+e)+(d+f)i]\\
&=[a(c+e)-b(d+f)]+[a(d+f)+b(c+e)]i\\
&=(ac+ae-bd-bf)+(ad+af+bc+be)i\\
&=[(ac-bd)+(ae-bf)]+[(ad+bc)+(af+be)]i\\
&=[(ac-bd)+(ad+bc)i]+[(ae-bf)+(af+be)i]\\
&=(a+bi)(c+di)+(a+bi)(e+fi)\\
&=z_1z_2+z_1z_3.
\end{align*}
\end{proof}

%
\rema{}
(1)
실수 \(a\), \(b\)에 대해
\[(a+b)^2=a^2+2ab+b^2\]
이 성립했다.
왜냐하면 
\begin{align*}
(a+b)^2
&=(a+b)(a+b)\\
&=(a+b)a+(a+b)b\\
&=a(a+b)+b(a+b)\\
&=(a^2+ab)+(ba+b^2)\\
&=a^2+(ab+ba)+b^2\\
&=a^2+(ab+ab)+b^2\\
&=a^2+ab+ab+b^2\\
&=a^2+2ab+b^2
\end{align*}
이기 때문인데 이떄 실수의 분배법칙, 곱셈에 대한 교환법칙, 덧셈에 대한 결합법칙 등을 사용했다.
그런데 방금 복소수에 대해서도 실수와 마찬가지로 결합법칙, 교환법칙, 분배법칙이 성립한다는 것을 보였으므로, 복소수 \(z\)와 \(w\)에 대해서도
\[(z+w)^2=z^2=2zw+w^2\]
가 성립한다는 것을 알 수 있다(\(z^2=z\cdot z\)로 정의한다.).
마찬가지로 이전 단원들에서 배웠던 많은 인수분해 공식들 모두를 복소수에 적용시킬 수 있다.

(2) 정의 \ref{operations}-(4)를 다시 보자.
두 복소수 사이의 나눗셈을 다음과 같이 정의했었다.
\[\frac{a+bi}{c+di}=\frac{ac+bd}{c^2+d^2}+\frac{bc-ad}{c^2+d^2}i.\]

분모가 무리수인 경우에 `분모의 유리화'를 했듯이, 좌변의 분모가 복소수이므로 `분모의 실수화'를 한 번 해보자.
\(c-di\)가 \(0\)이 아니므로(만약 \(c-di=0\)이면 \(c+di=0\)이 되므로 모순이다.) \(c-di\)를 좌변의 분모와 분자에 각각 곱해보자.
그러면
\begin{align*}
\frac{a+bi}{c+di}
&=\frac{(a+bi)(c-di)}{(c+di)(c-di)}\\
&=\frac{(ac+bd)+(bc-ad)i}{c^2-(di)^2}\\
&=\frac{ac+bd}{c^2+d^2}+\frac{bc-ad}{c^2+d^2}i\\
\end{align*}
이 되어 정의 \ref{operations}-(4)와 같다.
그러니까 아까 정의한 복소수의 나눗셈이라는 것은, 실은, `분모의 실수화'의 결과인 셈이다.

%
\defi{켤레복소수\label{conjugate}}
복소수 \(z=a+bi\)에 대해 (\(a\), \(b\)는 실수) 새로운 복소수 \(\bar z\)를
\[\bar z=a-bi\]
로 정의하자.
\(\bar z\)는 \(z\)의 \emph{켤레복소수}라고 불린다.

%
\theo{}
두 복소수 \(z\), \(w\)에 대해서(\(z=a+bi\))
\begin{enumerate}[(1)]
\item
\(\overline{(\bar z)}=z\)
\item
\(z+\bar z=2a\), \(z\bar z=a^2+b^2\)
\item
\(\overline{z+w}=\bar z+\bar w\), \(\overline{z-w}=\bar z-\bar w\)
\item
\(\overline{zw}=\bar z\bar w\), \(\overline{\left(\frac zw\right)}=\frac{\bar z}{\bar w}\).
\end{enumerate}

\begin{proof}
증명은 생략한다.
정의 \ref{conjugate}에 의해 쉽게 증명할 수 있다.
\end{proof}
\end{document}