\documentclass{article}
\usepackage{amsmath,amssymb,amsthm,kotex,mdframed,paralist,chngcntr}

\newcounter{num}
%\newcommand{\defi}[1]
%{\bigskip\noindent\refstepcounter{num}\textbf{정의 \arabic{num}) #1}\par}
%\newcommand{\theo}[1]
%{\bigskip\noindent\refstepcounter{num}\textbf{정리 \arabic{num}) #1}\par}
%\newcommand{\exam}[1]
%{\bigskip\noindent\refstepcounter{num}\textbf{예시 \arabic{num}) #1}\par}
%\newcommand{\prob}[1]
%{\bigskip\noindent\refstepcounter{num}\textbf{문제 \arabic{num}) #1}\par}
\newcommand{\howo}[1]
{\bigskip\noindent\refstepcounter{num}\textbf{숙제 \arabic{num}) #1}\par\bigskip}


\renewcommand{\proofname}{증명)}
\counterwithout{subsection}{section}


%%%
\begin{document}

\title{영석 : 06 숙제(\(\sim\)2015. 6. 9. 화요일)}
\author{}
\date{\today}
\maketitle
%\tableofcontents
\newpage


%
\howo{}
점 \((1,3)\)를 지나고 \(y=2x+1\)에 평행한 직선의 방정식을 구하여라.

%
\howo{}
점 \((3,-2)\)를 지나고 \(y=x+3\)에 수직인 직선의 방정식을 구하여라.

%
\howo{}
점 \((0,0)\)과 직선 \(x+2y+5=0\) 사이의 거리를 구하여라.

%
\howo{}
점 \((-2,1)\)과 직선 \(3x-4y-5=0\) 사이의 거리를 구하여라.

%
\howo{}
점 \((1,2)\)과 직선 \(y=3x+4\) 사이의 거리를 구하여라.

%
\howo{}
두 직선 \(y=2x+1\), \(y=2x+5\) 사이의 거리를 구하여라.

%
\howo{}
두 직선 \(x-y+3=0\), \(2x-y-3=0\)의 교점을 지나는 직선이 제 2 사분면을 지나지 않을 때, 그 직선 중 기울기가 최소인 직선의 방정식을 구하시오.

%
\howo{}
직선 \(3x+2y+6=0\)과 평행하고 점 \((-2,2)\)를 지나는 직선이 점 \((-4,k)\)를 지날 때, 실수 \(k\)의 값을 구하여라.

%
\howo{}
세 점 \(O(0,0)\), \(A(3,-4)\), \(B(5,2)\)를 꼭지점으로 하는 삼각형 \(OAB\)의 넓이를 구하여라.
\end{document}