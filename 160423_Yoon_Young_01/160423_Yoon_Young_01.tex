\documentclass{article}
\usepackage[a4paper,margin=3cm,footskip=.5cm]{geometry}
\usepackage{amsmath,amssymb,amsthm,amsfonts,mdframed,kotex}
\newcounter{num}
\newcommand{\bp}
{\stepcounter{num}
\begin{mdframed}
[frametitle={\thenum},skipabove=10pt,skipbelow=10pt]}
\newcommand{\ep}
{\vspace{0.1\textheight}
\par
\end{mdframed}}
\newcommand{\parall}{\mathbin{\!/\mkern-5mu/\!}}
\mdfsetup{nobreak=true}

\title{윤영 : 01 중간고사 대비(3학년 1학기)(1)}
\date{\today}
\author{}

\begin{document}
\maketitle
\newpage

\bp
아래 그림과 같이 한 변의 길이가 각각 \(a\), \(b\)(\(0<a<b\))인 두 정삼각형이 있다.
\(\overline{AD}\)의 중점을 \(F\)라고 할 때, \(\overline{AF}\), \(\overline{FB}\)를 한 변으로 하는 정삼각형의 넓이를 \(S_1\), \(S_2\)라고 하자.
이때 \(S_1-S_2\)의 값은?
\ep

\bp

\ep


\end{document}