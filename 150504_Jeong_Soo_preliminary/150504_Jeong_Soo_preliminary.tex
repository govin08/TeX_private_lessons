\documentclass{article}
\usepackage{amsmath,amssymb,amsthm,mdframed,kotex,paralist}
\usepackage{tabto}
%\TabPositions{0.5\textwidth}
\TabPositions{0.33\textwidth,0.66\textwidth}
\newcommand\bp[1]{\begin{mdframed}[frametitle={#1},skipabove=10pt,skipbelow=20pt,innertopmargin=5pt,innerbottommargin=40pt]}
\newcommand\ep{\end{mdframed}\par}
\newcommand\ov[1]{\ensuremath{\overline{#1}}}
\newcommand{\vs}{\vspace{0.05\textheight}}
\newcommand{\vvs}{\vspace{0.1\textheight}}
\newcommand{\vvvs}{\vspace{0.15\textheight}}

\begin{document}
\title{정수00 : 문제들}
\author{}
\date{\today}
\maketitle

\section{이차방정식}
\bp{01}
이차방정식 \(p\), \(q\)의 두 근을 \(\alpha\), \(\beta\)라고 할 때, \(|\alpha-\beta|=2\), \(\alpha^2+\beta^2=34\)을 만족시키는 상수 \(p\), \(q\)에 대하여 \(p^2+q^2\)의 값을 구하면?
\vs\vs\ep

\bp{02}
\(x\)에 관한 이차방정식 \(x^2-2(a+k)x+k^2-4k+2b=0\)이 실수 \(k\)의 값에 관계없이 항상 중근을 가질 때, 살수 \(a\)와 \(b\)의 합을 구하면?
\vs\vs\vs\ep

\section{함수와 그래프}
\bp{03}
다음 함수의 그래프를 그리시오.\par
(1) \(y=|x^2-2x-3|\).\par\vvvs
(2) \(|x|+2|y|=1\).\par\vvvs
(3) \(y=x^3-3x+6\).\par\vvvs
(4) \(y=\sin x+\cos x\)\vvvs
\ep

\bp{04}
\(x\neq-1\)일 때,
\[f(x)=\lim_{n\to\infty}\frac{x^{n+1}-1}{x^n+1}\]
이고 \(f(-1)=-1\)일 때, 다음 물음에 답하시오.\\
(1) 이 함수의 그래프를 그리시오.\\
(2) 불연속점의 개수를 구하시오.\\
(3) 미분불가능한 지점의 개수를 구하시오.
\vvvs\ep

\section{수열}
\bp{05}
\(S_n=2n^2+4n\)일 때, \(a_n\)을 구하시오.
\ep

\bp{06}
\(a_{n+1}=2a_n-3\)이고 \(a_1=5\)일 때, \(\displaystyle\sum_{k=1}^{20}a_k\)의 값을 구하시오.
\vs\ep

\section{미분과 적분}
\bp{07}
(1) \(f(x)=(2x+1)^4\)일 때, \(f'(-1)\)을 구하시오.

\vvs\noindent
(2) \(f(x)=x^3-6x^2+5\)이고 \(0\le x\le 6\)일 때 \(f(x)\)의 최댓값을 구하시오.\vvvs
\ep

\bp{08}
(1)
\(y=x^2\)에 접하고 \((-1,-3)\)을 지나는 두 접선을 구하시오.\\
(2)
(1)에서 구한 두 접선과, 원래의 곡선이 만드는 영역의 넓이를 구하시오.\vvvs\vvs
\ep
\end{document}