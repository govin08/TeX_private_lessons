\documentclass{article}
\usepackage{amsmath,amssymb,amsthm,mdframed,kotex,paralist}
\newcounter{num}
\newcommand\bp{\refstepcounter{num}\begin{mdframed}[frametitle={\textbf{\arabic{num}.}},skipabove=10pt,skipbelow=20pt,innertopmargin=5pt,innerbottommargin=40pt]
%\noindent\textbf{\arabic{num}.}\par
}
\newcommand\ep{\end{mdframed}\par}
\newcommand\ov[1]{\ensuremath{\overline{#1}}}
\newcommand{\vs}{\vspace{0.05\textheight}}
\newcommand{\vvs}{\vspace{0.1\textheight}}
\newcommand{\vvvs}{\vspace{0.15\textheight}}

\begin{document}
\title{현빈16 : 수1 문제들}
\author{}
\date{\today}
\maketitle

\bp
모든 모서리의 길이의 합이 \(48\)이고, 대각선의 길이가 \(\sqrt{54}\)인 직육면체의 겉넓이를 구하여라.
\ep
%답 : 90

\bp
\(x+y=-1\), \(xy=-3\)일 때, \(x^5+y^5+x^6+x^6\)의 값은?
\ep
%답 : 93

\bp
\(x\)에 대한 다항식 \(f(x)\)를 \((x-1)^2\)으로 나누었을 때의 나머지는 \(x+2\)이고, \(x-2\)로 나누었을 때의 나머지는 \(3\)이다.
\(f(x)\)를 \((x-1)^2(x-2)\)로 나누었을 때의 나머지는?
\ep

\bp
자연수 \(n\)에 대하여 \(n^2+n+17\)이 어떤 자연수 \(m\)의 제곱이 될 때, \(mn\)의 값을 구하여라.
\ep
%답 : 272

\bp
\(x^4+ax+b\)가 \((x-1)^2\)을 인수로 가질 때, 상수 \(a\), \(b\)에 대하여 \(ab\)의 값은?
\ep
%답 : -12

\bp
다음 식을 만족하는 실수 \(x\), \(y\)에 대해 \(xy\)의 값을 구하시오.
\[\frac{x}{1+i}+\frac{y}{1-i}=\frac{5}{2+i}\]
\vvs\ep
%답 : \(\frac34\)
%답은 3인듯

\bp
이차방정식 \(x^2+px+q=0\)의 두 근을 \(\alpha\), \(\beta\)라고 할 때, \(|\alpha-\beta|=2\), \(\alpha^2+\beta^2=34\)을 만족시키는 상수 \(p\), \(q\)에 대하여 \(p^2+q^2\)의 값을 구하면?
\vvs\ep
%답 : 289

\bp
\(x\)에 관한 이차방정식 \(x^2-2(a+k)x+k^2-4k+2b=0\)이 실수 \(k\)의 값에 관계없이 항상 중근을 가질 때, 살수 \(a\)와 \(b\)의 합을 구하면?
\vvs\ep
%답 : 0

\bp
음이 아닌 두 실수 \(x\), \(y\)가 \(x+y=4\)를 만족시킬 때, \(2x+y^2\)의 최댓값과 최솟값의 합을 구하여라.
\ep
%답 : 23

\bp
두 실수 \(x\), \(y\)가
\[
\begin{cases}
2x^2-3xy+y^2=3\\
 x^2+2xy-3y^2=5
\end{cases}
\]
을 만족할 때, \(x+y\)의 최댓값을 구하시오.
\ep
%답 : 3

\bp
이차방정식 \(x^2+(2a+1)x+4=0\)의 한 근은 \(-1\)보다 크고 다른 한 근은 \(-1\)보다 작도록 하는 실수 \(a\) 값의 범위는?
\ep
%답 : \(a>2\)

\bp
다음 부등식을 풀어라\par
(1) \(|x+1|+|x-2|<5\)\par
(2) \(2[x]^2-9[x]+4<0\) (단 \([x]\)는 \(x\)를 넘지 않는 최대의 정수)
\vvs\ep
%답 : (1) \(-2\le x\le 3\), (2) \(1\le x<4\)

\bp
삼각형 \(A(2,7)\), \(B(-2,-1)\), \(C(4,-3)\)을 꼭짓점으로 하는 삼각형 \(ABC\)의 넓이는?
\ep
%답 : 28

\bp
두 점 \(A(-3,0)\), \(B(1,0)\)으로부터의 거리의 비가 \(3:1\)인 점 \(P\)에 대하여 삼각형 \(PAB\)의 넓이의 최댓값을 구하여라.
\ep
%답 : 3

\bp
점 \((2,5)\)에서 원 \((x+1)^2+(y-4)^2=5\)에 그은 두 접선의 기울기의 곱을 구하여라.
\ep
%답 : -1

\bp
\(l:y=\frac43x+4\sqrt3\)을 원점에 대하여 대칭이동한 직선을 \(m\)이라고 할 때, \(l\)과 \(m\) 사이의 거리는?
\ep
%답 : \(6\sqrt3\)
%답은 \(\frac{24\sqrt5}5\)인듯

\bp
직선 \(y=2x-1\)을 직선 \(y=x+2\)에 대하여 대칭이동한 직선의 \(x\)절편을 구하여라.
\ep
%답 : -7

\bp
\(x\le1\), \(y\le1\), \(y\ge-x+1\)을 만족시키는 \(x\), \(y\)에 대하여 \(3x+2y\)의 최댓값을 구하시오.
\ep
%답 : 5


\end{document}