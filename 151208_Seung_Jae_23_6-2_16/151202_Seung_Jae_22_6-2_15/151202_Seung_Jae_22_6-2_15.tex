\documentclass{oblivoir}
\usepackage{amsmath,amssymb,amsthm,kotex,mdframed,paralist,kswrapfig}

\newcounter{num}
\newcommand{\prob}
{\bigskip\noindent\refstepcounter{num}\textbf{문제 \arabic{num})}\par}

\newcommand{\ans}{\par{\raggedleft\textbf{답 : (\qquad\qquad\qquad\qquad\qquad\qquad)}
\par}\bigskip}


%%%
\begin{document}
\Large

\title{승재 22 - 6학년 2학기 - 15}
\author{}
\date{\today}
\maketitle
%\tableofcontents

\newpage

\prob
어떤 물건을 원래 가격의 5\%를 인상하여 4620원에 팔았습니다. 원래 가격은 얼마입니까?
\ans

\prob
어떤 물건을 원래 가격의 10\%를 인상하여 4620원에 팔았습니다. 원래 가격은 얼마입니까?
\ans

\prob
어떤 물건을 원래 가격의 20\%를 인상하여 4620원에 팔았습니다. 원래 가격은 얼마입니까?
\ans

\prob
어떤 물건을 원래 가격의 25\%를 인상하여 4620원에 팔았습니다. 원래 가격은 얼마입니까?
\ans

\prob
어떤 물건을 원래 가격의 40\%를 인상하여 4620원에 팔았습니다. 원래 가격은 얼마입니까?
\ans

\prob
어떤 물건을 원래 가격의 50\%를 인상하여 4620원에 팔았습니다. 원래 가격은 얼마입니까?
\ans

\newpage

\prob
어떤 물건을 원래 가격의 1\%를 할인하여 8712원에 팔았습니다. 원래 가격은 얼마입니까?
\ans

\prob
어떤 물건을 원래 가격의 4\%를 할인하여 8712원에 팔았습니다. 원래 가격은 얼마입니까?
\ans

\prob
어떤 물건을 원래 가격의 10\%를 할인하여 8712원에 팔았습니다. 원래 가격은 얼마입니까?
\ans

\prob
어떤 물건을 원래 가격의 12\%를 할인하여 8712원에 팔았습니다. 원래 가격은 얼마입니까?
\ans

\prob
어떤 물건을 원래 가격의 25\%를 할인하여 8712원에 팔았습니다. 원래 가격은 얼마입니까?
\ans


\end{document}