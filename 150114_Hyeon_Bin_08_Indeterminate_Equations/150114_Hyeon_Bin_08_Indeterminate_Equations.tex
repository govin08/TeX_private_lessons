\documentclass{memoir}
\usepackage{amsmath,amssymb,amsthm,kotex,paralist,mathrsfs}

\begin{document}

\title{현빈 : 08 부정방정식}
\author{}
\date{\today}
\maketitle

--------------------- 예 제 ---------------------

31.
이차방정식 \(x^2-2ax+2a+4=0\)의 두 근이 모두 정수일 때, 정수 \(a\)의 값과 그 때의 두 근을 구하여라.

\bigskip
32.
방정식 \(x^2-2xy+2y^2+2x+2=0\)을 만족시키는 실수 \(x\), \(y\)의 값을 구하여라.

--------------------- 연습 문제 ---------------------

193.
이차방정식 \(x^2+(m-1)x+m+1=0\)의 두 근이 정수가 되도록 하는 \(m\)의 값과 그 때의 두 근을 구하여라.
\bigskip

194.
방정식 \(2x^2+2xy+y^2+2x+1=0\)을 만족시키는 실수 \(x\), \(y\)의 값을 구하여라.

\bigskip
195.
방정식 \(x^2+y^2-4x-2y+5=0\)을 만족시키는 실수 \(x\), \(y\)의 값을 구하여라.

--------------------- 추가 문제 ---------------------

223.
방정식 \(x^2+5y^2+4xy-2y+1=0\)을 만족시키는 실수 \(x\), \(y\)의 합 \(x+y\)의 값은?
\end{document}