\documentclass[a4paper]{oblivoir}
\usepackage{amsmath,amssymb,kotex,kswrapfig,mdframed,paralist}
\usepackage{fapapersize}
\usefapapersize{210mm,297mm,20mm,*,20mm,*}

\usepackage{tabto,pifont}
\TabPositions{0.2\textwidth,0.4\textwidth,0.6\textwidth,0.8\textwidth}
\newcommand\tabb[5]{\par\noindent
\ding{172}\:{\ensuremath{#1}}
\tab\ding{173}\:\:{\ensuremath{#2}}
\tab\ding{174}\:\:{\ensuremath{#3}}
\tab\ding{175}\:\:{\ensuremath{#4}}
\tab\ding{176}\:\:{\ensuremath{#5}}}

\usepackage{tabu}

\pagestyle{empty}
%%% Counters
\newcounter{num}

%%% Commands
\newcommand\defi[1]
{\bigskip\par\noindent\stepcounter{num} \textbf{정의 \thenum) #1}\par\noindent}
\newcommand\theo[1]
{\bigskip\par\noindent\stepcounter{num} \textbf{정리 \thenum) #1}\par\noindent}
\newcommand\exam[1]
{\bigskip\par\noindent\stepcounter{num} \textbf{예시 \thenum) #1}\par\noindent}
\newcommand\prob[1]
{\bigskip\par\noindent\stepcounter{num} \textbf{문제 \thenum) #1}\par\noindent}

\newcommand\pb[1]{\ensuremath{\fbox{\phantom{#1}}}}

\newcommand\an[1]{\bigskip\par\noindent\textbf{문제 #1)}\par\noindent}

\newcommand\ba{\ensuremath{\:|\:}}

\newcommand\vs[1]{\vspace{60pt}}

%%% Meta Commands
\let\oldsection\section
\renewcommand\section{\clearpage\oldsection}

\let\emph\textsf

\begin{document}
\begin{center}
\LARGE민형, 미니테스트 01
\end{center}
\begin{flushright}
날짜 : 2017년 \(\pb3\)월 \(\pb{10}\)일 \(\pb{월}\)요일
,\qquad
제한시간 : \pb{17년}분
,\qquad
점수 : \pb{20} / \pb{20}
\end{flushright}

%미3유4
\prob{}
\(\sin\theta+\cos\theta=\frac{3\sqrt2}4\)일 때, \(16\sin\theta\cos\theta\)의 값은?
\tabb{-2}{-1}123
\vs

%미3유6
\prob{}
함수 \(y=a\cos bx+2\)의 최댓값과 최솟값이 각각 \(6\), \(-2\)이고 주기가 \(\pi\)일 때, 두 상수 \(a\), \(b\)에 대하여 \(a+b\)의 값을 구하시오. (단 \(a>0\), \(b>0\))
\vs

%미3유7
\prob{}
방정식 \(\sqrt2\cos x+1=0\)의 해가 \(x=\alpha\) 또는 \(x=\beta\)일 때, \(\alpha+\beta\)의 값은?
(단, \(0\le x<2\pi\))
\tabb{\pi}{\frac32\pi}{2\pi}{\frac52\pi}{3\pi}
\vs

%미3유8
\prob{}
부등식 \(\tan x<1\)의 해가 \(0\le x<\alpha\), \(\;\frac\pi2<x<\beta\), \(\;\gamma<x<2\pi\)일 때, \(\alpha+\beta+\gamma\)의 값은?
(단, \(0\le x<2\pi\))
\tabb{\pi}{\frac32\pi}{2\pi}{\frac52\pi}{3\pi}
\vs

%미3확3
\prob{}
중심각의 크기가 \(60^\circ\)인 부채꼴의 넓이가 \(\frac83\pi\)cm\(^2\)일 때, 이 부채꼴의 반지름의 길이를 구하시오.
\vs

%미3확5
\prob{}
\(\sin\theta=0.4226\)일 때, \(\displaystyle\cot\theta\left(
\frac{\sin\theta\cos\theta}{\csc\theta+1}+\frac{\sin\theta\cos\theta}{\csc\theta-1}
\right)\)
의 값은?
\tabb{0.4226}{0.6339}{0.8452}{1.0565}{1.2678}
\vs

\par
\begin{tabu}{|c|c|c|c|c|c|}
\hline
문제1&문제2&문제3&문제4&문제5&문제6
\\\hline
\ding{174}&\(6\)&\ding{174}&\ding{176}&\(4\)cm&\ding{174}
\\\hline
\end{tabu}

\end{document}