\documentclass{article}
\usepackage{amsmath,amssymb,amsthm,kotex,mdframed,paralist,chngcntr}

\newcounter{num}
\newcommand{\defi}[1]
{\bigskip\noindent\refstepcounter{num}\textbf{정의 \arabic{num}) #1}\par}
\newcommand{\theo}[1]
{\bigskip\noindent\refstepcounter{num}\textbf{정리 \arabic{num}) #1}\par}
\newcommand{\exam}[1]
{\bigskip\noindent\refstepcounter{num}\textbf{예시 \arabic{num}) #1}\par}
\newcommand{\prob}[1]
{\bigskip\noindent\refstepcounter{num}\textbf{문제 \arabic{num}) #1}\par}


\renewcommand{\proofname}{증명)}
\newcommand{\mo}[1]{\ensuremath{\:(\text{mod}\:\:#1)}}
\counterwithout{subsection}{section}


%%%
\begin{document}

\title{준수 : 02 사원수(四元數, Quaternion)}
\author{}
\date{\today}
\maketitle
\tableofcontents
\newpage

%%
\subsection{군(群, Group)과 환(環, Ring)}
%
\defi{군(群, Group)}\label{group}
다음 조건들을 만족시키면 집합 \(G\)는 연산 \(\circ\)에 대한 \textbf{군}이다.
\begin{enumerate}
\item
\(a\in G\), \(b\in G\)이면 \(a\circ b\in G\)이다.
\item
\(a\in G\), \(b\in G\), \(c\in G\)이면 \((a\circ b)\circ c=a\circ(b\circ c)\)이다.
\item
임의의 \(a\in G\)에 대해 \(a\circ e=e\circ a=e\)를 만족시키는 \(e\in G\)가 존재한다.
\item
임의의 \(a\in G\)에 대해 \(a\circ x=e\)를 만족시키는 \(x\in G\)가 존재한다.
\end{enumerate}

%
\defi{가환군(可換群, Abelian Group)}\label{abelian group}
군 \(G\)가 다음 추가조건을 만족시키면 \textbf{가환군}이라고 부른다.
\begin{enumerate}
\item[5. ]
\(a\in G\), \(b\in G\)이면 \(a\circ b=b\circ a\)이다.
\end{enumerate}

%
\defi{}
정의 \ref{group}, \ref{abelian group}에서 1번 조건은, `\(G\)가 연산 \(\circ\)에 대해 닫혀있다.'라고도 말할 수 있다.
2번 조건은 \textbf{결합법칙(associative law)}이라고 부른다.
3번 조건을 만족시키는 원소 \(e\)는 \textbf{항등원}이라고 부른다.
4번 조건을 만족시키는 원소 \(x\)는 \(a\)의 \textbf{역원}이라고 부르며, \(a^{-1}\)이라고 쓴다.
5번 조건은 \textbf{교환법칙(commutative law)}이라고 부른다.

%
\exam{}
(1)
자연수의 집합 \(\mathbb N\)은 덧셈에 대해 닫혀있다.
즉 \(a\in\mathbb N\)이고 \(b\in\mathbb N\)이면 \(a+b\in\mathbb N\)이다.
또 결합법칙이 성립한다.
즉 \(a+(b+c)=(a+b)+c\)이다.
하지만 항등원이 존재하지 않는다.
즉 임의의 \(a\in\mathbb N\)에 대해 \(a+e=e+a=e\)를 만족시키는 \(e\)의 값은 \(0\)인데 \(0\not\in\mathbb N\)이기 때문이다.

따라서 \(\mathbb N\)은 덧셈에 대한 군이 아니다.
\medskip

(2)
정수의 집합 \(\mathbb Z\)는 덧셈에 대해 닫혀있고 결합법칙이 성립한다.
또 항등원이 존재한다.
\(0\in\mathbb Z\)이기 때문이다.
또한 임의의 정수 \(a\)에 대해서 \(a+x=0\)을 만족시키는 \(x\)는 \(-a\)이며, \(-a\in\mathbb Z\)이다.

따라서 \(\mathbb Z\)는 덧셈에 대한 군이다.
심지어 \(\mathbb Z\)는 덧셈에 대한 교환법칙을 만족시키므로 가환군이라고 볼 수 있다.
\medskip

(3)
마찬가지로 유리수의 집합 \(\mathbb Q\), 실수의 집합 \(\mathbb R\), 복소수의 집합 \(\mathbb C\)도 덧셈에 대한 가환군이라는 것을 쉽게 밝힐 수 있다.
\medskip

(4)
정수의 집합 \(\mathbb Z\)는 곱셈에 대해 닫혀있다.
즉 \(ab\in\mathbb Z\)이다.
또 결합법칙이 성립한다.
즉 \(a(bc)=(ab)c\)이다.
또한 항등원이 존재한다.
즉 임의의 \(a\in\mathbb Z\)에 대해, \(ae=ea=e\)를 만족하는 \(e\)의 값은 \(1\)인데 \(1\in\mathbb Z\)이다.
하지만 모든 \(\mathbb Z\)의 원소에 대해 역원이 존재하지는 않는다.
예를 들어 \(2\in\mathbb Z\)에 대한 역원 \(x\)는 \(2x=1\)을 만족시키는 값으로서 \(x=\frac12\)인데 \(\frac12\not\in\mathbb Z\)이기 때문이다.

따라서 \(\mathbb Z\)는 곱셈에 대한 군이 아니다.
\medskip

(5)
유리수의 집합에서 0을 뺀 집합인 \(\mathbb Q-\{0\}\)는 곱셈에 대해 닫혀있고, 결합법칙이 성립하며, 항등원이 존재한다.
이 때의 항등원은 \(1\)이다.
또 임의의 \(\frac pq\in\mathbb Q-\{0\}\)에 대해 역원 \(x\)는 \(\frac pqx=1\)를 만족시키는 값이다.
그런데 \(p\neq0\)이고 \(q\neq0\)이므로 \(x\)를 \(x=\frac qp\)로 잡으면
\[\frac pq\cdot\frac qp=1\]
이 되어 성립하며, 이 때 \(\frac qp\in\mathbb Q-\{0\}\)이다.

따라서 \(\mathbb Q-\{0\}\)은 곱셈에 대한 군이며, 교환법칙도 만족하므로 가환군이다.
\medskip


(6)
마찬가지로 \(\mathbb R-\{0\}\), \(\mathbb C-\{0\}\)도 곱셈에 대한 가환군이라는 것을 쉽게 밝힐 수 있다.

%
\exam{}
실수의 집합 \(\mathbb R\)에 대해 연산 \(\circ\)를
\[a\circ b=a+b+ab\]
로 정의하자.
그러면 \(a\circ b\in\mathbb R\)이고 \((a\circ b)\circ c=a\circ(b\circ c)\)이며, \(a\circ b=b\circ a\)이다.
따라서 \(\mathbb R\)은 \(\circ\)에 대해 닫혀있고 결합법칙과 교환법칙이 성립한다.

항등원이 존재하려면 \(a\)의 값에 상관없이
\[a\circ e=a\]
를 만족시키는 \(e\)가 존재해야 하고 또 \(e\)가 실수여야 한다.
즉
\[a+e+ae=a\]
이고
\[e(1+a)=0\]
이다.
이 식이 \(a\)의 값에 상관없이 성립해야 하므로 \(e=0\)이며 \(0\)은 실수이다.
따라서 항등원이 존재한다.

마지막 조건을 따지기 위해서는 임의의 실수 \(a\)에 대해서 \(a\circ x=e\)를 만족시키는 \(x\)가 존재하는 지 살펴야 한다.
식을 풀면
\[a+x+ax=0\]
이다.
좀 더 정리하면
\[x=-\frac{a}{a+1}\]
이다.
\end{document}