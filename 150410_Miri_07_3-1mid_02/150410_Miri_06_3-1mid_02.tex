\documentclass{article}
\usepackage[a4paper,margin=3cm,footskip=.5cm]{geometry}
\usepackage{amsmath,amssymb,amsthm,amsfonts,mdframed,kotex}
\newcounter{num}
\newcommand{\bp}
{\stepcounter{num}
\begin{mdframed}
[frametitle={최상위\thenum},skipabove=10pt,skipbelow=10pt]}
\newcommand{\ep}
{\vspace{0.1\textheight}
\par
\end{mdframed}}
\mdfsetup{nobreak=true}

\title{미리 : 07 중간고사 대비(3학년 1학기)(2)}
\date{\today}
\author{}

\begin{document}
\maketitle
\newpage
\setcounter{num}{32}

\bp
\(\frac1{1+\sqrt2+\sqrt3}\)을 유리화하여라.
\ep

\bp
\(\frac1{\sqrt2+\sqrt3+\sqrt5}\)을 유리화하여라.
\ep

\bp
\(\frac1{2+\sqrt3+\sqrt7}\)을 유리화하여라.
\ep

\bp
\(\frac1{1-\sqrt2+\sqrt3}\)을 유리화하여라.
\ep

\bp
\(\frac1{1+\sqrt2-\sqrt3}\)을 유리화하여라.
\ep

\bp
\((1-\sqrt3+\sqrt2)^2(1+\sqrt3-\sqrt2)^2\)를 계산하면 \(a+b\sqrt6\)이고 \(a\), \(b\)가 유리수일 때, \(a+b\)의 값을 구하여라.
\ep

\bp
\((1+\sqrt2+\sqrt3)^2(1+\sqrt2-\sqrt3)^2\)를 계산하면 \(a+b\sqrt2\)이고 \(a\), \(b\)가 유리수일 때, \(a+b\)의 값을 구하여라.
\ep

\bp
\((1-\sqrt3+\sqrt5)^2(1+\sqrt3-\sqrt5)^2\)를 계산하면 \(a+b\sqrt{15}\)이고 \(a\), \(b\)가 유리수일 때, \(a+b\)의 값을 구하여라.
\ep

\bp
\((2-\sqrt3+\sqrt7)^2(2-\sqrt3-\sqrt7)^2\)를 계산하면 \(a+b\sqrt2\)이고 \(a\), \(b\)가 유리수일 때, \(a+b\)의 값을 구하여라.
\ep

\bp
\(x\)가 유리수일 때, \((2+x\sqrt2)(3-\sqrt2)\)가 유리수가 되도록 \(x\)의 값을 정하여라.
\ep

\bp
\(x\)가 유리수일 때, \((x+\sqrt3)(2\sqrt3+5)\)가 유리수가 되도록 \(x\)의 값을 정하여라.
\ep

\bp
\(a\)가 유리수일 때, \(\frac{a+\sqrt2}{3\sqrt2+1}\)이 유리수가 되도록 \(a\)의 값을 정하여라.
\ep

\bp
\(a\)가 유리수일 때, \(\frac{a+\sqrt3}{2\sqrt3-1}\)이 유리수가 되도록 \(a\)의 값을 정하여라.
\ep

\bp
\((3-\sqrt3)(2a-b\sqrt3)=4\)일 때, 유리수 \(a\), \(b\)의 값을 각각 구하여라.
\ep

\bp
\((2-\sqrt2)(a+b\sqrt3)=\sqrt2+2\)일 때, 유리수 \(a\), \(b\)의 값을 각각 구하여라.
\ep

\bp
\(\sqrt2\)의 소수 부분을 \(a\), \(a\)의 역수를 \(b\)라고 할 때, \((a-1)x+(b+1)y-4\sqrt2-2=0\)을 만족하는 유리수 \(x\), \(y\)의 값을 각각 구하여라.
\ep

\bp
\(\sqrt3\)의 소수 부분을 \(a\), \(a\)의 역수를 \(b\)라고 할 때, \((a-1)x+2(b+3)y+1=0\)을 만족하는 유리수 \(x\), \(y\)의 값을 각각 구하여라.
\ep

\bp
\(\sqrt6\)의 소수 부분을 \(a\), \(a\)의 역수를 \(b\)라고 할 때, \(ax+by+1=3\sqrt6-2\)을 만족하는 유리수 \(x\), \(y\)의 값을 각각 구하여라.
\ep

\bp
\(\displaystyle\frac{\sqrt{9^{13}+81^{4}}}{\sqrt{27^6+9^{14}}}\)
의 값을 구하여라.
\ep

\bp
\(\displaystyle\frac{\sqrt{4^5+2^{14}}}{\sqrt{4^4+16^2}}\)
의 값을 구하여라.
\ep

\bp
\(\displaystyle\frac{\sqrt{32^5+16^4}}{\sqrt{16^6+8^5}}\)
의 값을 구하여라.
\ep

\bp
\(\displaystyle\frac{\sqrt{16^7-2^{23}}}{\sqrt{8^5+4^5}}\times\sqrt2\)
의 값을 구하여라.
\ep

\bp
\(\displaystyle\frac{\sqrt{9^{13}-27^{7}}}{\sqrt{81^4-3^{11}}}\times\frac{\sqrt{2^5+4^2}}{\sqrt{16}}\)
의 값을 구하여라.
\ep

\bp
\(\frac{\sqrt n+3}{\sqrt n-3}\)의 정수 부분이 \(3\)이 되도록 하는 자연수 \(n\)의 개수를 구하여라.
\ep

\bp
\(\frac{\sqrt n+4}{\sqrt n-2}\)의 정수 부분이 \(3\)이 되도록 하는 자연수 \(n\)의 개수를 구하여라.
\ep

\bp
\(\sqrt n\) 이하의 자연수의 개수를 \(f(n)\)이라고 할 때, \(f(1)+f(2)+\cdots+f(40)\)의 값을 구하여라.
\ep

\bp
\(\sqrt{2n}\) 이하의 자연수의 개수를 \(f(n)\)이라고 할 때, \(f(1)+f(2)+\cdots+f(30)\)의 값을 구하여라.
\ep

\bp
\(\sqrt{n+10}\) 이하의 자연수의 개수를 \(f(n)\)이라고 할 때, \(f(1)+f(2)+\cdots+f(20)\)의 값을 구하여라.
\ep

\bp
\(x=\frac{\sqrt6+\sqrt2}2\), \(y=\frac{\sqrt6-\sqrt2}2\)일 때, \(\frac{\sqrt x+\sqrt y}{\sqrt x-\sqrt y}\)의 값을 구하여라.
\ep

\bp
\(x=\frac{4+\sqrt2}2\), \(y=\frac{4-\sqrt2}2\)일 때, \(\frac{\sqrt x+\sqrt y}{\sqrt x-\sqrt y}\)의 값을 구하여라.
\ep

\bp
\(x=4+2\sqrt3\), \(y=4-2\sqrt3\)일 때, \(\frac{\sqrt x+\sqrt y}{\sqrt x-\sqrt y}\)의 값을 구하여라.
\ep

\bp
\(x=\frac1{\sqrt2}\)일 때, \(\frac{\sqrt{1+x}}{\sqrt{1-x}}+\frac{\sqrt{1-x}}{\sqrt{1+x}}\)의 값을 구하여라.
\ep

\bp
\(x=\frac1{\sqrt3}\)일 때, \(\frac{\sqrt{1+x}}{\sqrt{1-x}}+\frac{\sqrt{1-x}}{\sqrt{1+x}}\)의 값을 구하여라.
\ep

\bp
\(x+y=\sqrt{7\sqrt5-\sqrt3}\), \(x-y=\sqrt{7\sqrt3-\sqrt5}\)일 때, \(x^2+xy+y^2\)의 값을 구하여라.
\ep

\bp
\(x+y=\sqrt{5\sqrt3-3\sqrt2}\), \(x-y=\sqrt{5\sqrt2-3\sqrt3}\)일 때, \(x^2+y^2\)의 값을 구하여라.
\ep
\end{document}