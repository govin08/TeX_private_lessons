\documentclass[a4paper]{oblivoir}
\usepackage{amsmath,amssymb,kotex,kswrapfig,mdframed,paralist,graphicx}
\usepackage{fapapersize}
%\usefapapersize{210mm,297mm,10mm,*,10mm,*}

\usepackage{tabto,pifont}
\TabPositions{0.2\textwidth,0.4\textwidth,0.6\textwidth,0.8\textwidth}
\newcommand\tabb[5]{\par\noindent
\ding{172}\:{\ensuremath{#1}}
\tab\ding{173}\:\:{\ensuremath{#2}}
\tab\ding{174}\:\:{\ensuremath{#3}}
\tab\ding{175}\:\:{\ensuremath{#4}}
\tab\ding{176}\:\:{\ensuremath{#5}}}

%\pagestyle{empty}

%%% Counters
\newcounter{num}
\newcounter{type}

%%% Commands
\newcommand\type[1]
{\clearpage\noindent\stepcounter{type} \textbf{유형 \thetype) #1}\par\noindent}

\newcommand\prob[1]
{\vs\par\noindent\stepcounter{num} \textbf{문제 \thetype-\thenum) #1}\par\noindent}

\newenvironment{expl}{\begin{mdframed}[frametitle=풀이]}{\end{mdframed}}

\newcommand\pb[1]{\ensuremath{\fbox{\phantom{#1}}}}

\newcommand\vs[1]{\vspace{30pt}}

\newcommand\an[1]{\bigskip\par\noindent\textbf{문제 #1)}\par\noindent}

%%% Meta Commands
\let\oldsection\section
\renewcommand\section{\clearpage\oldsection}

\let\emph\textsf

\begin{document}
\title{다현, 서진 : 3-1 제곱근-절댓값 관련 문제들}
\date{\today}
\author{}
\maketitle

%%
\bigskip\bigskip\noindent\stepcounter{type} \textbf{유형 \thetype)}\par\noindent
\(a>0,\quad b<0\)일 때, \(\sqrt{a^2}+\sqrt{4b^2}\)을 간단히 하여라.
\begin{expl}
\begin{align*}
\sqrt{a^2}+\sqrt{4b^2}
&=\sqrt{a^2}+\sqrt{(2b)^2}\\
&=|a|+|2b|\\
&=a+(-2b)=a-2b
\end{align*}
\end{expl}

%
\prob{}
\(a>0,\quad b>0\)일 때, \(\sqrt{a^2}+\sqrt{b^2}\)을 간단히 하여라.

%
\prob{}
\(a<0,\quad b<0\)일 때, \(\sqrt{a^2}+\sqrt{(-b)^2}\)을 간단히 하여라.

%
\prob{}
\(a<0,\quad b>0\)일 때, \(\sqrt{4a^2}+\sqrt{(9b)^2}\)을 간단히 하여라.

%
\prob{개념+유형, 유형편 p8 \#20}
\(a>0,\quad b<0\)일 때, \(\sqrt{16a^2}-\sqrt{(-3b)^2}+\sqrt{b^2}\)을 간단히 하여라.
%\begin{expl}
%\begin{align*}
%\sqrt{16a^2}-\sqrt{(-3b)^2}+\sqrt{b^2}
%&=|4a|-|-3b|+|b|\\
%&=4a-(-3b)+(-b)=4a+2b
%\end{align*}
%\end{expl}

%%
%\bigskip\bigskip\noindent\stepcounter{type} \textbf{유형 \thetype) {개념+유형, 개념편 p26 \#8}}\par\noindent
\type{개념+유형, 개념편 p26 \#8}
\(a>0,\quad ab<0\)일 때, \(\sqrt{(-a)^2}+\sqrt{9a^2}-\sqrt{4b^2}\)을 간단히 하여라.
\begin{expl}
\(a>0\)이고 \(b<0\)이다.
따라서
\begin{align*}
\sqrt{(-a)^2}+\sqrt{9a^2}-\sqrt{4b^2}
&=|-a|+|3a|-|2b|\\
&=-(-a)+(3a)-(-2b)=4a+2b
\end{align*}
\end{expl}

%
\prob{}
\(a>0,\quad ab>0\)일 때, \(\sqrt{4a^2}-\sqrt{9b^2}\)를 간단히 하여라.

%
\prob{}
\(a<0<b\)일 때, \(\sqrt{(-3a)^2}+\sqrt{b^2}+(\sqrt a)^2\)를 간단히 하여라.

%
\prob{개념+유형, 유형편 p8 \#20}
\(a>b,\quad ab<0\)일 때, \((-\sqrt{a})^2-\sqrt{(-a)^2}+\sqrt{9b^2}\)을 간단히 하면?

%
\prob{}
\(a<b<0\)일 때, \(\sqrt{(-b)^2}+\sqrt{4a^2}\)를 간단히 하여라.

%
\prob{}
\(a<b,\quad ab<0\)일 때, \(\sqrt{\frac14a^2}+\sqrt{(-b)^2}-\sqrt{4a^2}\)을 간단히 하면?

%
\type{개념+유형, 유형편 p9 \#24}
\(a>b,\quad ab<0\)일 때, \(\sqrt{a^2}+\sqrt{(-2a)^2}+\sqrt{(b-a)^2}\)을 간단히 하여라.
\begin{expl}
\(a>0\), \(b<0\)이다.
또 \(b-a<0\)이다.
따라서
\begin{align*}
\sqrt{a^2}+\sqrt{(-2a)^2}+\sqrt{(b-a)^2}
&=|a|+|-2a|+|b-a|\\
&=a-(-2a)-(b-a)=4a-b
\end{align*}
\end{expl}

%
\prob{}
\(a>b,\quad a>0\)일 때, \(\sqrt{(-a)^2}+\sqrt{(b-a)^2}\)을 간단히 하여라.


%
\prob{개념+유형, 유형편 p21 \#14}
\(a-b>0,\quad ab<0\)일 때, \(\sqrt{(-2a)^2}-\sqrt{(b-a)^2}+\sqrt{9b^2}\)을 간단히 하여라.

%
\prob{}
\(a<b,\quad ab<0\)일 때, \(\sqrt{(a-b)^2}+\sqrt{a^2}+(\sqrt b)^2\)을 간단히 하여라.

%
\prob{}
\(a-b<0,\quad ab<0\)일 때, \(\sqrt{(a-b)^2}+\sqrt{(b-a)^2}+(-\sqrt b)^2\)을 간단히 하여라.

%
\type{개념+유형, 유형편 p9 \#25}
\(a>b>c>0\)일 때,\\
\(\sqrt{(a-b)^2}-\sqrt{(b-a)^2}-\sqrt{(c-a)^2}\)을 간단히 하면?
\begin{expl}
\(a-b>0\), \(b-a<0\), \(c-a<0\)이므로
\begin{align*}
\sqrt{(a-b)^2}-\sqrt{(b-a)^2}-\sqrt{(c-a)^2}
&=|a-b|-|b-a|-|c-a|\\
&=(a-b)-\{-(b-a)\}-\{-(c-a)\}\\
&=a-b+b-a+c-a=c-a
\end{align*}
\end{expl}

%
\prob{}
\(a>b>c>0\)일 때,\\
\(\sqrt{(a-b)^2}+\sqrt{(b-c)^2}+\sqrt{(c-a)^2}\)을 간단히 하면?

%
\prob{}
\(a<b<c\)일 때,\\
\(\sqrt{(a-b)^2}+\sqrt{(b-c)^2}-\sqrt{(c-a)^2}\)을 간단히 하면?

%
\prob{}
\(a<b<c<0\)일 때,\\
\(\sqrt{a^2}+\sqrt{b^2}+\sqrt{(c-b)^2}\)을 간단히 하면?

%
\prob{}
\(a<0<b<c\)일 때,\\
\((-\sqrt{b})^2+\sqrt{(c-a)^2}-\sqrt{(b+c)^2}\)을 간단히 하면?

%%
\type{개념+유형, 개념편 p28 \#28}
\(0<a<1\)일 때,\\
\(\displaystyle\sqrt{\left(a+\frac1a\right)^2}-\sqrt{\left(a-\frac1a\right)^2}-\sqrt{(2a)^2}\)을 간단히 하여라.
\begin{expl}
\(a>0\)이므로
\[a+\frac1a>0\tag{1}\]
\(0<a<1\)이므로 \(\frac1a>1\)이고, 따라서\(\frac1a>a\).
그러므로
\[a-\frac1a<0\tag{2}\]
또한
\[2a>0\]
그러므로
\begin{align*}
\sqrt{\left(a+\frac1a\right)^2}-\sqrt{\left(a-\frac1a\right)^2}-\sqrt{(2a)^2}
&=\left|a+\frac1a\right|-\left|a-\frac1a\right|-|2a|\\
&=\left(a+\frac1a\right)-\left\{-\left(a-\frac1a\right)\right\}-2a\\
&=\left(a+\frac1a\right)+\left(a-\frac1a\right)-2a\\
&=0
\end{align*}
\end{expl}

%
\prob{}
\(0<a<1\)일 때,\\
\(\displaystyle\sqrt{\left(\frac1a+a\right)^2}+\sqrt{\left(\frac1a-a\right)^2}\)을 간단히 하여라.

%
\prob{}
\(a>1\)일 때,\\
\(\displaystyle\sqrt{a^2}-\sqrt{\left(a-\frac1a\right)^2}\)\를 간단히 하여라.

%%
\type{개념+유형, 유형편 p21 \#21}
\(a<0<b<1\)일 때,
\(\displaystyle
\sqrt{(a-b)^2}+\sqrt{\left(b-\frac1b\right)^2}-\sqrt{\left(b+\frac1b\right)^2}-\sqrt{(-a)^2}\)
을 간단히 하여라.
\begin{expl}
\(a<b\)로부터
\[a-b<0\tag{1}\]
\(0<b<1\)로부터 \(\frac1b>1\).
따라서 \(b<\frac1b\).
그러므로
\[b-\frac1b<0\tag{2}\]
\(b>0\)으로부터
\[b+\frac1b>0\tag{3}\]
\(a<0\)으로부터
\[-a>0\tag{4}\]
따라서
\begin{align*}
\sqrt{(a-b)^2}+\sqrt{\left(b-\frac1b\right)^2}-\sqrt{\left(b+\frac1b\right)^2}-\sqrt{(-a)^2}
&=|a-b|+\left|b-\frac1b\right|-\left|b+\frac1b\right|-|-a|\\
&=-(a-b)-\left(b-\frac1b\right)-\left(b+\frac1b\right)-(-a)\\
&=-a+b-b+\frac1b-b-\frac1b+a=-b
\end{align*}
\end{expl}

%
\prob{}
\(a<0<b<1\)일 때,
\(\displaystyle
\sqrt{a^2}+\sqrt{\left(\frac1b-b\right)^2}+\sqrt{\left(\frac1b+b\right)^2}\)
을 간단히 하여라.


%
\prob{}
\(0<a<1<b\)일 때,
\(\displaystyle
\sqrt{b^2}-\sqrt{(a-b)^2}+\sqrt{\left(a-\frac1a\right)^2}\)
을 간단히 하여라.


\section*{답}
\begin{minipage}{0.49\textwidth}
%
\an{1-1}
\(a+b\)

%
\an{1-2}
\(-a-b\)

%
\an{1-3}
\(-2a+3b\)

%
\an{1-4}
\(4a+2b\)

%
\an{2-5}
\(2a-3b\)

%
\an{2-6}
\(-2a+b\)

%
\an{2-7}
\(-3b\)

%
\an{2-8}
\(-2a-b\)

%
\an{2-9}
\(\frac32a+b\)

%
\an{3-10}
\(2a-b\)
\end{minipage}
\begin{minipage}{0.49\textwidth}

%
\an{3-11}
\(a-2b\)

%
\an{3-12}
\(-2a+2b\)

%
\an{3-13}
\(-2a+3b\)

%
\an{4-14}
\(2a-2c\)

%
\an{4-15}
\(0\)

%
\an{4-16}
\(-a-2b+c\)

%
\an{4-17}
\(-a\)

%
\an{5-18}
\(\frac2a\)

%
\an{5-19}
\(\frac1a\)

%
\an{6-20}
\(-a+\frac2b\)

%
\an{6-21}
\(-a\)

\end{minipage}


\end{document}