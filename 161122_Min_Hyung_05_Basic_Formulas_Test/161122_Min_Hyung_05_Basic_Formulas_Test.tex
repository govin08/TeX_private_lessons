\documentclass{oblivoir}
\usepackage{amsmath,amssymb,amsthm,kotex,paralist,kswrapfig,tabu}

\usepackage[skipabove=10pt,skipbelow=10pt,innertopmargin=10pt]{mdframed}

\usepackage{tabto,pifont}
\TabPositions{0.2\textwidth,0.4\textwidth,0.6\textwidth,0.8\textwidth}
\newcommand\tabb[5]{\par\bigskip\noindent
\ding{172}\:{\ensuremath{#1}}
\tab\ding{173}\:\:{\ensuremath{#2}}
\tab\ding{174}\:\:{\ensuremath{#3}}
\tab\ding{175}\:\:{\ensuremath{#4}}
\tab\ding{176}\:\:{\ensuremath{#5}}}

\usepackage{enumitem}
\setlist[enumerate]{label=(\arabic*)}

\newcounter{num}
\newcommand{\defi}[1]
{\noindent\refstepcounter{num}\textbf{정의 \arabic{num}) #1}\par\noindent}
\newcommand{\theo}[1]
{\noindent\refstepcounter{num}\textbf{정리 \arabic{num}) #1}\par\noindent}
\newcommand{\exam}[1]
{\bigskip\bigskip\noindent\refstepcounter{num}\textbf{예시 \arabic{num}) #1}\par\noindent}
\newcommand{\prob}[1]
{\bigskip\bigskip\noindent\refstepcounter{num}\textbf{문제 \arabic{num}) #1}\par\noindent}
\newcommand{\proo}
{\bigskip\textsf{증명)}\par}

\newcommand{\procedure}[1]{\begin{mdframed}\vspace{#1\textwidth}\end{mdframed}}
\newcommand{\ans}{
{\par\raggedleft\textbf{답 : (\qquad\qquad\qquad\qquad\qquad\qquad)}\par}\bigskip\bigskip}
\newcommand\an[1]{\par\bigskip\noindent\textbf{문제 #1)}\\}

\newcommand{\pb}[1]%\Phantom + fBox
{\fbox{\phantom{\ensuremath{#1}}}}

\newcommand\ba{\,|\,}

\let\oldsection\section
\renewcommand\section{\clearpage\oldsection}

\newenvironment{talign}
 {\let\displaystyle\textstyle\align}
 {\endalign}
\newenvironment{talign*}
 {\let\displaystyle\textstyle\csname align*\endcsname}
 {\endalign}

\let\emph\textsf

\usepackage{fapapersize}
\usefapapersize{210mm,297mm,15mm,15mm,15mm,15mm}
%%%%
\begin{document}

\title{민형 : 05 기본 공식 테스트}
\author{}
\date{\today}
\maketitle

\prob{}
다음 빈칸을 채우시오
\begin{align*}
(1)&\:\:
\int\sin x\,dx=-\cos x+C\\
(2)&\:\:
\int\cos x\,dx=\sin x+C\\
(3)&\:\:
\int\sec^2 x\,dx=\tan x+C\\
(4)&\:\:
\int\csc x\,dx=-\cot x+C\\
(5)&\:\:
\int\tan x\sec x\,dx=\sec x+C\\
(6)&\:\:
\int\cot x\csc x\,dx=-\csc x+C\\
(7)&\:\:
\int e^x\,dx=e^x+C\\
(8)&\:\:
\int\frac1x\,dx=\ln x+C\\
(9)&\:\:
\int\sqrt x\,dx=\frac2{3\sqrt x}+C\\
(10)&\:\:
\int\cos^2 x\,dx=\frac12x+\frac14\sin2x+C\\
(11)&\:\:
\int\sin^2 x\,dx=\frac12x-\frac14\sin2x+C\\
(12)&\:\:
\int\ln x\,dx=x\ln x-x+C\\
(13)&\:\:
\int xe^x\,dx=xe^x-e^x+C\\
(14)&\:\:
\int_a^bf(x)\,dx
=
\lim_{n\to\infty}\sum_{k=1}^nf\left(a+\frac{b-a}nk\right)\frac{b-a}n
\end{align*}

\prob{}
다음 급수의 합을 정적분을 이용하여 구하여라.
\[
\lim_{n\to\infty}\frac1n\sum_{k=1}^n\ln\frac{n+k}n=2\ln2-1
\]
\end{document}