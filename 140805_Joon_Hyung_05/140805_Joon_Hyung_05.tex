\documentclass{article}
\usepackage{amsmath,amssymb,amsthm,mdframed,kotex,paralist}
\usepackage{tabto}
%\TabPositions{0.5\textwidth}
\TabPositions{0.33\textwidth,0.66\textwidth}
\newcommand\bp[1]{\begin{mdframed}[frametitle={#1},skipabove=10pt,skipbelow=20pt,innertopmargin=5pt,innerbottommargin=40pt]}
\newcommand\ep{\end{mdframed}\bigskip\bigskip\bigskip\bigskip\bigskip\bigskip\bigskip\bigskip}

\begin{document}
\title{준형05, 조건을 만족시키는 도형의 방정식}
\author{}
\date{\today}
\maketitle
%\section{이차함수의 개형}

\bp{01}
원점 \(O\)와 \(y=x+4\)위의 임의의 점 \(P\)에 대해, \(\overline{OP}\)의 연장선 위의 점 \(Q\)가 \(\overline{OP}:\overline{OQ}=1:2\)를 만족시킬 때 \(Q\)가 그리는 도형의 방정식은?
\ep

\bp{02}
점 \(A(0,1)\)와 \(y=2x-2\)위의 임의의 점 \(P\)에 대해, \(\overline{AP}\)의 연장선 위의 점 \(Q\)가 \(\overline{AP}=\overline{PQ}\)를 만족시킬 때, \(Q\)가 그리는 도형의 방정식은?
\ep

\bp{03}
점 \(A(1,2)\)와 \(y=x-1\)위의 임의의 점 \(P\)에 대해, \(\overline{AP}\)의 연장선 위의 점 \(Q\)가 \(\overline{AP}=3\overline{AQ}\)를 만족시킬 때, \(Q\)가 그리는 도형의 방정식은?
\ep

\bp{04}
점 \(A(1,1)\)와 \(y=x+2\)위의 임의의 점 \(P\)에 대해, \(\overline{AP}\)의 연장선 위의 점 \(Q\)가 \(\overline{AP}\)의 \(1:2\) 외분점 \(Q\)가 그리는 도형의 방정식은?
\ep

\bp{05}
원점 \(O\)와 \(y=x^2\)위의 임의의 점 \(P\)에 대해, \(\overline{OP}\)의 연장선 위의 점 \(Q\)가 \(\overline{OP}:\overline{OQ}=1:2\)를 만족시킬 때, \(Q\)가 그리는 도형의 방정식은?
\ep

\bp{06}
점 \(A(0,1)\)와 \(y=x^2\)위의 임의의 점 \(P\)에 대해, \(\overline{AP}\)의 연장선 위의 점 \(Q\)가 \(\overline{AP}=\overline{PQ}\)를 만족시킬 때, \(Q\)가 그리는 자취의 방정식은?
\ep


\newpage
\section*{답}
01 : $y=x+8$\\
02 : $y=2x-5$\\
03 : $y=x-5$\\
04 : $y=x-2$\\
05 : $y=\frac12x^2$\\
06 : $y=\frac12x^2-1$.
\end{document}