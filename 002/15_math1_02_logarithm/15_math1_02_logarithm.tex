\documentclass{oblivoir}
\usepackage{amsmath,amssymb,kotex,paralist,graphicx}
\usepackage{mdframed}
\usepackage{../kswrapfig}
\usepackage{fapapersize}
\usefapapersize{210mm,297mm,20mm,*,20mm,*}
%\pagestyle{empty}
\usepackage{multicol}
\setlength{\columnsep}{30pt}
\setlength{\columnseprule}{1pt}
%\def\columnseprulecolor{\color{blue}}

%%% 객관식 선지

\usepackage{tabto,pifont}
\TabPositions{0.2\textwidth,0.4\textwidth,0.6\textwidth,0.8\textwidth}

\newcommand\one{\ding{172}}
\newcommand\two{\ding{173}}
\newcommand\three{\ding{174}}
\newcommand\four{\ding{175}}
\newcommand\five{\ding{176}}

\newcommand\taba[5]{\par\bigskip\noindent
\one\:{\ensuremath{#1}}
\tab\two\:\:{\ensuremath{#2}}
\tab\three\:\:{\ensuremath{#3}}
\tab\four\:\:{\ensuremath{#4}}
\tab\five\:\:{\ensuremath{#5}}}

\newcommand\tabb[5]{\par\bigskip\noindent
\one\:{\ensuremath{#1}}
\tabto{0.16\textwidth}\two\:\:{\ensuremath{#2}}
\tabto{0.33\textwidth}\three\:\:{\ensuremath{#3}}\medskip\par\noindent
\four\:\:{\ensuremath{#4}}.
\tabto{0.16\textwidth}\five\:\:{\ensuremath{#5}}}

\newcommand\tabc[5]{\par\bigskip\noindent
\one\:{\ensuremath{#1}}
\tabto{0.25\textwidth}\two\:\:{\ensuremath{#2}}\medskip\par\noindent
\three\:\:{\ensuremath{#3}}
\tabto{0.25\textwidth}\four\:\:{\ensuremath{#4}}\medskip\par\noindent
\five\:\:{\ensuremath{#5}}}

\newcommand\tabd[5]{\par\bigskip\noindent
\one\:{#1}\medskip\par\noindent
\two\:\:{#2}\medskip\par\noindent
\three\:\:{#3}\medskip\par\noindent
\four\:\:{#4}\medskip\par\noindent
\five\:\:{#5}}

%%% Counters
\newcounter{num}

%%% Commands
\newcommand{\prob}[1]
{\vs\par\noindent\refstepcounter{num}\textbf{문제 \arabic{num})}\label{#1}\par\noindent}

\newcommand\vs[1]{\vspace{70pt}}

\newcommand\inc[1]{\begin{center}\includegraphics[width=0.95\columnwidth]{#1}\end{center}}

\newcommand\pb[1]{\ensuremath{\fbox{\phantom{#1}}}}

\newcommand\ba{\ensuremath{\:|\:}}

\newcommand\an[2]{\par\bigskip\noindent\textbf{문제 \ref{#1})} #2\\}

\newcommand\ans[1]{\begin{flushright}\textbf{답 : }#1\end{flushright}}

\renewcommand{\arraystretch}{1.5}

%%% Meta Commands
\let\oldsection\section
\renewcommand\section{\clearpage\oldsection}
\let\emph\textsf

%%%%
\begin{document}

\title{수학Ⅰ : 02 로그}
\author{}
\date{\today}
\maketitle
\tableofcontents
\newpage

%%%
\section{로그의 뜻}

%
\prob{다음 등식을 만족시키는 \(x\)의 값을 각각 구하여라.}
\begin{enumerate}\label{log1}
\item
\(2^x=8\)
\item
\(2^x=\frac12\)
\item
\(2^x=1\)
\item
\(2^x=2\sqrt2\)
\end{enumerate}

%
\exam{등식 \(2^x=7\)를 만족시키는 \(x\)는 어떤 값인지 알아보자.}\label{log2}
계산기를 사용해 계산하면,
\begin{align*}
&2^2=4				\text{ 이고 }&&2^3=8				\text{ 이므로 }&&2<x<3\text{이다.}\\
&2^{2.8}=6.964\cdots 	\text{ 이고 }&&2^{2.9}=7.464\cdots 	\text{ 이므로 }&&2.8<x<2.9\text{이다.}\\
&2^{2.80}=6.964\cdots 	\text{ 이고 }&&2^{2.81}=7.012\cdots 	\text{ 이므로 }&&2.80<x<2.81\text{이다.}\\
&2^{2.807}=6.998\cdots 	\text{ 이고 }&&2^{2.808}=7.003\cdots 	\text{ 이므로 }&&2.807<x<2.808\text{이다.}\\
\end{align*}
%따라서, \(2^x=7\)을 만족시키는 \(x\)의 값은
따라서, \(x=2.807\cdots\)이다.
%복잡한 값이긴 하지만 \(2^x=7\)를 만족시키는 실수 \(x\)의 값은 존재할 것이다.
이 값 \(x\)를 \(\log_27\)로 표시한다.
\begin{mdframed}[rightmargin=.2\textwidth,leftmargin=.2\textwidth]
\centering
\(2^x=7\qquad\iff\qquad x=\log_27\)
\end{mdframed}
위의 계산에서 \(\log_27\)을 소수점 둘째자리까지 구하면,
\[\log_27\approx2.81\]
인 것이다.

%
\prob{등식 \(3^x=5\)를 만족시키는 \(x\)의 값을 로그로 표시하고, 소수점 둘째자리까지 구하여라(계산기).}\label{log3}

\newpage
\begin{mdframed}
%
\defi{로그의 정의}\label{log4}
\(a>0\), \(a\neq1\), \(N>0\)일 때, 
\begin{mdframed}[rightmargin=.2\textwidth,leftmargin=.2\textwidth]
\centering
\(a^x=N\qquad\iff\qquad x=\log_aN\)
\end{mdframed}
%\[a^x=N\]
%을 만족시키는 \(x\)의 값을
%\[x=\log_aN\]
%으로 표시한다.
이때 \(a\)를 \fbox밑, \(N\)을 \fbox{진수}라고 부른다.
\end{mdframed}


%
\exam{다음 등식들을 만족시키는 \(x\)의 값을 생각해보자.}\label{log5}
%\begin{enumerate}
%\item
%\(2^x=-7\)을 만족시키는 \(x\)의 값을 구하여라.
%\item
%\(2^x=0\)을 만족시키는 \(x\)의 값을 구하여라.
%\item
%\((\frac12)^x=7\)을 만족시키는 \(x\)의 값을 구하여라.
%\item
%\((-2)^x=7\)을 만족시키는 \(x\)의 값을 구하여라.
%\item
%\(1^x=7\)을 만족시키는 \(x\)의 값을 구하여라.
%\item
%\(0^x=7\)을 만족시키는 \(x\)의 값을 구하여라.
%\end{enumerate}
\par\vspace{-10pt}\noindent
(1)\:\:{\(2^x=-7\)}
\tabto{0.33\textwidth}
(2)\:\:{\((\frac12)^x=7\)}
\tabto{0.66\textwidth}
(3)\:\:{\((-2)^x=7\)}
\medskip\par\noindent
(4)\:\:{\(1^x=7\)}
\tabto{0.33\textwidth}
(5)\:\:{\(1^x=1\)}
\tabto{0.66\textwidth}
(6)\:\:{\(0^x=7\)}
%\begin{enumerate}
%\item[(1), (2)]
\begin{mdframed}
\begin{enumerate}
\item
\(2^x>0\)이므로 (1)을 만족시키는 실수 \(x\)는 존재하지 않는다.
\item
\((\frac12)^x=(2^{-1})^x=2^{-x}\)이다.
\(2^{-x}=7\)에서 \(-x=2.807\cdots\)이다.
따라서 \(x=-2.807\cdots\)이다.
\item
\(x\)는 정수가 아니다.
정수가 아닌 지수가 정의되려면 밑은 음수여야 하는데 이 경우에 밑은 \(-2\)이다.
따라서 위 등식을 만족시키는 \(x\)는 존재하지 않는다.
\item
\(1^x=1\)이므로 위 등식을 만족시키는 \(x\)는 존재하지 않는다.
\item
\(x\)의 값에 관계없이 \(1^x=1\)이 성립하므로 \(x\)는 모든 실수가 가능하다.
\item
\(0^x=0\)이므로 위 등식을 만족시키는 \(x\)는 존재하지 않는다.
%\end{enumerate}
\end{enumerate}
\end{mdframed}
위의 예시에서 보듯 \(a\le0\), \(a=1\), \(N\le0\)이면 \(a^x=N\)를 만족시키는 \(x\)의 값이 존재하지 않거나 유일하지 않다.
한편, \(a>0\), \(a\neq1\)이고 \(N>0\)이면 \(x\)의 값은 유일하게 하나 존재한다는 것이 알려져 있다.

%
\prob{다음 등식들을 만족시키는 \(x\)의 값을 소수점 둘째자리까지 구하여라.}
\begin{enumerate}\label{log6}
\item
\(3^x=\frac15\)
\item
\(9^x=5\)
\end{enumerate}

%
\exam{}
\begin{enumerate}\label{log7}
\item
\(2^3=8\)을 로그를 사용하여 나타내어라.
\item
\(-2=\log_3{\frac19}\)를 \(a^x=N\)의 꼴로 나타내어라.
\end{enumerate}
\ans{(1)\;\(3=\log_28\)\qquad(2)\;\(3^{-2}=\frac19\)}

%
\prob{다음 등식을 로그를 사용하여 나타내어라.}\label{log8}
\par\noindent
\begin{enumerate*}[itemjoin=\tabto{.5\textwidth}]
\item
\(5^3=125\)
\item
\(3^{\frac12}=\sqrt3\)
\end{enumerate*}

%
\prob{다음 등식을 \(a^x=N\)의 꼴로 나타내어라.}\label{log9}
\\
\begin{enumerate*}[itemjoin=\tabto{.5\textwidth}]
\item
\(\log_31=0\)
\item
\(\log_{\frac13}9=-2\)
\end{enumerate*}

\bigskip\bigskip
%
\exam{\(\log_{\frac13}27\)의 값을 구하여라.}\label{log10}
\begin{mdframed}
\(x=\log_{\frac13}27\)로 두면 \((\frac13)^x=27\)이다.\\
\((3^{-1})^x=3^3\), \(3^{-x}=3^3\)에서 \(-x=3\), \(x=-3\)이다.
\end{mdframed}
\ans{\(-3\)}

%
\prob{다음 값을 구하여라.}\label{log11}
\\
\begin{enumerate*}[itemjoin=\qquad\qquad\qquad]
\item
\(\log_525\)
\item
\(\log_7{\frac1{49}}\)
\item
\(\log_{\frac14}8\)
\end{enumerate*}

\newpage
%
\exam{다음 등식을 만족시키는 \(N\), \(a\)의 값을 각각 구하여라.}\label{log12}
\\
\begin{enumerate*}[itemjoin=\tabto{.5\textwidth}]
\item
\(\log_2N=-5\)
\item
\(\log_a81=4\)
\end{enumerate*}
\bigskip
\begin{mdframed}
\begin{enumerate}
\item
\(N=2^{-5}\)이므로 \(N=\frac1{32}\).
\item
\(a^4=81\)이다.
\(a^4-81=0\), \((a+3)(a-3)(a+3i)(a-3i)=0\)에서 \(a=\pm3,\pm3i\)이다.
이때, \(a>0\), \(a\neq1\)이므로 \(a=3\)이다.
\end{enumerate}
\end{mdframed}
\ans{(1)\;\(N=\frac1{32}\)\qquad(2)\;\(a=3\)}

%
\prob{다음 빈 칸에 알맞은 수를 넣어라.}
\begin{enumerate}\label{log13}
\item
\(\log_3\square=3\)
\item
\(\log_8\square=\frac16\)
\item
\(\log_{\frac23}\square=-2\)
\item
\(\log_\square64=3\)
\item
\(\log_\square25=2\)
\item
\(\log_2(\log_3\square)=2\)
\end{enumerate}

%%%
\section{로그의 성질}
%%
%\prob{다음 값들을 구하여라.}
%\begin{enumerate*}[itemjoin=\qquad\qquad\quad]
%\item
%\(\log_33\)
%\item
%\(\log_{\frac15}{\frac15}\)
%\item
%\(\log_41\)
%\item
%\(\log_71\)
%\end{enumerate*}

%
\exam{}
\begin{enumerate}\label{prop1}
\item
\(3^0=1\)이고 \((\frac15)^0=1\)이다.
따라서
\begin{center}
\fbox{\(\log_31=0\)}
\qquad
\fbox{\(\log_{\frac15}1=0\)}
\end{center}
한편, \(3^1=3\)이고 \((\frac15)^1=\frac15\)이다.
따라서
\begin{center}
\fbox{\(\log_33=1\)}
\qquad
\fbox{\(\log_{\frac15}{\frac15}=1\)}
\end{center}
\item
\(\log_28=3\), \(\log_24=2\), \(\log_232=5\)이다.
\(3+2=5\)에서
\begin{mdframed}[rightmargin=.2\textwidth,leftmargin=.2\textwidth]
\centering
\(\log_28+\log_24=\log_232\)
\end{mdframed}
\item
또, \(\log_22=1\)이므로 \(3-2=1\)에서
\begin{mdframed}[rightmargin=.2\textwidth,leftmargin=.2\textwidth]
\centering
\(\log_28-\log_24=\log_22\)
\end{mdframed}
\item
\(\log_232=5\), \(\log_22=1\)이므로
\begin{mdframed}[rightmargin=.2\textwidth,leftmargin=.2\textwidth]
\centering
\(\log_2{2^5}=5\log_22\)
\end{mdframed}
\end{enumerate}

\begin{mdframed}
%
\defi{로그의 성질(1)}\label{prop2}
\(a>0\), \(a\neq1\), \(M>0\), \(N>0\)일 때,
\begin{enumerate}[label=(\alph*)]
\item
\(\log_a1=0,\quad\log_aa=1\)
\item
\(\log_aM+\log_aN=\log_aMN\)
\item
\(\log_aM-\log_aN=\log_a{\frac MN}\)
\item
\(\log_a{M^k}=k\log_aM\)\quad(단, \(k\)는 실수)
\end{enumerate}
\end{mdframed}

\newpage
%
\prob{다음은 위의 정리를 증명하는 과정이다. 빈 칸을 완성하여라.}\label{prop3}
\begin{mdframed}
\begin{enumerate}[label=(\alph*)]
\item
\(a^{\scalebox{.4}{\fbox{(가)}}}=1\)이고 \(a^1=a\)이다.
따라서
\(\log_a1=\fbox{(가)}\)이고 \(\log_aa=1\)이다.
\item
\(\log_aM=m\), \(\log_aN=n\)이라고 하면,
\[a^m=M,\quad a^n=N\tag{$*$}\]
이다.
두 식을 곱하면
\[MN=a^m\times a^n=a^{m+n}\]
이다.
따라서
\[m+n=\log_a{MN}\]
이고, (b)가 성립한다.
\item
($*$)의 두 식을 나누면
\[\frac MN=\frac{a^m}{a^n}=a^{\scalebox{.4}{\fbox{(나)}}}\]
이다.
따라서
\[{\fbox{(나)}}=\log_a{\frac MN}\]
이고, (c)가 성립한다.
\item
($*$)의 첫번째식의 양변을 \(k\)제곱하면,
\begin{gather*}
(a^m)^k=M^k\\
a^{\scalebox{.4}{\fbox{(다)}}}=M^k
\end{gather*}
이다.
따라서
\(\fbox{(다)}=\log_a{M^k}\)이고 (d)가 성립한다.
\end{enumerate}
\end{mdframed}
\newpage

%
\exam{다음 식을 간단히 하시오.}
\begin{enumerate}\label{prop4}
\item
\(\log_612+\log_63\)
\item
\(\log_3\sqrt{27}\)
\item
\(\log_23-2\log_2\sqrt6\)
\item
\(\log_{10}\sqrt5+\frac12\log_{10}2\)
\end{enumerate}
\begin{mdframed}[innerleftmargin=0pt]
\begin{enumerate}
\item
\(\log_612+\log_63
\stackrel{(b)}=
\log_636=2\)
\item
\(\log_3\sqrt{27}=\log_3{3^{\frac32}}
\stackrel{(d)}=
\frac32\log_33=\frac32\)
\item
\(\log_23-2\log_2\sqrt6
\stackrel{(d)}=
\log_23-\log_26
\stackrel{(b)}=
\log_2\frac12=-1\)
\item
\(\log_{10}\sqrt5+\frac12\log_{10}2
\stackrel{(d)}=
\log_{10}\sqrt5+\log_{10}\sqrt2
\stackrel{(b)}=
\log_{10}\sqrt{10}=\frac12\)
\end{enumerate}
\end{mdframed}

%
\prob{다음 식을 간단히 하시오.}
\begin{enumerate}\label{prop5}
\item
\(\log_3\sqrt[3]{81}\)
\item
\(\log_798-\log_72\)
\item
\(\log_{\frac23}27-\log_{\frac23}8\)
\item
\(\log_3{\frac{\sqrt3}5}+\log_345\)
\item
\(\log_212+\log_26-2\log_23\)
\item
\(\frac12\log_3\frac95+\log_3\sqrt5\)
\end{enumerate}

%
\prob{\(\log_52=a\), \(\log_53=b\)라고 할 때, 다음을 \(a\)와 \(b\)로 나타내어라.}
\begin{enumerate}\label{prop6}
\item
\(\log_518=\log_5(2\times3^2)=\log_52+2\log_53=a+2b\)
\item
\(\log_5{60}\)
\item
\(\log_5\frac89\)
%\item
%\(\log_50.25\)
\item
\(\log_5\sqrt{1000}\)
\end{enumerate}

\newpage
%
\prob{다음 값들을 구하여라.}\label{prop7}
\vspace{-10pt}\par\noindent
\begin{enumerate*}[itemjoin=\qquad\qquad\qquad]
\item
\(\log_{16}256\)
\item
\(\frac{\log_2256}{\log_216}\)
\item
\(\frac{\log_4256}{\log_416}\)
\end{enumerate*}

%
\exam{}
\begin{enumerate}\label{prop8}
\item
위의 문제에서의 세 값은 서로 같다.
즉,
\begin{center}
\fbox{\(\displaystyle\log_{16}256=\frac{\log_2256}{\log_216}\)}
\qquad
\fbox{\(\displaystyle\log_{16}256=\frac{\log_4256}{\log_416}\)}
\end{center}
이다.
\(\log_{16}256\)의 밑은 \(16\)이었지만, 밑을 2로 변환할 수 있고 4로도 변환할 수 있는 것이다.
일반적으로는 다음 공식이 성립한다.
\[\fbox{\(\displaystyle\log_ab=\frac{\log_cb}{\log_ca}\)}\tag{$*$}\]
이것을 `밑의 변환 공식'이라고 부른다.
\item
밑의 변환 공식을 이용하여 \(\log_23\)과 \(\log_32\)를 곱해보자.\\
밑을 \(5\)로 통일하면,
\[\log_23\stackrel{(*)}=\frac{\log_53}{\log_52},\quad\log_32\stackrel{(*)}=\frac{\log_52}{\log_53}\]
이다.
따라서
\[\log_23\times\log_32=\frac{\log_53}{\log_52}\times\frac{\log_52}{\log_53}=1\]
즉 \(\log_23\)과 \(\log_32\)는 서로 역수관계이다.
%\begin{mdframed}[innertopmargin=5pt,innerbottommargin=5pt,rightmargin=.3\textwidth,leftmargin=.3\textwidth]
%\centering
%\(\displaystyle\log_23=\frac1{\log_32}\)
%\end{mdframed}
\begin{center}
\fbox{\(\displaystyle\log_23=\frac1{\log_32}\)}
\end{center}
\item
한편 밑의 변환 공식을 두 번 적용하여 다음과 같은 계산을 할 수도 있다.
\[\log_825\stackrel{(*)}=\frac{\log_725}{\log_78}=\frac{2\log_75}{3\log_72}\stackrel{(*)}=\frac23\log_25\]
따라서
%\begin{mdframed}[rightmargin=.3\textwidth,leftmargin=.3\textwidth]
%\centering
%\(\log_{2^3}{5^2}=\frac23\log_25\)
%\end{mdframed}
\begin{center}
\fbox{\(\log_{2^3}{5^2}=\frac23\log_25\)}
\end{center}
\end{enumerate}

\begin{mdframed}
%
\theo{로그의 성질(2)}
\begin{enumerate}[label=(\alph*)]\label{prop9}
\setcounter{enumi}{4}
\item
\(a>0\), \(a\neq1\), \(b>0\), \(c>0\), \(c\neq1\)일 때,
\[\phantom{\qquad(\text{밑의 변환 공식})}\log_ab=\frac{\log_cb}{\log_ca}\qquad(\text{밑의 변환 공식})\]
\item
\(a>0\), \(a\neq1\), \(b>0\), \(b\neq1\)일 때,
\[\log_ab=\frac1{\log_ba}\]
\item
\(a>0\), \(a\neq1\), \(b>0\)이고 \(m\), \(n\)이 실수일 때,
\[\log_{a^n}{b^m}=\frac mn \log_ab\]
\end{enumerate}
\end{mdframed}

%
\exam{다음 식을 간단히 하시오.}\label{prop10}
\\[-5pt]
\begin{enumerate*}[itemjoin=\tabto{.5\textwidth}]
\item
\(\frac1{\log_26}+\frac1{\log_36}\)
\item
\(\log_927\)
\end{enumerate*}
\begin{mdframed}
\begin{enumerate}
\item
\(\frac1{\log_26}+\frac1{\log_36}
\stackrel{(f)}=\log_62+\log_63
\stackrel{(b)}=\log_66=1\)
\item
\(\log_927=\log_{3^2}{3^3}
\stackrel{(g)}=\frac32\log_33\)
\end{enumerate}
\end{mdframed}
\ans{(1)\;1\quad(2)\;\(\frac32\)}

%
\prob{다음 식을 간단히 하시오.}
\begin{enumerate}\label{prop11}
\item
\(\log_8\frac1{16}\)
\item
\(\log_220-\frac1{\log_52}\)
\item
\(\log_52\cdot\log_225\)
\item
\(\log_45\cdot\log_56\cdot\log_64\)
\item
\((\log_38+\log_94)\cdot\log_23\)
\item
\(\log_2(\log_23)+\log_2(\log_34)\)
\end{enumerate}

\newpage
%
\prob{\(\log_72=a\), \(\log_73=b\)라고 할 때, 다음을 \(a\)와 \(b\)로 나타내어라.}
\begin{enumerate}\label{prop12}
\item
\(\log_23\)
\item
\(\log_6{28}\)
\end{enumerate}

%
\prob{다음은 밑의 변환 공식를 증명하는 과정이다. 빈칸을 완성하여라.}\label{prop13}
\begin{mdframed}[nobreak=false]
\(\fbox{(가)}=x\), \(\fbox{(나)}=y\)라고 하면
\[a^x=b,\qquad c^y=a\]
이다.
지수의 성질에 따라
\[b=a^x=(c^y)^x=c^{\scalebox{.4}{\fbox{(다)}}}\]
이다.
그러면 로그의 정의에 의해
\[\fbox{(다)}=\log_cb\]
이다.
따라서
\[\fbox{(가)}\times\fbox{(나)}=\log_cb\]
이다.
그런데 \(a\neq1\)로부터 \(\fbox{(나)}\neq0\)이므로 위의 식의 양변을 \(\fbox{(나)}\)로 나누면
\[\log_ab=\frac{\log_cb}{\log_ca}\]
이 얻어진다.
\end{mdframed}

\newpage
%
\exam{}
\begin{enumerate}\label{prop14}
\item
\(2^x=5\)라고 하자.
그러면 \(x=\log_25\)이다.
두 번째 식의 \(x\)를 첫 번째 식의 \(x\)에 대입하면
\begin{center}
\fbox{\(\displaystyle2^{\log_25}=5\)}
\end{center}
이다.
\item
지수와 로그의 성질을 사용하여 \(\displaystyle8^{\log_32}\)를 계산하면
\[8^{\log_32}=(2^3)^{\log_32}=2^{3\log_32}=2^{\log_38}\]
따라서
\begin{center}
\fbox{\(\displaystyle8^{\log_32}=2^{\log_38}\)}
\end{center}
이다.
즉 \(\displaystyle8^{\log_32}\)에서 8과 2의 위치를 바꿀 수 있다.
\end{enumerate}

\begin{mdframed}
%
\theo{로그의 성질(3)}\label{prop15}
\(a\), \(b\), \(c\)가 1이 아닌 양수일 때,
\begin{enumerate}[label=(\alph*)]
\setcounter{enumi}{7}
\item
\(\displaystyle a^{\log_ab}=b\)
\item
\(\displaystyle a^{\log_bc}=c^{\log_ba}\)
\end{enumerate}
\end{mdframed}

\newpage
%
\prob{다음은 위의 정리를 증명하는 과정이다. 빈칸을 완성하여라.}\label{prop16}
\begin{mdframed}
먼저 (i)를 증명하자.
\(a^{\log_bc}=x\)라 하면 로그의 정의에 의해
\[\log_bc=\fbox{(가)}\tag{$*$}\]
($*$)의 양변을 각각 \(c\)를 밑으로 하는 로그로 바꾸면\\
\(\displaystyle\log_bc=\frac1{\fbox{(나)}}\), \(\displaystyle\fbox{(가)}=\frac{\log_cx}{\fbox{(다)}}\)이므로
\[\frac1{\fbox{(나)}}=\frac{\log_cx}{\fbox{(다)}}\]
따라서
\[\log_cx=\frac{\fbox{(다)}}{\fbox{(나)}}=\log_ba\]
로그의 정의에 의해
\[x=c^{\log_ba}\]
그러므로
\[a^{\log_bc}=c^{\log_ba}\]
가 성립한다.

\bigskip
(i)의 식에 \(b\) 대신 \fbox{(라)}를, \(c\)대신 \fbox{(마)}를 대입하면 (h)가 나온다.
\end{mdframed}

%
\prob{다음 식을 간단히 하시오.}
\begin{enumerate}\label{prop17}
\item
\(3^{\log_310}+7^{\log_72}\)
\item
\(27^{\log_35}\)
\end{enumerate}

%%%
\section{상용로그}

밑이 \(10\)인 로그를 \fbox{상용로그}라고 한다.
양수 \(N\)의 상용로그 \(\log_{10}N\)은 보통 밑을 생략하여
\[\log N\]
과 같이 나타낸다.

%
\prob{다음 상용로그의 값을 구하시오.}\label{common1}
\vspace{-10pt}\par\noindent
\begin{enumerate*}[itemjoin=\qquad\qquad\qquad]
\item
\(\log100\)
\item
\(\log\sqrt{10}\)
\item
\(\log\frac1{1000}\)
\end{enumerate*}

\bigskip\bigskip
%이 소책자의 맨 뒤 수록된 
상용로그표란 \(1.00\)부터 \(9.99\)까지의 수에 대한 상용로그의 값을 표시해놓은 표이다.
%이
이 표를 사용하면 유효숫자가 최대 세 자리인 모든 양수 \(N\)에 대한 상용로그의 값을 계산할 수 있다.

%
\exam{상용로그표를 이용하여 다음 값들을 구하여라.}\label{common2}
\begin{enumerate*}[itemjoin=\tabto{.5\textwidth}]
\item
\(\log 3.12\)
\item
\(\log 312\)
\end{enumerate*}
\begin{mdframed}[nobreak=false]
\begin{enumerate}
\item
상용로그표에서 \(\log3.12\)의 값을 구하려면 3.1의 가로줄과 2의 세로줄이 만나는 곳에 있는 \(.4942\)를 찾으면 된다.
즉 \(\log 3.12=0.4942\)이다.
\begin{center}
\begin{tabu}to.5\textwidth{X[c]|X[c]X[c]X[c]X[c]}
수	&0		&1		&2		&3\\\hline
1.0	&.0000	&.0043	&.0086	&.0128\\
1.1	&.0414	&.0453	&.0492	&.0531\\
\(\vdots\)&\(\vdots\) 	&\(\vdots\)	&\(\vdots\)	&\(\vdots\)\\
3.1	&.4914	&.4928	&\fbox{.4942}	&.4955\\
3.2	&.5051	&.5065	&.5079	&.5092
\end{tabu}
\end{center}
\item
\(\log312=\log(3.12\times100)=\log3.12+\log100=0.4942+2=2.4942\)
\end{enumerate}
\end{mdframed}

%
\prob{상용로그표를 이용하여 다음 값들을 구하여라.}\label{common3}
\vspace{-10pt}\par\noindent
\begin{enumerate*}[itemjoin=\qquad\quad]
\item
\(\log8.02\)
\item
\(\log8020\)
\item
\(\log0.00802\)
\item
\(\log\sqrt[3]{8.02}\)
\end{enumerate*}

%
\prob{상용로그표를 이용하여 다음 값들을 구하여라}\label{common4}
\vspace{-10pt}\par\noindent
\begin{enumerate*}[itemjoin=\tabto{.5\textwidth}]
\item
\(\log41\)
\item
\(\log0.007\)
\end{enumerate*}

%
\exam{실수 \(x\)의 정수부분을 \([x]\), 소수부분을 \(\langle x\rangle\)이라고 하자.
이때, 다음 값의 정수부분과 소수부분을 각각 계산하여라.}\label{common5}
\vspace{-10pt}\par\noindent
\begin{enumerate*}[itemjoin=\tabto{.5\textwidth}]
\item
\(\log3240\)
\item
\(\log0.324\)
\end{enumerate*}
\begin{mdframed}[nobreak=false]
\(\log3.24=0.5105\)이다.
따라서
\begin{enumerate}
\item
\(\log3240=3+\log3.24=3.5105\)
그러므로
\[[\log3240]=3,\qquad\langle\log3240\rangle=0.5105\]
\item
\(\log0.324=-1+\log3.24=-0.4895\)
그러므로
\[[\log0.324]=-1,\qquad\langle\log0.324\rangle=0.5105\]
\end{enumerate}
\end{mdframed}

%
\prob{다음 값의 정수부분과 소수부분을 각각 계산하여라.}\label{common6}
\vspace{-10pt}\par\noindent
\begin{enumerate*}[itemjoin=\tabto{.5\textwidth}]
\item
\(\log52000\)
\item
\(\log0.052\)
\end{enumerate*}

%
\prob{\(a=\log2\), \(b=\log3\)이라고 하자.
이때, 다음 수들을 \(a\)와 \(b\)로 각각 나타내어라.}
\begin{enumerate}\label{common7}
\item
\(\log12\)
\item
\(\log5\)
\item
\(\log_43\)
\end{enumerate}

%%%
\section{상용로그의 활용}
%
\exam{}
\begin{enumerate}\label{app1}
\item
365는 세자리수이다.
왜냐하면 365은 100 이상 1000 미만인 수이기 때문이다.
\[100\le365<1000\]
이 식의 양변에 상용로그를 취하면
\[2\le\log365<3\]
이다.
즉
\[[\log365]=2\]
인 것이다.
\begin{center}
\fbox{
\(A\)가 \(n\)자리수이다.
\qquad\(\iff\)\qquad
\([\log A]=n-1\)
}
\end{center}
\item
\(0.00032\)는 소수 넷째자리에서 처음으로 0이 아닌 숫자가 나타난다.
왜냐하면 0.00032는 0.0001 이상 0.001 미만인 수이기 때문이다.
\[0.0001\le0.00032<0.001\]
이 식의 양변에 상용로그를 취하면
\[-4\le\log0.00032<-3\]
이다.
즉
\[[\log0.00032]=-4\]
인 것이다.
\begin{center}
\fbox{
\parbox{0.43\textwidth}{\(A\)가 소수 \(n\)째자리에서 처음으로\\ 0이 아닌 수가 나타난다.}
\qquad\(\iff\)\qquad
\([\log A]=-n\)
}
\end{center}
\end{enumerate}

\newpage
%
\exam{\(\log2=0.3010\), \(\log3=0.4771\)로 계산할 때,}
\begin{enumerate}\label{app2}
\item
\(6^{100}\)은 몇 자리수인가?
\item
\(5^{-30}\)는 소수 몇째 자리에서 처음으로 0이 아닌 숫자가 나타나는가?
\end{enumerate}
\begin{mdframed}
\begin{enumerate}
\item
\(6^{100}\)에 상용로그를 취하면
\[\log6^{100}=100\log6=100(\log2+\log3)=100(0.3010+0.4771)=77.81\]
따라서 \([\log6^{100}]=77\)이고 \(6^{100}\)은 78자리 수이다.
\item
\(5^{-30}\)에 상용로그를 취하면
\[\log5^{-30}=-30\log5=-30(1-\log2)=-30(1-0.3010)=-20.97\]
따라서 \([\log5^{-30}]=-21\)이고 \(5^{-30}\)은 소수 21번째 자리에서 처음으로 0이 아닌 숫자가 나타난다.
\end{enumerate}
\end{mdframed}

%
\prob{\(\log2=0.3010\), \(\log3=0.4771\)로 계산할 때,}
\begin{enumerate}\label{app3}
\item
\(2^{100}\times3^{10}\)은 몇 자리수인가?
\item
\(3^{-20}\)는 소수 몇째 자리에서 처음으로 0이 아닌 숫자가 나타나는가?
\end{enumerate}


%%%
\section*{답}
\addcontentsline{toc}{chapter}{\protect\numberline{*}답}
\begin{multicols*}{2}

%%
\anssec{1. 로그의 뜻}
%
\an{log1}
\begin{enumerate*}[itemjoin=\qquad]
\item
3
\item
\(-1\)
\item
0
\item
\(\frac32\)
\end{enumerate*}

%
\an{log3}
\(x=\log_35\approx1.46\)

%
\an{log6}
\begin{enumerate*}[itemjoin=\qquad]
\item
\(x\approx-1.46\)
\item
\(x\approx1.23\)
\end{enumerate*}

%
\an{log8}
\begin{enumerate*}[itemjoin=\qquad]
\item
\(3=\log_5125\)
\item
\(\frac12=\log_3\sqrt3\)
\end{enumerate*}

%
\an{log9}
\begin{enumerate*}[itemjoin=\qquad]
\item
\(3^0=1\)
\item
\((\frac13)^{-2}=9\)
\end{enumerate*}

%
\an{log11}
\begin{enumerate*}[itemjoin=\qquad]
\item
2
\item
\(-2\)
\item
\(-\frac32\)
\end{enumerate*}

\an{log13}\\[-15pt]
(1)\:27
\tabto{.33\columnwidth}
(2)\:\(\sqrt2\)
\tabto{.66\columnwidth}
(3)\:\(\frac94\)
\par\noindent
(4)\:4
\tabto{.33\columnwidth}
(5)\:5
\tabto{.66\columnwidth}
(6)\:81

%%
\anssec{2. 로그의 성질}
%
\an{prop3}\\[-10pt]
\((가)=0\)\\
\((나)=m-n\)\\
\((다)=km\)

%
\an{prop5}\\[-15pt]
(1)\:\(\frac43\)
\tabto{.33\columnwidth}
(2)\:2
\tabto{.66\columnwidth}
(3)\:\(-3\)
\par\noindent
(4)\:\(\frac52\)
\tabto{.33\columnwidth}
(5)\:3
\tabto{.66\columnwidth}
(6)\:1

\columnbreak
%
\an{prop6}
\begin{enumerate}
\setcounter{enumi}1
\item
\(2a+b+1\)
\item
\(3a-2b\)
\item
\(\frac32a+\frac32\)
\end{enumerate}

%
\an{prop7}
\begin{enumerate*}[itemjoin=\qquad]
\item
2
\item
2
\item
2
\end{enumerate*}

%
\an{prop11}\\[-15pt]
(1)\:\(-\frac43\)
\tabto{.33\columnwidth}
(2)\:2
\tabto{.66\columnwidth}
(3)\:2
\par\noindent
(4)\:1
\tabto{.33\columnwidth}
(5)\:4
\tabto{.66\columnwidth}
(6)\:1

%
\an{prop12}
\begin{enumerate*}[itemjoin=\qquad]
\item
\(\frac ba\)
\item
\(\frac{2a+1}{a+b}\)
\end{enumerate*}

%
\an{prop13}
\\
\((가)=\log_ab\)\\
\((나)=\log_ca\)\\
\((다)=xy\)

%
\an{prop16}
\\
\((가)=\log_ax\)\\
\((나)=\log_cb\)\\
\((나)=\log_ca\)\\
\((라)=a\)\\
\((마)=b\)
%\(\log_ax\), \(\log_cb\), \(\log_ax\), \(\log_ca\), \(\log_cb\), \(\log_ca\), \(\log_ca\), \(\log_cb\), \(a\), \(b\)

%
\an{prop17}
\begin{enumerate*}[itemjoin=\qquad\qquad]
\item
12
\item
125
\end{enumerate*}

\columnbreak
%%
\anssec{3. 상용로그}

%
\an{common1}
\begin{enumerate*}[itemjoin=\qquad\qquad]
\item
2
\item
\(\frac12\)
\item
\(-3\)
\end{enumerate*}

%
\an{common3}
\begin{enumerate}
\item
0.9042
\item
3.9042
\item
-2.0958
\item
0.3014
\end{enumerate}

%
\an{common4}
\begin{enumerate*}[itemjoin=\tabto{.5\columnwidth}]
\item
1.6128
\item
-2.1549
\end{enumerate*}

%
\an{common6}
\begin{enumerate}
\item
정수부분 : 4\\
소수부분 : 0.7160
\item
정수부분 : -2\\
소수부분 : 0.7160
\end{enumerate}

%
\an{common7}
\begin{enumerate}
\item
\(2a+b\)
\item
\(1-a\)
\item
\(\frac b{2a}\)
\end{enumerate}

\columnbreak
%%
\anssec{4. 상용로그의 활용}

%
\an{app3}
\begin{enumerate*}[itemjoin=\qquad\qquad]
\item
35
\item
소수 10째자리
\end{enumerate*}
\end{multicols*}


%
\section*{요약}
\addcontentsline{toc}{chapter}{\protect\numberline{*}요약}
\begin{enumerate}[label=\arabic*.,itemsep=40pt]
\item
로그의 정의\\
\(a>0\), \(a\neq1\), \(N>0\)일 때,
\[a^x=N\qquad\iff\qquad x=\log_aN\]
\item
로그의 성질
\begin{enumerate}[itemsep=10pt]
\item
\(\log_a1=0,\quad\log_aa=1\)
\item
\(\log_aM+\log_aN=\log_aMN\)
\item
\(\displaystyle\log_aM-\log_aN=\log_a{\frac MN}\)
\item
\(\log_a{M^k}=k\log_aM\)
\item
\(\displaystyle\log_ab=\frac{\log_cb}{\log_ca}\qquad(\text{밑의 변환 공식})\)
\item
\(\displaystyle\log_ab=\frac1{\log_ba}\)
\item
\(\displaystyle\log_{a^n}{b^m}=\frac mn \log_ab\)
\item
\(\displaystyle a^{\log_ab}=b\)
\item
\(\displaystyle a^{\log_bc}=c^{\log_ba}\)
\end{enumerate}
\item
상용로그
\[\log N=\log_{10}N\]
\item
상용로그의 활용 : 자릿수문제
\end{enumerate}
\end{document}