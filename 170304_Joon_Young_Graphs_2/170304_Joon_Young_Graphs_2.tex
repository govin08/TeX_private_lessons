\documentclass[a4paper]{oblivoir}
\usepackage{amsmath,amssymb,kotex,kswrapfig,mdframed,tabto,paralist}
\usepackage{fapapersize}
\usefapapersize{210mm,297mm,10mm,*,10mm,*}

%%% Counters
\newcounter{num}

%%% Commands
\newcommand\defi[1]
{\bigskip\par\noindent\stepcounter{num} \textbf{정의 \thenum) #1}\par\noindent}
\newcommand\theo[1]
{\bigskip\par\noindent\stepcounter{num} \textbf{정리 \thenum) #1}\par\noindent}
\newcommand\exam[1]
{\bigskip\par\noindent\stepcounter{num} \textbf{예시 \thenum) #1}\par\noindent}
\newcommand\prob[1]
{\bigskip\par\noindent\stepcounter{num} \textbf{문제 \thenum) #1}\par\noindent}

\newcommand\pb[1]{\ensuremath{\fbox{\phantom{#1}}}}

\newcommand\ba{\ensuremath{\:|\:}}

\newcommand\procedure[1]{\begin{mdframed}\vspace{#1\textheight}\end{mdframed}\bigskip}

\newcommand\an[1]{\bigskip\par\noindent\textbf{문제 #1)}\par\noindent}

%%% Meta Commands
\let\oldsection\section
\renewcommand\section{\clearpage\oldsection}

\let\emph\textsf

%%% Title
\title{준영, 절댓값을 포함한 함수 그래프 그리기(2)}
\date{\today}
\author{}

\begin{document}
\maketitle

다음 그래프를 그리시오.
\begin{enumerate}[(1)]
\item
\(y=|2x|\)
\item
\(y=|x-2|\)
\item
\(y=|x-2|-2\)
\item
\(y=-|x|\)
\item
\(y=|x|+|x-3|\)
\item
\(y=|x+3|+|x-1|\)
\item
\(y=|x^2-4|\)
\item
\(y=|x^2-4x+3|\)
\end{enumerate}
\clearpage
\begin{figure*}
\noindent\includegraphics[width=0.47\textwidth]{55}
\hspace{30pt}\includegraphics[width=0.47\textwidth]{55}
\par\vspace{40pt}
\noindent\includegraphics[width=0.47\textwidth]{55}
\hspace{30pt}\includegraphics[width=0.47\textwidth]{55}
\end{figure*}

\clearpage
\begin{figure*}
\noindent\includegraphics[width=0.47\textwidth]{55}
\hspace{30pt}\includegraphics[width=0.47\textwidth]{55}
\par\vspace{40pt}
\noindent\includegraphics[width=0.47\textwidth]{55}
\hspace{30pt}\includegraphics[width=0.47\textwidth]{55}
\end{figure*}

\end{document}