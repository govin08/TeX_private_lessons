\documentclass{article}
\usepackage{amsmath,amssymb,amsthm,mdframed,kotex,paralist}
\usepackage{tabto}
%\TabPositions{0.5\textwidth}
\TabPositions{0.33\textwidth,0.66\textwidth}
\newcommand\bp[1]{\begin{mdframed}[frametitle={#1},skipabove=10pt,skipbelow=20pt,innertopmargin=5pt,innerbottommargin=40pt]}
\newcommand\ep{\end{mdframed}\par}
\newcommand\ov[1]{\ensuremath{\overline{#1}}}
\newcommand{\vs}{\vspace{0.05\textheight}}
\newcommand{\vvs}{\vspace{0.1\textheight}}
\newcommand{\vvvs}{\vspace{0.2\textheight}}

\begin{document}
\title{영승00 : 문제들}
\author{}
\date{\today}
\maketitle

%%
\section{다항식의 연산}
\bp{01}
\(x^2+y^2=7\), \(x+y=3\)일 때, 다음 식의 값을 구하여라.\\
(1) \(xy\),\quad (2) \(x-y\),\quad (3) \(x^2-y^2\),\quad (4) \(x^3+y^3\)
\vvs\ep

\bp{02}
모든 모서리의 길이의 합이 \(24\)이고, 겉넓이가 \(20\)인 직육면체가 있다.
이 직육면체의 대각선의 길이는?
\vvs\vs\ep

%%
\section{항등식과 나머지 정리}
\bp{03}
다음 등식이 \(x\)에 대한 항등식일 때, 상수 \(a\), \(b\), \(c\)의 값을 각각 구하여라.\\
(1) \((a+1)x+b-3=0\)\\
(2) \(3x^2-ax+4=bx^2+5x-c\)\\
(3) \(x^2+3=ax+bx(x-1)+c\)
\vvvs\vs\ep

%%
\bp{04}
다항식 \(x^3+ax^2+bx-2\)를 \(x-1\)로 나눈 나머지는 \(6\)이고, \(x+1\)로 나누면 나누어떨어진다.
이때, 이 다항식을 \(x+2\)로 나눈 나머지를 구하여라.
\vvvs\ep

%%
\section{인수분해}
\bp{05}
다음 식을 인수분해하여라.\\
(1) \(x^2+3x+2\)\\
(2) \(x^3+8\)\\
(3) \((x^2+5x+4)(x^2+5x+6)-8\)\\
(4) \(x^4-3x^2+1\)\\
(5) \(x^3-7x+6\)
\vvvs\vs\ep

\bp{06}
삼각형의 세 변의 길이 \(a\), \(b\), \(c\)에 대하여
\(a^3-a^2b+ab^2+ac^2-b^3-bc^2=0\)
이 성립할 때, 이 삼각형은 어떤 삼각형인지 말하여라.
\vvvs\ep

%%
\section{복소수와 이차방정식}
\bp{07}
\(\frac x{1-i}+\frac y{1+i}=(2-i)(i+3)\)을 만족하는 실수 \(x\), \(y\)의 값을 구하여라.
\vs\ep

\bp{08}
다음 이차방정식을 풀어라.\\
(1) \(x^2+3x-18=0\)\\
(2) \(x^2+4=0\)\\
(3) \(x^2+x-4=0\)\\
(4) \(8x^2-8x+2=0\)\\
(5) \(x^2-2x+2=0\)
\vvs\vs\ep

\bp{09}
두 양수 \(x\), \(y\)에 대하여 \(x+y=8\)일 때, \(xy\)의 최댓값은?
\vs\ep


%%
\section{고차방정식과 연립방정식}
\bp{10}
다음 방정식을 풀어라.\\
(1) \(x^4-2x^2-3=0\)\\
(2) \(x^3-4x^2+x+6=0\)
\vs\ep

\bp{11}
다음 연립방정식을 풀어라.\\
(1)
\(\begin{cases}
x-y+z=1\\
x+2y-z=2\\
-2x+y-3z=-4
\end{cases}\)\par\bigskip\noindent
(2)
\(\begin{cases}
x-y=1\\
x^2+y^2=5
\end{cases}\)
\vvs\ep


\bp{12}
방정식 \(xy-x-2y-3=0\)을 만족시키는 정수 \(x\), \(y\)를 모두 구하여라.
\vspace{0.03\textheight}\ep
\end{document}