\documentclass{oblivoir}
\usepackage{amsmath,amssymb,amsthm,kotex,mdframed,paralist,kswrapfig}

\newcounter{num}
\newcommand{\prob}
{\bigskip\noindent\refstepcounter{num}\textbf{문제 \arabic{num})}\par}

\newcommand{\ans}{{\raggedleft\textbf{답 : (\qquad\qquad\qquad\qquad\qquad\qquad)}
\par}\bigskip\bigskip}


%%%
\begin{document}
\Large

\title{승재 20 - 6학년 2학기 - 13}
\author{}
\date{\today}
\maketitle
%\tableofcontents

\prob
어떤 일을 마치는 데 도준이는 8시간 동안 해야 하고 효주는 10시간 동안 일을 해야 합니다.
도준이가 6시간 동안 일한 양만큼 효주가 일을 하려면 효주는 도준이보다 몇 시간 일을 더 해야 합니까?

\prob
어떤 일을 마치는 데 도준이는 12시간 동안 해야 하고 효주는 15시간 동안 일을 해야 합니다.
도준이가 8시간 동안 일한 양만큼 효주가 일을 하려면 효주는 도준이보다 몇 시간 일을 더 해야 합니까?

\prob
어떤 일을 마치는 데 도준이는 4시간 동안 해야 하고 효주는 3시간 동안 일을 해야 합니다.
도준이가 18시간 동안 일한 양만큼 효주가 일을 하려면 도준이는 효주보다 몇 시간 일을 더 해야 합니까?

\prob
100m를 달리는 데 지우는 16초가 걸리고 시우는 20초가 걸립니다.
지우가 20초 동안 달린 거리만큼 시우가 달리려면 시우는 지우보다 몇 초 더 달려야 합니까?

\prob
피자 한 판을 먹는데 진수 10분이 걸리고 윤진이는 12분이 걸립니다.
진수가 8분동안 먹은 만큼 윤진이가 먹으려면 윤진이는 진수보다 몇 분 더 먹어야 합니까?

\prob
농도가 서로 다른 두 소금물 (가)와 (나)가 있습니다.
100g의 소금물 (가)와 (나)를 증발시키면 각각 3g과 2g의 소금을 얻습니다.
이제, 같은 양의 소금물 (가)와 (나)를 증발시키는데 (가)에서는 7.5g의 소금을 얻었습니다.
그렇다면 (나)에서는 몇 g의 소금을 얻습니까?

\end{document}