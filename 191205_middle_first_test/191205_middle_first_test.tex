\documentclass[b4paper]{article}
\usepackage{amsmath,amssymb,kotex,paralist,graphicx,tabu}
\usepackage[margin=0.5in]{geometry}
\usepackage{multicol}
\setlength{\columnseprule}{0.4pt}

\usepackage[inline]{enumitem}
\setlist[enumerate,1]{label=(\arabic*)}

%%% Counters
\newcounter{num}
%\newcounter{answer}

%%% Commands
\newcommand\prob
{\par\bigskip\noindent\refstepcounter{num} \textbf{문제 \thenum) }}
\newcommand\ans[1]
{\begin{flushright}답 : #1\end{flushright}}

%\let\emph\textsf

\begin{document}
%\begin{center}
%\LARGE
%중학교 1학년 수학 문제들
%\end{center}
\begin{flushright}
\quad중학교 \qquad학년 \qquad반
\end{flushright}

\bigskip
\begin{multicols}{2}

%
\prob
두 수 \(48\), \(60\)의 최대공약수를 구하여라.
\begin{enumerate}
\item
48을 소인수분해하면 :
\(48=2^4\times3\)
\item
60을 소인수분해하면 :
\(60=2^2\times 3\times 5\)
\item
최대공약수\(=2^\times3=12\)
\end{enumerate}
\ans{12}

%
\prob
135의 약수의 개수를 구하여라.
\begin{enumerate}
\item
135를 소인수분해하면 : \(135=3^3\times5\)
\item
135의 약수들 : 1, 3, 9, 27, 5, 15, 45, 135
\item
135의 약수의 개수 : 8개
\end{enumerate}
\ans{8개}

%
\prob
다음 중 가장 큰 수와 가장 작은 수의 합은?
\[
\left(-\frac23\right)^2,
\quad
-\left(\frac23\right)^2,
\quad
-\frac{2^3}3,
\quad
\frac2{(-3)^3},
\quad
\frac{(-2)^2}3
\]
\begin{enumerate}
\item
다섯 개의 수를 각각 계산하면
\[
\frac49,
\quad
-\frac49,
\quad
-\frac83,
\quad
-\frac2{27},
\quad
\frac43
\]
\item
가장 큰 수는 \(\frac43\)이고 가장 작은 수는 \(-\frac83\)이다.
\item
\(\frac43+(-\frac83)=-\frac43\)
\end{enumerate}
\ans{\(-\frac43\)}

%
\prob
농도가 9\%인 소금물 200g과 농도가 4\%인 소금물 300g을 섞을 때 혼합된 소금물의 농도는?
\begin{enumerate}
\item
첫번째 소금물 속의 소금의 양은
\(200\text g\times\frac9{100}=18\text g\)
\item
두번째 소금물 속의 소금의 양은
\(300\text g\times\frac4{100}=12\text g\)
\item
혼합된 소금물 속의 소금의 양은
\(18\text g+12\text g=30\text g\)
\item
혼합된 소금물의 농도는
\(\frac{30}{200+300}\times100=6\%\)
\end{enumerate}
\ans{6\%}

%
\prob
농도가 6\%인 소금물 200g에 몇 g의 물을 넣으면 농도가 4\%인 소금물이 되는가?
\begin{enumerate}
\item
넣는 물의 양을 \(x\) g이라고 하자.
\item
처음 소금물 속의 소금의 양은
\(200\text g\times\frac6{100}=12\text g\)
\item
\(x\)에 관해서 식을 세우고 방정식을 풀면
\begin{gather*}
\frac{12}{200+x}\times100=4\\
1200=4(200+x)\\
1200=800+4x\\
400=4x\\
x=100
\end{gather*}
\end{enumerate}
\ans{100g}

%
\prob
유진이는 A에서 B까지 500m의 거리를 5m/s의 속력으로 뛰다가 B에서 C까지의 300m의 거리를 2m/s로 걸었다.
유진이가 A에서 C까지 이동할 때의 평균 속력을 구하여라.
\begin{enumerate}
\item
\(A\)에서 \(B\)까지 이동할 때 걸린 시간은 \(\frac{500\text m}{5\text{m/s}}=100\text s=100초\)
\item
\(B\)에서 \(C\)까지 이동할 때 걸린 시간은 \(\frac{300\text m}{2\text{m/s}}=150\text s=150초\)
\item
\(A\)에서 \(C\)까지 이동할 때 걸린 시간은 \(100초+150초=250초\)
\item
\(A\)에서 \(C\)까지 이동할 때의 평군 속력은
\[\frac{500\text m+300\text m}{250s}=\frac{16}5\text{m/s}=3.2\text{m/s}\]
\end{enumerate}
\ans{3.2m/s}

\vfill\null\columnbreak

%%
\setcounter{num}{0}
%
\prob
세 수 225, 105, 300의 최대공약수를 구하여라.
\vspace{50pt}
\ans{}

%
\prob
120의 약수의 개수를 구하여라.
\vspace{50pt}
\ans{}

%
\prob
다음 중 가장 큰 수와 가장 작은 수의 합은?
\[
\frac5{3^2},
\quad
\left(-\frac53\right)^2,
\quad
-\frac{5^2}3,
\quad
-\frac 5{(-3)^3},
\quad
\frac53
\]
\vspace{65pt}
\ans{}

%
\prob
농도가 4\%인 소금물 300g과 농도가 10\%인 소금물 600g을 섞을 때 혼합된 소금물의 농도는?
\vspace{65pt}
\ans{}

%
\prob
농도가 10\%인 소금물 100g에 몇 g의 소금을 넣으면 농도가 20\%인 소금물이 되겠는가?
\vspace{140pt}
\ans{}

%
\prob
준용이는 A에서 B까지 210m의 거리를 7m/s의 속력으로 뛰다가 B에서 C까지의 60m의 거리를 3m/s로 걸었다.
준용이가 A에서 C까지 이동할 때의 평균 속력을 구하여라.
\vspace{100pt}
\ans{}
\newpage
%%
%\prob
%집에서 서점까지 가는데 60km/h의 자동차로 가면 10km/h의 자전거로 가는 것보다 50분 빨리 도착한다고 한다. 그렇다면 40km/h의 오토바이로 가면 집에서 서점까지 몇 분 걸리겠는가?

\end{multicols}
\end{document}