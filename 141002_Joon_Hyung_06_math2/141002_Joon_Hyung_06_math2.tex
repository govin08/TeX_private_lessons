\documentclass{book}
\usepackage{amsmath,amssymb,amsthm,mdframed,kotex,paralist}
\newcommand{\bd}{\begin{description}}
\newcommand{\ed}{\end{description}}
\begin{document}

\tableofcontents

%새 章
\chapter{집합과 명제}

%새 節
\section{집합}

%새 小節
\subsection{집합의 뜻}
\bd
\item[정의)]
\textbf{집합}이란, 어떤 조건이나 기준에 의하여 그 대상을 분명히 알 수 있는 것의 모임. \textbf{원소}란 집합을 이루는 대상 각각을 지칭하는 말.
%
\item[예시)]
`6의 약수의 모임'은 그 대상이 1, 2, 3, 6이 분명하므로 집합이다.
이때 1, 2, 3, 6은 각각 이 집합의 원소이다.
`수학점수가 높은 학생의 모임'은 그 대상이 되는 기준이 모호하므로 집합이 아니다.
%
\item[기호)]
(a)
통상적으로 집합은 영어의 알파벳 대문자 \(A\), \(B\), \(C\), \(\cdots\) 등으로 나타내고 원소는 알파벳 소문자 \(a\), \(b\), \(c\), \(\cdots\).
\(a\)가 \(A\)의 원소일 때, `\(a\)가 \(A\)에 속한다'고 하며, 이것을
\[a\in A\]
로 나타낸다.
또 \(b\)가 \(A\)의 원소가 아닐 때 `\(b\)는 \(A\)에 속하지 않는다'라고 하며, 이것을 
\[b\notin A\]
로 나타낸다.
또한 원소가 하나도 없는 집합을 \textbf{공집합}이라고 하고, 이것을 기호로
\[\emptyset\]
로 나타낸다.

(b)
집합을 나타내는 방법에는 크게 두 가지가 있다. \textbf{원소나열법}과 \textbf{조건제시법}이 그것이다.
원소나열법은 집합의 원소를 중괄호 안에 모두 나열하는 방법이다.
각각의 원소를 구분하기 위해 쉼표(comma)를 사용하며, 원소의 수가 많고 원소 사이의 일정한 규칙이 있을 때에는 `\(\cdots\)'를 사용하기도 한다.
조건제시법은 원소를 결정하는 조건을 제시하여 나타내는 방법으로, 중괄호 안에 '|'를 중심으로, 왼쪽에는 원소의 형태를, 오른쪽에는 원소가 될 조건을 적는다.

(c)
원소나열법에서는 원소를 나열하는 순서를 생각하지 않는다.
따라서 집합과 순서쌍은 서로 다르다.
또, 같은 원소는 중복하여 쓰지 않는다. 
%
\item[예시)]
(a)
`8의 약수의 집합'을 \(A\)라고 하면, \(A\)는 원소나열법에 의해
\[A=\{1,2,4,8\}\]
로 나타낼 수도 있고, 조건제시법에 의해
\[A=\{x\;|\;x\text{는 8의 약수}\}\]
로 나타낼 수도 있다.

(b)
100보다 작은 자연수 중 짝수의 집합을 \(B\)라고 하면, \(B\)는 원소나열법에 의해
\[B=\{2,4,\cdots,98\}\]
로 나타낼 수도 있고, 조건제시법에 의해
\[B=\{2n\;|\;n\text{은 50보다 작은 자연수}\}\]
로 나타낼 수도 있다.

(c)
\[\{1,2\}=\{2,1\}=\{1,1,2\}=\{1,2,1,2\}\]
이다.
반면
\[(1,2)\neq(2,1),\quad(1,2,3)\neq(2,1,3)\]
이다.
%
\item[정의)]
집합과 원소를 나타내는 간단한 방법으로, \textbf{벤다이어그램}을 자주 사용한다.
\ed

%새 小節
\subsection{집합 사이의 포함관계}
\bd
\item[정의)]
두 집합 \(A\), \(B\)에 대해, 집합 \(A\)의 모든 원소가 집합 \(B\)에 포함될 때, \(A\)를 \(B\)의 \textbf{부분집합}이라고 하며, 기호로는
\[A\subset B\]
로 나타낸다.
\(A\)가 \(B\)의 부분집합이 아니면
\[A\not\subset B\]
로 나타낸다.
%
\item[성질)]
임의의 집합 \(A\)에 대해 \(\emptyset\subset A\)가 성립한다.
또 \(A\subset A\)도 성립한다.
%
\item[정의)(집합의 상등)]
두 집합 \(A\), \(B\)에 대하여 \(A\subset B\), \(B\subset A\)가 성립하면 `\(A\)와 \(B\)는 서로 같다.'고 말하며,
\[A=B\]
로 나타낸다.
%
\item[성질)]
임의의 집합 \(A\)에 대하여 \(A=A\)이다.
%
\item[정의)]
\(A\subset B\)이면서 \(A\neq B\)이면 \(A\)를 \(B\)의 \textbf{진부분집합}이라고 한다.
%
\item[예시)]
\(A=\{1,3,5\}\), \(B=\{1,2,3,4,5\}\), \(C=\{2n+1\;|\;n=0,1,2\}\)라고 하면, \(A\subset B=C\)가 성립한다.
특히 \(A\)는 \(B(=C)\)의 진부분집합이다.
반면 \(B\)는 \(C\)의 부분집합이지만, 진부분집합은 아니다.
%
\item[정리)]
집합 \(A\), \(B\), \(C\)에 대하여 \(A\subset B\), \(B\subset C\)이면 \(A\subset C\)이다.
\ed
\end{document}