\documentclass{article}
\usepackage{amsmath,amssymb,amsthm,kotex,paralist,mathrsfs,mdframed}
\newcommand{\ao}{\ensuremath{{\alpha_1}}}
\newcommand{\at}{\ensuremath{{\alpha_2}}}
\newcommand{\bo}{\ensuremath{{\beta_1}}}
\newcommand{\bt}{\ensuremath{{\beta_2}}}

%%%
\begin{document}

\title{현빈 : 05 이차방정식의 공통근}
\author{}
\date{\today}
\maketitle

\noindent
\begin{mdframed}[frametitle={17장의 한 정리(p191)의 역}]
근을 가지는 두 이차방정식(\(a_i\), \(b_i\), \(c_i\)는 실수, \(a_i\neq0\))
\begin{gather}
a_1x^2+b_1x+c_1=0\\
a_2x^2+b_2x+c_2=0
\end{gather}
에 대해
\begin{equation}
(a_1c_2-a_2c_1)^2=(a_1b_2-a_2b_1)(b_1c_2-b_2c_1)
\end{equation}
이면 두 이차방정식은 공통근을 가진다.
\end{mdframed}

\begin{proof}
(1)의 두 근을 \(\ao\), \(\bo\), (2)의 두 근을 \(\at\), \(\bt\)라고 하면(각각 서로다른 두 근을 가질 수도 있고 중근을 가질 수도 있다.)
\begin{gather}
\alpha_i+\beta_i=-\frac{b_i}{a_i}\\
\alpha_i\beta_i=\frac{c_i}{a_i}
\end{gather}
이다.
\[(\ao-\at)(\ao-\bt)(\bo-\at)(\bo-\bt)=A\]
라고 하자.
그러면 \(A=0\)이라는 것만 증명하면 된다.

\((a-x)(a-y)(b-x)(b-y)\)를 전개하면
\[
\begin{split}
&(a-x)(a-y)(b-x)(b-y)\\
=&a^2b^2-a^2by-a^2bx+a^2xy-ab^2y+aby^2+abxy-axy^2\\
&-ab^2x+abxy+abx^2-ax^2y+b^2xy-bxy^2-bx^2y+x^2y^2\\
=&(ab-xy)^2+ab(x+y)^2+xy(a+b)^2\\
&-[a^2b(x+y)+ab^2(x+y)]-[axy(x+y)+bxy(x+y)]\\
=&(ab-xy)^2+ab(x+y)^2+xy(a+b)^2-ab(a+b)(x+y)-xy(a+b)(x+y)\\
=&(ab-xy)^2+ab(x+y)^2+xy(a+b)^2-(ab-xy)(a+b)(x+y)\\
\end{split}
\]
이다.
따라서
\begin{multline*}
A=(\ao\bo-\at\bt)^2+\ao\bo(\at+\bt)^2+\at\bt(\ao+\bo)^2\\
-(\ao\bo-\at\bt)(\ao+\bo)(\at+\bt)
\end{multline*}
(4), (5)를 적용하면
\begin{align*}
A
&=\left(\frac{c_1}{a_1}-\frac{c_2}{a_2}\right)^2+\frac{{b_2}^2c_1}{{a_1}{a_2}^2}+\frac{{b_1}^2c_2}{{a_1}^2{a_2}}-\left(\frac{c_1}{a_1}+\frac{c_2}{a_2}\right)\frac{b_1b_2}{a_1a_2}\\
&=\frac{B}{{a_1}^2{a_2}^2}
\end{align*}
(3)을 적용하면
\begin{align*}
B
&=(a_2c_1-a_1c_2)^2+a_1{b_2}^2c_1+a_2{b_1}^2c_2-b_1b_2(a_2c_1+a_1c_2)\\
&=(a_1b_2-a_2b_1)(b_1c_2-b_2c_1)+a_1{b_2}^2c_1+a_2{b_1}^2c_2-b_1b_2(a_2c_1+a_1c_2)\\
&=0
\end{align*}
따라서 \(A=0\)이다.
\end{proof}
%따라서
%\[
%\begin{split}
%&(\ao-\at)(\ao-\bt)(\bo-\at)(\bo-\bt)\\
%=&\ao^2\bo^2-\ao^2\bo\bt-\ao^2\at\bo+\ao^2\at\bt
%-\ao\bo^2\bt+\ao\bo\bt^2+\ao\at\bo\bt-\ao\at\bt^2\\
%&-\ao\at\bo^2+\ao\at\bo\bt+\ao\at^2\bo-\ao\at^2\bt
%+\at\bo^2\bt-\at\bo\bt^2-\at^2\bo\bt+\at^2\bt^2\\
%=&(\ao+\bo)(\at+\bt)
%\end{split}
%\]
\end{document}