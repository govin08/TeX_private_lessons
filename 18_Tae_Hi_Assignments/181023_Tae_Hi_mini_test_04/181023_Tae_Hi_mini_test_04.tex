\documentclass[a4paper]{oblivoir}
\usepackage{amsmath,amssymb,kotex,mdframed,paralist}
\usepackage{fapapersize}
\usefapapersize{210mm,297mm,20mm,*,20mm,*}

\usepackage{tabto,pifont}
\TabPositions{0.2\textwidth,0.4\textwidth,0.6\textwidth,0.8\textwidth}

%%% 객관식 선지
\newcommand\one{\ding{172}}
\newcommand\two{\ding{173}}
\newcommand\three{\ding{174}}
\newcommand\four{\ding{175}}
\newcommand\five{\ding{176}}
\usepackage{tabto,pifont}
%\TabPositions{0.2\textwidth,0.4\textwidth,0.6\textwidth,0.8\textwidth}

\newcommand\taba[5]{\par\noindent
\one\:{#1}
\tabto{0.2\textwidth}\two\:\:{#2}
\tabto{0.4\textwidth}\three\:\:{#3}
\tabto{0.6\textwidth}\four\:\:{#4}
\tabto{0.8\textwidth}\five\:\:{#5}}

\newcommand\tabb[5]{\par\noindent
\one\:{#1}
\tabto{0.33\textwidth}\two\:\:{#2}
\tabto{0.67\textwidth}\three\:\:{#3}\medskip\par\noindent
\four\:\:{#4}
\tabto{0.33\textwidth}\five\:\:{#5}}

\newcommand\tabc[5]{\par\noindent
\one\:{#1}
\tabto{0.5\textwidth}\two\:\:{#2}\medskip\par\noindent
\three\:\:{#3}
\tabto{0.5\textwidth}\four\:\:{#4}\medskip\par\noindent
\five\:\:{#5}}

\newcommand\tabd[5]{\par\noindent
\one\:{#1}\medskip\par\noindent
\two\:\:{#2}\medskip\par\noindent
\three\:\:{#3}\medskip\par\noindent
\four\:\:{#4}\medskip\par\noindent
\five\:\:{#5}}

\usepackage{graphicx}

%\pagestyle{empty}

%%% Counters
\newcounter{num}

%%% Commands
\newcommand{\prob}[1]
{\bigskip\bigskip\noindent\refstepcounter{num}\textbf{문제 \arabic{num})} #1\par\noindent}

\newcommand\pb[1]{\ensuremath{\fbox{\phantom{#1}}}}

\newcommand\ba{\ensuremath{\:|\:}}

\newcommand\vs[1]{\par\vspace{30pt}}

\newcommand\an[1]{\bigskip\par\noindent\textbf{문제 #1)}\par\noindent}

%%% Meta Commands
\let\oldsection\section
\renewcommand\section{\clearpage\oldsection}

\let\emph\textsf

\begin{document}
\begin{center}
\LARGE태희, 미니테스트 04
\end{center}
\begin{flushright}
날짜 : 2018년 \(\pb3\)월 \(\pb{10}\)일 \(\pb{월}\)요일
,\qquad
제한시간 : \pb{17년}분
,\qquad
점수 : \pb{20} / \pb{20}
\end{flushright}

%
\prob{자연수 720의 양의 약수의 개수는?}
\taba{12}{16}{20}{24}{30}
\vs

%
\prob{180의 양의 약수 중 짝수의 개수를 \(p\), 3의 배수의 개수를 \(q\)라 할 때, \(p+q\)의 값은?}
\taba{6}{12}{18}{24}{36}
\vs

%
\prob{6개의 숫자 0, 1, 2, 3, 4, 5에서 서로 다른 4개의 숫자를 택하여 만들 수 있는 네 자리 자연수의 개수는?}
\taba{120}{180}{210}{240}{300}
\vs

%
\prob{5개의 숫자 0, 1, 2, 3, 4 중에서 서로 다른 세 개의 숫자를 사용하여 만들 수 있는 세 자리 정수 중 3의 배수는 모두 몇 개인가?}
\taba{20}{24}{28}{32}{40}
\vs

%
\prob{\(a\), \(b\), \(c\), \(d\), \(e\), \(f\), \(g\) 중 4개의 문자를 뽑을 때, 모음이 모두 뽑히는 경우의 수는?}
\taba{10}{16}{20}{24}{30}
\vs

%
\prob{대형 할인점에서 사은 행사를 실시하고 있다.
냉장고, TV, 세탁기 중 어느 한 제품을 구입하면, 5가지 사은품 중에서 2가지를 선택할 수 있다고 한다.
이때 제품과 사은품을 선택하는 방법의 수는?}
\taba{26}{30}{36}{45}{52}
\vs

\newpage
%
\prob{\(f(x)=a^x\)(단, \(a>0\))일 때, 다음 중 옳지 않은 것은?}
\tabd
{\(f(x)f(y)=f(x+y)\)}
{\(f(x)\div f(y)=f(x-y)\)}
{\(\{f(x)\}^y=f(xy)\)}
{\(f(x\div y)=f(x)-f(y)\)}
{\(f(2x)=\{f(x)\}^2\)}
\vs


%
\prob{}
\(\log_{10}2=a\), \(\log_{10}3=b\)일 때, 다음을 \(a\), \(b\)로 나타내어라.
\begin{enumerate}[(1)]
\item
\(\log_{10}48\)
\item
\(\log_{10}\frac1{25}\)
\item
\(\log_{10}0.072\)
\item
\(\log_612\)
\end{enumerate}

%
\prob{\(\{2,7\}\subset X\subset\{2,3,5,7,9\}\)를 만족시키는 집합 \(X\)의 개수는?}
\taba468{16}{32}

%
\prob{집합 \(A=\{1,2,3,4,5,6,7,8,9,10\}\)의 부분집합 중에서 적어도 하나의 홀수를 원소로 갖고, 3의 배수는 원소로 갖지 않는 부분집합의 개수는?}
\taba{96}{112}{120}{124}{126}

%
\prob{자연수 전체의 집합에서 자연수 \(k\)의 배수의 집합을 \(A_k\)라고 할 때, \((A_{18}\cup A_{36})\cap(A_{36}\cup A_{24})\)를 간단히 하면?}
\taba{\(A_{12}\)}{\(A_{18}\)}{\(A_{24}\)}{\(A_{36}\)}{\(A_{72}\)}

%
\prob{두 집합 \(X\), \(Y\)에 대하여 \(X\triangle Y\)를 \(X\triangle Y=(X-Y)\cup(Y-X)\)로 정의하자.
\(A=\{1,2,3,4,5\}\),
\(A^c\cap B=\{6,7\}\),
\(A\cap B=\{2,3,5\}\)일 때, \(A\triangle B\)의 원소들의 합을 구하여라.}
\taba{18}{19}{20}{21}{22}

%
\prob{학생 40명을 대상으로 음악과 영화에 대한 선호도를 조사하였다.
그 결과 음악을 좋아하는 학생이 26명,
영화를 좋아하는 학생이 18명이었다.
음악과 영화를 모두 좋아하는 학생의 수의 최댓값을 \(M\), 최솟값을 \(m\)이라 할 때, \(M-m\)의 값을 구하여라.}
\taba{11}{12}{13}{14}{!5}
\end{document}