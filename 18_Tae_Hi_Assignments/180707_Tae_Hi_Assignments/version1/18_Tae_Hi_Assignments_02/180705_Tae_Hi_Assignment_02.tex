\documentclass{oblivoir}
%%%Default packages
\usepackage{amsmath,amssymb,amsthm,kotex,tabu,graphicx,pifont}
\usepackage{../kswrapfig}

\usepackage{gensymb} %\degree

%%%More packages
%\usepackage{caption,subcaption}
%\usepackage[perpage]{footmisc}
%
\usepackage[skipabove=10pt,innertopmargin=10pt,nobreak=true]{mdframed}

\usepackage[inline]{enumitem}
\setlist[enumerate,1]{label=(\arabic*)}
\setlist[enumerate,2]{label=(\alph*)}

\usepackage{multicol}
\setlength{\columnsep}{30pt}
\setlength{\columnseprule}{1pt}
%
%\usepackage{forest}
%\usetikzlibrary{shapes.geometric,arrows.meta,calc}
%
%%%defi theo exam prob rema proo
%이 환경들 아래에 문단을 쓸 경우 살짝 들여쓰기가 되므로 \hspace{-.7em}가 필요할 수 있다.

\newcounter{num}
\newcommand{\defi}[1]
{\noindent\refstepcounter{num}\textbf{정의 \arabic{num})} #1\par\noindent}
\newcommand{\theo}[1]
{\noindent\refstepcounter{num}\textbf{정리 \arabic{num})} #1\par\noindent}
\newcommand{\revi}[1]
{\noindent\refstepcounter{num}\textbf{복습 \arabic{num})} #1\par\noindent}
\newcommand{\exam}[1]
{\bigskip\bigskip\noindent\refstepcounter{num}\textbf{예시 \arabic{num})} #1\par\noindent}
\newcommand{\prob}[1]
{\bigskip\bigskip\noindent\refstepcounter{num}\textbf{문제 \arabic{num})} #1\par\noindent}
\newcommand{\rema}[1]
{\bigskip\bigskip\noindent\refstepcounter{num}\textbf{참고 \arabic{num})} #1\par\noindent}
\newcommand{\proo}
{\bigskip\noindent\textsf{증명)}}

\newenvironment{talign}
 {\let\displaystyle\textstyle\align}
 {\endalign}
\newenvironment{talign*}
 {\let\displaystyle\textstyle\csname align*\endcsname}
 {\endalign}
%
%%%Commands

\newcommand{\procedure}[1]{\begin{mdframed}\vspace{#1\textheight}\end{mdframed}}

\newcommand\an[1]{\par\bigskip\noindent\textbf{문제 \ref{#1})}\par\noindent}

\newcommand\ann[2]{\par\bigskip\noindent\textbf{문제 \ref{#1})}\:\:#2\par\medskip\noindent}

\newcommand\ans[1]{\begin{flushright}\textbf{답 : }#1\end{flushright}}

\newcommand\anssec[1]{\bigskip\bigskip\noindent{\large\bfseries#1}}

\newcommand{\pb}[1]%\Phantom + fBox
{\fbox{\phantom{\ensuremath{#1}}}}

\newcommand\ba{\,|\,}

\newcommand\ovv[1]{\ensuremath{\overline{#1}}}
\newcommand\ov[2]{\ensuremath{\overline{#1#2}}}
%
%%%% Settings
%\let\oldsection\section
%
%\renewcommand\section{\clearpage\oldsection}
%
%\let\emph\textsf
%
%\renewcommand{\arraystretch}{1.5}
%
%%%% Footnotes
%\makeatletter
%\def\@fnsymbol#1{\ensuremath{\ifcase#1\or
%*\or **\or ***\or
%\star\or\star\star\or\star\star\star\or
%\dagger\or\dagger\dagger\or\dagger\dagger\dagger
%\else\@ctrerr\fi}}
%
%\renewcommand{\thefootnote}{\fnsymbol{footnote}}
%\makeatother
%
%\makeatletter
%\AtBeginEnvironment{mdframed}{%
%\def\@fnsymbol#1{\ensuremath{\ifcase#1\or
%*\or **\or ***\or
%\star\or\star\star\or\star\star\star\or
%\dagger\or\dagger\dagger\or\dagger\dagger\dagger
%\else\@ctrerr\fi}}%
%}   
%\renewcommand\thempfootnote{\fnsymbol{mpfootnote}}
%\makeatother
%
%%% 객관식 선지
\newcommand\one{\ding{172}}
\newcommand\two{\ding{173}}
\newcommand\three{\ding{174}}
\newcommand\four{\ding{175}}
\newcommand\five{\ding{176}}
\usepackage{tabto,pifont}
%\TabPositions{0.2\textwidth,0.4\textwidth,0.6\textwidth,0.8\textwidth}

\newcommand\taba[5]{\par\noindent
\one\:{#1}
\tabto{0.2\textwidth}\two\:\:{#2}
\tabto{0.4\textwidth}\three\:\:{#3}
\tabto{0.6\textwidth}\four\:\:{#4}
\tabto{0.8\textwidth}\five\:\:{#5}}

\newcommand\tabb[5]{\par\noindent
\one\:{#1}
\tabto{0.33\textwidth}\two\:\:{#2}
\tabto{0.67\textwidth}\three\:\:{#3}\medskip\par\noindent
\four\:\:{#4}
\tabto{0.33\textwidth}\five\:\:{#5}}

\newcommand\tabc[5]{\par\noindent
\one\:{#1}
\tabto{0.5\textwidth}\two\:\:{#2}\medskip\par\noindent
\three\:\:{#3}
\tabto{0.5\textwidth}\four\:\:{#4}\medskip\par\noindent
\five\:\:{#5}}

\newcommand\tabd[5]{\par\noindent
\one\:{#1}\medskip\par\noindent
\two\:\:{#2}\medskip\par\noindent
\three\:\:{#3}\medskip\par\noindent
\four\:\:{#4}\medskip\par\noindent
\five\:\:{#5}}
%
%%%% fonts
%
%\usepackage{fontspec, xunicode, xltxtra}
%\setmainfont[]{은 바탕}
%\setsansfont[]{은 돋움}
%\setmonofont[]{은 바탕}
%\XeTeXlinebreaklocale "ko"
%%%%
\begin{document}

\begin{center}
\LARGE태희, 추가 과제 01 - 풀이
\end{center}
\vspace{50pt}
\begin{multicols*}{2}

%
\an{01-1}{2018년 6월 고1 학력평가 26번}
\inc{c0101-1}
\inc{c0101-2}

%
\an{01-2}{2018년 6월 고1 학력평가 15번}
\inc{c0102-1}
\vfill\null\columnbreak

%
\an{013}{2018년 6월 고1 학력평가 21번}
\inc{c0103-1}
\inc{c0103-2}
\end{multicols*}

\newpage
\begin{center}
\LARGE태희, 추가 과제 02 - 문제
\end{center}
\vspace{50pt}

\begin{multicols*}{2}
%18번%5
\prob{02-1}
\inc{a0201}
%\begin{center}
%\includegraphics[width=0.95\columnwidth]{0201a}
%\end{center}
%복소수 \(z=a+bi\)(\(a\), \(b\)는 \(0\)이 아닌 실수)에 대하여
%\[iz=\bar z\]
%일 때, <보기>에서 옳은 것만을 있는 대로 고른 것은?\\
%(단, \(i=\sqrt{-1}\)이고 \(\bar z\)는 \(z\)의 켤레복소수이다.)
%\begin{mdframed}[frametitle=<보기>]
%ㄱ. \(z+\bar z=-2b\)
%\\
%ㄴ. \(i\bar z=-z\)
%\\
%ㄷ. \(\frac{\bar z}{z}+\frac{z}{\bar z}=0\)
%\end{mdframed}
%\tabb{\text{ㄱ}}{\text{ㄷ}}{\text{ㄱ, ㄴ}}{\text{ㄴ, ㄷ}}{\text{ㄱ, ㄴ, ㄷ}}

%21번%1
\prob{02-2}
\inc{a0202}
%\begin{center}
%\includegraphics[width=0.95\columnwidth]{0202a}
%\end{center}
%\(x\)에 대한 연립부등식
%\[\begin{cases}
%x^2-a^2x\ge0\\
%x^2-4ax+4a^2-1<0
%\end{cases}\]
%을 만족시키는 정수 \(x\)의 개수가 \(1\)이 되기 위한 모든 실수 \(a\)의 값의 합은?
%(단, \(0<a<\sqrt2\))
%\tabb{\frac32}{\frac{25}{16}}{\frac{13}8}{\frac{27}{16}}{\frac74}

\vfill\null
\columnbreak
%30번%27
\prob{02-3}
\inc{a0203}
%\begin{center}
%\includegraphics[width=0.95\columnwidth]{0203a}
%\end{center}

%다음 조건을 만족시키는 모든 이차다항식 \(P(x)\)의 합을 \(Q(x)\)라고 하자.
%\begin{mdframed}\begin{enumerate}[(가)]
%\item
%\(P(1)P(2)=0\)
%\item
%사차다항식 \(P(x)\{P(x)-3\}\)은 \(x(x-3)\)으로 나누어 떨어진다.
%\end{enumerate}\end{mdframed}
%\(Q(x)\)를 \(x-4\)로 나눈 나머지를 구하시오.
\end{multicols*}
\newpage

\begin{multicols*}{2}
%
\an{02-1}{2017년 6월 고1 학력평가 18번}
\inc{c0202-1}
%\begin{center}
%\includegraphics[width=0.95\columnwidth]{0201c}
%\end{center}
%\(iz=ai-b=-b+ai\), \(\bar z=a-bi\)으로부터
%\[-b+ai=a-bi\]
%따라서 \(-b=a\), \(a=-b\).
%\[b=-a\]
%를 대입하면
%\[z=a-ai=a(1-i)\]
%\begin{enumerate}
%\item[ㄱ.]
%\(z+\bar z=a(1-i)+a(1+i)=2a=-2b\)\quad ( 참 )
%\item[ㄴ.]
%\(iz=i\cdot a(1-i)=a(-1+i)=-z\)\quad ( 참 )
%\item[ㄷ.]
%\begin{align*}
%z\bar z		&=a^2(1-i)(1+i)=2a^2\\
%z^2			&=a^2(1-i)^2=-2a^2i\\
%{\bar z}^2	&=a^2(1+i)^2=2a^2i
%\end{align*}
%으로부터
%\[\frac{\bar z}{z}+\frac{z}{\bar z}=\frac{z^2+{\bar z}^2}{z\bar z}\quad\text{ ( 참 )}\]
%\end{enumerate}
%\ans{\five}

\vfill\null\columnbreak
%
\an{02-2}{2017년 6월 고1 학력평가 21번}
\inc{c0202-1}
%\noindent\includegraphics[width=0.95\columnwidth]{0202c}

\newpage
%
\an{02-3}{2017년 6월 고1 학력평가 30번}
%\begin{center}
%\includegraphics[width=0.9\columnwidth]{0203c}
%\end{center}
\inc{c0203-1}
\vfill\null
\inc{c0203-2}
\end{multicols*}
\end{document}