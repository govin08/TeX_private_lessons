\documentclass{oblivoir}
\usepackage{amsmath,amssymb,kotex,paralist,graphicx}
\usepackage{mdframed}
\usepackage{../kswrapfig}
\usepackage{fapapersize}
\usefapapersize{210mm,297mm,20mm,*,20mm,*}
%\pagestyle{empty}
\usepackage{multicol}
\setlength{\columnsep}{30pt}
\setlength{\columnseprule}{1pt}
%\def\columnseprulecolor{\color{blue}}

%%% 객관식 선지

\usepackage{tabto,pifont}
\TabPositions{0.2\textwidth,0.4\textwidth,0.6\textwidth,0.8\textwidth}

\newcommand\one{\ding{172}}
\newcommand\two{\ding{173}}
\newcommand\three{\ding{174}}
\newcommand\four{\ding{175}}
\newcommand\five{\ding{176}}

\newcommand\taba[5]{\par\bigskip\noindent
\one\:{\ensuremath{#1}}
\tab\two\:\:{\ensuremath{#2}}
\tab\three\:\:{\ensuremath{#3}}
\tab\four\:\:{\ensuremath{#4}}
\tab\five\:\:{\ensuremath{#5}}}

\newcommand\tabb[5]{\par\bigskip\noindent
\one\:{\ensuremath{#1}}
\tabto{0.16\textwidth}\two\:\:{\ensuremath{#2}}
\tabto{0.33\textwidth}\three\:\:{\ensuremath{#3}}\medskip\par\noindent
\four\:\:{\ensuremath{#4}}.
\tabto{0.16\textwidth}\five\:\:{\ensuremath{#5}}}

\newcommand\tabc[5]{\par\bigskip\noindent
\one\:{\ensuremath{#1}}
\tabto{0.25\textwidth}\two\:\:{\ensuremath{#2}}\medskip\par\noindent
\three\:\:{\ensuremath{#3}}
\tabto{0.25\textwidth}\four\:\:{\ensuremath{#4}}\medskip\par\noindent
\five\:\:{\ensuremath{#5}}}

\newcommand\tabd[5]{\par\bigskip\noindent
\one\:{#1}\medskip\par\noindent
\two\:\:{#2}\medskip\par\noindent
\three\:\:{#3}\medskip\par\noindent
\four\:\:{#4}\medskip\par\noindent
\five\:\:{#5}}

%%% Counters
\newcounter{num}

%%% Commands
\newcommand{\prob}[1]
{\vs\par\noindent\refstepcounter{num}\textbf{문제 \arabic{num})}\label{#1}\par\noindent}

\newcommand\vs[1]{\vspace{70pt}}

\newcommand\inc[1]{\begin{center}\includegraphics[width=0.95\columnwidth]{#1}\end{center}}

\newcommand\pb[1]{\ensuremath{\fbox{\phantom{#1}}}}

\newcommand\ba{\ensuremath{\:|\:}}

\newcommand\an[2]{\par\bigskip\noindent\textbf{문제 \ref{#1})} #2\\}

\newcommand\ans[1]{\begin{flushright}\textbf{답 : }#1\end{flushright}}

\renewcommand{\arraystretch}{1.5}

%%% Meta Commands
\let\oldsection\section
\renewcommand\section{\clearpage\oldsection}
\let\emph\textsf

%%%%
\begin{document}

\begin{center}
\LARGE태희, 추가 과제 01 - 풀이
\end{center}
\vspace{50pt}
\begin{multicols*}{2}

%
\an{01-1}{2018년 6월 고1 학력평가 26번}
\inc{c0101-1}
\inc{c0101-2}

%
\an{01-2}{2018년 6월 고1 학력평가 15번}
\inc{c0102-1}
\vfill\null\columnbreak

%
\an{013}{2018년 6월 고1 학력평가 21번}
\inc{c0103-1}
\inc{c0103-2}
\end{multicols*}

\newpage
\begin{center}
\LARGE태희, 추가 과제 02 - 문제
\end{center}
\vspace{50pt}

\begin{multicols*}{2}
%18번%5
\prob{02-1}
\inc{a0201}
%\begin{center}
%\includegraphics[width=0.95\columnwidth]{0201a}
%\end{center}
%복소수 \(z=a+bi\)(\(a\), \(b\)는 \(0\)이 아닌 실수)에 대하여
%\[iz=\bar z\]
%일 때, <보기>에서 옳은 것만을 있는 대로 고른 것은?\\
%(단, \(i=\sqrt{-1}\)이고 \(\bar z\)는 \(z\)의 켤레복소수이다.)
%\begin{mdframed}[frametitle=<보기>]
%ㄱ. \(z+\bar z=-2b\)
%\\
%ㄴ. \(i\bar z=-z\)
%\\
%ㄷ. \(\frac{\bar z}{z}+\frac{z}{\bar z}=0\)
%\end{mdframed}
%\tabb{\text{ㄱ}}{\text{ㄷ}}{\text{ㄱ, ㄴ}}{\text{ㄴ, ㄷ}}{\text{ㄱ, ㄴ, ㄷ}}

%21번%1
\prob{02-2}
\inc{a0202}
%\begin{center}
%\includegraphics[width=0.95\columnwidth]{0202a}
%\end{center}
%\(x\)에 대한 연립부등식
%\[\begin{cases}
%x^2-a^2x\ge0\\
%x^2-4ax+4a^2-1<0
%\end{cases}\]
%을 만족시키는 정수 \(x\)의 개수가 \(1\)이 되기 위한 모든 실수 \(a\)의 값의 합은?
%(단, \(0<a<\sqrt2\))
%\tabb{\frac32}{\frac{25}{16}}{\frac{13}8}{\frac{27}{16}}{\frac74}

\vfill\null
\columnbreak
%30번%27
\prob{02-3}
\inc{a0203}
%\begin{center}
%\includegraphics[width=0.95\columnwidth]{0203a}
%\end{center}

%다음 조건을 만족시키는 모든 이차다항식 \(P(x)\)의 합을 \(Q(x)\)라고 하자.
%\begin{mdframed}\begin{enumerate}[(가)]
%\item
%\(P(1)P(2)=0\)
%\item
%사차다항식 \(P(x)\{P(x)-3\}\)은 \(x(x-3)\)으로 나누어 떨어진다.
%\end{enumerate}\end{mdframed}
%\(Q(x)\)를 \(x-4\)로 나눈 나머지를 구하시오.
\end{multicols*}
\newpage

\begin{multicols*}{2}
%
\an{02-1}{2017년 6월 고1 학력평가 18번}
\inc{c0202-1}
%\begin{center}
%\includegraphics[width=0.95\columnwidth]{0201c}
%\end{center}
%\(iz=ai-b=-b+ai\), \(\bar z=a-bi\)으로부터
%\[-b+ai=a-bi\]
%따라서 \(-b=a\), \(a=-b\).
%\[b=-a\]
%를 대입하면
%\[z=a-ai=a(1-i)\]
%\begin{enumerate}
%\item[ㄱ.]
%\(z+\bar z=a(1-i)+a(1+i)=2a=-2b\)\quad ( 참 )
%\item[ㄴ.]
%\(iz=i\cdot a(1-i)=a(-1+i)=-z\)\quad ( 참 )
%\item[ㄷ.]
%\begin{align*}
%z\bar z		&=a^2(1-i)(1+i)=2a^2\\
%z^2			&=a^2(1-i)^2=-2a^2i\\
%{\bar z}^2	&=a^2(1+i)^2=2a^2i
%\end{align*}
%으로부터
%\[\frac{\bar z}{z}+\frac{z}{\bar z}=\frac{z^2+{\bar z}^2}{z\bar z}\quad\text{ ( 참 )}\]
%\end{enumerate}
%\ans{\five}

\vfill\null\columnbreak
%
\an{02-2}{2017년 6월 고1 학력평가 21번}
\inc{c0202-1}
%\noindent\includegraphics[width=0.95\columnwidth]{0202c}

\newpage
%
\an{02-3}{2017년 6월 고1 학력평가 30번}
%\begin{center}
%\includegraphics[width=0.9\columnwidth]{0203c}
%\end{center}
\inc{c0203-1}
\vfill\null
\inc{c0203-2}
\end{multicols*}
\end{document}