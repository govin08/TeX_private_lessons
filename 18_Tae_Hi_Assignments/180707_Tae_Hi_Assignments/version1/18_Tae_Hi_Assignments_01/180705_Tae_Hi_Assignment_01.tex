\documentclass[a4paper]{oblivoir}
\usepackage{amsmath,amssymb,kotex,paralist,graphicx}
\usepackage{mdframed}
\usepackage{../kswrapfig}
%\usepackage{fapapersize}
%\usefapapersize{210mm,297mm,20mm,*,20mm,*}
\pagestyle{empty}

%%% 객관식 선지

\usepackage{tabto,pifont}
\TabPositions{0.2\textwidth,0.4\textwidth,0.6\textwidth,0.8\textwidth}

\newcommand\one{\ding{172}}
\newcommand\two{\ding{173}}
\newcommand\three{\ding{174}}
\newcommand\four{\ding{175}}
\newcommand\five{\ding{176}}

\newcommand\taba[5]{\par\bigskip\noindent
\one\:{\ensuremath{#1}}
\tab\two\:\:{\ensuremath{#2}}
\tab\three\:\:{\ensuremath{#3}}
\tab\four\:\:{\ensuremath{#4}}
\tab\five\:\:{\ensuremath{#5}}}

\newcommand\tabb[5]{\par\bigskip\noindent
\one\:{\ensuremath{#1}}
\tabto{0.33\textwidth}\two\:\:{\ensuremath{#2}}
\tabto{0.67\textwidth}\three\:\:{\ensuremath{#3}}\medskip\par\noindent
\four\:\:{\ensuremath{#4}}.
\tabto{0.33\textwidth}\five\:\:{\ensuremath{#5}}}

\newcommand\tabc[5]{\par\bigskip\noindent
\one\:{\ensuremath{#1}}
\tabto{0.5\textwidth}\two\:\:{\ensuremath{#2}}\medskip\par\noindent
\three\:\:{\ensuremath{#3}}
\tabto{0.5\textwidth}\four\:\:{\ensuremath{#4}}\medskip\par\noindent
\five\:\:{\ensuremath{#5}}}

\newcommand\tabd[5]{\par\bigskip\noindent
\one\:{#1}\medskip\par\noindent
\two\:\:{#2}\medskip\par\noindent
\three\:\:{#3}\medskip\par\noindent
\four\:\:{#4}\medskip\par\noindent
\five\:\:{#5}}

%%% Counters
\newcounter{num}

%%% Commands
\newcommand{\prob}[1]
{\vs\par\noindent\refstepcounter{num}\textbf{문제 \arabic{num})}\label{#1}\par\noindent}

\newcommand\vs[1]{\vspace{70pt}}

\newcommand\pb[1]{\ensuremath{\fbox{\phantom{#1}}}}

\newcommand\ba{\ensuremath{\:|\:}}

\newcommand\an[2]{\par\bigskip\noindent\textbf{문제 \ref{#1})} #2\\}

\newcommand\ans[1]{\begin{flushright}\textbf{답 : }#1\end{flushright}}

\renewcommand{\arraystretch}{1.5}

%%% Meta Commands
\let\oldsection\section
\renewcommand\section{\clearpage\oldsection}
\let\emph\textsf

\begin{document}
\begin{center}
\LARGE태희, 추가 과제 01
\end{center}
\begin{flushright}
\today
\end{flushright}
%40
\prob{01-1}
\(x\)에 대한 다항식 \(x^4+ax+b\)가 \((x-2)^2\)으로 나누어떨어질 때, 몫을 \(Q(x)\)라고 하자.
두 상수 \(a\), \(b\)에 대하여 \(a+b+Q(2)\)의 값을 구하여라.

%1
\prob{01-2}
\(2018^3-27\)을 \(2018\times2021+9\)로 나눈 몫은?
\taba{2015}{2025}{2035}{2045}{2055}

%5
\prob{01-3}
모든 실수 \(x\)에 대하여 두 이차다항식 \(P(x)\), \(Q(x)\)가 다음 조건을 만족시킨다.
\begin{mdframed}\begin{enumerate}[(가)]
\item
\(P(x)+Q(x)=4\)
\item[(나)]
\(\{P(x)\}^3+\{Q(x)\}^3=12x^4+24x^3+12x^2+16\)
\end{enumerate}\end{mdframed}
\(P(x)\)의 최고차항의 계수가 음수일 때, \(P(2)+Q(3)\)의 값은?
\taba6789{10}

\newpage
%
\an{01-1}{2018년 6월 고1 교육청 모의고사 26번}
\[x^4+ax+b=(x-2)^2Q(x)\tag{1}\]
에서 \(x=2\)를 대입하면
\[16+2a+b=0\]
따라서
\(b=-2a-16\)이다. 이것을 (1)에 대입하면
\[x^4+ax-2a-16=(x-2)^2Q(x)\]
이다.
조립제법을 사용하여 좌변을 인수분해하면

\[\begin{array}{cccccc}
\multicolumn{1}{c|}{2}	&1	&0	&0	&a	&-2a-16\\
\multicolumn{1}{c|}{}		&	&2	&4	&8	&2a+16\\
\cline{2-6}
							&1	&2	&4	&a+8	&\multicolumn1{|c}{0}\\
\cline{6-6}
\end{array}\]
\[(x-2)(x^3+2x^2+4x+a+8)=(x-2)^2Q(x)\]
이다.
이 식을 \(x-2\)로 나누면
\[x^3+2x^2+4x+a+8=(x-2)Q(x)\tag{2}\]
여기에 \(x=2\)를 대입하면 \(a+32=0\), \(a=-32\).
또한 \(b=48\).
따라서 (2)는
\[x^3+2x^2+4x-24=(x-2)Q(x)\]
이 된다.
한편
\[\begin{array}{ccccc}
\multicolumn{1}{c|}{2}	&1	&2	&4	&-24\\
\multicolumn{1}{c|}{}		&	&2	&8	&24	\\
\cline{2-5}
							&1	&4	&12	&\multicolumn1{|c}{0}\\
\cline{5-5}
\end{array}\]
에서
\[(x-2)(x^2+4x+12)=(x-2)Q(x)\]
이고 다시 양변을 \(x-2\)로 나누면
\[Q(x)=x^2+4x+12\tag{3}\]
이다.
따라서 \(Q(2)=24\).
그러므로
\[a+b+Q(2)=-32+48+24=40\]
\ans{40}

%
\an{01-2}{2018년 6월 고1 교육청 모의고사 15번}
\begin{mdframed}
정수 \(a\)와 자연수 \(b\)에 대하여
\[a=bq+r\quad(0\le r<b)\]
이면 \(a\)를 \(b\)로 나누었을 때 몫이 \(q\)이고 나머지가 \(r\)이다.
\end{mdframed}
곱셈공식 \(a^3-b^3=(a-b)(a^2+ab+b^2)\)으로부터
\begin{align*}
x^3-27
&=(x-3)(x^2+3x+9)\\
&=(x-3)\{x(x+3)+9\}
\end{align*}
\(x=2018\)을 대입하면
\[2018^3-27=2015\cdot(2018\times2021+9)\]
따라서 \(2018^3-27\)를 \(2018\times2021+9\)로 나누면 몫이 \(2015\)이고 나머지가 \(0\)이다.
\ans{\one}

%
\an{01-3}{2018년 6월 고1 교육청 모의고사 21번}
(나)의 좌변을 인수분해하여 정리하면(\(P(x)=P\), \(Q(x)=Q\)로 표기)
\begin{align*}
P^3+Q^3
&=(P+Q)(P^2-PQ+Q^2)\\
&\stackrel{(가)}=4(P^2-PQ+Q^2)
\end{align*}
따라서
\[P^2-PQ+Q^2=3x^4+6x^3+3x^2+4\]
(가)를 변형한 \(Q=4-P\)를 대입하면
\[3P^2-12P+16=3x^4+6x^3+3x^2+4\]
\[3P^2-12P=3x^4+6x^3+3x^2-12\]
\[P(P-4)=x^4+2x^3+x^2-4\]
조립제법을 써서 우변을 인수분해하면
\[P(P-4)=(x-1)(x+2)(x^2+x+2)\]
이것을 다시 정리하면
\[P(P-4)=(x^2+x-2)(x^2+x+2)\]
따라서
\[P(x)=x^2+x+2,\qquad P(x)-4=x^2+x-2\]
혹은
\[P(x)=-x^2-x+2,\qquad P(x)-4=-x^2-x-2\]
\(P(x)\)의 최고차항의 계수가 음수이므로
\[P(x)=-x^2-x+2\]
또한 \(Q=4-P\)에서
\[Q=x^2+x+2\]
\(P(2)+Q(3)=(-4)+14=10\)
\ans{\five}


%(가)로부터 \(Q(x)=4-P(x)\)이다.
%이것을 (나)에 대입하면(\(P(x)=P\), \(Q(x)=Q\)로 표기)
%\begin{align*}
%P^3+(4-P)^3=12x^4+24x^3
%\end{align*}


\end{document}