\documentclass[a4paper]{oblivoir}
\usepackage{amsmath,amssymb,kotex,mdframed,paralist}
\usepackage{fapapersize}
\usefapapersize{210mm,297mm,20mm,*,20mm,*}

\usepackage{tabto,pifont}
\TabPositions{0.2\textwidth,0.4\textwidth,0.6\textwidth,0.8\textwidth}

%%% 객관식 선지
\newcommand\one{\ding{172}}
\newcommand\two{\ding{173}}
\newcommand\three{\ding{174}}
\newcommand\four{\ding{175}}
\newcommand\five{\ding{176}}
\usepackage{tabto,pifont}
%\TabPositions{0.2\textwidth,0.4\textwidth,0.6\textwidth,0.8\textwidth}

\newcommand\taba[5]{\par\noindent
\one\:{#1}
\tabto{0.2\textwidth}\two\:\:{#2}
\tabto{0.4\textwidth}\three\:\:{#3}
\tabto{0.6\textwidth}\four\:\:{#4}
\tabto{0.8\textwidth}\five\:\:{#5}
\vspace{30pt}}

\newcommand\tabb[5]{\par\noindent
\one\:{#1}
\tabto{0.33\textwidth}\two\:\:{#2}
\tabto{0.67\textwidth}\three\:\:{#3}\medskip\par\noindent
\four\:\:{#4}
\tabto{0.33\textwidth}\five\:\:{#5}
\vspace{30pt}}

\newcommand\tabc[5]{\par\noindent
\one\:{#1}
\tabto{0.5\textwidth}\two\:\:{#2}\medskip\par\noindent
\three\:\:{#3}
\tabto{0.5\textwidth}\four\:\:{#4}\medskip\par\noindent
\five\:\:{#5}
\vspace{30pt}}

\newcommand\tabd[5]{\par\noindent
\one\:{#1}\medskip\par\noindent
\two\:\:{#2}\medskip\par\noindent
\three\:\:{#3}\medskip\par\noindent
\four\:\:{#4}\medskip\par\noindent
\five\:\:{#5}
\vspace{30pt}}

\newcommand\vs[1]{\par\vspace{30pt}}

\usepackage{graphicx}

%\pagestyle{empty}

%%% Counters
\newcounter{num}

%%% Commands
\newcommand{\prob}[1]
{\bigskip\bigskip\noindent\refstepcounter{num}\textbf{문제 \arabic{num})} #1\par\noindent}

\newcommand\pb[1]{\ensuremath{\fbox{\phantom{#1}}}}

\newcommand\ba{\ensuremath{\:|\:}}

\newcommand\an[1]{\bigskip\par\noindent\textbf{문제 #1)}\par\noindent}

%%% Meta Commands
\let\oldsection\section
\renewcommand\section{\clearpage\oldsection}

\let\emph\textsf

\begin{document}
\begin{center}
\LARGE태희, 미니테스트 05
\end{center}
\begin{flushright}
날짜 : 2018년 \(\pb3\)월 \(\pb{10}\)일 \(\pb{월}\)요일
,\qquad
제한시간 : \pb{17년}분
,\qquad
점수 : \pb{20} / \pb{20}
\end{flushright}

%
\prob{다음 중 옳은 것은?}
\tabd
{\(a<0\)일 때, \((\sqrt[3]{-a})^3=a\)이다.}
{\((-2)^2\)의 제곱근은 \(2\)이다.}
{\(\sqrt{256}\)의 네제곱근은 \(\pm2\)이다.}
{\(n\)이 짝수이고 \(a>0\)일 떄, \(x^n=a\)를 만족시키는 실수 \(x\)의 값은 \(n\)개이다.}
{\(n\)이 홀수일 때, \(-3\)의 \(n\)제곱근 중 실수인 것은 \(-\sqrt[n]3\)이다.}


%
\prob{\(2^{8x}=9\)일 때, \(\displaystyle\frac{2^{6x}-2^{-6x}}{2^{2x}+2^{-2x}}\)의 값은?}
\taba{\(\frac53\)}{\(\frac{11}6\)}2{\(\frac{13}6\)}{\(\frac73\)}


%
\prob{세 수 \(A=\sqrt{\sqrt5}\), \(B=\sqrt[3]3\), \(C=\sqrt{\sqrt[3]{10}}\)의 대소 관계를 바르게 나타낸 것은?}
\tabb{\(A<B<C\)}{\(B<A<C\)}{\(A<C<B\)}{\(B<C<A\)}{\(C<B<A\)}


%
\prob{\(\log_{\sqrt3}a=4\), \(\log_{\frac13}27=b\)일 때, \(ab\)의 값은?}
\tabb{\(-27\)}{\(-21\)}{\(-18\)}{\(-12\)}{\(-9\)}


%
\prob{\(\log_7(\log_3(\log_2x))=0\)일 때, \(x\)의 값은?}
\taba2468{10}


\newpage
%
\prob{1이 아닌 양수 \(x\)에 대하여 등식}
\[\frac1{\log_2x}+\frac1{\log_3x}+\frac1{\log_5x}=\frac1{\log_ax}\]
이 성립할 때, \(a\)의 값을 구하시오.
\vs

%
\prob{\(\log_35\cdot\log_57\cdot\log_79\)의 값은?}
\taba12345


%
\prob{\(\log_72=a\), \(\log_73=b\)일 때, \(\log_{12}\sqrt{24}\)를 \(a\), \(b\)로 나타내면?}
\tabb
{\(\frac{2(3a+b)}{2a+b}\)}
{\(\frac{3a+b}{2(2a+b)}\)}
{\(\frac{2(2a+b)}{3a+b}\)}
{\(\frac{2a+b}{2(3a+b)}\)}
{\(\frac{3a+b}{3(2a+b)}\)}


%
\prob{\(10^a=x\), \(10^b=y\), \(10^c=z\)일 때, \(\log_{10}\frac{x^2z^4}{y^3}\)을 \(a\), \(b\), \(c\)로 나타내면?}
\tabb
{\(2a-3b+4c\)}
{\(\frac{a^2c^4}{b^3}\)}
{\(\frac{8ac}{3b}\)}
{\(\frac{2a+4c}{3b}\)}
{\(4ab+2bc-3ab\)}


%
\prob{\(25^x=4^y=10\)일 때, \(\frac1x+\frac1y\)의 값은?}
\taba12345


%
\prob{0이 아닌 실수 \(x\), \(y\), \(z\)에 대하여 \(5^x=2^y=\sqrt{10^z}\)일 때, \(\frac1x+\frac1y-\frac2z\)의 값은?}
\taba01234
\end{document}