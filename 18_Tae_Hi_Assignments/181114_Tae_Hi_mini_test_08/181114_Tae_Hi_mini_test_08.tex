\documentclass[a4paper]{oblivoir}
\usepackage{amsmath,amssymb,kotex,mdframed,paralist,tabu}
\usepackage{fapapersize}
\usefapapersize{210mm,297mm,20mm,*,20mm,*}

\usepackage{tabto,pifont}
\TabPositions{0.2\textwidth,0.4\textwidth,0.6\textwidth,0.8\textwidth}

%%% 객관식 선지
\newcommand\one{\ding{172}}
\newcommand\two{\ding{173}}
\newcommand\three{\ding{174}}
\newcommand\four{\ding{175}}
\newcommand\five{\ding{176}}
\usepackage{tabto,pifont}
%\TabPositions{0.2\textwidth,0.4\textwidth,0.6\textwidth,0.8\textwidth}

\newcommand\taba[5]{\par\noindent
\one\:{#1}
\tabto{0.2\textwidth}\two\:\:{#2}
\tabto{0.4\textwidth}\three\:\:{#3}
\tabto{0.6\textwidth}\four\:\:{#4}
\tabto{0.8\textwidth}\five\:\:{#5}}

\newcommand\tabb[5]{\par\noindent
\one\:{#1}
\tabto{0.33\textwidth}\two\:\:{#2}
\tabto{0.67\textwidth}\three\:\:{#3}\medskip\par\noindent
\four\:\:{#4}
\tabto{0.33\textwidth}\five\:\:{#5}}

\newcommand\tabc[5]{\par\noindent
\one\:{#1}
\tabto{0.5\textwidth}\two\:\:{#2}\medskip\par\noindent
\three\:\:{#3}
\tabto{0.5\textwidth}\four\:\:{#4}\medskip\par\noindent
\five\:\:{#5}}

\newcommand\tabd[5]{\par\noindent
\one\:{#1}\medskip\par\noindent
\two\:\:{#2}\medskip\par\noindent
\three\:\:{#3}\medskip\par\noindent
\four\:\:{#4}\medskip\par\noindent
\five\:\:{#5}}

\newcommand\vs[1]{\par\vspace{30pt}}

\usepackage{graphicx}

%\pagestyle{empty}

%%% Counters
\newcounter{num}

%%% Commands
\newcommand{\prob}[1]
{\bigskip\bigskip\noindent\refstepcounter{num}\textbf{문제 \arabic{num})} #1\par\noindent}

\newcommand\pb[1]{\ensuremath{\fbox{\phantom{#1}}}}

\newcommand\ba{\ensuremath{\:|\:}}

\newcommand\an[1]{\bigskip\par\noindent\textbf{문제 #1)}\par\noindent}

%%% Meta Commands
\let\oldsection\section
\renewcommand\section{\clearpage\oldsection}

\let\emph\textsf

\begin{document}
\begin{center}
\LARGE태희, 미니테스트 08
\end{center}
\begin{flushright}
날짜 : 2018년 \(\pb3\)월 \(\pb{10}\)일 \(\pb{월}\)요일
,\qquad
제한시간 : \pb{17년}분
,\qquad
점수 : \pb{20} / \pb{20}
\end{flushright}

%
\prob{\(2^{8x}=9\)일 때, \(\displaystyle\frac{2^{6x}-2^{-6x}}{2^{2x}+2^{-2x}}\)의 값은?}
\taba{\(\frac53\)}{\(\frac{11}6\)}2{\(\frac{13}6\)}{\(\frac73\)}
\vs

%
\prob{\(\log_7(\log_3(\log_2x))=0\)일 때, \(x\)의 값은?}
\taba2468{10}
\vs


%
\prob{\(12^x=18^y=6\)일 때, \(\frac1x+\frac1y\)의 값은?}
\taba12345
\vs

%
\prob{\(4^a+4^{-a}=14\)일 때, \(\displaystyle\frac{2^{6a}+1}{2^{4a}+2^{2a}}\)의 값을 구하시오.}
\vs

%
\prob{\(9^a+9^{-a}=11\)일 때, \(\displaystyle\frac{3^{8a}-1}{3^{5a}-3^{3a}}\)의 값을 구하시오.
(단, \(a>0\))}
\vs

%
\prob{\(a\), \(b\)가 실수일 때, 다음 <보기> 중 옳은 것을 모두 고른 것은?}
\begin{mdframed}[frametitle=<보기>]
\begin{enumerate}
\item[ㄱ.]
\(a^2-ab+b^2\ge0\)
\item[ㄴ.]
\(|a+b|\ge|a-b|\)
\item[ㄷ.]
\(a\ge0\), \(b\ge0\)일 때, \(\sqrt a+\sqrt b\ge\sqrt{a+b}\)
\end{enumerate}
\end{mdframed}
\tabb{\text{ㄱ}}{\text{ㄴ}}{\text{ㄱ, ㄴ}}{\text{ㄱ, ㄷ}}{\text{ㄴ, ㄷ}}
\vs

\newpage
%
\prob{다음 <보기>에서 옳은 것만을 있는 대로 고른 것은?}
\begin{mdframed}[frametitle=<보기>]
\begin{enumerate}
\item[ㄱ.]
실수 \(a\), \(b\)에 대하여 \(|a+b|\ge|a|-|b|\)
\item[ㄴ.]
\(a\ge b\ge0\)이면 \(\sqrt{a-b}\ge\sqrt a-\sqrt b\)
\item[ㄷ.]
\(a\), \(b\), \(c\)가 양수이면 \(\frac{b+c}a+\frac{c+a}b+\frac{a+b}c\ge6\)
\end{enumerate}
\end{mdframed}
\tabb{\text{ㄱ}}{\text{ㄷ}}{\text{ㄱ, ㄴ}}{\text{ㄴ, ㄷ}}{\text{ㄱ, ㄴ, ㄷ}}
\vs

%
\prob{실수 \(x\), \(y\)에 대하여 \(x^2+y^2=13\)일 때, \(2x-3y\)의 최댓값은 \(M\)이고 최솟값은 \(m\)이다.
\(M-m\)의 값은?}
\taba{13}{26}{39}{52}{65}
\vs

%
\prob{두 함수 \(f(x)=3x-1\), \(g(x)=-2x+k\)에 대하여 \(f\circ g=g\circ f\)가 성립할 때, \((g^{-1}\circ g^{-1})\left(\frac52\right)\)의 값을 구하여라.}
\taba12345
\vs


%
\prob{함수 \(f(x)=2|x-2|+(k+1)x+1\)가 일대일대응이 되도록 하는 실수 \(k\)의 값의 범위는?}
\tabb
{\(-1<k<3\)}
{\(k<-1\) 또는 \(k>3\)}
{\(-2<k<2\)}
{\(k<-3\) 또는 \(k>1\)}
{\(-3<k<1\)}
\vs

%
\prob{두 집합 \(X=\{x\ba 1\le x\le 3\}\), \(Y=\{y\ba a\le y\le b\}\)에 대하여 \(X\)에서 \(Y\)로의 함수 \(f(x)=2^{x-1}-1\)의 역함수가 존재할 때, \(a+b\)의 값을 구하시오. (단, \(a\), \(b\)는 상수)}
\taba12345
\vs

%
\prob{두 집합 \(X=\{x\ba a\le x\le 4\}\), \(Y=\{y\ba -4\le y\le4\}\)에 대하여 \(X\)에서 \(Y\)로의 함수 \(f(x)=-x^2+2x+b\)의 역함수가 존재할 때, \(a+b\)의 값을 구하시오. (단, \(a\), \(b\)는 상수)}
\taba34567
\vs
 





\end{document}