\documentclass[a4paper]{oblivoir}
\usepackage{amsmath,amssymb,kotex,mdframed,paralist,tabu}
\usepackage{fapapersize}
\usefapapersize{210mm,297mm,20mm,*,20mm,*}

\usepackage{tabto,pifont}
\TabPositions{0.2\textwidth,0.4\textwidth,0.6\textwidth,0.8\textwidth}

%%% 객관식 선지
\newcommand\one{\ding{172}}
\newcommand\two{\ding{173}}
\newcommand\three{\ding{174}}
\newcommand\four{\ding{175}}
\newcommand\five{\ding{176}}
\usepackage{tabto,pifont}
%\TabPositions{0.2\textwidth,0.4\textwidth,0.6\textwidth,0.8\textwidth}

\newcommand\taba[5]{\par\noindent
\one\:{#1}
\tabto{0.2\textwidth}\two\:\:{#2}
\tabto{0.4\textwidth}\three\:\:{#3}
\tabto{0.6\textwidth}\four\:\:{#4}
\tabto{0.8\textwidth}\five\:\:{#5}}

\newcommand\tabb[5]{\par\noindent
\one\:{#1}
\tabto{0.33\textwidth}\two\:\:{#2}
\tabto{0.67\textwidth}\three\:\:{#3}\medskip\par\noindent
\four\:\:{#4}
\tabto{0.33\textwidth}\five\:\:{#5}}

\newcommand\tabc[5]{\par\noindent
\one\:{#1}
\tabto{0.5\textwidth}\two\:\:{#2}\medskip\par\noindent
\three\:\:{#3}
\tabto{0.5\textwidth}\four\:\:{#4}\medskip\par\noindent
\five\:\:{#5}}

\newcommand\tabd[5]{\par\noindent
\one\:{#1}\medskip\par\noindent
\two\:\:{#2}\medskip\par\noindent
\three\:\:{#3}\medskip\par\noindent
\four\:\:{#4}\medskip\par\noindent
\five\:\:{#5}}

\newcommand\vs[1]{\par\vspace{20pt}}

\usepackage{graphicx}

\pagestyle{empty}

%%% Counters
\newcounter{num}

%%% Commands
\newcommand{\prob}[1]
{\bigskip\bigskip\noindent\refstepcounter{num}\textbf{문제 \arabic{num})} #1\par\noindent}

\newcommand\pb[1]{\ensuremath{\fbox{\phantom{#1}}}}

\newcommand\ba{\ensuremath{\:|\:}}

\newcommand\an[1]{\bigskip\par\noindent\textbf{문제 #1)}\par\noindent}

%%% Meta Commands
\let\oldsection\section
\renewcommand\section{\clearpage\oldsection}

\let\emph\textsf

\begin{document}
\begin{center}
\LARGE태희, 미니테스트 12
\end{center}
\begin{flushright}
날짜 : 2018년 \(\pb3\)월 \(\pb{10}\)일 \(\pb{월}\)요일
,\qquad
제한시간 : \pb{17년}분
,\qquad
점수 : \pb{20} / \pb{20}
\end{flushright}

%
\prob{\(a_1=1\), \(a_{n+1}=a_n+n\)일 때(\(n=1,2,3,\cdots\)), \(a_n\)을 구하여라.}
\vs

%
\prob{\(a_1=3\), \(a_{n+1}=a_n+2n+3\)일 때(\(n=1,2,3,\cdots\)), \(a_n\)을 구하여라.}
\vs

%
\prob{\(a_1=1\), \(a_{n+1}=\frac{n+3}{n+1}a_n\)일 때(\(n=1,2,3,\cdots\)), \(a_n\)을 구하여라.}
\vs

%
\prob{\(a_1=1\), \(a_{n+1}=2^na_n\)일 때(\(n=1,2,3,\cdots\)), \(a_n\)을 구하여라.}
\vs

%
\prob{\(a_1=2\), \(a_{n}=\left(1-\frac1{n^2}\right)a_{n-1}\)일 때(\(n=2,3,4,\cdots\)), \(a_n\)을 구하여라.}
\vs

%
\prob{\(a_1=1\), \(a_2=2\), \(3a_{n+2}-4a_{n+1}+a_n=0\)일 때(\(n=1,2,3,\cdots\)), \(a_n\)을 구하여라.}
\vs

%
\prob{\(a_1=3\), \(a_2=6\), \(a_{n+2}=3a_{n+1}-2a_n\)일 때(\(n=1,2,3,\cdots\)), \(a_n\)을 구하여라.}
\vs

%
\prob{\(a_1=2\), \(a_{n+1}=2a_n+1\)일 때(\(n=1,2,3,\cdots\)), \(a_n\)을 구하여라.}
\vs

%
\prob{\(a_1=1\), \(2a_{n+1}-a_n+2=0\)일 때(\(n=1,2,3,\cdots\)), \(a_n\)을 구하여라.}
\vs

\end{document}