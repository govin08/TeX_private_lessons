\documentclass{oblivoir}
\usepackage{amsmath,amssymb,amsthm,kotex,paralist,kswrapfig,tabu}

\usepackage[skipabove=10pt,innertopmargin=10pt,nobreak]{mdframed}

\usepackage{tabto,pifont}
\TabPositions{0.2\textwidth,0.4\textwidth,0.6\textwidth,0.8\textwidth}
\newcommand\tabb[5]{\par\bigskip\noindent
\ding{172}\:{\ensuremath{#1}}
\tab\ding{173}\:\:{\ensuremath{#2}}
\tab\ding{174}\:\:{\ensuremath{#3}}
\tab\ding{175}\:\:{\ensuremath{#4}}
\tab\ding{176}\:\:{\ensuremath{#5}}}

\usepackage{enumitem}
\setlist[enumerate]{label=(\arabic*)}

\newcounter{num}
\newcommand{\defi}[1]
{\noindent\refstepcounter{num}\textbf{정의 \arabic{num}) #1}\par\noindent}
\newcommand{\theo}[1]
{\noindent\refstepcounter{num}\textbf{정리 \arabic{num}) #1}\par\noindent}
\newcommand{\exam}[1]
{\bigskip\bigskip\noindent\refstepcounter{num}\textbf{예시 \arabic{num}) #1}\par\noindent}
\newcommand{\prob}[1]
{\bigskip\bigskip\noindent\refstepcounter{num}\textbf{문제 \arabic{num}) #1}\par\noindent}
\newcommand{\proo}
{\bigskip\textsf{증명)}\par}

\newcommand{\procedure}[1]{\begin{mdframed}\vspace{#1\textwidth}\end{mdframed}}

\newcommand{\ans}{
{\par
\raggedleft\textbf{답 : (\qquad\qquad\qquad\qquad\qquad\qquad)}
\par}\bigskip}

\newcommand{\pb}[1]%\Phantom + fBox
{\fbox{\phantom{\ensuremath{#1}}}}

\newcommand\ba{\,|\,}

\newcommand\an[1]{\par\bigskip\noindent\textbf{문제 #1)}\\}

\let\oldsection\section
\renewcommand\section{\clearpage\oldsection}

\renewcommand{\arraystretch}{1.5}

\let\emph\textsf
%%%%
\begin{document}

\title{준영 : 07 지수}
\author{}
\date{\today}
\maketitle
\tableofcontents
\newpage

%%
\section{복습}

%
\prob{다항식의 전개}
다음 식을 전개하시오.
\[(a+b)(a^2-ab+b^2)\]
\vspace{-20pt}
\begin{mdframed}
\vspace{0.05\textheight}
\end{mdframed}
\ans

%
\prob{인수분해}
다음 식을 인수분해하시오.
\par\noindent
(1)\:\:\(a^3-b^3\)
\tabto{.5\textwidth}
(2)\:\:\(x^3-27\)
\par\noindent
(3)\:\:\(x^2-1\)
\tabto{.5\textwidth}
(4)\:\:\(x^4-16\)
\begin{mdframed}
\vspace{0.3\textheight}
\end{mdframed}

\begin{mdframed}[frametitle={기본적인 인수분해 공식}]
\begin{enumerate}
\item
\(a^2-b^2=(a+b)(a-b)\)
\item
\(a^3+b^3=(a+b)(a^2-ab+b^2)\)
\item
\(a^3-b^3=(a-b)(a^2+ab+b^2)\)
\end{enumerate}
\end{mdframed}

%
\prob{이차방정식}
다음 이차방정식을 푸시오.
\par\noindent
(1)\:\:\(x^2-x-2=0\)
\tabto{.333\textwidth}
(2)\:\:\(x^2-x-1=0\)
\tabto{.666\textwidth}
(3)\:\:\(x^2-x+2=0\)
\par\noindent
(4)\:\:\(x^2=4\)
\tabto{.333\textwidth}
(5)\:\:\(x^2=0\)
\tabto{.666\textwidth}
(6)\:\:\(x^2=-4\)
\begin{mdframed}
\vspace{0.45\textheight}
\end{mdframed}
{\par\raggedleft\textbf{답 :
(1)\:\:\(x=\)\qquad\qquad(2)\:\:\(x=\)\qquad\qquad(3)\:\:\(x=\)\qquad\qquad\qquad}
\par}
{\par\raggedleft\textbf{
(4)\:\:\(x=\)\qquad\qquad(5)\:\:\(x=\)\qquad\qquad(6)\:\:\(x=\)\qquad\qquad\qquad}
\par}
\bigskip

\begin{mdframed}[frametitle={이차방정식의 풀이}]
\begin{enumerate}
\item
이차방정식 \(x^2=A\)의 근은
\[x=\pm\sqrt A\]
\item
이차방정식 \(ax^2+bx+c=0\)의 근은
\[x=\frac{-b\pm\sqrt{b^2-4ac}}2\]
\end{enumerate}
\end{mdframed}
%%
\section{거듭제곱근}

%
\exam{3의 제곱근}
제곱해서 3이 되는 수를 `3의 \emph{제곱근}'이라고 한다.
즉 \(x^2=3\)의 근인데 이것을 풀면
\begin{gather*}
x^2=3\\
x^2-3=0\\
(x+\sqrt3)(x-\sqrt3)=0\\
x=-\sqrt3,\sqrt3
\end{gather*}
%\(x=\pm\sqrt3\)이다.
그러므로 3의 제곱근은 \(\sqrt3\)와 \(-\sqrt3\)의 두 개이다.

\begin{mdframed}
\[
3\text{의 제곱근}
\quad\Rightarrow\quad
\text{제곱해서 3이 되는 수}
\quad\Rightarrow\quad
x^2=3
\]
\end{mdframed}

%
\exam{\(-3\)의 제곱근}
\(-3\)의 제곱근은 제곱해서 \(-3\)이 되는 수이다.
즉 \(x^2=-3\)의 근인데 이것을 풀면
\begin{gather*}
x=\pm\sqrt{-3}=\pm\sqrt3i
\end{gather*}
이다.
따라서 \(-3\)의 제곱근은 \(\sqrt3i\)와 \(-\sqrt3i\)의 두 개이다.

\begin{mdframed}
\[
-3\text{의 제곱근}
\quad\Rightarrow\quad
\text{제곱해서 \(-3\)이 되는 수}
\quad\Rightarrow\quad
x^2=-3
\]
\end{mdframed}

\clearpage
%
\prob{}
\begin{enumerate}
\item
\(4\)의 제곱근을 구하여라.
\procedure{0.18}
\ans
\item
\(8\)의 제곱근을 구하여라.
\procedure{0.18}
\ans
\item
\(0\)의 제곱근을 구하여라.
\procedure{0.18}
\ans
\item
\(-4\)의 제곱근을 구하여라.
\procedure{0.18}
\ans
\end{enumerate}

%
\exam{8의 세제곱근}
세제곱해서 8이 되는 수를 8의 \emph{세제곱근}이라고 한다.
즉 \(x^3=8\)의 근인데 이것을 풀면
\begin{gather*}
x^3=8\\
x^3-8=0\\
(x-2)(x^2+2x+4)=0\\
x=2\text{ 이거나 }x^2+2x+4=0
\end{gather*}
\(x^2+2x+4=0\)에서 근의공식을 쓰면
\[
x=\frac{-2\pm\sqrt{2^2-4\cdot1\cdot4}}2=\frac{-2\pm\sqrt{-12}}2
=\frac{-2\pm\sqrt{12}i}2=\frac{-2\pm2\sqrt3i}2=-1\pm\sqrt3i
\]
이다.
따라서 \(8\)의 세제곱근은 \(2\), \(-1+\sqrt3i\), \(1+\sqrt3i\)의 세 개이다.

\begin{mdframed}
\[
8\text{의 세제곱근}
\quad\Rightarrow\quad
\text{세제곱해서 8이 되는 수}
\quad\Rightarrow\quad
x^3=8
\]
\end{mdframed}

%
\prob{}
\begin{enumerate}
\item
\(-8\)의 세제곱근을 구하여라.
\vspace{0.1\textwidth}
\ans
\item
\(27\)의 세제곱근을 구하여라.
\vspace{0.1\textwidth}
\ans
\item
\(-1\)의 세제곱근을 구하여라.
\vspace{0.1\textwidth}
\ans
\end{enumerate}

%
\exam{81의 네제곱근}
네제곱해서 \(81\)이 되는 수를 \(81\)의 \emph{네제곱근}이라고 한다.
즉 \(x^4=81\)의 근인데 이것을 풀면
\begin{gather*}
x^4=81\\
x^4-81=0\\
(x^2-9)(x^2+9)=0\\
(x-3)(x+3)(x^2+9)=0\\
x=3,\:-3\text{ 이거나 }x^2=-9
\end{gather*}
\(x^2=-9\)이면 \(x=\pm\sqrt{-9}=\pm\sqrt9i=\pm3i\)
따라서 \(81\)의 네제곱근은 \(3\), \(-3\), \(3i\), \(-3i\)의 네 개이다.

\begin{mdframed}
\[
81\text{의 네제곱근}
\quad\Rightarrow\quad
\text{네제곱해서 \(81\)이 되는 수}
\quad\Rightarrow\quad
x^4=81
\]
\end{mdframed}

%
\prob{}
\begin{enumerate}
\item
\(16\)의 네제곱근을 구하여라.
\procedure{0.15}
\ans
\item
\(1\)의 네제곱근을 구하여라.
\procedure{0.15}
\ans
\end{enumerate}

%%
%\begin{mdframed}
%\defi{거듭제곱근}
%제곱해서 \(a\)가 되는 수를 `\(a\)의 \emph{\(n\)제곱근}'이라고 한다.
%\end{mdframed}

%\bigskip
%거듭제곱근 중에 실수인 것의 개수를 구해보자.
%
%%
%\exam{제곱근 중에 실수인 것의 개수}
%아까 구한 예에서
%\begin{enumerate}
%\item
%\(3\)의 제곱근 중 실수인 것은 \(\sqrt3\)과 \(-\sqrt3\)의 두 개였다.
%\item
%\(-3\)의 제곱근 중 실수인 것은 0개였다.
%\item
%\(4\)의 제곱근 중 실수인 것은 \(2\)와 \(-2\)의 두 개였다.
%\item
%\(8\)의 제곱근 중 실수인 것은 \(2\sqrt2\)와 \(-2\sqrt2\)의 두 개였다.
%\item
%\(0\)의 제곱근 중 실수인 것은 \(0\)의 한 개였다.
%\end{enumerate}
%이것을 정리하면
%
%\begin{mdframed}[skipbelow=10pt]
%\(a\)의 제곱근 중 실수인 것은
%\(
%\begin{cases}
%a>0\text{ 이면 2개}\\
%a=0\text{ 이면 1개}\\
%a<0\text{ 이면 0개}
%\end{cases}
%\)
%이다.
%\end{mdframed}
%
%\(a>0\)일때, 두 제곱근을 각각 \(\sqrt a\), \(-\sqrt a\)라고 쓴다.
%그러므로 \(4\)의 제곱근이 \(2\)와 \(-2\)이므로 \(\sqrt4=2\)이다.
%
%%
%\exam{세제곱근 중에 실수인 것의 개수}
%아까 구한 예에서
%\begin{enumerate}
%\item
%\(8\)의 세제곱근 중 실수인 것은 \(2\)의 한 개였다.
%\item
%\(-8\)의 세제곱근 중 실수인 것은 \(-2\)의 한 개였다.
%\item
%\(27\)의 세제곱근 중 실수인 것은 \(3\)의 한 개였다.
%\item
%\(-1\)의 세제곱근 중 실수인 것은 \(-1\)의 한 개였다.
%\end{enumerate}
%이것을 정리하면
%
%\begin{mdframed}[skipbelow=10pt]
%\(a\)의 세제곱근 중 실수인 것은 1개이다.
%\end{mdframed}
%
%하나밖에 없는 이 세제곱근을 각각 \(\sqrt[3]a\)라고 쓴다.
%그러므로
%\begin{enumerate}
%\item
%\(8\)의 세제곱근이 \(2\)이므로 \(\sqrt[3]8=2\)이다.
%\item
%\(-8\)의 세제곱근이 \(-2\)이므로 \(\sqrt[3]{-8}=-2\)이다.
%\item
%\(1\)의 세제곱근이 \(1\)이므로 \(\sqrt[3]1=1\)이다.
%\end{enumerate}
%
%또한
%\[\sqrt[3]2\]
%와 같은 표현도 쓸 수 있다.
%이것은 `\(2\)의 세제곱근', `세제곱해서 2가 되는 수'라는 뜻이다.
%
%
%%
%\exam{네제곱근 중에 실수인 것의 개수}
%아까 구한 예에서
%\begin{enumerate}
%\item
%\(81\)의 네제곱근 중 실수인 것은 \(3\)과 \(-3\)의 두 개였다.
%\item
%\(16\)의 네제곱근 중 실수인 것은 \(2\)와 \(-2\)으 두 개였다.
%\item
%\(1\)의 네제곱근 중 실수인 것은 \(1\)와 \(-1\)의 두 개였다.
%\end{enumerate}
%또 조금 더 생각해보면, \(0\)의 네제곱근 중 실수인 것은 \(0\)의 한 개이고, 음수의 네제곱근 중 실수인 것은 없을 것이다.
%이것을 정리하면
%
%\begin{mdframed}[skipbelow=10pt]
%\(a\)의 제곱근 중 실수인 것은
%\(
%\begin{cases}
%a>0\text{ 이면 2개}\\
%a=0\text{ 이면 1개}\\
%a<0\text{ 이면 0개}
%\end{cases}
%\)
%이다.
%\end{mdframed}
%
%\(a>0\)일때, 두 네제곱근을 각각 \(\sqrt[4]a\), \(-\sqrt[4]a\)라고 쓴다.
%그러므로 \(81\)의 네제곱근이 \(3\)와 \(-3\)이므로 \(\sqrt[4]{81}=3\)이다.
%
%또한
%\[\sqrt[4]2\]
%와 같은 표현도 쓸 수 있다.
%이것은 `\(2\)의 네제곱근 중 양수', `네제곱해서 2가 되는 양수'라는 뜻이다.

\clearpage
이번에는 \(\sqrt2\), \(\sqrt3\)에서 쓰이는 근호(루트)와 비슷한 기호를 알아보자.

%이처럼 \(n\)이 짝수일 때와 홀수일 때, \(\sqrt[n]{~~}\)의 의미가 조금씩 다르다.
\begin{mdframed}
%
\defi{}
\(\sqrt[n]a=
\begin{cases}
\text{\(a\)의 \(n\)제곱근 중 양수}	&(n=짝수)\\
\text{\(a\)의 \(n\)제곱근 중 실수}	&(n=홀수)
\end{cases}\)
\end{mdframed}

%
\exam{}
\begin{enumerate}
\item
\(\sqrt4\)\는 `\(4\)의 제곱근 중 양수인 것'을 뜻한다.
즉 \(x^2=4\)의 두 근 \(-2\), \(2\) 중 양수인 \(2\)를 뜻하므로
\[\sqrt4=2\]
\item
\(\sqrt[3]8\)은 `\(8\)의 세제곱근 중 실수인 것'을 뜻한다.
즉 \(x^3=8\)의 세 근 \(2\), \(1+\sqrt3i\), \(-1+\sqrt3i\) 중 실수인 \(2\)를 뜻하므로
\[\sqrt[3]8=2\]
\item
\(\sqrt[4]{16}\)은  `\(16\)의 네제곱근 중 양수인 것'을 뜻한다.
즉 \(x^4=16\)의 네 근 \(2\), \(-2\), \(2i\), \(-2i\) 중 양수인 \(2\)를 뜻하므로
\[\sqrt[4]{16}=2\]
\end{enumerate}

%일반적으로
%%
%\begin{mdframed}
%\defi{\(a\)의 \(n\)제곱근 중 실수인 것}
%\begin{center}
%\begin{tabu}{X[1,c]|X[1,c]|X[1,c]|X[1,c]}
%\hline
%				&\(a>0\)						&\(a=0\)	&\(a<0\)\\\hline
%\(n\)이 홀수일 때	&\(\sqrt[n]a\)				&0		&\(\sqrt[n]a\)\\\hline
%\(n\)이 짝수일 때	&\(\sqrt[n]a\), \(\sqrt[n]a\)	&0		&없다
%\end{tabu}
%\end{center}
%\end{mdframed}

\clearpage
%
\exam{}
다음 값을 구하여라.
\par\noindent
(1)\:\:\(\sqrt[3]{-27}\)
\tabto{.25\textwidth}
(2)\:\:\(\sqrt[5]{100000}\)
\tabto{.5\textwidth}
(3)\:\:\(\sqrt[4]{\frac{16}{81}}\)
\tabto{.75\textwidth}
(4)\:\:\(-\sqrt[4]{0.0625}\)
\begin{mdframed}
\begin{enumerate}
\item
\(\sqrt[3]{-27}\)은 \(-27\)의 세제곱근 중 실수인 \(-3\)이다.
\item
\(\sqrt[5]{100000}\)은 \(100000\)의 다섯제곱근 중 실수인 \(10\)이다.
\item
\(\sqrt[4]{\frac{16}{81}}\)은 \(\frac{16}{81}\)의 네제곱근 중 양수인 \(\frac23\)이다.
\item
\(\sqrt[4]{0.0625}\)은 \(0.0625\)의 네제곱근 중 양수인 \(\frac12\)이다.
따라서 \(-\sqrt[4]{0.0625}=-\frac12\)
\end{enumerate}
\end{mdframed}
{\par\raggedleft\textbf{답 :
(1)\:\:\(\sqrt[3]{-27}=-3\)\quad(2)\:\:\(\sqrt[5]{100000}=10\)}\par}
{\par\raggedleft\textbf{
(3)\:\:\(\sqrt[4]{\frac{16}{81}}=\frac23\)\quad(4)\:\:\(-\sqrt[4]{0.0625}=-\frac12\)}\par}\bigskip

%
\prob{}
다음 값을 구하여라.
\par\noindent
(1)\:\:\(\sqrt[5]{32}\)
\tabto{.25\textwidth}
(2)\:\:\(\sqrt[3]{0.008}\)
\tabto{.5\textwidth}
(3)\:\:\(\sqrt[3]{-8}\)
\tabto{.75\textwidth}
(4)\:\:\(-\sqrt[4]{0.0001}\)
\procedure{0.4}
{\par\raggedleft\textbf{답 :
(1)\:\:\qquad\qquad
(2)\:\:\qquad\qquad\qquad}\par}
{\par\raggedleft\textbf{
(3)\:\:\qquad\qquad
(4)\:\:\qquad\qquad\qquad}\par}\bigskip

%%
\section{지수의 확장과 지수법칙}

\[a^x\]

와 같이 생긴 것을 \emph{거듭제곱}이라고 부른다.
이때 \(a\)를 \emph{밑}, \(x\)를 \emph{지수}라고 부른다.
이러한 지수에 대해 다음 법칙이 항상 성립한다.

%
\begin{mdframed}
\theo{지수법칙}
다음이 성립한다.
\begin{enumerate}
\item
\(a^x\times a^y=a^{x+y}\)
\item
\(a^x\div a^y=a^{x-y}\)
\item
\((a^x)^y=a^{xy}\)
\item
\((ab)^x=a^xb^y\)
\item
\(\left(\frac ab\right)^x=\frac{a^x}{b^x}\)
\end{enumerate}
\end{mdframed}

%지금까지는 지수가 자연수였다.
%\(x\)에 \(0\)이나 음의 정수, 유리수나 무리수를 넣을 수는 없었다.
%이 단원에서는 지수의 범위를 자연수, 정수, 유리수, 실수 순으로 확장해나간다.

%%
\subsection{정수 지수}

%
\exam{}\label{infer}
다음 나열된 수를 보고 빈칸에 들어갈 수를 유추해보자.
\begin{align*}
2^4		&=16,\\
2^3		&=8,\\
2^2		&=4,\\
2^1		&=2,\\
2^0		&=\fbox{?}\\
2^{-1}	&=\fbox{?}\\
2^{-2}	&=\fbox{?}
\end{align*}
\(16\), \(8\), \(4\), \(2\)는 공비가 \(\frac12\)인 등비수열을 이루고 있으므로 그 다음에 나올 수는 차례로 \(1\), \(\frac12\), \(\frac14\)가 되면 자연스러울 것임을 알 수 있다.
\[2^0=1,\quad2^{-1}=\frac12,\quad2^{-2}=\frac14\]

%
\begin{mdframed}
\defi{정수인 지수}
\(a\neq0\)이고 \(n\)이 자연수이면
\begin{enumerate}
\item
\(a^0=1\)
\item
\(a^{-n}=\frac1{a^n}\)
\end{enumerate}
\end{mdframed}

%
\exam{}
\begin{enumerate}
\item
\(2^0=1\)
\item
\((-3)^0=1\)
\item
\(2^{-3}=\frac1{2^3}=\frac18\)
\end{enumerate}

%
\prob{}
다음 값을 구하여라.
\par\noindent
(1)\:\:\((\sqrt3)^2\)
\tabto{.25\textwidth}
(2)\:\:\(\left(\frac23\right)^0\)
\tabto{.5\textwidth}
(3)\:\:\((-2)^{-3}\)
\tabto{.75\textwidth}
(4)\:\:\(\left(\frac12\right)^{-1}\)
\procedure{0.4}
{\par\raggedleft\textbf{답 :
(1)\:\:\qquad\qquad
(2)\:\:\qquad\qquad
(3)\:\:\qquad\qquad
(4)\:\:\qquad\qquad\qquad}\par}\bigskip

%
\exam{}
\begin{enumerate}
\item
\(3^{-3}\div3^{-2}\times 9^2=3^{(-3)-(-2)}\times (3^2)^2=3^{-1}\times3^4=3^{-1+4}=3^3\)
\item
\((25^4\times125^{-3})^{-2}=\left((5^2)^4\times(5^3)^{-3}\right)^{-2}=(5^8\times5^{-9})^{-2}
=(5^{-1})^{-2}=5^2=25\)
\end{enumerate}

%
\prob{}
다음을 간단히 하여라
\par\noindent
(1)\:\:\(2^4\times4^{-1}\div6^2\)
\tabto{.5\textwidth}
(2)\:\:\((3^3\times9^{-2})^{-1}\)
\procedure{0.4}
{\par\raggedleft\textbf{답 :
(1)\:\:\qquad\qquad
(2)\:\:\qquad\qquad\qquad}\par}


%%
%\prob{}
%다음 식을 전개하여라.
%\par\noindent
%(1)\:\:\((a+a^{-1})^2\)
%\tabto{.5\textwidth}
%(2)\:\:\((a+a^{-1})^3\)
%\procedure{0.4}
%{\par\raggedleft\textbf{답 :
%(1)\:\:\((a+a^{-1})^2=\)\qquad\qquad
%(2)\:\:\((a+a^{-1})^3=\)\qquad\qquad\qquad}\par}

%%
\subsection{유리수 지수}

%
\exam{}
예시 \ref{infer}\과 같은 추론을 다시 해보자.
\begin{align*}
2^2		&=4,\\
2^{1.5}	&=\fbox{?},\\
2^1		&=2,\\
2^{0.5}	&=\fbox{?},\\
2^0		&=1
\end{align*}

\(2^0\), \(2^1\), \(2^2\)가 등비수열을 이루므로\(2^0\), \(2^0.5\), \(2^1\), \(2^{1.5}\), \(2^2\)도 등비수열을 이루어야 자연스러울 것이다.
\(2^{0,5}\)는 \(2^0=1\)\과 \(2^1=2\)의 등비중항이므로 \(2^{0.5}=\sqrt{1\times 2}=\sqrt2\).
\(0.5=\frac12\)이므로
\[2^{\frac12}=\sqrt2\]
이다.
마찬가지로
\[2^{\frac13}=\sqrt[3]2,\quad 2^{\frac14}=\sqrt[4]2,\quad\cdots\]
등이다.

%
\begin{mdframed}
\defi{유리수인 지수}
\(a>0\)이고 \(m\)은 정수, \(n\)은 \(2\)이상의 정수일 때,
\begin{enumerate}
\item
\(a^{\frac1n}=\sqrt[n]a\)
\item
\(a^{\frac mn}=\sqrt[n]{a^m}\)
\end{enumerate}
\end{mdframed}

%
\prob{}
다음을 근호를 사용하여 나타내어라.
\par\noindent
(1)\:\:\(2^{\frac34}\)
\tabto{.25\textwidth}
(2)\:\:\(3^{0.5}\)
\tabto{.5\textwidth}
(3)\:\:\(5^{-\frac32}\)
\tabto{.75\textwidth}
(4)\:\:\(7^{-1.2}\)
\procedure{0.33}
{\par\raggedleft\textbf{답 :
(1)\:\:\qquad\qquad
(2)\:\:\qquad\qquad
(3)\:\:\qquad\qquad
(4)\:\:\qquad\qquad\qquad}\par}\bigskip

%
\prob{}
다음을 \(a^{\frac mn}\)의 꼴로 나타내어라.
\par\noindent
(1)\:\:\(\sqrt[3]{2^4}\)
\tabto{.25\textwidth}
(2)\:\:\(\sqrt{3^4}\)
\tabto{.5\textwidth}
(3)\:\:\(\sqrt[4]{5^{-3}}\)
\tabto{.75\textwidth}
(4)\:\:\(\frac1{\sqrt[5]{7^3}}\)
\procedure{0.33}
{\par\raggedleft\textbf{답 :
(1)\:\:\qquad\qquad
(2)\:\:\qquad\qquad
(3)\:\:\qquad\qquad
(4)\:\:\qquad\qquad\qquad}\par}\bigskip

%
\prob{}
다음을 간단히 하여라.
\par\noindent
(1)\:\:\(\sqrt[3]{2^4}\times2^{\frac23}\)
\tabto{.5\textwidth}
(2)\:\:\((\sqrt[3]3)^2\div3^{-\frac13}\)
\par\noindent
(3)\:\:\(12^{\frac13}\times\sqrt[3]9\times4^{-\frac13}\)
\tabto{.5\textwidth}
(4)\:\:\(\sqrt[3]2\div4^{\frac13}\times8^{\frac19}\)
\procedure{0.4}
{\par\raggedleft\textbf{답 :
(1)\:\:\qquad\qquad
(2)\:\:\qquad\qquad
(3)\:\:\qquad\qquad
(4)\:\:\qquad\qquad\qquad}\par}\bigskip
%
\exam{}
\(\sqrt[4]{a^3b}\times\sqrt{ab}\div\sqrt[4]{ab^3}\)을 간단히 하여라.
(단, \(a>0\), \(b>0\))
\begin{mdframed}
\begin{align*}
\sqrt[4]{a^3b}\times\sqrt{ab}\div\sqrt[4]{ab^3}
&=(a^3b)^{\frac14}\times(ab)^{\frac12}\div(ab^3)^{\frac14}\\
&=\left(a^{\frac34}b^{\frac14}\right)\times\left(a^{\frac12}b^{\frac12}\right)
\div\left(a^{\frac14}b^{\frac34}\right)\\
&=a^{\frac34+\frac12-\frac14}b^{\frac14+\frac12-\frac34}\\
&=a^1b^0\\
&=a
\end{align*}
\end{mdframed}
{\par\raggedleft\textbf{답 : \(a\)}\par}\bigskip

\clearpage
%
\prob{}
다음을 간단히 하여라.
(단, \(a>0\))
\par\noindent
(1)\:\:\(a^{\frac12}\div\left(a^{-\frac12}\right)^4\)
\tabto{.5\textwidth}
(2)\:\:\(\sqrt{\sqrt[6]{a}\times\sqrt{a^5}}\)
\par\noindent
(3)\:\:\(\left(a^{\frac12}+a^{-\frac12}\right)\left(a^{\frac12}-a^{-\frac12}\right)\)
\tabto{.5\textwidth}
(4)\:\:\(\left(a^{\frac13}+a^{-\frac13}\right)
\left(a^{\frac23}-a^{\frac13}a^{-\frac13}+a^{\frac23}\right)\)
\procedure{0.25}
{\par\raggedleft\textbf{답 :
(1)\:\:\qquad\qquad
(2)\:\:\qquad\qquad
(3)\:\:\qquad\qquad
(4)\:\:\qquad\qquad\qquad}\par}\bigskip

%%
\subsection{실수인 지수}
\vspace{-30pt}
%
\exam{}
지수의 범위를 실수까지 확장시켜보자.

예를 들어 \(3^{\sqrt2}\)에 대하여 알아보자.
무리수 \(\sqrt2\)는 \(\sqrt2=1.414213\cdots\)이므로 \(\sqrt2\)에 가까워지는 유리수
\[1,\quad1.4\quad1.41,\quad1.414,\quad1.4142,\quad1.41421,\quad\cdots\]
을 지수로 하는 수들은 일정한 수에 가까워진다는 사실이 알려져있다.
\begin{align*}
3^1			&=3				\\
3^{1.4}		&=4.65553\cdots	\\
3^{1.41}		&=4.70696\cdots	\\
3^{1.414}		&=4.72769\cdots	\\
3^{1.4142}		&=4.72873\cdots	\\
3^{1.41421}	&=4.72878\cdots	\\
3^{1.414213}	&=4.72880\cdots
\end{align*}
이 일정한 수를 \(3^{\sqrt2}\)로 정한다.

이러한 방식으로 무리수 \(x\)에 대해 \(3^x\)를 정할 수 있다.

%
\exam{}
\begin{enumerate}
\item
\(2^{\sqrt2}\times2^{-\sqrt2}=2^{\sqrt2-\sqrt2}=2^0=1\)
\item
\(\left(2^{\sqrt8}\div2^{\sqrt2}\right)^{\sqrt2}=\left(2^{2\sqrt2-\sqrt2}\right)^{\sqrt2}
=(2^{\sqrt2})^{\sqrt2}=2^2=4\)
\end{enumerate}

%
\prob{}
다음을 간단히 하여라.
\par\noindent
(1)\:\:\(2^{\sqrt2}\times4^{\sqrt2}\)
\tabto{.5\textwidth}
(2)\:\:\(\left(\sqrt2\right)^{3\sqrt2}\div2^{\sqrt2}\)
\par\noindent
(3)\:\:\(\left(3^{\sqrt2+1}\right)^{\sqrt2-1}\)
\tabto{.5\textwidth}
(4)\:\:\(\left(2^{\sqrt2}\div3^{\sqrt6}\right)^{\sqrt2}\times9^{\sqrt3}\)
\procedure{0.4}
{\par\raggedleft\textbf{답 :
(1)\:\:\qquad\qquad
(2)\:\:\qquad\qquad
(3)\:\:\qquad\qquad
(4)\:\:\qquad\qquad\qquad}\par}\bigskip

%%
\section{보충·심화 문제}
%
\prob{}
\(64\)의 세제곱근 중 실수인 것을 \(a\), \(\sqrt[4]{81}\)을 \(b\)라고 할 때, \(a+b\)의 값을 구하시오.
\vspace{0.1\textheight}

%
\prob{}
다음을 간단히 하여라.
\par\noindent
(1)\:\:\(\left(-\frac12\right)^{-3}\)
\tabto{.5\textwidth}
(2)\:\:\(2^3\div4\times(-2)^{-2}\)

\vspace{0.1\textheight}
%
\prob{}
다음을 간단히 하여라.
\par\noindent
(1)\:\:\(\displaystyle\frac{\sqrt[3]2}{\sqrt[3]{16}}\)
\tabto{.5\textwidth}
(2)\:\:\(\sqrt{2\sqrt[3]4}\)
\par\noindent
(3)\:\:\(\sqrt{2\sqrt{2\sqrt2}}\)
\tabto{.5\textwidth}
(4)\:\:\(\sqrt{2\sqrt2\sqrt[3]2}\)

\vspace{0.1\textheight}
%
\prob{}
다음 세 수의 대소를 비교하여라.
\begin{mdframed}
\centering
\(\sqrt2,\qquad\sqrt[3]3,\qquad\sqrt[6]6\)
\end{mdframed}

\vspace{0.1\textheight}
%
\prob{}
\(a+a^{-1}=3\)일 때, 다음 식의 값을 구하여라. (단 \(a>0\))
\par\noindent
(1)\:\:\(a^2+a^{-2}\)
\tabto{.5\textwidth}
(2)\:\:\(a^3+a^{-3}\)
\par\noindent
(3)\:\:\(a^{\frac12}+a^{-\frac12}\)
\tabto{.5\textwidth}
(4)\:\:\(a^{\frac32}+a^{-\frac32}\)

\vspace{0.1\textheight}
%
\prob{}
\(2^{x+1}=a\), \(3^x=b\)일 때, \(72^x\)를 \(a\)와 \(b\)에 관한 식으로 나타내어라.

\vspace{0.1\textheight}
%
\prob{}
\(x>0\)이고 \((a-b)(b-c)(c-a)\neq0\)일 때, 다음을 간단히 하여라.
\[
\left(x^{\frac a{a-b}}\right)^{\frac a{c-a}}
\times
\left(x^{\frac b{b-c}}\right)^{\frac b{a-b}}
\times
\left(x^{\frac c{c-a}}\right)^{\frac c{b-c}}
\]

\vspace{0.1\textheight}
%
\prob{}
사람 피부의 겉넓이 \(A\) m\(^2\)는 몸무게 \(W\) kg와 키 \(H\) cm에 대하여 다음 식이 성립한다고 한다.
\[
A=0.007W^{0.425}H^{0.725}
\]
몸무게 \(81\)kg, 키 \(192\)cm인 사람이 피부의 넓이를 \(0.007\times 2^a\times3^b\)로 나타내었을 때, 두 실수 \(a\), \(b\)의 값을 구하여라.

%%
\section*{답}

\begin{minipage}{0.33\textwidth}
%
\an{1}
\(a^3+b^3\)

%
\an{2}
(1) \((a-b)(a^2+ab+b^2)\)\\
(2) \((x-3)(x^2+3x+9)\)\\
(3) \((x+1)(x-1)\)\\
(4) \((x+2)(x-2)(x^2+4)\)

%
\an{3}
(1) \(x=-1\), \(2\)\\
(2) \(x=\frac{1\pm\sqrt5}2\)\\
(3) \(x=\frac{1\pm\sqrt7i}2\)\\
(4) \(x=2\), \(-2\)\\
(5) \(x=0\)\\
(6) \(x=2i\), \(-2i\)

%
\an{6}
(1) \(2\), \(-2\)\\
(2) \(2\sqrt2\), \(-2\sqrt2\)\\
(3) \(0\)\\
(4) \(2i\), \(-2i\)\\

%
\an{8}
(1) \(-2,1\pm\sqrt3i\)\\
(2) \(3,\frac{-3\pm3\sqrt3i}2\)\\
(3) \(-1,\frac{1\pm\sqrt3i}2\)

%
\an{10}
(1) \(\pm2\),\(\pm2i\)\\
(2) \(\pm1\), \(\pm i\)
\end{minipage}
\begin{minipage}{0.33\textwidth}

%
\an{14}
(1) \(2\)\\
(2) \(0.2\)\\
(3) \(-2\)\\
(4) \(-0.1\)

%
\an{19}
(1) \(3\)\\
(2) \(1\)\\
(3) \(-\frac18\)\\
(4) \(2\)

%
\an{21}
(1) \(\frac19\)\\
(2) \(3\)

%
\an{24}
(1) \(\sqrt[4]{2^3}\)\\
(2) \(\sqrt{3}\)\\
(3) \(\frac1{\sqrt{5^3}}\)\\
(4) \(\frac1{\sqrt[5]{7^6}}\)

%
\an{25}
(1) \(2^{\frac43}\)\\
(2) \(3^2\)\\
(3) \(5^{-\frac34}\)\\
(4) \(7^{-\frac35}\)

%
\an{26}
(1) \(4\)\\
(2) \(3\)\\
(3) \(3\)\\
(4) \(1\)
\end{minipage}
\begin{minipage}{0.33\textwidth}

%
\an{28}
(1) \(a^{\frac52}\)\\
(2) \(a^{\frac43}\)\\
(3) \(a-a^{-1}\)\\
(4) \(a+a^{-1}\)

%
\an{31}
(1) \(2^{3\sqrt2}\)\\
(2) \(2^{\frac{\sqrt2}2}\)\\
(3) \(3\)\\
(4) \(4\)

%
\an{32}
\(7\)

%
\an{33}
(1) \(-8\)\\
(2) \(\frac12\)

%
\an{34}
(1) \(\frac12\)\\
(2) \(2^{\frac56}\)\\
(3) \(2^{\frac78}\)\\
(4) \(2^{\frac{11}{12}}\)

%
\an{35}
\(\sqrt[6]6<\sqrt2<\sqrt[3]3\)

%
\an{36}
(1) \(7\)\\
(2) \(18\)\\
(3) \(\sqrt5\)\\
(4) \(2\sqrt5\)
\end{minipage}

%
\an{37}
\(\frac18a^3b^2\)

%
\an{38}
\(\frac1x\)

%
\an{39}
\(a=\frac{87}{20}\), \(b=\frac{97}{40}\)


\end{document}