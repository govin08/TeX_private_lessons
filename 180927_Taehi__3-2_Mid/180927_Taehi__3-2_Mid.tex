\documentclass[a4paper,twocolumn]{oblivoir}
\usepackage{amsmath,amssymb,kotex,kswrapfig,mdframed,paralist,graphicx,tabu}
\usepackage{fapapersize}
%\usefapapersize{210mm,297mm,10mm,*,10mm,*}

\usepackage[inline]{enumitem}
\setlist[enumerate,1]{label=(\arabic*)}

\usepackage{tabto,pifont}
%\TabPositions{0.1\textwidth,0.2\textwidth,0.3\textwidth,0.4\textwidth}
\newcommand\taba[5]{\par\noindent
\ding{172}\:{\ensuremath{#1}}
\tabto{0.1\textwidth}\ding{173}\:\:{\ensuremath{#2}}
\tabto{0.2\textwidth}\ding{174}\:\:{\ensuremath{#3}}
\tabto{0.3\textwidth}\ding{175}\:\:{\ensuremath{#4}}
\tabto{0.4\textwidth}\ding{176}\:\:{\ensuremath{#5}}}

\newcommand\tabb[5]{\par\bigskip\noindent
\ding{172}\:\:{\ensuremath{#1}}
\tabto{0.16\textwidth}\ding{173}\:\:{\ensuremath{#2}}
\tabto{0.33\textwidth}\ding{174}\:\:{\ensuremath{#3}}\medskip\par\noindent
\ding{175}\:\:{\ensuremath{#4}}
\tabto{0.16\textwidth}\ding{176}\:\:{\ensuremath{#5}}}

\newcommand\tabc[5]{\par\bigskip\noindent
\ding{172}\:\:{\ensuremath{#1}}
\tabto{0.25\textwidth}\ding{173}\:\:{\ensuremath{#2}}\medskip\par\noindent
\ding{174}\:\:{\ensuremath{#3}}
\tabto{0.25\textwidth}\ding{175}\:\:{\ensuremath{#4}}\medskip\par\noindent
\ding{176}\:\:{\ensuremath{#5}}}

\newcommand\tabd[5]{\par\bigskip\noindent
\ding{172}\:{#1}\medskip\par\noindent
\ding{173}\:\:{#2}\medskip\par\noindent
\ding{174}\:\:{#3}\medskip\par\noindent
\ding{175}\:\:{#4}\medskip\par\noindent
\ding{176}\:\:{#5}}

%
\newcommand\one{\ding{172}}
\newcommand\two{\ding{173}}
\newcommand\three{\ding{174}}
\newcommand\four{\ding{175}}
\newcommand\five{\ding{176}}


%\pagestyle{empty}

%%% Counters
\newcounter{num}
%\newcounter{answer}

%%% Commands
\newcommand\prob[1]
{\vs\par\noindent\refstepcounter{num} \textbf{문제 \thenum) #1}\par\noindent}

\newcommand\exam[1]
{\vs\par\noindent\refstepcounter{num} \textbf{예시 \thenum) #1}\par\noindent}

\newenvironment{expl}{\begin{mdframed}[frametitle=풀이]}{\end{mdframed}}

\newcommand\pb[1]{\ensuremath{\fbox{\phantom{#1}}}}

\newcommand\vs[1]{\vspace{30pt}}

\newcommand\an[1]{\par\bigskip\noindent\textbf{문제 \ref{#1})}\par\noindent}
\newcommand\ann[2]{\par\bigskip\noindent\textbf{문제 \ref{#1})}\:\:#2\par\medskip\noindent}

\newcommand\ov[1]{\ensuremath{\overline{#1}}}


%%% Footnotes
\makeatletter
\def\@fnsymbol#1{\ensuremath{\ifcase#1\or
*\or **\or ***\or
\star\or\star\star\or\star\star\star\or
\dagger\or\dagger\dagger\or\dagger\dagger\dagger
\else\@ctrerr\fi}}

\renewcommand{\thefootnote}{\fnsymbol{footnote}}
\makeatother

%\let\emph\textsf

\begin{document}
\title{중학교 3-2 기말 대비(천재교육 교과서)}
\date{\today}
\author{}
\maketitle

%%%
\section{이차방정식}

%
\prob{162-1}
\label{162-1}%2
네 수 \(m\), 5, 9, 16에 대하여 다음을 구하여라.
\begin{enumerate}
\item
평균이 10일 때, \(m\)의 값
\item
중앙값이 10일 때, \(m\)의 값
\end{enumerate}

%
\prob{162-2}
\label{162-2}%2
다음이 모두 성립할 때, \(a\)의 범위를 구하여라.
\begin{mdframed}\small
\begin{enumerate}
\item[ㄱ.]
5개의 수 10, 14, 20, 24, \(a\)의 중앙값은 14이다.
\item[ㄴ.]
4개의 수 2, 5, 11, \(a\)의 중앙값은 8이다.
\end{enumerate}
\end{mdframed}

%
\prob{162-3}
\label{162-3}%2
다음이 모두 성립할 때, \(a\)의 범위를 구하여라.
\begin{mdframed}\small
\begin{enumerate}
\item[ㄱ.]
5개의 수 21, 28, 35, 42, \(a\)의 중앙값은 35이다.
\item[ㄴ.]
4개의 수 40, 50, 60, \(a\)의 중앙값은 45이다.
\end{enumerate}
\end{mdframed}

\newpage
%
\prob{162-4}
\label{162-4}%2
다음이 모두 성립할 때, \(a\)의 범위를 구하여라.
\begin{mdframed}\small
\begin{enumerate}
\item[ㄱ.]
5개의 수 5, 10, 15, 20, \(a\)의 평균은 13보다 작다.
\item[ㄴ.]
4개의 수 5, 9, 13, \(a\)의 중앙값은 11이다.
\end{enumerate}
\end{mdframed}

%
\prob{162-5}
\label{162-5}%2
다음이 모두 성립할 때, \(a\)의 범위를 구하여라.
\begin{mdframed}\small
\begin{enumerate}
\item[ㄱ.]
5개의 수 4, 10, 16, 22, \(a\)의 중앙값을 \(m\)이라고 할 때, \(13\le m\le 16\)이다.
\item[ㄴ.]
4개의 수 18, 27, 36, \(a\)의 평균은 25보다 작다.
\end{enumerate}
\end{mdframed}


%
\prob{162-6}
\label{162-6}%2
다음이 모두 성립할 때, \(a\)의 범위를 구하여라.
\begin{mdframed}\small
\begin{enumerate}
\item[ㄱ.]
5개의 수 2, 5, 7, 11, \(a\)의 중앙값을 \(m\)은 5이다.
\item[ㄴ.]
8개의 수 4, 5, 5, 6, 6, 6, 7 \(a\)의 최빈값은 6이다.
\end{enumerate}
\end{mdframed}

\newpage
%
\prob{165}\label{165}
다음은 수현이가 이번 중간고사에서 얻은 점수들을 나열한 것이다.
\begin{center}
\begin{tabu}{c|cccc}
과목	&국어	&영어	&수학	&과학	\\\hline
점수	&100	&90		&78		&76		
\end{tabu}
\end{center}
이 점수들에 대해 다음 값들을 구하여라.
\begin{enumerate}
\item
최솟값
\item
최댓값
\item
평균
\item
중앙값
\item
최빈값
\item
점수의 최댓값과 최솟값의 차
\item
각 점수와 평균의 차의 합
\item
각 점수의 평균의 차의 제곱의 합
\item
분산
\item
표준편차
\end{enumerate}

%
\prob{167}\label{167}
다음 중 틀린 것을 고르시오.
\tabd
{대푯값에는 평균, 중앙값, 최빈값 등이 있다.}
{자료에 매우 크거나 작은 값이 포함되어 있을 때에는 평균보다는 중앙값을 사용하는 것이 좋다.}
{자료들이 대푯값 주위에 몰려있으면 산포도가 작다.}
{편차는 평균에서 편차를 뺀 값이다.}
{편차들을 모두 더한 값은 0이다.}

\newpage
%
\prob{173-1}\label{173-1}
다섯 개의 수 4, 7, 11, \(x\), \(y\)의 평균이 9, 분산이 10일 때, \(x\)와 \(y\)의 값을 각각 구하여라. (단, \(x<y\))

%
\prob{173-2}\label{173-2}
여섯 개의 수 1, 4, 5, 11, \(x\), \(y\)의 평균이 5, 분산이 11일 때, \(x\)와 \(y\)의 값을 각각 구하여라. (단, \(x<y\))

%
\prob{173-3}\label{173-3}
다음은 다섯 학생의 수학 점수를 나타낸 것이다.
\begin{center}\begin{tabu}{c|ccccc}
학생	&일장	&지인	&소진	&진태	&태희\\\hline
점수	&94		&\(x\)	&88		&82		&\(y\)
\end{tabu}\end{center}
수학 점수의 평균이 \(90\)점이고 표준편차는 \(40\)점일 때, \(xy\)의 값을 구하여라.

\newpage
%
\exam{173-4}\label{173-4}
다음은 지현이네 모둠 4명 각각의 쪽지 시험 점수에서 지현이의 점수를 뺀 값이다.
\begin{center}\begin{tabu}{c|ccccc}
학생		&지영	&민우	&지현	&현식\\\hline
뺀점수	&\(-3\)	&\(-2\)	&0		&1
\end{tabu}\end{center}
\begin{mdframed}
네 학생의 점수를 각각 \(a\), \(b\), \(c\), \(d\)라고 하자.
표를 해석하면
\[a-c=-3,\quad b-c=-2,\quad c-c=0,\quad d-c=1\]
이다.
따라서
\[a=c-3,\quad b=c-2,\quad c=c,\quad d=c+1\]
이다.

\bigskip
만약 \(c=8\)이면
\vspace{-10pt}
\[a=5,\quad b=6,\quad c=8,\quad d=9\]
이고 \(c=10\)이면
\vspace{-10pt}
\[a=7,\quad b=8,\quad c=10,\quad d=11\]
인 것이다.

\bigskip
네 점수의 평균은
\[\frac{(c-3)+(c-2)+c+(c+1)}4=c-1\]
이다.
따라서
\begin{center}\begin{tabu}{c|ccccc}
학생	&지영	&민우	&지현	&현식\\\hline
점수	&\(c-3\)	&\(c-2\)	&c		&\(c+1\)\\
편차&\(-2\)	&\(-1\)	&1		&\(2\)
\end{tabu}\end{center}
이고 분산은
\[\frac{(-2)^2+(-1)^2+1^2+2^2}4=\frac52\]
이다.
\end{mdframed}

%
\prob{173-5}\label{173-5}
다음은 지현이네 모둠 5명 각각의 쪽지 시험 점수에서 지현이의 점수를 뺀 값이다.
\begin{center}\begin{tabu}{c|ccccc}
학생		&지영	&민우	&지현	&현식	&상철\\\hline
뺀점수	&\(-7\)	&\(-3\)	&0		&1		&4
\end{tabu}\end{center}
\begin{enumerate}
\item
지현이의 점수가 11점일 때, 상철이의 점수를 구하여라.
\item
지현이의 점수가 13점일 때, 점수들의 평균을 구하여라.
\item
점수들의 분산을 구하여라.
\end{enumerate}

%
\prob{173-6}\label{173-6}
다음은 지현이네 모둠 5명 각각의 쪽지 시험 점수에서 지현이의 점수를 뺀 값이다.
\begin{center}\begin{tabu}{c|ccccc}
학생		&지영	&민우	&지현	&현식	&상철\\\hline
뺀점수	&\(-3\)	&\(-1\)	&0		&6		&8
\end{tabu}\end{center}
이때, 다섯 점수들의 분산을 구하여라.

\newpage
%
\exam{175-1}\label{175-1}%1,4,7
3개의 변량 \(a\), \(b\), \(c\)의 평균이 4, 분산이 6이다.
이때, \(2a-3\), \(2b-3\), \(2c-3\)의 평균과 분산을 구하여라.
\begin{mdframed}
평균이 \(4\)이므로
\[a+b+c=12\tag{1}\]
이다.
또 분산이 6이므로
\[(a-4)^2+(b-4)^2+(c-4)^2=18\]
이다.
즉
\[(a^2+b^2+c^2)-8(a+b+c)+48=18\]
이다.
(1)를 적용하면
\[a^2+b^2+c^2=66\tag{2}\]

\bigskip
이제  \(2a-3\), \(2b-3\), \(2c-3\)의 평균과 분산을 구하자.
\begin{align*}
평균
&=\frac{(2a-3)+(2b-3)+(2c-3)}3\\
&=\frac{2(a+b+c)-9}3=5
\end{align*}
이다.
변량들의 편차는 각각
\begin{align*}
(2a-3)-5&=2a-8,\\
(2b-3)-5&=2b-8,\\
(2c-3)-5&=2c-8
\end{align*}
이므로
\begin{align*}
분산
&=\frac{(2a-8)^2+(2b-8)^2+(2c-8)^2}3\\
&=\frac{4(a^2+b^2+c^2)-32(a+b+c)+192}3\\
&=\frac{4\cdot66-32\cdot12+192}3=\frac{72}3=24
\end{align*}
\end{mdframed}

%
\prob{175-2}\label{175-2}%1,2,3,4,5
다섯 개의 변량 \(a\), \(b\), \(c\), \(d\), \(e\)의 평균이 \(3\), 분산이 \(2\)이다.
이때 \(3a\), \(3b\), \(3c\), \(3d\), \(3e\)의 평균과 분산을 구하여라.

%
\prob{175-3}\label{175-3}%1,3,5,7
네 개의 변량 \(a\), \(b\), \(c\), \(d\)의 평균이 \(4\), 분산이 \(5\)이다.
이때 \(a+2\), \(b+2\), \(c+2\), \(d+2\)의 평균과 분산을 구하여라.

%
\prob{175-4}\label{175-4}%1, 7, 9, 15
네 개의 변량 \(a\), \(b\), \(c\), \(d\)의 평균이 \(8\), 분산이 \(25\)이다.
이때 \(3a+4\), \(3b+4\), \(3c+4\), \(3d+4\)의 평균과 분산을 구하여라.

%
\prob{178-1}\label{178-1}%2,4,6,8
4개의 수 \(a\), \(b\), \(c\), \(d\)의 평균이 \(5\), 표준편차가 \(\sqrt5\)일 때,
\(a^2\), \(b^2\), \(c^2\), \(d^2\)의 평균을 구하여라.

\newpage
%
\prob{189-1}\label{189-1}
\begin{enumerate}
\item
다음 식이 성립함을 확인하여라.
\[(2mn)^2+(m^2-n^2)^2=(m^2+n^2)^2\]
\item
\(a^2+b^2=c^2\)을 만족시키는 세 자연수 \(a\), \(b\), \(c\)를 \fbox{피타고라스의 수}라고 한다.
예를 들어 3, 4, 5는 \(3^2+4^2=5^2\)을 만족시키므로 피타고라스의 수이다.
위의 식에 적당한 자연수 \(m\), \(n\)을 넣어 다른 피타고라스의 수를 다섯 개 이상 찾아라.
\end{enumerate}

%
\prob{189-2}\label{189-2}
\(p^2+q^2=1\)을 만족시키는 유리수 \(p\), \(q\)를 다섯 개 이상 찾아라.


%
\prob{189-3}\label{189-3}
\(a^2+b^2+c^2=d^2\)을 만족시키는 네 자연수 \(a\), \(b\), \(c\), \(d\)를 구하여라.

%
\prob{189-4}\label{189-4}
\(a^3+b^3=d^3\)을 만족시키는 세 자연수 \(a\), \(b\), \(c\)를 구하여라.

%
\prob{189-5}\label{189-5}
\(a^4+b^4=d^4\)을 만족시키는 세 자연수 \(a\), \(b\), \(c\)를 구하여라.


\clearpage
%%
\section*{답}

\an{162-1}
\begin{enumerate}
\item
10
\item
11
\end{enumerate}

%
\ann{162-2}{\(11\le a\le 14\)}

%
\ann{162-3}{\(35\le a\le 40\)}

%
\ann{162-4}{\(13\le a<15\)}

%
\ann{162-5}{\(13\le a<19\)}

%
\ann{162-6}{\(a<5\)}

%
\an{165}
\begin{enumerate}
\item
78
\item
100
\item
86
\item
84
\item
없다.
\item
24
\item
36
\item
376
\item
94
\item
\(\sqrt{94}\)
\end{enumerate}

%
\ann{167}\four

%
\an{173-1}
\(x=10\), \(y=13\)

%
\an{173-2}
\(x=2\), \(y=7\)

%
\ann{173-3}{8600}

%
\an{173-5}
\begin{enumerate}
\item
15점
\item
12점
\item
14
\end{enumerate}

%
\ann{173-6}{18}

%
\an{175-2}
평균=9, 분산=18

%
\an{175-3}
평균=6, 분산=5

%
\an{175-4}
평균=28, 분산=225

%
\ann{178-1}{30}
\newpage
%
\an{189-1}
\begin{enumerate}
\item
\begin{align*}
&(2mn)^2+(m^2-n^2)^2\\
=&4m^2n^2+m^4-2m^2n^2+n^4\\
=&m^4+2m^2n^2+n^4\\
=&(m^2+n^2)^2
\end{align*}
\item
\begin{gather*}
\begin{aligned}
(m,n)=(2,1)&\qquad\Longrightarrow\qquad (a,b,c)=(3,4,5)\\
(m,n)=(4,1)&\qquad\Longrightarrow\qquad (a,b,c)=(8,15,17)\\
(m,n)=(6,1)&\qquad\Longrightarrow\qquad (a,b,c)=(12,35,37)\\
(m,n)=(3,2)&\qquad\Longrightarrow\qquad (a,b,c)=(5,12,13)\\
(m,n)=(5,2)&\qquad\Longrightarrow\qquad (a,b,c)=(20,21,29)\\
(m,n)=(4,3)&\qquad\Longrightarrow\qquad (a,b,c)=(7,24,25)\\
(m,n)=(5,4)&\qquad\Longrightarrow\qquad (a,b,c)=(9,40,41)\\
(m,n)=(6,5)&\qquad\Longrightarrow\qquad (a,b,c)=(11,60,61)\\
\end{aligned}\\
\vdots
\end{gather*}
\end{enumerate}

%
\an{189-2}
\begin{align*}
(p,q)=
&\left(\frac35,\frac45\right),\left(\frac{5}{13},\frac{12}{13}\right),\left(\frac{7}{25},\frac{24}{25}\right),\\
&\left(\frac{8}{17},\frac{15}{17}\right),\left(\frac{9}{41},\frac{40}{41}\right)\cdots
\end{align*}

%
\an{189-3}
\begin{align*}
(a,b,c,d)=
&(1,2,2,3),(2,10,11,15),\\
&(2,3,6,7),(1,4,8,9),\cdots
\end{align*}

\newpage
%
\an{189-4}
없다.

%
\an{189-5}
없다.\footnotemark

\footnotetext{
\begin{mdframed}
\[a^n+b^n=c^n(\text{단, \(n\)은 3 이상의 자연수})\]
를 만족시키는 자연수 \(a\), \(b\), \(c\)는 존재하지 않는다.
\end{mdframed}
이것은 \fbox{페르마의 마지막 정리}라고 불리는 정리이다.
이 정리는 17세기 프랑스의 수학자 페르마의 이름을 따서 만들어진 정리로, 오랜 세월을 거쳐 1995년 영국의 수학자 앤드류 와일즈에 의해 완전히 증명되었다.
}


\end{document}