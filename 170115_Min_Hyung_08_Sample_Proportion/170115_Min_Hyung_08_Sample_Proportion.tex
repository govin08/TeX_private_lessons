\documentclass{oblivoir}
\usepackage{amsmath,amssymb,amsthm,kotex,paralist,kswrapfig,tabu}

\usepackage[skipabove=10pt,skipbelow=10pt,innertopmargin=10pt]{mdframed}

\usepackage{tabto,pifont}
\TabPositions{0.2\textwidth,0.4\textwidth,0.6\textwidth,0.8\textwidth}
\newcommand\tabb[5]{\par\bigskip\noindent
\ding{172}\:{\ensuremath{#1}}
\tab\ding{173}\:\:{\ensuremath{#2}}
\tab\ding{174}\:\:{\ensuremath{#3}}
\tab\ding{175}\:\:{\ensuremath{#4}}
\tab\ding{176}\:\:{\ensuremath{#5}}}

\usepackage{enumitem}
\setlist[enumerate]{label=(\arabic*)}

\newcounter{num}
\newcommand{\defi}[1]
{\noindent\refstepcounter{num}\textbf{정의 \arabic{num}) #1}\par\noindent}
\newcommand{\theo}[1]
{\noindent\refstepcounter{num}\textbf{정리 \arabic{num}) #1}\par\noindent}
\newcommand{\exam}[1]
{\bigskip\bigskip\noindent\refstepcounter{num}\textbf{예시 \arabic{num}) #1}\par\noindent}
\newcommand{\prob}[1]
{\bigskip\bigskip\noindent\refstepcounter{num}\textbf{문제 \arabic{num}) #1}\par\noindent}
\newcommand{\proo}
{\bigskip\textsf{증명)}\par}

\newcommand{\procedure}[1]{\begin{mdframed}\vspace{#1\textwidth}\end{mdframed}}
\newcommand{\ans}{
{\par\raggedleft\textbf{답 : (\qquad\qquad\qquad\qquad\qquad\qquad)}\par}\bigskip\bigskip}
\newcommand\an[1]{\par\bigskip\noindent\textbf{문제 #1)}\\}

\newcommand{\pb}[1]%\Phantom + fBox
{\fbox{\phantom{\ensuremath{#1}}}}

\newcommand\ba{\,|\,}

\let\oldsection\section
\renewcommand\section{\clearpage\oldsection}
\counterwithout{subsection}{section}

\newenvironment{talign}
 {\let\displaystyle\textstyle\align}
 {\endalign}
\newenvironment{talign*}
 {\let\displaystyle\textstyle\csname align*\endcsname}
 {\endalign}

\let\emph\textsf

%\usepackage{fapapersize}
%\usefapapersize{210mm,297mm,45mm,45mm,15mm,15mm}
%%%

\begin{document}

\title{민형 : 08 모비율과 표본비율}
\author{}
\date{\today}
\maketitle

어느 고등학교의 학생 \(1000\)명 중 축구를 좋아하는 학생이 \(900\)명이라고 한다.
\begin{enumerate}
\item
이 고등학교 학생들 중 축구를 좋아하는 학생의 비율을 \(p\)라고 할 때, \(p\)의 값은
\[p=\frac{900}{1000}=0.9\]
\item
이 고등학교의 학생들 중 \(100\)명을 임의추출했을 때, \(100\)명 중 축구를 좋아하는 학생의 수를 \(X\)라고 하자.
%이때 확률질량함수는
%
%\begin{tabu}{|X[c,3]|X[c,2]|X[c,2]|X[c,2]|X[c,2]|}
%\hline
%\(X\) 		&\(0\)	&\(1\)	&\(\cdots\)	&\(100\)\\
%\hline
%\(P(X=x)\)	&		&		&			&\\
%\hline
%\end{tabu}
%이다.

각각의 학생이 축구를 좋아할 확률이 \(0.9\)이므로 \(X\)는 이항분포
\[B(100,0.9)\]
\vspace{-23pt}\tabto{0.8\textwidth}\(=B(n,p)\)\\
를 따른다.
따라서
\[E(X)=100\times0.9=90\]
\vspace{-24pt}\tabto{0.8\textwidth}\(=np\)
\[V(X)=100\times0.9\times0.1=9\]
\vspace{-31pt}\tabto{0.8\textwidth}\(=npq\)\\
%\[N(\phantom{90},\phantom{3^2})\]
이다.

\(100\)명 중 축구를 좋아하는 학생의 비율을 \(\hat p\)라고 하면
\[\hat p=\frac X{100}\]
\vspace{-28pt}\tabto{0.8\textwidth}\(=\frac Xn\)\\
이고 따라서
\[E(\hat p)=E\left(\frac X{100}\right)=\frac1{100}E(X)=0.9\]
\vspace{-25pt}\tabto{0.8\textwidth}\(=p\)\\
\[V(\hat p)=V\left(\frac X{100}\right)=\frac1{10000}V(X)=0.0009\]
\vspace{-25pt}\tabto{0.8\textwidth}\(=\frac{pq}n\)\\
\end{enumerate}

\thispagestyle{empty}
\end{document}