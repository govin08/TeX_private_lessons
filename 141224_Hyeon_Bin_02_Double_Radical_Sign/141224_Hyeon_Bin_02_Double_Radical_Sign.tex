\documentclass{article}
\usepackage{amsmath,amssymb,amsthm,kotex,paralist,mathrsfs}

%%%
\begin{document}

\title{현빈 : 02 이중근호}
\author{}
\date{\today}
\maketitle

\noindent
1. 다음 이중근호를 풀어라.
\begin{enumerate}[(1)]
\item
\(\sqrt{3-2\sqrt2}=\)
\item
\(\sqrt{4+2\sqrt3}=\)
\item
\(\sqrt{7-2\sqrt6}=\)
\item
\(\sqrt{8-2\sqrt{15}}=\)
\item
\(\sqrt{13-2\sqrt{30}}=\)
\item
\(\sqrt{13+2\sqrt{42}}=\)
\item
\(\sqrt{\frac{5+2\sqrt6}6}=\)
\end{enumerate}

\noindent
2. 다음 이중근호를 풀어라.
\begin{enumerate}[(1)]
\item
\(\sqrt{11+6\sqrt2}=\)
\item
\(\sqrt{28-10\sqrt3}=\)
\item
\(\sqrt{6+4\sqrt2}=\)
\item
\(\sqrt{15-4\sqrt{14}}=\)
\item
\(\sqrt{30-12\sqrt6}=\)
\item
\(\sqrt{28-8\sqrt{10}}=\)
\end{enumerate}

\noindent
3. 다음 이중근호를 풀어라.
\begin{enumerate}[(1)]
\item
\(\sqrt{2+\sqrt3}=\)
\item
\(\sqrt{4+\sqrt7}=\)
\item
\(\sqrt{\frac{11}2-3\sqrt2}=\)
\item
\(\sqrt{7+3\sqrt5}=\)
\item
\(\sqrt{1-\frac{\sqrt3}2}=\)
\item
\(\sqrt{6+\sqrt{35}}=\)
\end{enumerate}

4. 다음 이중근호를 풀어라.
\begin{enumerate}[(1)]
\item
\(\sqrt{3\sqrt3+2\sqrt6}=\)
\item
\(\sqrt{7\sqrt2-4\sqrt5}=\)
\item
\(\sqrt{12\sqrt2+2\sqrt{70}}=\)
\end{enumerate}

5. 다음 이중근호를 풀어라((1)에서 \(a>b>0\)이다.).
\begin{enumerate}[(1)]
\item
\(\sqrt{a+\sqrt{a^2-b^2}}=\)
\item
\(\sqrt{a^2+b^2+\sqrt{a^4+a^2b^2+b^4}}=\)
\end{enumerate}
\end{document}