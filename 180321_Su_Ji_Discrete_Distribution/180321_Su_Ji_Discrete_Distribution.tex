\documentclass[a4paper]{oblivoir}
\usepackage{amsmath,amssymb,amsthm,kotex,mdframed,paralist,tabu}
%\usefapapersize{210mm,297mm,20mm,*,20mm,*}

\usepackage{tabto,pifont}
\TabPositions{0.2\textwidth,0.4\textwidth,0.6\textwidth,0.8\textwidth}
\newcommand\tabb[5]{\par\noindent
\ding{172}\:{\ensuremath{#1}}
\tab\ding{173}\:\:{\ensuremath{#2}}
\tab\ding{174}\:\:{\ensuremath{#3}}
\tab\ding{175}\:\:{\ensuremath{#4}}
\tab\ding{176}\:\:{\ensuremath{#5}}}

%\pagestyle{empty}
%%% Counters
\newcounter{num}

%%% Commands
\newcommand\defi[1]
{\bigskip\par\noindent\stepcounter{num} \textbf{정의 \thenum) #1}\par\noindent}
\newcommand\theo[1]
{\bigskip\par\noindent\stepcounter{num} \textbf{정리 \thenum) #1}\par\noindent}
\newcommand\exam[1]
{\bigskip\par\noindent\stepcounter{num} \textbf{예시 \thenum) #1}\par\noindent}
\newcommand\prob[1]
{\bigskip\par\noindent\stepcounter{num} \textbf{문제 \thenum) #1}\par\noindent}
\newcommand{\proo}
{\bigskip\textsf{증명)}\par}

\newcommand\pb[1]{\fbox{$\phantom{#1}$}}

\newcommand\ba{\ensuremath{\:|\:}}

\newcommand\vs[1]{\vspace{60pt}}

\newcommand\summ{\ensuremath{\sum_{i=1}^n}}

%%% Meta Commands
\let\oldsection\section
\renewcommand\section{\clearpage\oldsection}
\let\emph\textsf
\tabulinesep=2pt

\begin{document}

\title{수지, 이산확률분포}
\author{}
\date{\today}
\maketitle

%
\exam{중학교 통계 복습}
다섯 개의 숫자 \(1\), \(2\), \(3\), \(4\), \(5\)의 평균 \(m\)은
\[m=\frac{1+2+3+4+5}5=3\]
이다.
분산 \(V\)는 `편차의 제곱의 평균'이다.
이때, 편차란 `\(변량-평균\)'이다.
따라서
\begin{center}
\begin{tabu}to.6\textwidth{|X[2$c]|X[$c]|X[$c]|X[$c]|X[$c]|X[$c]|} 
\hline
변량	&1	&2	&3	&4	&5\\\hline
편차	&-2	&-1	&3	&1	&2\\\hline
편차^2	&4	&1	&0	&1	&4\\\hline
\end{tabu}
\end{center}
이고
\[V=\frac{4+1+0+1+4}5=\fbox{(1)}\]
표준편차 \(\sigma\)는
\(\sigma=\sqrt V=\fbox{(2)}\)
이다.

%
\exam{확률질량함수}\label{stone}
검은 바둑돌이 2개, 흰 바둑돌이 2개 들어있는 주머니에서 2개의 바둑돌을 동시에 꺼낼 때, 꺼낸 흰 바둑돌의 개수를 \(X\)라고 하자.
그러면
\begin{align*}
P(X=0)&=\frac{_2C_2\times_2C_0}{_4C_2}	=\frac16\\
P(X=1)&=\fbox{(3)}\\
P(X=2)&=\fbox{(4)}
\end{align*}
이다.
이 함수 \(P(X=x)\)를 \emph{확률질량함수}라고 부르고 다음과 같이 표로 표현한다.

\begin{center}
\begin{tabu}to.8\textwidth{X[1.5$c]|X[$c]X[$c]X[$c]|X[$c]}
\hline
X		&0			&1				&2				&합계\\\hline
P(X=x)	&\frac16	&\fbox{(3)}	&\fbox{(4)}	&1\\\hline
\end{tabu}
\end{center}

이때 \(X\)의 평균 \(E(X)\)는
\[E(X)=0\times\frac16+1\times\fbox{(3)}+2\times\fbox{(4)}=1\]
이고, \(X\)의 분산 \(V(X)\)는 편차 \(X-1\)의 제곱의 평균이므로

\begin{center}
\begin{tabu}to.8\textwidth{X[1.5$c]|X[$c]X[$c]X[$c]|X[$c]}
\hline
(X-1)^2	&1			&0				&1				&합계\\\hline
P(X=x)	&\frac16	&\fbox{(3)}	&\fbox{(4)}	&1\\\hline
\end{tabu}
\end{center}
\noindent에서
\[V(X)=1\times\frac16+0\times\fbox{(3)}+1\times\fbox{(4)}=\fbox{(5)}\]
이다.
또 표준편차 \(\sigma(X)\)는
\[\sigma(X)=\sqrt{V(X)}=\frac{\sqrt3}3.\]

%
\begin{mdframed}
\defi{}
확률변수 \(X\)의 확률질량함수가

\begin{center}
\begin{tabu}to.8\textwidth{X[3$c]|X[$c]X[$c]X[$c]X[$c]X[$c]|X[$c]}
\hline
X		&x_1	&x_2	&x_3	&\cdots		&x_n	&합계\\\hline
P(X=x)	&p_1	&p_2	&p_3	&\cdots		&p_n	&1\\\hline
\end{tabu}
\end{center}
로 주어질 때, \(X\)의 평균 \(E(X)\)는
\begin{align*}
E(X)
&=x_1p_1+x_2p_2+\cdots+x_np_n\\
&=\summ x_ip_i\
\end{align*}
이다.
\(E(X)=m\)이라고 하면,
\(X\)의 분산 \(V(X)\)는 편차 \(X-m\)의 제곱의 평균이므로
%\begin{center}
%\begin{tabu}{X[2.8$c]|X[$c]X[$c]X[$c]|X[$c0.5]}
%\hline
%(X-m)^2									&(x_1-m)^2	&\cdots		&(x_n-m)^2	&합계\\\hline
%P\left((X-m)^2=(x-m)^2\right)		&p_1		&\cdots		&p_n		&1\\\hline
%\end{tabu}
%\end{center}
%에서
\begin{align*}
V(X)
&=\summ(x_i-m)^2p_i\
\end{align*}
이다.
\(X\)의 표준편차 \(\sigma(X)\)는
\[\sigma(X)=\sqrt{V(X)}\]
이다.
\end{mdframed}

%
\begin{mdframed}
\theo{}\label{var}
\[V(X)=E(X^2)-\{E(X)\}^2\]
\end{mdframed}

%
\proo{}
\begin{align*}
V(X)
&=\summ(x_i-m)^2p_i\\
&=\summ({x_i}^2-2mx_i+m^2)p_i\\
&=\summ({x_i}^2p_i-2mx_ip_i+m^2p_i)\\
&=\summ{x_i}^2p_i-2m\summ x_ip_i+m^2\summ p_i
\end{align*}
이때,
\[\summ{x_i}^2p_i=\fbox{(6)},\quad \summ x_ip_i=\fbox{(7)},\qquad \summ p_i=\fbox{(8)}\]
이므로
\[V(X)=E(X^2)-\{E(X)\}^2\]
\qed

%
\exam{}
예시 2)%\ref{stone}
의 분산을 정리 4)%\ref{var}
의 방법으로 구하자.
먼저 \(E(X^2)\)를 구하면,
\begin{center}\centering
\begin{tabu}to.8\textwidth{X[2$c]|X[$c]X[$c]X[$c]|X[$c]}
\hline
X^2			&0			&1			&4			&합계\\\hline
P(X^2=x^2)	&\frac16	&\frac23	&\frac16	&1\\\hline
\end{tabu}
\end{center}
에서
\[E(X^2)=0\times\frac16+1\times\frac23+4\times\frac16=\frac43\]
따라서
\[V(X)=E(X^2)-\{E(X)\}^2=\frac43-1^2=\frac13\]
이다. 이것은 예시 2)%\ref{stone}
의 결과와도 일치한다.

%
\begin{mdframed}
\theo{}
\(Y=aX+b\)일 때,
\[E(Y)=aE(X)+b,\quad V(Y)=a^2V(X)\]
이다.
\end{mdframed}

%
\proo{}
\(X=x_i\)이면 \(Y=ax_i+b\)이다.
따라서 \(Y\)의 확률질량함수는

\begin{center}
\begin{tabu}{X[2$c]|X[$c]X[$c]X[$c]X[$c]X[$c]|X[$c]}
\hline
Y				&ax_1+b	&ax_2+b	&ax_3+b	&\cdots		&ax_n+b	&합계\\\hline
P(Y=y)	&p_1	&p_2	&p_3	&\cdots		&p_n	&1\\\hline
\end{tabu}
\end{center}

따라서
\begin{align*}
E(Y)
&=\summ(ax_i+b)p_i\\
&=\summ(ax_ip_i+bp_i)\\
&=a\summ x_ip_i+b\summ p_i\\
\end{align*}
이때,
\[\summ x_ip_i=\fbox{(7)},\qquad \summ p_i=\fbox{(8)}\]
이므로
\[E(Y)=aE(X)+b\]
이다.

\clearpage
또
\[V(Y)=E(Y^2)-\{E(Y)\}^2\]
에서
\begin{align*}
E(Y^2)
&=\summ(ax_i+b)^2p_i\\
&=\summ(a^2{x_i}^2+2abx_i+b^2)p_i\\
&=a^2\summ{x_i}^2p_i+2ab\summ x_ip_i+b^2\summ p_i\\
&=\fbox{\qquad\qquad(9)\qquad\qquad}
\end{align*}
이고
\begin{align*}
\{E(Y)\}^2
&=\{aE(X)+b\}^2\\
&=(am+b)^2\\
&=\fbox{\qquad\qquad(10)\qquad\qquad}
\end{align*}
이므로
\begin{align*}
V(Y)
&=E(Y^2)-\{E(Y)\}^2\\
&=\fbox{\qquad\qquad(9)\qquad\qquad}-\fbox{\qquad\qquad(10)\qquad\qquad}\\
&=a^2E(X^2)-a^2m^2\\
&=a^2\{E(X^2)-m^2\}\\
&=a^2\left[E(X^2)-\{E(X)\}^2\right]\\
&=a^2V(X)
\end{align*}
\qed

\clearpage
%
\exam{}
주사위를 90번 던져서 1의 눈이 나온 횟수를 \(X\)라고 하자.
그러면
\begin{align*}
P(X=0)&=\left(\frac56\right)^{90}\\
P(X=1)&=_{90}C_1\left(\frac16\right)^1\left(\frac56\right)^{89}\\
P(X=2)&=_{90}C_2\left(\frac16\right)^2\left(\frac56\right)^{88}\\
\vdots\\
P(X=89)&=_{90}C_{89}\left(\frac16\right)^{89}\left(\frac56\right)^1\\
P(X=90)&=\left(\frac16\right)^{90}\\
\end{align*}
즉
\[P(X=x)=_{90}C_x\left(\frac16\right)^x\left(\frac56\right)^{90-x}\]
이다.

한 번 던질 때마다 1의 눈이 나올 확률은 \(\frac16\)이므로, 1의 눈이 나오는 횟수는 평균적으로 \(90\times\frac16=15\)번 임을 예상할 수 있다.
즉 \[E(X)=15\]일 것이다.

%
\begin{mdframed}
\defi{}
일어날 확률이 \(p\)인 사건의 시행을 \(n\)회 반복할 때, 이 사건이 일어난 횟수를 \(X\)라고 하자.
\(X\)의 확률질량함수는
\[P(X=x)=_nC_xp^xq^{n-x}\]
이다.
이와 같은 \(X\)의 확률분포를 \emph{이항분포}라고 하며 기호로
\[X\sim B(n,p)\]
라고 쓴다.
\end{mdframed}

%
\begin{mdframed}
\theo{}
\(X\sim B(n,p)\)이면 
\[E(X)=np,\quad V(X)=npq\]
이다.
(\(q=1-p\))
\end{mdframed}

이 식의 증명에는 \(r\cdot_nC_r=n\cdot_{n-1}C_{r-1}\)이 필요하다.
그리고 이 식은 쉽게 증명된다 ;
\begin{align*}
r\cdot_nC_r
&=r\cdot\frac{n!}{r!(n-r)!}=\frac{n!}{(r-1)!(n-r)!}\\
&=n\cdot\frac{\fbox{(11)}}{(r-1)!\{(n-1)-(r-1)\}!}\\
&=n\cdot_{n-1}C_{r-1}
\end{align*}

\proo{}
\(X\)의 확률질량함수는
\begin{center}
\begin{tabu}to.8\textwidth{X[3$c]|X[$c]X[$c]X[$c]X[$c]X[$c]|X[$c]}
\hline
X		&0		&1		&2		&\cdots		&n	&합계\\\hline
P(X=x)	&p_0	&p_1	&p_2	&\cdots		&p_n	&1\\\hline
\end{tabu}
\end{center}
이다.
이때,
\[p_r=P(X=r)=_nC_rp^rq^{n-r}\]이다.
그러면
\begin{align*}
E(X)
&=\sum_{r=0}^nr\cdot p_r=\sum_{r=0}^nr\cdot_nC_rp^rq^{n-r}\\
&=\fbox{(12)}+\sum_{r=1}^nr\cdot_nC_rp^rq^{n-r}\\
&=\sum_{r=1}^nn\cdot_{n-1}C_{r-1}p^rq^{n-r}\\
&=np\sum_{r=1}^n{}_{n-1}C_{r-1}p^{r-1}q^{(n-1)-(r-1)}
\end{align*}
\(r-1=s\)로 치환하면 \(s=r+1\)이므로
\begin{align*}
E(X)
&=np\sum_{s=0}^{n-1}{}_{n-1}C_sp^sq^{(n-1)-s}\\
&=np(p+q)^{n-1}=\fbox{(13)}.
\end{align*}

한편,
\begin{align*}
E(X^2)
&=\sum_{r=0}^nr^2\cdot p_r\\
&=\sum_{r=0}^nr^2\cdot_nC_rp^rq^{n-r}\\
&=\sum_{r=0}^n\{r(r-1)+r\}\cdot_nC_rp^rq^{n-r}\\
&=\sum_{r=0}^nr(r-1)\cdot_nC_rp^rq^{n-r}+\sum_{r=0}^nr\cdot_nC_rp^rq^{n-r}\\
&=\fbox{(12)}+\fbox{(12)}+\sum_{r=2}^nr(r-1)\cdot_nC_rp^rq^{n-r}+\fbox{(13)}
\end{align*}
이때,
\begin{align*}
r(r-1)\cdot_nC_r
&=(r-1)\times n\cdot_{n-1}C_{r-1}=n(n-1)\cdot_{n-2}C_{r-2}
\end{align*}
이므로
\begin{align*}
E(X^2)
&=\sum_{r=2}^nn(n-1)\cdot_{n-2}C_{r-2}p^rq^{n-r}+\fbox{(13)}\\
&=n(n-1)p^2\sum_{r=2}^n\cdot_{n-2}C_{r-2}p^{r-2}q^{(n-2)-(r-2)}+\fbox{(13)}
\end{align*}
\(r-2=s\)로 치환하면 \(s=r+2\)이므로
\begin{align*}
E(X^2)
&=n(n-1)p^2\sum_{s=0}^{n-2}{}_{n-2}C_sp^sq^{(n-2)-s}+\fbox{(13)}\\
&=n(n-1)p^2(p+q)^{n-2}+\fbox{(13)}\\
&=\fbox{(14)}+\fbox{(13)}.
\end{align*}
이다.

따라서
\begin{align*}
V(X)
&=E(X^2)-\{E(X)\}^2\\
&=\left(\fbox{(14)}+\fbox{(13)}\right)-\fbox{(13)}^2\\
&=npq.
\end{align*}

%
\section*{답}
\begin{enumerate}[(1)]
\item
\(2\)
\item
\(\sqrt2\)
\item
\(\frac23\)
\item
\(\frac16\)
\item
\(\frac13\)
\item
\(E(X^2)\)
\item
\(E(X\) 또는 \(m\)
\item
\(1\)
\item
\(a^2E(X^2)+2abm+b^2\)
\item
\(a^2m^2+2abm+b^2\)
\item
\((n-1)!\)
\item
\(0\)
\item
\(np\)
\item
\(n(n-1)p^2\)
\end{enumerate}

\end{document}