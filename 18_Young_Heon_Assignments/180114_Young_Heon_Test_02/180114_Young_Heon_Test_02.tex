\documentclass[twocolumn]{article}
\usepackage{amsmath,amssymb,kotex,mdframed,paralist}
\usepackage{geometry}
\geometry{a4paper,margin=1in}

\usepackage{tabto,pifont}
\TabPositions{0.2\textwidth,0.4\textwidth,0.6\textwidth,0.8\textwidth}
\newcommand\tabb[5]{\par\noindent
\ding{172}\:{\ensuremath{#1}}
\tab\ding{173}\:\:{\ensuremath{#2}}
\tab\ding{174}\:\:{\ensuremath{#3}}
\tab\ding{175}\:\:{\ensuremath{#4}}
\tab\ding{176}\:\:{\ensuremath{#5}}}

\newcounter{num}
\newcommand\pb[1]{\ensuremath{\fbox{\phantom{#1}}}}

%\pagestyle{empty}

\newcommand\prob[1]
{\vspace{40pt}\par\noindent\stepcounter{num} \textbf{문제 \thenum) #1}\par\medskip\noindent}

\begin{document}
\begin{center}
\LARGE영헌, 추가과제 02
\end{center}
\begin{flushright}
\today
\end{flushright}

%
\prob{}
\(\displaystyle\lim_{n\to\infty}\frac{3n+4}{2n-1}=\)

%
\prob{}
\(\displaystyle\lim_{n\to\infty}\frac{2n+7}{2n^2+3n+2}=\)

%
\prob{}
\(\displaystyle\lim_{n\to\infty}\frac{-n^2+2n+7}{2n^2+3n+2}=\)

%
\prob{}
\(\displaystyle\lim_{n\to\infty}\frac{3n^3-n^2+2n+7}{2n^2+3n+2}=\)

%
\prob{}
\(\displaystyle\lim_{n\to\infty}\frac{(n+2)(2n+1)}{(n-1)(3n+4)}=\)

%
\prob{}
\(\displaystyle\lim_{n\to\infty}\frac{(2n+5)(4n-3)}{(5n-2)(n-1)}=\)

%
\prob{}
\(\displaystyle\lim_{n\to\infty}\frac{2n^2+4}{1+2+3+\cdots+n}=\)

%
\prob{}
\(\displaystyle\lim_{n\to\infty}\frac{2+4+6+\cdots+2n}{5n^2+2}=\)

%
\prob{}
\(\displaystyle\lim_{n\to\infty}\frac{3n^3+2n-4}{1^2+2^2+3^2+\cdots+n^2}=\)

%
\prob{}
\(\displaystyle\lim_{n\to\infty}\frac{\sqrt{3n+5}}{\sqrt{4n+1}}=\)

%
\prob{}
\(\displaystyle\lim_{n\to\infty}\frac{\sqrt{n-2}}{\sqrt{3n-1}}=\)

%
\prob{}
\(\displaystyle\lim_{n\to\infty}\frac{\sqrt{n-3}+1}{\sqrt{n+4}}=\)

%
\prob{}
\(\displaystyle\lim_{n\to\infty}\frac{\sqrt{6n+2}}{2+\sqrt{3n}}=\)

%
\prob{}
\(\displaystyle\lim_{n\to\infty}\frac{\sqrt{n-2}+\sqrt{n+3}}{\sqrt{4n}}=\)

%
\prob{}
\(\displaystyle\lim_{n\to\infty}\frac{\sqrt{4n+3}}{\sqrt{n}+\sqrt{n+3}}=\)

%
\prob{}
\(\displaystyle\lim_{n\to\infty}\frac{\sqrt{5n^2+3}}{\sqrt{n^2-2n+3}}=\)

%
\prob{}
\(\displaystyle\lim_{n\to\infty}\frac{\sqrt{n^2+4n+3}}{\sqrt{9n^2+5}}=\)

%
\prob{}
\(\displaystyle\lim_{n\to\infty}\frac{\sqrt{4n^2+1}+2n}{\sqrt{n^2-1}}=\)

%
\prob{}
\(\displaystyle\lim_{n\to\infty}\frac{\sqrt{16n-1}}{\sqrt{n-2}+\sqrt{n+2}}=\)

%
\prob{}
\(\displaystyle\lim_{n\to\infty}(-n^3+2n)=\)

%
\prob{}
\(\displaystyle\lim_{n\to\infty}\left(\frac14n+6\right)=\)

%
\prob{}
\(\displaystyle\lim_{n\to\infty}(\sqrt{n^2+5n+1}-n)=\)

%
\prob{}
\(\displaystyle\lim_{n\to\infty}(\sqrt{n+3}-\sqrt n)=\)

%
\prob{}
\(\displaystyle\lim_{n\to\infty}\sqrt{4n+2}(\sqrt{n+2}-\sqrt{n-1})=\)

%
\prob{}
\(\displaystyle\lim_{n\to\infty}\frac1{\sqrt{n+3}-\sqrt n}=\)

%
\prob{}
\(\displaystyle\lim_{n\to\infty}\frac1{n-\sqrt{n^2-n}}=\)

%
\prob{}
\(\displaystyle\lim_{n\to\infty}\left(\sqrt{n^2+an+1}-n\right)=4\)일 때, 상수 \(a\)의 값을 구하여라.

%
\prob{}
\(\displaystyle\lim_{n\to\infty}\sqrt n\left(\sqrt{n}-\sqrt{n-a}\right)=3\)일 때, 상수 \(a\)의 값을 구하여라.

%
\prob{}
수열 \(\{a_n\}\)이 \(\displaystyle\lim_{n\to\infty}\frac{2a_n-7}{5a_n+2}=1\)를 만족시킬 때, \(\displaystyle\lim_{n\to\infty}a_n\)의 값은?

%
\prob{}
수열 \(\{a_n\}\)이 \(\displaystyle\lim_{n\to\infty}\frac{4-3a_n}{2a_n+1}=3\)를 만족시킬 때, \(\displaystyle\lim_{n\to\infty}a_n\)의 값은?

%
\prob{}
수열 \(a_n\)이 \(\displaystyle\lim_{n\to\infty}(5n+1)a_n=3\)을 만족시킬 때, \(\displaystyle\lim_{n\to\infty}(2n-3)a_n\)의 값은?

%
\prob{}
수열 \(a_n\)이 \(\displaystyle\lim_{n\to\infty}(n^2-2)a_n=1\)을 만족시킬 때, \(\displaystyle\lim_{n\to\infty}(4n^2+5n+1)a_n\)의 값은?

%
\prob{}
수열 \(a_n\)이 \(\displaystyle\lim_{n\to\infty}na_n=\frac13\)을 만족시킬 때,\\ \(\displaystyle\lim_{n\to\infty}(3n+2)a_n\)의 값은?

%
\prob{}
수열 \(a_n\)이 \(\displaystyle\lim_{n\to\infty}(n^2+3n+1)a_n=2\)을 만족시킬 때,\\ \(\displaystyle\lim_{n\to\infty}n^2a_n\)의 값은?

%
\prob{}
수열 \(\{a_n\}\)에 대하여 \(\displaystyle\lim_{n\to\infty}(a_n+2)=5\)일 때, \(\displaystyle\lim_{n\to\infty}\left({a_n}^2+a_n+4\right)\)의 값은?

%
\prob{}
수열 \(\{a_n\}\)에 대하여 \(\displaystyle\lim_{n\to\infty}(a_n-2)=1\)일 때, \(\displaystyle\lim_{n\to\infty}a_n\left(a_n-1\right)\)의 값은?

%
\prob{}
수렴하는 두 수열 \(\{a_n\}\), \(\{b_n\}\)에 대하여
\[\lim_{n\to\infty}\left(a_n+b_n\right)=6,\quad\lim_{n\to\infty}a_nb_n=4\]
일 때, \(\displaystyle\lim_{n\to\infty}\left({a_n}^2+{b_n}^2\right)\)의 값은?

%
\prob{}
수렴하는 두 수열 \(\{a_n\}\), \(\{b_n\}\)에 대하여
\[\lim_{n\to\infty}\left(a_n+b_n\right)=4,\quad\lim_{n\to\infty}a_nb_n=3\]
일 때, \(\displaystyle\lim_{n\to\infty}\left({a_n}^2+{b_n}^2\right)\)의 값은?

%
\prob{}
수렴하는 두 수열 \(\{a_n\}\), \(\{b_n\}\)에 대하여
\[\lim_{n\to\infty}\left(a_n-b_n\right)=6,\quad\lim_{n\to\infty}a_nb_n=-2\]
일 때, \(\displaystyle\lim_{n\to\infty}\left({a_n}^2+{b_n}^2\right)\)의 값은?

%
\prob{}
\(\displaystyle\lim_{n\to\infty}\frac{an^2+bn+4}{3n+1}=2\)일 때, 상수 \(a\), \(b\)에 대하여 \(a+b\)의 값은?

%
\prob{}
\(\displaystyle\lim_{n\to\infty}\frac{4n+1}{an^2+bn+4}=2\)일 때, 상수 \(a\), \(b\)에 대하여 \(a+b\)의 값은?


%
\prob{}
\(\displaystyle\lim_{n\to\infty}\frac{3n^2+4n+1}{an^2-1}=6\)일 때, 상수 \(a\)의 값을 구하여라.

%
\prob{}
\(\displaystyle\lim_{n\to\infty}\frac{a(n+1)(2n+1)}{n^2+4n+1}=14\)일 때, 상수 \(a\)의 값을 구하여라.

%
\prob{}
수열 \(\{a_n\}\)이 모든 자연수 \(n\)에 대하여 부등식
\[\frac{3n-2}{n+4}<a_n<\frac{3n+1}{n+4}\]
을 만족시킬 때, \(\displaystyle\lim_{n\to\infty}a_n\)의 값을 구하여라.

%
\prob{}
수열 \(\{a_n\}\)이 모든 자연수 \(n\)에 대하여 부등식
\[5-\frac2{n+1}<a_n<5-\frac2{n+4}\]
을 만족시킬 때, \(\displaystyle\lim_{n\to\infty}a_n\)의 값을 구하여라.

%
\prob{}
수열 \(\{a_n\}\)이 모든 자연수 \(n\)에 대하여 부등식
\[n-1<(n+2)a_n<n+1\]
을 만족시킬 때, \(\displaystyle\lim_{n\to\infty}a_n\)의 값을 구하여라.

%
\prob{}
수열 \(\{a_n\}\)이 모든 자연수 \(n\)에 대하여 부등식
\[\sqrt{4n^2-1}<(n+2)a_n<\sqrt{4n^2+3}\]
을 만족시킬 때, \(\displaystyle\lim_{n\to\infty}a_n\)의 값을 구하여라.

%
\prob{}
수열 \(\{a_n\}\)이 모든 자연수 \(n\)에 대하여 부등식
\[2n+3<\sqrt{n^2+2}a_n<2n+5\]
을 만족시킬 때, \(\displaystyle\lim_{n\to\infty}a_n\)의 값을 구하여라.

\end{document}