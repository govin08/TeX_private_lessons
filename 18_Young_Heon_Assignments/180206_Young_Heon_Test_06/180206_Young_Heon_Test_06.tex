\documentclass[twocolumn]{article}
\usepackage{amsmath,amssymb,kotex,mdframed,paralist}
\usepackage{geometry}
\geometry{a4paper,margin=1in}

\usepackage{tabto,pifont}
\TabPositions{0.2\textwidth,0.4\textwidth,0.6\textwidth,0.8\textwidth}
\newcommand\tabb[5]{\par\noindent
\ding{172}\:{\ensuremath{#1}}
\tab\ding{173}\:\:{\ensuremath{#2}}
\tab\ding{174}\:\:{\ensuremath{#3}}
\tab\ding{175}\:\:{\ensuremath{#4}}
\tab\ding{176}\:\:{\ensuremath{#5}}}

\newcounter{num}
\newcommand\pb[1]{\ensuremath{\fbox{\phantom{#1}}}}

%\pagestyle{empty}

\newcommand\prob[1]
{\vspace{40pt}\par\noindent\stepcounter{num} \textbf{문제 \thenum) #1}\par\medskip\noindent}
\newcommand\exam[1]
{\vspace{40pt}\par\noindent\stepcounter{num} \textbf{예시 \thenum) #1}\par\medskip\noindent}

\begin{document}
\begin{center}
\LARGE영헌, 추가과제 06
\end{center}
\begin{flushright}
\today
\end{flushright}

%%
\section{수열의 극한}

%
\prob{}
\(\displaystyle\lim_{n\to\infty}\frac{1^2+2^2+3^2+\cdots+n^2}{n(1+2+3+\cdots+n)}=\)

%
\prob{}
\(\displaystyle\lim_{n\to\infty}\frac{\sqrt n-\sqrt{n+3}}{\sqrt n-\sqrt{n-3}}=\)

%
\prob{}
\(\displaystyle
\lim_{n\to\infty}\frac{(3n+2)^2}{4-3n^2}
+
\lim_{n\to\infty}\frac{6n+1}{\sqrt{n^2+1}+2n}=\)

%
\prob{}
\(\displaystyle\lim_{n\to\infty}\frac{\sqrt{an}}{n\left(\sqrt{2n+1}-\sqrt{2n}\right)}=4\)일 때, 상수 \(a\)의 값은?

%
\prob{}
\(\displaystyle\lim_{n\to\infty}\frac{1+9^{n+1}}{3^{n+2}+9^n}\)

%
\prob{}
\(\displaystyle\lim_{n\to\infty}\frac{3^n+4^n}{3^{n+1}+4^{n+1}}\)

\bigskip
%
\prob{}
\(\displaystyle\lim_{n\to\infty}\frac{3^{n+1}(5^n+1)}{5^n(3^{n+1}+1)}\)

%
\prob{}
\(\displaystyle\lim_{n\to\infty}\frac{(3^n+2)(2^n+1)}{6^{n+1}+3^n}\)

%
\prob{}
\(\displaystyle\lim_{n\to\infty}\frac{15^{n+1}}{(3^n+1)(5^n-1)}\)

%
\prob{}
\(\displaystyle\lim_{n\to\infty}\frac{1+2+2^2+\cdots+2^{n-1}}{2^n}\)

%
\prob{}
\(\displaystyle\lim_{n\to\infty}\frac{3+6+12+\cdots+3\cdot2^{n-1}}{2^n}\)

%
\prob{}
\(\displaystyle\lim_{n\to\infty}\frac{3+3^2+3^3+\cdots+3^n}{3^n}\)

%
\prob{}
두 수열 \(\{a_n\}\), \(\{b_n\}\)에 대하여
\[a_n+b_n=3^{n},\qquad a_n-b_n=4^{n+1}\]
일 때, \(\displaystyle\lim_{n\to\infty}\frac{b_n}{a_n}\)의 값은?

%
\prob{}
두 수열 \(\{a_n\}\), \(\{b_n\}\)에 대하여
\[a_n+b_n=3^{2n},\qquad a_n-b_n=2^{3n+1}\]
일 때, \(\displaystyle\lim_{n\to\infty}\frac{b_n}{a_n}\)의 값은?

%
\prob{}
등비수열 \(\displaystyle\left\{\left(\frac{4x-1}5\right)^n\right\}\)이 수렴하도록 하는 정수 \(x\)의 개수를 구하여라.

%
\prob{}
등비수열 \(\displaystyle\left\{\left(\frac13x+2\right)^n\right\}\)이 수렴하도록 하는 정수 \(x\)의 개수를 구하여라.

%
\prob{}
등비수열 \(\displaystyle\left\{\left(\frac{6-x}3\right)^n\right\}\)이 수렴하도록 하는 정수 \(x\)의 개수를 구하여라.

\bigskip
%
\prob{}
수열
\[(x+5),\quad(x-3)(x+5),\quad(x-3)(x+5)^2,\cdots\]
이 수렴하도록 하는 모든 정수 \(x\)의 합을 구하여라.

%
\prob{}
수열
\[(x+3),\quad(x+3)(7-x),\quad(x+3)(7-x)^2,\cdots\]
이 수렴하도록 하는 모든 정수 \(x\)의 합을 구하여라.

%
\prob{}
수열
\[(x-5),\quad(x-5)\left(\frac x3\right),\quad(x-5)\left(\frac x3\right)^2,\cdots\]
이 수렴하도록 하는 모든 정수 \(x\)의 합을 구하여라.

\clearpage
%%
\section{급수}

%
\prob{}
\(\displaystyle2+4+8+16+32=\)

%
\prob{}
\(\displaystyle3+6+12+24+48=\)

%
\prob{}
\(\displaystyle1+3+9+27+81=\)

%
\prob{}
\(\displaystyle2+1+\frac12+\frac14+\frac18=\)

%
\prob{}
\(\displaystyle1+(-2)+4+(-8)+16=\)

\bigskip\bigskip
\begin{mdframed}[frametitle=등비수열의 합 공식]
첫항이 \(a\)이고 공비가 \(r\)인 등비수열의 합은
\[S_n=\frac{a(r^n-1)}{r-1}=\frac{a(1-r^n)}{1-r}\]
\end{mdframed}

%
\exam{}
\(\displaystyle1+4+16+64+256=\frac{1(4^5-1)}{4-1}=\frac{1023}3=341\)

%
\prob{}
\(\displaystyle2+4+8+16+32=\)

%
\prob{}
\(\displaystyle3+6+12+24+48=\)

%
\prob{}
\(\displaystyle1+3+9+27+81=\)

%
\exam{}
\begin{align*}
&\frac23+\frac29+\frac2{27}+\frac2{81}+\frac2{243}
=\frac{\frac23(1-\left(\frac13\right)^5)}{1-\frac13}
=\frac{\frac23(1-\left(\frac13\right)^5)}{\frac23}\\
&=\frac23\left(1-\left(\frac13\right)^5\right)\div\frac23
=\frac23\left(1-\left(\frac13\right)^5\right)\times\frac32\\
&=1-\left(\frac13\right)^5
\end{align*}

%
\prob{}
\(\displaystyle2+1+\frac12+\frac14+\frac18=\)

%
\prob{}
\(\displaystyle1+(-2)+4+(-8)+16=\)

%%
%\exam{}
%\begin{align*}
%&1+\frac13+\frac19+\frac1{27}+\cdots\\
%=&\lim_{n\to\infty}\left(1+\frac13+\frac19+\frac1{27}+\cdots+\right)
%\end{align*}
%
%%
%\prob{}
%\(\displaystyle1+\frac12+\frac14+\frac18+\cdots=\)
%
%%
%\prob{}
%\(\displaystyle2+\frac23+\frac29+\frac2{27}+\cdots=\)
%
%%
%\prob{}
%\(\displaystyle1+\frac23+\frac49+\frac8{27}+\cdots=\)
%
%%
%\prob{}
%\(\displaystyle1-\frac13+\frac19-\frac1{27}+\cdots=\)



\end{document}