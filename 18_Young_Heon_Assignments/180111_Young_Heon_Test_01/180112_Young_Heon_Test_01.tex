\documentclass[twocolumn]{article}
\usepackage{amsmath,amssymb,kotex,mdframed,paralist}
\usepackage{geometry}
\geometry{a4paper,margin=1in}

\usepackage{tabto,pifont}
\TabPositions{0.2\textwidth,0.4\textwidth,0.6\textwidth,0.8\textwidth}
\newcommand\tabb[5]{\par\noindent
\ding{172}\:{\ensuremath{#1}}
\tab\ding{173}\:\:{\ensuremath{#2}}
\tab\ding{174}\:\:{\ensuremath{#3}}
\tab\ding{175}\:\:{\ensuremath{#4}}
\tab\ding{176}\:\:{\ensuremath{#5}}}

\newcounter{num}
\newcommand\pb[1]{\ensuremath{\fbox{\phantom{#1}}}}

%\pagestyle{empty}

\newcommand\prob[1]
{\vspace{40pt}\par\noindent\stepcounter{num} \textbf{문제 \thenum) #1}\par\medskip\noindent}

\begin{document}
\begin{center}
\LARGE영헌, 추가과제 01
\end{center}
\begin{flushright}
\today
\end{flushright}

%
\prob{}
\(\displaystyle\lim_{n\to\infty}\frac n{n+3}=\)

%
\prob{}
\(\displaystyle\lim_{n\to\infty}\frac{2n}{3n+1}=\)

%
\prob{}
\(\displaystyle\lim_{n\to\infty}\frac{4n+9}{n-2}=\)

%
\prob{}
\(\displaystyle\lim_{n\to\infty}\frac{-n+4}{3n+1}=\)

%
\prob{}
\(\displaystyle\lim_{n\to\infty}\frac{\frac13n+\frac23}{\frac32n-\frac12}=\)

%
\prob{}
\(\displaystyle\lim_{n\to\infty}\frac{n^2}{2n^2+3}=\)

%
\prob{}
\(\displaystyle\lim_{n\to\infty}\frac{2n^2+1}{n+2}=\)

%
\prob{}
\(\displaystyle\lim_{n\to\infty}\frac{3n+8}{n^2+1}=\)

%
\prob{}
\(\displaystyle\lim_{n\to\infty}\frac{n^2+2n-1}{-3n+1}=\)

%
\prob{}
\(\displaystyle\lim_{n\to\infty}\frac{2n^2-3n+1}{5n^2+3}=\)

%
\prob{}
\(\displaystyle\lim_{n\to\infty}\frac{-n^2+4n}{(n+3)(2n+1)}=\)

%
\prob{}
\(\displaystyle\lim_{n\to\infty}\frac{3n^2+4n+1}{\frac12n^2+4n-5}=\)

%
\prob{}
\(\displaystyle\lim_{n\to\infty}\frac{3n^2+4n+1}{\frac12n^2+4n-5}=\)

%
\prob{}
\(\displaystyle\lim_{n\to\infty}\frac{1+2+3+\cdots+n}{2n^2-n}=\)

%
\prob{}
\(\displaystyle\lim_{n\to\infty}\frac{n^2+3n}{1+2+3+\cdots+n}=\)

%
\prob{}
\(\displaystyle\lim_{n\to\infty}\frac{(n+1)(n+2)}{1+2+3+\cdots+n}=\)

%
\prob{}
\(\displaystyle\lim_{n\to\infty}\frac{2+4+6+\cdots+2n}{n^2+1}=\)

%
\prob{}
\(\displaystyle\lim_{n\to\infty}\frac{3n^2-2n-5}{3+6+9+\cdots+3n}=\)

%
\prob{}
\(\displaystyle\lim_{n\to\infty}\frac{1+2+3+\cdots+2n}{n^2-3}=\)

%
\prob{}
\(\displaystyle\lim_{n\to\infty}\frac{1+2+3+\cdots+3n}{3n^2+4n+1}=\)

%
\prob{}
\(\displaystyle\lim_{n\to\infty}\frac{1^2+2^2+3^2+\cdots+n^2}{n^3+3n-1}=\)

%
\prob{}
\(\displaystyle\lim_{n\to\infty}\frac{\sqrt{n+1}}{\sqrt n}=\)

%
\prob{}
\(\displaystyle\lim_{n\to\infty}\frac{\sqrt{2n+3}}{\sqrt{n+2}}=\)

%
\prob{}
\(\displaystyle\lim_{n\to\infty}\frac{\sqrt{n+4}}{\sqrt{4n-5}}=\)

%
\prob{}
\(\displaystyle\lim_{n\to\infty}\frac{\sqrt{n+1}}{\sqrt n}=\)

%%
%\prob{}
%\(\displaystyle\lim_{n\to\infty}\frac{\sqrt{n-3}+1}{\sqrt{n+4}}=\)
%
%%
%\prob{}
%\(\displaystyle\lim_{n\to\infty}\frac{\sqrt{6n+2}}{2+\sqrt{3n}}=\)

%
\prob{}
\(\displaystyle\lim_{n\to\infty}\frac{\sqrt{n-1}+\sqrt{n+1}}{\sqrt{2n}}=\)

%
\prob{}
\(\displaystyle\lim_{n\to\infty}\frac{\sqrt{2n-1}}{\sqrt{n+1}+\sqrt{n+2}}=\)

%
\prob{}
\(\displaystyle\lim_{n\to\infty}\frac{\sqrt{4n+1}}{\sqrt{9n-4}}=\)

%
\prob{}
\(\displaystyle\lim_{n\to\infty}\frac{\sqrt{16n^2+1}}{\sqrt{4n^2+3}}=\)

%
\prob{}
\(\displaystyle\lim_{n\to\infty}\frac{\sqrt{n^2+4n+3}}{\sqrt{9n^2+5}}=\)

%
\prob{}
\(\displaystyle\lim_{n\to\infty}\frac{\sqrt{n^2+2}+3n}{\sqrt{4n^2-1}}=\)

%
\prob{}
\(\displaystyle\lim_{n\to\infty}\frac{\sqrt{5n+3}}{\sqrt{n+3}+\sqrt n}=\)

%
\prob{}
\(\displaystyle\lim_{n\to\infty}(2n^2-n)=\)

%
\prob{}
\(\displaystyle\lim_{n\to\infty}\left(3n-\frac12n^2\right)=\)

%
\prob{}
\(\displaystyle\lim_{n\to\infty}(2-3n)=\)

%
\prob{}
\(\displaystyle\lim_{n\to\infty}(n-3\sqrt n)=\)

%
\prob{}
\(\displaystyle\lim_{n\to\infty}(\sqrt{n^2-2n}-n)=\)

%
\prob{}
\(\displaystyle\lim_{n\to\infty}(\sqrt{n^2+5n+3}-n)=\)

%
\prob{}
\(\displaystyle\lim_{n\to\infty}(\sqrt{n+1}-\sqrt n)=\)

%
\prob{}
\(\displaystyle\lim_{n\to\infty}\sqrt n(\sqrt{n+4}-\sqrt n)=\)

%
\prob{}
\(\displaystyle\lim_{n\to\infty}\frac1{\sqrt{n-2}-\sqrt n}=\)

%
\prob{}
\(\displaystyle\lim_{n\to\infty}\frac1{n-\sqrt{n^2-3n}}=\)

%
\prob{}
\(\displaystyle\lim_{n\to\infty}\left(2n-\sqrt{4n^2+7n+1}\right)=\)

%
\prob{}
\(\displaystyle\lim_{n\to\infty}\left(\sqrt{4n^2+an+1}-2n\right)=\frac12\)일 때, 상수 \(a\)의 값을 구하여라.

%
\prob{}
\(\displaystyle\lim_{n\to\infty}\sqrt n\left(\sqrt{n+a}-\sqrt{n-1}\right)=6\)일 때, 상수 \(a\)의 값을 구하여라.

%
\prob{}
\(\displaystyle\lim_{n\to\infty}\frac1{\sqrt{n^2+an}-n}=1\)일 때, 상수 \(a\)의 값을 구하여라.

%
\prob{}
수열 \(\{a_n\}\)이 \(\displaystyle\lim_{n\to\infty}\frac{a_n+3}{a_n+1}=2\)를 만족시킬 때, \(\displaystyle\lim_{n\to\infty}a_n\)의 값은?

%
\prob{}
수열 \(\{a_n\}\)이 \(\displaystyle\lim_{n\to\infty}\frac{3a_n+1}{a_n+2}=4\)를 만족시킬 때, \(\displaystyle\lim_{n\to\infty}a_n\)의 값은?

%
\prob{}
수열 \(\{a_n\}\)이 \(\displaystyle\lim_{n\to\infty}\frac{-2a_n-3}{3a_n+3}=1\)를 만족시킬 때, \(\displaystyle\lim_{n\to\infty}a_n\)의 값은?

%
\prob{}
수열 \(a_n\)이 \(\displaystyle\lim_{n\to\infty}(2n+3)a_n=3\)을 만족시킬 때, \(\displaystyle\lim_{n\to\infty}(n-2)a_n\)의 값은?

%
\prob{}
수열 \(a_n\)이 \(\displaystyle\lim_{n\to\infty}(5n-3)a_n=1\)을 만족시킬 때, \(\displaystyle\lim_{n\to\infty}(3n+5)a_n\)의 값은?

%
\prob{}
수열 \(a_n\)이 \(\displaystyle\lim_{n\to\infty}na_n=\frac12\)을 만족시킬 때,\\ \(\displaystyle\lim_{n\to\infty}(4n-1)a_n\)의 값은?

%
\prob{}
수열 \(a_n\)이 \(\displaystyle\lim_{n\to\infty}(n^2+2)a_n=2\)을 만족시킬 때,\\ \(\displaystyle\lim_{n\to\infty}(n^2+2n-2)a_n\)의 값은?

%
\prob{}
수열 \(a_n\)이 \(\displaystyle\lim_{n\to\infty}(2n^2+3n+2)a_n=6\)을 만족시킬 때,\\ \(\displaystyle\lim_{n\to\infty}(3n^2+7)a_n\)의 값은?

%
\prob{}
수열 \(a_n\)이 \(\displaystyle\lim_{n\to\infty}\sqrt{n^2+1}a_n=2\)을 만족시킬 때,\\ \(\displaystyle\lim_{n\to\infty}2na_n\)의 값은?


\end{document}