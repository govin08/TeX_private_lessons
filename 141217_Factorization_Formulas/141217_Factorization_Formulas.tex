\documentclass{article}
\usepackage{amsmath,amssymb,amsthm,kotex,paralist,wasysym}
%\newcounter{num}[section]
%\newcommand{\defi}[1]
%{\bigskip\noindent\refstepcounter{num}\textbf{정의 \arabic{section}. \arabic{num}) #1}\par}
%\newcommand{\theo}[1]
%{\bigskip\noindent\refstepcounter{num}\textbf{정리 \arabic{section}. \arabic{num}) #1}\par}
%\newcommand{\axio}[1]
%{\bigskip\noindent\refstepcounter{num}\textbf{공리 \arabic{section}. \arabic{num}) #1}\par}
%
%\newcommand{\notiff}{%
%  \mathrel{{\ooalign{\hidewidth$\not\phantom{"}$\hidewidth\cr$\iff$}}}}
%\newcommand{\LHS}{\text{LHS}}
%\newcommand{\RHS}{\text{RHS}}
%\newcommand{\irange}{\ensuremath{1\le i\le n}}
%\newcommand{\jrange}{\ensuremath{1\le j\le n}}
%\newcommand{\bb}[2]{\ensuremath{(^{#1}_{#2})}}
%\newcommand{\cc}[2]{\ensuremath{_{#1}C_{#2}}}
%
%\renewcommand{\figurename}{그림.}
%\renewcommand{\proofname}{증명.}
%\renewcommand{\contentsname}{목차}
%\renewcommand\emph{\textbf}

%%%
\begin{document}

\title{인수분해 공식}
\author{}
\date{\today}
\maketitle
%\tableofcontents

%%

%
실수 \(a\), \(b\), \(c\), \(d\), \(x\)에 대해 다음 식들이 성립한다.
\begin{enumerate}[(1)]
\item
\((a+b)^2=a^2+2ab+b^2\)
\item
\((a-b)^2=a^2-2ab+b^2\)
\item
\(a^2+b^2=(a+b)^2-2ab=(a-b)^2+2ab\)
\item
\((a+b)^2=(a-b)^2+4ab\)
\item
\((x+\frac1x)^2=x^2+\frac1{x^2}+2\)
\item
\((x-\frac1x)^2=x^2+\frac1{x^2}-2\)
\item
\(x^2+\frac1{x^2}=(x+\frac1x)^2-2=(x-\frac1x)^2+2\)
\item
\((x+\frac1x)^2=(x-\frac1x)^2+4\)
\item
\((a-b)(a+b)=a^2-b^2\)
\item
\((x+\frac1x)(x-\frac1x)=x^2-\frac1{x^2}\)
\item
\((x+a)(x+b)=x^2+(a+b)x+ab\)
\item
\((ax+b)(cx+d)=acx^2+(ad+bc)x+bd\)
\item
\((x+a)(x+b)(x+c)=x^3+(a+b+c)x^2+(ab+bc+ca)x+abc\)
\item
\((x-a)(x-b)(x-c)=x^3-(a+b+c)x^2+(ab+bc+ca)x-abc\)
\item
\((a+b+c)^2=a^2+b^2+c^2+2(ab+bc+ca)\)
\item
\(a^2+b^2+c^2=(a+b+c)^2-2(ab+bc+ca)\)
\item
\((a+b)^3=a^3+3a^2b+3ab^2+b^3=a^3+b^3+3ab(a+b)\)
\item
\((a-b)^3=a^3-3a^2b+3ab^2-b^3=a^3-b^3-3ab(a-b)\)
\item
\(a^3+b^3=(a+b)^3-3ab(a+b)\)
\item
\(a^3-b^3=(a-b)^3+3ab(a-b)\)
\item
\(a^3+b^3=(a+b)(a^2-ab+b^2)\)
\item
\(a^3-b^3=(a-b)(a^2+ab+b^2)\)
\item
\((a+b+c)(a^2+b^2+c^2-ab-bc-ca)=a^3+b^3+c^3-3abc\)\\
\(*\;a+b+c=0\)이면 \(a^3+b^3+c^3=3abc\)\\
\(*\;a^2+b^2+c^2-ab-bc-ca=\frac12[(a-b)^2+(b-c)^2+(c-a)^2]\)
\item
\(a^4+a^2b^2+b^4=(a^2+ab+b^2)(a^2-ab+b^2)\).
\item
\((a+b)^4=a^4+4a^3b+6a^2b^2+4ab^3+b^4\).
\item
\((a-b)^4=a^4-4a^3b+6a^2b^2-4ab^3+b^4\).
\end{enumerate}

실수 \(a\), \(b\), \(x\), 자연수 \(n, k\)에 대해
\begin{enumerate}
\item[(27)]
\(a^n-b^n=(a-b)(a^{n-1}+a^{n-2}b+a^{n-3}b^2+\cdots +a^2b^{n-3}+ab^{n-2}+b^{n-1})\)
\item[(28)]
\(x^n-1=(x-1)(x^{n-1}+x^{n-2}+x^{n-3}+\cdots+x^2+x+1)\)
\item[(29)]
\(a^{2k+1}+b^{2k+1}=(a+b)(a^{2k}-a^{2k-1}b+a^{2k-2}b^2-\cdots-ab^{2k-1}+b^{2k})\)
\item[(30)]
\(x^{2k+1}+1=(x+1)(x^{2k}-x^{2k-1}+x^{2k-2}-\cdots-x+1)\)
\end{enumerate}

\end{document}