\documentclass[t,8pt]{beamer}
%\usetheme{boxes}
%\usecolortheme{default}

%%% packages
\geometry{paperwidth=140mm,paperheight=105mm}
\usefonttheme[onlymath]{serif}

\usepackage{kotex,amsmath,tabto,setspace}

\usepackage{xcolor}

%%% counters, commands, environments
\newcounter{num}
\resetcounteronoverlays{num}

\newenvironment{defi}[1]{\refstepcounter{num}\begin{block}{정의 \arabic{num}#1}}{\end{block}}
\newenvironment{theo}[1]{\refstepcounter{num}\begin{block}{정리 \arabic{num}#1}}{\end{block}}
\newenvironment{prob}[1]{\refstepcounter{num}\begin{block}{문제 \arabic{num}#1}}{\end{block}}
\newenvironment{exam}[1]{\refstepcounter{num}\begin{block}{예시 \arabic{num}#1}}{\end{block}}

\newcommand{\pb}[1]%\Phantom + fBox
{\fbox{\phantom{\ensuremath{#1}}}}
\newcommand{\rb}[2]%\Red+fBox
{\fbox{\uncover<#1>{\red{\ensuremath{#2}}}}}
\renewcommand{\arraystretch}{1.5}
\newcommand{\red}[1]{\color{red}{#1}}
\newcommand{\ivs}{\centering\strut\vspace*{-\baselineskip}\newline}%image vertical setting

%%% title
\title{수열}
\institute[ibedu]{아이비에듀}
\date{\today}

%%% toc
\AtBeginSection[]
{\begin{frame}
    \frametitle{목차}
    \tableofcontents[currentsection]
  \end{frame}}


\begin{document}
%%
\frame{\titlepage}

%%%%
\section{등차수열과 등비수열}

%%%
\subsection{수열의 뜻}

%%
\begin{frame}[t]{\subsecname}
%
\begin{minipage}[t]{.5\textwidth}
\begin{prob}{) 빈칸에 알맞은 숫자를 넣어라.}\par\:
\begin{enumerate}[(1)]
\setlength{\itemsep}{25pt}
\item<1-5>
3, 5, 7, 9, \rb{2-}{11}, \rb{2-}{13}, \(\cdots\)
\item<1-3,6-9>
5, 10, 15, 20, \rb{2-}{25}, \rb{2-}{30}, \(\cdots\)
\item<1-3,10-13>
2, 4, 8, 16, \rb{2-}{32}, \rb{2-}{64}, \(\cdots\)
\item<1-3>
2, 6, 18, 54, \rb{2-}{162}, \rb{2-}{486}, \(\cdots\)
\item<1-3,14-17>
1, 4, 9, 16, \rb{2-}{25}, \rb{2-}{36}, \(\cdots\)
\item<1-3,18-21>
\(1\), \(\frac12\), \(\frac13\), \(\frac14\), \rb{2-}{\frac15}, \rb{2-}{\frac16}, \(\cdots\)
\end{enumerate}
\end{prob}
\end{minipage}
\begin{minipage}[t]{.45\textwidth}
\uncover<3->{
%
\begin{defi}{) 수열}
숫자들의 나열을 \alert{수열}이라고 한다.
그리고 수열의 각 숫자들을 \alert{항}이라고 한다.
\end{defi}}

\uncover<4-5>{
%
\begin{exam}{) }
첫째항은 3이고 둘째항은 5이다.\\
이것을 \(a_1=3\), \(a_2=5\) 등으로 표시한다.
따라서 \(a_3=\rb{5}{7}\), \(a_4=\rb{5}{9}\) 등이다.
\end{exam}}

\uncover<6->{
%
\begin{prob}{) }
\end{prob}}
\uncover<6-9>{
\(a_1=\rb{7-9}{5}\), \(a_2=\rb{7-9}{10}\), \(a_3=\rb{7-9}{15}\), \(a_4=\rb{7-9}{20}\).}
\uncover<8-9>{
따라서, 자연수 \(n\)에 대하여 \(a_n=\rb{9}{5n}\) 이다.}

\vspace{-25pt}
\uncover<10-13>{
\(a_1=\rb{11-13}{2}\), \(a_2=\rb{11-13}{4}\), \(a_3=\rb{11-13}{8}\), \(a_4=\rb{11-13}{16}\).}
\uncover<12-13>{
따라서, 자연수 \(n\)에 대하여 \(a_n=\rb{13}{2^n}\) 이다.}

\vspace{-25pt}
\uncover<14-17>{
\(a_1=\rb{15-17}{1}\), \(a_2=\rb{15-17}{4}\), \(a_3=\rb{15-17}{9}\), \(a_4=\rb{15-17}{16}\).}
\uncover<16-17>{
따라서, 자연수 \(n\)에 대하여 \(a_n=\rb{17}{n^2}\) 이다.}

\vspace{-25pt}
\uncover<18-21>{
\(a_1=\rb{19-21}{1}\), \(a_2=\rb{19-21}{\frac12}\), \(a_3=\rb{19-21}{\frac13}\), \(a_4=\rb{19-21}{\frac14}\).}
\uncover<20-21>{
따라서, 자연수 \(n\)에 대하여 \(a_n=\rb{21}{\frac1n}\) 이다.}
\end{minipage}

\end{frame}

%%%
\subsection{등차수열}
%%
\begin{frame}[t]{\subsecname}
\begin{minipage}[t]{.53\textwidth}
%
\begin{prob}{) 다음 수열의 일반항을 구하여라.}\:\par\vspace{-5pt}
\begin{enumerate}[(1)]
\setlength{\itemsep}{5pt}
\item
3, 5, 7, 9, 11, 13, 15, \(\cdots\)
\item
5, 10, 15, 20, 25, 30, 35, \(\cdots\)
\end{enumerate}
\end{prob}
\end{minipage}
\pause
\begin{minipage}[t]{.45\textwidth}
\vspace{18pt}
\begin{enumerate}[(1)]
\setlength{\itemsep}{5pt}
\item
\(\red{a_n=2n+1}\)
\item
\(\red{a_n=5n}\)
\end{enumerate}
\end{minipage}

\pause\bigskip
\begin{minipage}[t]{.53\textwidth}
%
\begin{theo}{) 등차수열의 일반항}\:\par\vspace{-5pt}
첫항이 \(a\)이고 공차가 \(d\)인 수열 \(\{a_n\}\)의 일반항은\par\smallskip
\(a_n=a+(n-1)d\)이다.
\end{theo}
\end{minipage}

\pause\bigskip
\begin{minipage}[t]{.53\textwidth}
%
\begin{prob}{) 다음 수열의 일반항을 구하여라.}\:\par\vspace{-5pt}
\begin{enumerate}[(1)]
\setcounter{enumi}{2}
\setlength{\itemsep}{5pt}
\item
\(-2\), \(2\), \(6\), \(10\), \(14\), \(18\), \(22\), \(\cdots\)
\item
\(9\), \(7\), \(5\), \(3\), \(1\), \(-1\), \(-3\), \(-5\), \(\cdots\)
\item
\(1\), \(\frac32\), \(2\), \(\frac52\), \(3\), \(\frac72\), \(4\), \(\cdots\)
\end{enumerate}
\end{prob}
\pause\bigskip
\end{minipage}
\pause
\begin{minipage}[t]{.45\textwidth}
\vspace{18pt}
\begin{enumerate}[(1)]
\setcounter{enumi}{2}
\setlength{\itemsep}{5pt}
\item
\(\red{a_n=4n-6}\)
\item
\(\red{a_n=-2n+11}\)
\item
\(\red{a_n=\frac12n+\frac12}\)
\end{enumerate}
\end{minipage}

\pause
\begin{minipage}[t]{.53\textwidth}
%
\begin{theo}{) 등차중항}
세 숫자 \(a\), \(b\), \(c\)가 차례대로 등차수열을 이룰 때, \(b\)를 \(a\)와 \(c\)의 \alert{등차중항}이라고 부른다.
\[b=\frac{a+c}2\]
\end{theo}
\end{minipage}
\end{frame}

%%%
\subsection{등차수열의 합}

%%
\begin{frame}[t]{\subsecname}
%
\begin{exam}{) \(3+5+7+9+\cdots+21\)을 계산하여라.}\:\par
\begin{spacing}{1.3}
\(a_1=3\), \(a_2=5\), \(a_3=7\), \(\cdots\)라고 하면, \(a=3\), \(d=2\)이다.
따라서 \[a_n=3+(n-1)2=2n+1\]이다.
\(2n+1=21\)로부터 \(n=10\).
\(S=3+5+7+9+\cdots+21\)이라고 하면,
\begin{align*}
S&= 3+ 5 +\cdots+19+21\\
S&=21+19+\cdots+ 5+ 3
\end{align*}
이다.
두 식을 더하면
\begin{align*}
2S
&=(3+21)+(5+19)+\cdots+(19+5)+(21+3)\\
&=24+24+\cdots+24+24\\
&=24\times 10\\
&=240
\end{align*}
이다.
따라서 \(S=120\)이다.
\end{spacing}
\end{exam}
\end{frame}

%%
\begin{frame}[t]{\subsecname}
%
\begin{prob}{) \(5+10+15+20+\cdots+100\)을 계산하여라.}\:\par
\begin{spacing}{1.3}
\(a_1=\rb25\), \(a_2=\rb2{10}\), \(a_3=\rb2{15}\), \(\cdots\)라고 하면, \(a=\rb25\), \(d=\rb25\)이다.
따라서 \[a_n=\rb2{5n}\]이다.
\(a_n=100\)로부터 \(n=\rb2{20}\).
\(S=5+10+15+20+\cdots+100\)이라고 하면,
\begin{align*}
S&= 5  +10+\cdots+95+100\\
S&=100+95+\cdots+10+ 5
\end{align*}
이다.
두 식을 더하면
\begin{align*}
2S=\rb2{2100}
\end{align*}
이다.
따라서 \(S=\rb2{1050}\)이다.
\end{spacing}
\end{prob}
\end{frame}

%%
\begin{frame}[t]{\subsecname}
%
\begin{theo}{) 등차수열의 합}
첫항이 \(a\)이고 공차가 \(d\)인 수열의 \(n\)항까지의 합을 \(S\)라고 하면 (\(l=a_n\)),
\[S=\frac{n\left[2a+(n-1)d\right]}2=\frac{n(a+l)}2\]
\end{theo}
%\begin{spacing}{1.3}
%\(a_1=a\), \(a_2=a+d\), \(a_3=a+2d\), \(\cdots\), \(a_{n-1}=a+(n-2)d\), \(a_n=a+(n-1)d\)이므로,
%\(S=a+(a+d)+(a+2d)+\cdots+a_{n-1}+a+(n-2)d+a_n+a+(n-1)d\)이다.
%따라서
%\begin{align*}
%S&=a 		+&&a+d		+&&\cdots 	&&+a+(n-2)d 	+&&a+(n-1)d\\
%S&=a+(n-1)d 	+&&a+(n-2)d	+&&\cdots 	&&+a+d 		+&&a
%\end{align*}
%이다.
%두 식을 더하면
%\begin{align*}
%2S
%&=(3+21)+(5+19)+\cdots+(19+5)+(21+3)\\
%&=24+24+\cdots+24+24\\
%&=24\times 10\\
%&=240
%\end{align*}
%이다.
%따라서 \(S=120\)이다.
%\end{spacing}
\pause
%
\begin{prob}{) 위의 공식을 활용하여 다음을 계산하여라.}\:\par
\begin{enumerate}[(1)]\setlength{\itemsep}{25pt}
\item
\(3+5+7+\cdots+21\)
\begin{itemize}\setlength{\itemsep}{10pt}
\item
\(a=\rb33\), \(d=\rb32\), \(n=\rb3{10}\), \(l=\rb3{21}\)
\item
\(S=\rb3{120}\)
\end{itemize}
\item
\(5+10+15+\cdots+100\)
\begin{itemize}\setlength{\itemsep}{10pt}
\item
\(a=\rb35\), \(d=\rb35\), \(n=\rb3{20}\), \(l=\rb3{100}\)
\item
\(S=\rb3{1050}\)
\end{itemize}
\end{enumerate}
\end{prob}
\end{frame}

%%%
\subsection{등비수열}
%%
\begin{frame}[t]{\subsecname}
\begin{minipage}[t]{.53\textwidth}
%
\begin{prob}{) 다음 수열의 일반항을 구하여라.}\:\par\vspace{-5pt}
\begin{enumerate}[(1)]
\setlength{\itemsep}{5pt}
\item
2, 4, 8, 16, 32, 64, \(\cdots\)
\item
2, 6, 18, 54, 162, 486, \(\cdots\)
\end{enumerate}
\end{prob}
\end{minipage}
\pause
\begin{minipage}[t]{.45\textwidth}
\vspace{18pt}
\begin{enumerate}[(1)]
\setlength{\itemsep}{5pt}
\item
\(\red{a_n=2^n}\)
\item
\(\red{a_n=2\times3^{n-1}}\)
\end{enumerate}
\end{minipage}

\pause\bigskip
\begin{minipage}[t]{.53\textwidth}
%
\begin{theo}{) 등비수열의 일반항}\:\par\vspace{-5pt}
첫항이 \(a\)이고 공비가 \(r\)인 수열 \(\{a_n\}\)의 일반항은\par\smallskip
\(a_n=a\times r^{n-1}\)이다.
\end{theo}
\end{minipage}

\pause\bigskip
\begin{minipage}[t]{.53\textwidth}
%
\begin{prob}{) 다음 수열의 일반항을 구하여라.}\:\par\vspace{-5pt}
\begin{enumerate}[(1)]
\setcounter{enumi}{2}
\setlength{\itemsep}{5pt}
\item
\(3\), \(6\). \(12\), \(24\), \(48\), \(\cdots\)
\item
\(2\), \(-2\), \(2\), \(-2\), \(2\), \(\cdots\)
\item
\(6\), \(3\), \(\frac32\), \(\frac34\), \(\frac38\), \(\cdots\)
\end{enumerate}
\end{prob}
\pause\bigskip
\end{minipage}
\pause
\begin{minipage}[t]{.45\textwidth}
\vspace{18pt}
\begin{enumerate}[(1)]
\setcounter{enumi}{2}
\setlength{\itemsep}{5pt}
\item
\(\red{a_n=3\times2^{n-1}}\)
\item
\(\red{a_n=2\times(-1)^{n-1}}\)
\item
\(\red{a_n=6\times\left(\frac12\right)^{n-1}}\)
\end{enumerate}
\end{minipage}

\pause
\begin{minipage}[t]{.53\textwidth}
%
\begin{theo}{) 등비중항}
세 숫자 \(a\), \(b\), \(c\)가 차례대로 등비수열을 이룰 때, \(b\)를 \(a\)와 \(c\)의 \alert{등비중항}이라고 부른다.
\[b=\sqrt{ac}\]
\end{theo}
\end{minipage}
\end{frame}

%%%
\subsection{등비수열의 합}

%%
\begin{frame}[t]{\subsecname}
%
\begin{exam}{) \(2+6+18+54+\cdots+486\)을 계산하여라.}\:\par
\begin{spacing}{1.3}
\(a_1=2\), \(a_2=6\), \(a_3=18\), \(\cdots\)라고 하면, \(a=2\), \(r=3\)이다.
따라서 \[a_n=2\times3^{n-1}\]이다.
\(2\times3^{n-1}=486\)로부터 \(3^{n-1}=243\)이고, 따라서 \(n=6\) 이다.
\(S=2+6+18+54+\cdots+486\)이라고 하면,
\begin{align*}
S 	&= 2+ 6 +\cdots+486\\
3S 	&=\phantom{2+\;\;} 6+\cdots +486+1458
\end{align*}
이다.
두 식을 빼면
\begin{align*}
2S
&=1458-2\\
&=1456
\end{align*}
이다.
따라서 \(S=728\)이다.
\end{spacing}
\end{exam}
\end{frame}

%%
\begin{frame}[t]{\subsecname}
%
\begin{prob}{) \(3+6+12+24+\cdots+384\)을 계산하여라.}\:\par
\begin{spacing}{1.3}
\(a_1=\rb23\), \(a_2=\rb26\), \(a_3=\rb2{12}\), \(\cdots\)라고 하면, \(a=\rb23\), \(r=\rb22\)이다.
따라서 \[a_n=\rb2{3\times2^{n-1}}\]이다.
\(a_n=384\)로부터 \(n=\rb2{8}\).
\(S=3+6+12+24+\cdots+384\)이라고 하면,
\begin{align*}
S 		&= 3  +			6+\cdots+384\\
\rb22S 	&=\phantom{3+\;}	6+\cdots+384+768
\end{align*}
이다.
두 식을 빼면
\begin{align*}
S=\rb2{765}
\end{align*}
이다.
\end{spacing}
\end{prob}
\end{frame}

%%
\begin{frame}[t]{\subsecname}
%
\begin{theo}{) 등비수열의 합}
첫항이 \(a\)이고 공비가 \(r\)인 수열의 \(n\)항까지의 합을 \(S\)라고 하면(\(r\neq1\)),
\[S=\frac{a(r^n-1)}{r-1}=\frac{a(1-r^n)}{1-r}\]
\end{theo}
\pause
%
\begin{prob}{) 위의 공식을 활용하여 다음을 계산하여라.}\:\par
\begin{enumerate}[(1)]\setlength{\itemsep}{25pt}
\item
\(2+4+8+\cdots+128\)
\begin{itemize}\setlength{\itemsep}{10pt}
\item
\(a=\rb32\), \(r=\rb32\), \(n=\rb37\)
\item
\(S=\rb3{254}\)
\end{itemize}
\item
\(6+3+\frac32+\cdots+\frac3{64}\)
\begin{itemize}\setlength{\itemsep}{10pt}
\item
\(a=\rb36\), \(r=\rb3{\frac12}\), \(n=\rb38\)
\item
\(S=\rb3{12\left(1-\left(\frac12\right)^8\right)}\)
\end{itemize}
\end{enumerate}
\end{prob}
\end{frame}

%%%%
\section{수열의 합}

%%%
\subsection{합의 기호}

%%
\begin{frame}[t]{\subsecname}
%
\begin{defi}{) 합의 기호}\:\par
수열 \(\{a_n\}\)의 첫째항부터 \(n\)번째 항까지의 합은 다음과 같이 나타낸다.
\[\sum_{k=1}^na_k=a_1+a_2+\cdots+a_n.\]
\end{defi}
\pause\par\bigskip
%
\begin{exam}{}%\vspace{-30pt}
\begin{align*}
(1)	\;\;&\sum_{k=1}^{10}(2k+1)=\rb33+\rb35+\rb37+\cdots+\rb3{21}\\
(2)	\;\;&3+9+27+81+243=\sum_{\rb3{k=1}}^{\rb3{5}}\rb3{3^k}
\end{align*}
\end{exam}
\end{frame}

%%
\begin{frame}[t]{\subsecname}
%
\begin{prob}{}\vspace{-30pt}
\begin{align*}
(1)	\;\;&\sum_{k=1}^{15}(3k)=\rb23+\rb26+\rb29+\cdots+\rb2{45}\\
(2)	\;\;&\sum_{k=1}^{10}\frac1k=\rb21+\rb2{\frac12}+\rb2{\frac13}+\cdots+\rb2{\frac1{10}}\\
(3)	\;\;&\sum_{i=1}^4{4^i}=\rb24+\rb2{16}+\rb2{64}+\rb2{256}\\[10pt]
(4)	\;\;&3+7+11+\cdots+99=\sum_{\rb2{k=1}}^{\rb2{25}}\rb2{4k-1}\\[10pt]
(5)	\;\;&1+5+25+\cdots+5^8=\sum_{\rb2{k=1}}^{\rb2{9}}\rb2{5^{k-1}}\\[10pt]
(6)	\;\;&1+4+9+\cdots+100=\sum_{\rb2{k=1}}^{\rb2{10}}\rb2{k^2}
\end{align*}
\end{prob}
\end{frame}

%%%
\subsection{자연수의 거듭제곱의 합}

%%
\begin{frame}[t]{\subsecname}
%
\begin{theo}{}\vspace{-30pt}
\begin{align*}
(1)\;\;&\sum_{k=1}^n k=\frac{n(n+1)}2\\
(2)\;\;&\sum_{k=1}^n k^2=\frac{n(n+1)(2n+1)}6\\
(3)\;\;&\sum_{k=1}^n k^3=\left\{\frac{n(n+1)}2\right\}^2
\end{align*}
\end{theo}
\pause
%
\begin{exam}{}
\begin{align*}
(1)\;\;\sum_{k=1}^{10}(2k+1)
&=2\sum_{k=1}^{10}k+\sum_{k=1}^{10}1\\
&=2\cdot\frac{10\times11}2+10\cdot1=65\\
(2)\;\;\sum_{k=1}^6k^2(k-1)
&=\sum_{k=1}^6k^3-\sum_{k=1}^6k^2\\
&=\left(\frac{6\times7}2\right)^2-\frac{6\times7\times13}6=350
\end{align*}
\end{exam}
\end{frame}

%%
\begin{frame}[t]{\subsecname}
%
\begin{prob}{}
\begin{align*}
(1)\;\;	&\sum_{k=1}^{5}(4k+3)=\rb2{75}\\
(2)\;\;	&\sum_{k=1}^{15}\left(\frac13k-2\right)=\rb2{10}\\
(3)\;\;	&\sum_{k=1}^{10}k(k+1)=\rb2{440}
\end{align*}
\end{prob}
\end{frame}

%%%%
\section{수학적 귀납법}

%%%
\subsection{수열의 귀납적 정의}

%%
\begin{frame}[t]{\subsecname}
\end{frame}

%%
\begin{frame}[t]{\subsecname}
\end{frame}

%%
\begin{frame}[t]{\subsecname}
\end{frame}

%%%
\subsection{수학적 귀납법}

%%
\begin{frame}[t]{\subsecname}
\end{frame}

%%
\begin{frame}[t]{\subsecname}
\end{frame}

%%
\begin{frame}[t]{\subsecname}
\end{frame}

\end{document}