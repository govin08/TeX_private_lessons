\documentclass[t,8pt]{beamer}

%%% packages
\geometry{paperwidth=140mm,paperheight=105mm}
\usefonttheme[onlymath]{serif}

\usepackage{kotex,amsmath,tabto,setspace,tikz}

\usepackage{tabularx}
%X is not not centered, text moded
\newcolumntype{Y}{>{\centering\arraybackslash}X} % centered, text-moded
\newcolumntype{M}{>{$}X<{$}} % math-moded
\newcolumntype{N}{>{$}Y<{$}} %centered and math-moded
\setlength\tabcolsep{1pt}


%%% counters, commands, environments
\newcounter{num}
\resetcounteronoverlays{num}

\newenvironment{defi}[1]{\refstepcounter{num}\begin{block}{정의 \arabic{num}#1}}{\end{block}}
\newenvironment{theo}[1]{\refstepcounter{num}\begin{block}{정리 \arabic{num}#1}}{\end{block}}
\newenvironment{prob}[1]{\refstepcounter{num}\begin{block}{문제 \arabic{num}#1}}{\end{block}}
\newenvironment{exam}[1]{\refstepcounter{num}\begin{block}{예시 \arabic{num}#1}}{\end{block}}

\newcommand{\pb}[1]%\Phantom + fBox
{\fbox{\phantom{\ensuremath{#1}}}}
\newcommand{\rb}[2]%\Red+fBox
{\fbox{\uncover<#1>{\red{\ensuremath{#2}}}}}
\renewcommand{\arraystretch}{1.5}
\newcommand{\red}[1]{\color{red}{#1}}
\newcommand{\ivs}{\centering\strut\vspace*{-\baselineskip}\newline}%image vertical setting
\newcommand*\circled[1]{\tikz[baseline=(char.base)]{\node[shape=circle,draw,inner sep=2pt] (char) {#1};}}

%%% title
\title{수학(하) : 04 집합}
\institute[ibedu]{아이비에듀}
\date{\today}

%%% toc
\AtBeginSection[]
{\begin{frame}
    \frametitle{목차}
    \tableofcontents[currentsection]
  \end{frame}}


\begin{document}
%%
\frame{\titlepage}

%%%%
\section{집합}

%%%
\subsection{집합의 뜻과 표현}

%%
\begin{frame}[t]{\subsecname}
\begin{minipage}{.47\textwidth}
%
\begin{prob}{) 10 이하의 홀수를 모두 구하시오.}
\end{prob}
%
\begin{prob}{) 오른쪽 표는 우리나라 시 단위의 광역자치단체 목록이다.}
\begin{enumerate}[(1)]
\item
광역시들을 모두 구하여라.
\item
특별시를 모두 구하여라.
\end{enumerate}
\end{prob}
\end{minipage}
\begin{minipage}{.47\textwidth}
\begin{tabularx}{\textwidth}{|Y|@{}Y@{}|}
\hline
도시		&분류\\\hline
서울		&특별시\\\hline
부산		&광역시\\\hline
대구		&광역시\\\hline
인천		&광역시\\\hline
\end{tabularx}
\end{minipage}

%%
%\begin{defi}{) 집합과 원소}
%어떤 기준에 따라 그 대상을 분명하게 알 수 있는 것들의 모임을 \alert{집합}이라고 한다.
%집합을 이루는 대상 하나하나를 \alert{원소}라고 한다.
%\end{defi}
%
\begin{exam}{}
10 이하의 홀수들의 모임을 \(A\)라고 하면, 
\[A=\{1,3,5,7,9\}\]
이다.
또한, \(1\in A\), \(3\in A\), \(5\in A\), \(7\in A\), \(9\in A\)이고, \(2\notin A\) 이다.
집합 \(A\)는 5개의 원소를 가지고 있다.
이것을 \(n(A)=5\)라고 표시한다.
\end{exam}
%
\begin{prob}{}
우리나라 광역시들의 모임을 \(B\), 특별시들의 모임을 \(C\)라고 하자.
그러면
\begin{align*}
B&=\{\text{부산}, \text{대구}, \text{인천}, \text{광주}, \text{대전}, \text{울산}\}\\
C&=\{\text{서울}\}
\end{align*}
이다.
따라서 \(\text{세종}\notin B\), \(\text{대전}\in B\), \(\text{서울}\in C\), \(\text{울산}\notin C\)이다.
그리고 \(n(B)=6\), \(n(C)=1\)이다.
\end{prob}

\end{frame}

%%%
\subsection{부분집합}
%%
\begin{frame}[t]{\subsecname}
\end{frame}

%%%
\subsection{부분집합의 개수}
%%
\begin{frame}[t]{\subsecname}
\end{frame}

%%%%
\section{집합의 연산}

%%%
\subsection{집합의 연산}
%%
\begin{frame}[t]{\subsecname}
\end{frame}

%%%
\subsection{집합의 연산 법칙}
%%
\begin{frame}[t]{\subsecname}
\end{frame}

%%%
\subsection{유한집합의 원소의 개수}
%%
\begin{frame}[t]{\subsecname}
\end{frame}



\end{document}