\documentclass{oblivoir}
\usepackage{amsmath,amssymb,kotex,paralist,graphicx}
\usepackage{mdframed}
\usepackage{../kswrapfig}
\usepackage{fapapersize}
\usefapapersize{210mm,297mm,20mm,*,20mm,*}
%\pagestyle{empty}
\usepackage{multicol}
\setlength{\columnsep}{30pt}
\setlength{\columnseprule}{1pt}
%\def\columnseprulecolor{\color{blue}}

%%% 객관식 선지

\usepackage{tabto,pifont}
\TabPositions{0.2\textwidth,0.4\textwidth,0.6\textwidth,0.8\textwidth}

\newcommand\one{\ding{172}}
\newcommand\two{\ding{173}}
\newcommand\three{\ding{174}}
\newcommand\four{\ding{175}}
\newcommand\five{\ding{176}}

\newcommand\taba[5]{\par\bigskip\noindent
\one\:{\ensuremath{#1}}
\tab\two\:\:{\ensuremath{#2}}
\tab\three\:\:{\ensuremath{#3}}
\tab\four\:\:{\ensuremath{#4}}
\tab\five\:\:{\ensuremath{#5}}}

\newcommand\tabb[5]{\par\bigskip\noindent
\one\:{\ensuremath{#1}}
\tabto{0.16\textwidth}\two\:\:{\ensuremath{#2}}
\tabto{0.33\textwidth}\three\:\:{\ensuremath{#3}}\medskip\par\noindent
\four\:\:{\ensuremath{#4}}.
\tabto{0.16\textwidth}\five\:\:{\ensuremath{#5}}}

\newcommand\tabc[5]{\par\bigskip\noindent
\one\:{\ensuremath{#1}}
\tabto{0.25\textwidth}\two\:\:{\ensuremath{#2}}\medskip\par\noindent
\three\:\:{\ensuremath{#3}}
\tabto{0.25\textwidth}\four\:\:{\ensuremath{#4}}\medskip\par\noindent
\five\:\:{\ensuremath{#5}}}

\newcommand\tabd[5]{\par\bigskip\noindent
\one\:{#1}\medskip\par\noindent
\two\:\:{#2}\medskip\par\noindent
\three\:\:{#3}\medskip\par\noindent
\four\:\:{#4}\medskip\par\noindent
\five\:\:{#5}}

%%% Counters
\newcounter{num}

%%% Commands
\newcommand{\prob}[1]
{\vs\par\noindent\refstepcounter{num}\textbf{문제 \arabic{num})}\label{#1}\par\noindent}

\newcommand\vs[1]{\vspace{70pt}}

\newcommand\inc[1]{\begin{center}\includegraphics[width=0.95\columnwidth]{#1}\end{center}}

\newcommand\pb[1]{\ensuremath{\fbox{\phantom{#1}}}}

\newcommand\ba{\ensuremath{\:|\:}}

\newcommand\an[2]{\par\bigskip\noindent\textbf{문제 \ref{#1})} #2\\}

\newcommand\ans[1]{\begin{flushright}\textbf{답 : }#1\end{flushright}}

\renewcommand{\arraystretch}{1.5}

%%% Meta Commands
\let\oldsection\section
\renewcommand\section{\clearpage\oldsection}
\let\emph\textsf

\usepackage{movie15}

%%% Title
\title{미적분 1 : 03 미분계수와 도함수}
\date{\today}
\author{}

\begin{document}
\maketitle

\tableofcontents
\clearpage

%%
\section{복습}
%
\exam{직선의 기울기}
\begin{enumerate}
\item
직선 \(y=2x+1\)의 기울기는 \(2\)이고 \(y\)절편은 \(1\)이다.
이때 \emph{기울기}란 직선이 기울어진 정도로서
\[\text{기울기}=\frac{x\text{의 값의 증가량}}{y\text{의 값의 증가량}}\]
으로 계산한다.
이때 기울기는 항상 일정한 값 \(2\)를 가진다는 것을 관찰할 수 있다.
\[\text{기울기}=\frac{\Delta y}{\Delta x}=\frac21=\frac42.\]
\item
만약 \(x\)의 값이 증가할 때 \(y\)의 값이 감소한다면, \(y\)의 값의 증가량은 음수로 나타낸다.
예를 들어, 직선 \(y=-x+4\)의 기울기는 \(-1\)이다.
\[\text{기울기}=\frac{\Delta y}{\Delta x}=\frac{-1}1=\frac{-2}2.\]
\end{enumerate}
\begin{center}
\begin{tabular}{cc}
\includegraphics[width=.48\textwidth]{slope1}
&\includegraphics[width=.48\textwidth]{slope2}
\end{tabular}
\end{center}

%
\prob{다음 직선들에 대하여 기울기를 계산하여라.}
\begin{center}
\begin{tabular}{cc}
(1) \(y=-3x+1\)
&(2) \(y=\frac12x+2\)\\
\includegraphics[width=.35\textwidth]{55}
&\includegraphics[width=.35\textwidth]{55}
\end{tabular}
\end{center}

%
\exam{이차함수 \(y=x^2+2\) 위의 한 점 \((1,3)\)에서의 접선의 방정식을 구하여라.}

\begin{mdframed}
접선의 기울기를 \(m\)이라고 두면, 이 접선의 방정식을
\[y=m(x-1)+3\]
이라고 둘 수 있다.

이때, 포물선과 직선이 접하기 위해서는 교점의 개수가 한 개여야 한다.
따라서 연립방정식
\[\begin{cases}
y&=x^2+2\\
y&=m(x-1)+3
\end{cases}\]
은 단 하나의 근을 가져야 한다.
즉 이차방정식
\begin{gather*}
x^2+2=mx-m+3\\
x^2-mx+m-1=0
\end{gather*}
의 판별식의 값이 0이어야 한다;
\[D=(-m)^2-4\cdot1\cdot(m-1)=m^2-4m+4=0\]
따라서 \(m=2\)이고, 접선의 방정식은 \(y=2x+1\)이다.
\end{mdframed}
\begin{center}
\includegraphics[width=0.4\textwidth]{tangent}
\end{center}

%
\prob{이차함수 \(y=-x^2+6x-4\) 위의 한 점 \((1,1)\)에서의 접선의 방정식을 구하여라.}
\bigskip

\clearpage
%%
\section{평균변화율}
함수 \(y=f(x)\)에서 \(x\)의 값이 \(a\)에서 \(b\)까지 변할 때, 함숫값 \(y\)는 \(f(a)\)에서 \(f(b)\)까지 변한다.
이때 \(x\)의 변화량인 \(b-a\)를 \(x\)의 \emph{증분}, \(y\)의 변화량인 \(f(b)-f(a)\)를 \(y\)의 증분이라고 부르며 각각 \(\Delta x\), \(\Delta y\)라고 표시한다.
\begin{align*}
\Delta x &= b-a\\
\Delta y &= f(b)-f(a)
\end{align*}
이때, \emph{평균변화율}이란 \(\Delta y\)를 \(\Delta x\)로 나눈 값을 말한다;
\[\frac{\Delta y}{\Delta x}=\frac{f(b)-f(a)}{b-a}=\frac{f(a+\Delta x)-f(a)}{\Delta x}.\]

\begin{center}
\includegraphics[width=.4\textwidth]{average_rate}
\end{center}

\vspace{-15pt}
%
\exam{다음 함수들에 대하여 \(x=1\)에서 \(x=3\)까지의 평균변화율을 구하여라.}
\begin{tabularx}{\textwidth}{XX}
(1) \(f(x)=2x^2+1\)
&
(2) \(g(x)=2x+3\)
\end{tabularx}
\vspace{-15pt}
\begin{mdframed}
\vspace{-20pt}
\begin{align*}
(1)&\:\text{평균변화율}=\frac{f(3)-f(1)}{3-1}=\frac{19-3}2=8\\
(2)&\:\text{평균변화율}=\frac{g(3)-g(1)}{3-1}=\frac{9-5}2=2
\end{align*}
\end{mdframed}

%
\prob{다음 함수들에 대하여 \(x=-1\)에서 \(x=3\)까지의 평균변화율을 구하여라.}
\begin{tabularx}{\textwidth}{XXX}
(1) \(f(x)=-\frac23x+1\)
&
(2) \(g(x)=x^2\)
&
(3) \(h(x)=|2x-2|+1\)
\end{tabularx}

%%
\section{순간변화율(미분계수)}

\end{document}