\documentclass[t,8pt]{beamer}
%\usetheme{boxes}
%\usecolortheme{default}

%%% packages
\geometry{paperwidth=140mm,paperheight=105mm}
\usefonttheme[onlymath]{serif}

\usepackage{kotex,amsmath,tabto}

\usepackage{tabularx}
\newcolumntype{Y}{>{\centering\arraybackslash}X} % centered
\newcolumntype{M}{>{$}X<{$}} % math-moded
\newcolumntype{N}{>{$}Y<{$}} %centered and math-moded

%\usepackage[inline]{enumitem}
%\setlist[enumerate,1]{label=(\arabic*)}
%\setlist[enumerate,2]{label=(\alph*)}

\usepackage{xcolor}

\usepackage{gensymb} %\degree

%%% counters, commands, environments
\newcounter{num}
\resetcounteronoverlays{num}

\newenvironment{defi}[1]{\refstepcounter{num}\begin{block}{정의 \arabic{num}#1}}{\end{block}}
\newenvironment{theo}[1]{\refstepcounter{num}\begin{block}{정리 \arabic{num}#1}}{\end{block}}
\newenvironment{prob}[1]{\refstepcounter{num}\begin{block}{문제 \arabic{num}#1}}{\end{block}}
\newenvironment{exam}[1]{\refstepcounter{num}\begin{block}{예시 \arabic{num}#1}}{\end{block}}

\newcommand{\pb}[1]%\Phantom + fBox
{\fbox{\phantom{\ensuremath{#1}}}}
\renewcommand{\arraystretch}{1.5}
\newcommand{\red}[1]{\color{red}{#1}}
\newcommand{\ivs}{\centering\strut\vspace*{-\baselineskip}\newline}%image vertical setting

%%% title
\title[삼각함수]{삼각함수의 그래프}
\institute[ibedu]{아이비에듀}
\date{\today}

%%% toc
\AtBeginSection[]
{\begin{frame}
    \frametitle{목차}
    \tableofcontents[currentsection]
  \end{frame}}


\begin{document}

%%
\frame{\titlepage}

%%%
\section{삼각함수의 기본 그래프}

%%
\begin{frame}{\secname}
\setlength{\tabcolsep}{0pt}
%
\begin{prob}{) 다음 표를 완성하여라.}
\begin{tabularx}{\textwidth}{|@{\quad}N@{\quad\;\;}|N|N|N|N|N|N|N|N|N|N|N|N|N|}
\hline
\theta
&0
&\frac\pi6
&\frac\pi3
&\frac\pi2
&\frac23\pi
&\frac56\pi
&\pi
&\frac76\pi
&\frac43\pi
&\frac32\pi
&\frac53\pi
&\frac{11}6\pi
&2\pi
\\\hline
\sin\theta
&\uncover<2>{\red{0}}
&\frac12
&\uncover<2>{\red{\frac{\sqrt3}2}}
&\uncover<2>{\red{1}}
&\uncover<2>{\red{\frac{\sqrt3}2}}
&\uncover<2>{\red{\frac12}}
&\uncover<2>{\red{0}}
&\uncover<2>{\red{-\frac12}}
&\uncover<2>{\red{-\frac{\sqrt3}2}}
&\uncover<2>{\red{-1}}
&\uncover<2>{\red{-\frac{\sqrt3}2}}
&\uncover<2>{\red{-\frac12}}
&\uncover<2>{\red{0}}
\\\hline
\end{tabularx}
\end{prob}
%
\begin{prob}{) \(y=\sin x\)의 그래프를 그려라.}
\begin{minipage}{.7\textwidth}
\includegraphics<1>[width=\textwidth]{graph_1_grid}
\includegraphics<2>[width=\textwidth]{graph_1-1}
\end{minipage}
\begin{minipage}{.29\textwidth}
\uncover<2>{\small
\begin{itemize}
\item
\(-1\le\sin x\le 1\)
\item
\(\text{주기}=2\pi\)
\item
원점 대칭 (기함수)
\end{itemize}}
\end{minipage}
\end{prob}
\end{frame}

%%
\begin{frame}{\secname}
\setlength{\tabcolsep}{0pt}
%
\begin{prob}{) 다음 표를 완성하여라.}
\begin{tabularx}{\textwidth}{|@{\quad}N@{\quad\;\;}|N|N|N|N|N|N|N|N|N|N|N|N|N|}
\hline
\theta
&0
&\frac\pi6
&\frac\pi3
&\frac\pi2
&\frac23\pi
&\frac56\pi
&\pi
&\frac76\pi
&\frac43\pi
&\frac32\pi
&\frac53\pi
&\frac{11}6\pi
&2\pi
\\\hline
\cos\theta
&\uncover<2>{\red{1}}
&\frac{\sqrt3}2
&\uncover<2>{\red{\frac12}}
&\uncover<2>{\red{0}}
&\uncover<2>{\red{-\frac12}}
&\uncover<2>{\red{-\frac{\sqrt3}2}}
&\uncover<2>{\red{-1}}
&\uncover<2>{\red{-\frac{\sqrt3}2}}
&\uncover<2>{\red{-\frac12}}
&\uncover<2>{\red{0}}
&\uncover<2>{\red{\frac12}}
&\uncover<2>{\red{\frac{\sqrt3}2}}
&\uncover<2>{\red{1}}
\\\hline
\end{tabularx}
\end{prob}
%
\begin{prob}{) \(y=\cos x\)의 그래프를 그려라.}
\begin{minipage}{.7\textwidth}
\includegraphics<1>[width=\textwidth]{graph_1_grid}
\includegraphics<2>[width=\textwidth]{graph_1-2}
\end{minipage}
\begin{minipage}{.29\textwidth}
\uncover<2>{\small
\begin{itemize}
\item
\(-1\le\cos x\le 1\)
\item
\(\text{주기}=2\pi\)
\item
\(y\)축 대칭 (우함수)
\end{itemize}}
\end{minipage}
\end{prob}
\end{frame}

%%
\begin{frame}{\secname}
\setlength{\tabcolsep}{0pt}
%
\begin{prob}{) 다음 표를 완성하여라.}
\begin{tabularx}{\textwidth}{|@{\quad}N@{\quad\;\;}|N|N|N|N|N|N|N|N|N|N|}
\hline
\theta
&0
&\frac\pi4
&\frac\pi2
&\frac34\pi
&\pi
&\frac54\pi
&\frac32\pi
&\frac74\pi
&2\pi
\\\hline
\tan\theta
&\uncover<2>{\red{0}}
&1
&\uncover<2>{\red{\times}}
&\uncover<2>{\red{-1}}
&\uncover<2>{\red{0}}
&\uncover<2>{\red{1}}
&\uncover<2>{\red{\times}}
&\uncover<2>{\red{-1}}
&\uncover<2>{\red{0}}
\\\hline
\end{tabularx}
\end{prob}
%
\begin{prob}{) \(y=\tan x\)의 그래프를 그려라.}
\begin{minipage}{.7\textwidth}
\includegraphics<1>[width=\textwidth]{graph_1_grid}
\includegraphics<2>[width=\textwidth]{graph_1-3}
\end{minipage}
\begin{minipage}{.29\textwidth}
\uncover<2>{\small
\begin{itemize}
\item
\(-\infty<\tan x<\infty\)
\item
\(\text{주기}=\pi\)
\item
원점 대칭
\end{itemize}}
\end{minipage}
\end{prob}
\end{frame}

%%%
\section{삼각함수의 평행이동}

%%
\begin{frame}{\secname}
%
\begin{prob}{) 다음 삼각함수들의 그래프를 그려라.}
(1-1) \(y=\sin(x-\frac\pi4)\)\\
\includegraphics<1>[width=.6\textwidth]{graph_2_grid}
\includegraphics<2>[width=.6\textwidth]{graph_2-1-1}\\
(1-2) \(y=\sin x-1\)\\
\includegraphics<1>[width=.6\textwidth]{graph_2_grid}
\includegraphics<2>[width=.6\textwidth]{graph_2-1-2}\\
\end{prob}
\end{frame}

%%
\begin{frame}{\secname}
(1-3) \(y=\cos(x+\frac\pi2)\)\\
\includegraphics<1>[width=.6\textwidth]{graph_2_grid}
\includegraphics<2>[width=.6\textwidth]{graph_2-1-3}\\
(1-4) \(y=\cos x+\frac12\)\\
\includegraphics<1>[width=.6\textwidth]{graph_2_grid}
\includegraphics<2>[width=.6\textwidth]{graph_2-1-4}\\
\end{frame}

%%
\begin{frame}{\secname}
(1-5) \(y=\sin(x-\frac\pi4)+1\)\\
\includegraphics<1>[width=.6\textwidth]{graph_2_grid}
\includegraphics<2>[width=.6\textwidth]{graph_2-1-5}\\
(1-6) \(y=\cos\left(x+\frac\pi2\right)+\frac12\)\\
\includegraphics<1>[width=.6\textwidth]{graph_2_grid}
\includegraphics<2>[width=.6\textwidth]{graph_2-1-6}\\
\end{frame}

%%
\begin{frame}{\secname}
(1-7) \(y=\cos(x-\pi)-1\)\\
\includegraphics<1>[width=.6\textwidth]{graph_2_grid}
\includegraphics<2>[width=.6\textwidth]{graph_2-1-7}\\
(1-8) \(y=\sin\left(x+\frac\pi2\right)-1\)\\
\includegraphics<1>[width=.6\textwidth]{graph_2_grid}
\includegraphics<2>[width=.6\textwidth]{graph_2-1-8}\\
\end{frame}


%%%
\section{삼각함수의 대칭이동과 확대변환}

%%
\begin{frame}{\secname}
(2-1) \(y=2\sin x\)\\
\includegraphics<1>[width=.6\textwidth]{graph_2_grid}
\includegraphics<2>[width=.6\textwidth]{graph_2-2-1}\\
(2-2) \(y=\frac12\sin x\)\\
\includegraphics<1>[width=.6\textwidth]{graph_2_grid}
\includegraphics<2>[width=.6\textwidth]{graph_2-2-2}\\
\end{frame}

%%
\begin{frame}{\secname}
(2-3) \(y=2\cos x\)\\
\includegraphics<1>[width=.6\textwidth]{graph_2_grid}
\includegraphics<2>[width=.6\textwidth]{graph_2-2-3}\\
(2-4) \(y=\frac12\cos x\)\\
\includegraphics<1>[width=.6\textwidth]{graph_2_grid}
\includegraphics<2>[width=.6\textwidth]{graph_2-2-4}\\
\end{frame}

%%
\begin{frame}{\secname}
(2-5) \(y=-\sin x\)\\
\includegraphics<1>[width=.6\textwidth]{graph_2_grid}
\includegraphics<2>[width=.6\textwidth]{graph_2-2-5}\\
(2-6) \(y=-\cos x\)\\
\includegraphics<1>[width=.6\textwidth]{graph_2_grid}
\includegraphics<2>[width=.6\textwidth]{graph_2-2-6}\\
\end{frame}

%%
\begin{frame}{\secname}
(2-7) \(y=\sin 2x\)\\
\includegraphics<1>[width=.6\textwidth]{graph_2_grid}
\includegraphics<2>[width=.6\textwidth]{graph_2-2-7}\\
(2-8) \(y=\sin\frac12x\)\\
\includegraphics<1>[width=.6\textwidth]{graph_2_grid}
\includegraphics<2>[width=.6\textwidth]{graph_2-2-8}\\
\end{frame}

%%
\begin{frame}{\secname}
(2-9) \(y=\cos 2x\)\\
\includegraphics<1>[width=.6\textwidth]{graph_2_grid}
\includegraphics<2>[width=.6\textwidth]{graph_2-2-9}\\
(2-10) \(y=\cos\frac12x\)\\
\includegraphics<1>[width=.6\textwidth]{graph_2_grid}
\includegraphics<2>[width=.6\textwidth]{graph_2-2-10}\\
\end{frame}

%%%
\section{삼각함수의 일반형}

%%
\begin{frame}{\secname}
(3-1) \(y=2\sin\left(x-\pi\right)\)\\
\includegraphics<1>[width=.6\textwidth]{graph_2_grid}
\includegraphics<2>[width=.6\textwidth]{graph_2-3-1}\\
(3-2) \(y=\frac12\sin\left(x+\frac\pi2\right)\)\\
\includegraphics<1>[width=.6\textwidth]{graph_2_grid}
\includegraphics<2>[width=.6\textwidth]{graph_2-3-2}\\
\end{frame}

%%
\begin{frame}{\secname}
(3-3) \(y=2\cos x-1\)\\
\includegraphics<1>[width=.6\textwidth]{graph_2_grid}
\includegraphics<2>[width=.6\textwidth]{graph_2-3-3}\\
(3-4) \(y=\frac12\cos x+\frac12\)\\
\includegraphics<1>[width=.6\textwidth]{graph_2_grid}
\includegraphics<2>[width=.6\textwidth]{graph_2-3-4}\\
\end{frame}

%%
\begin{frame}{\secname}
(3-5) \(y=-\cos x+1\)\\
\includegraphics<1>[width=.6\textwidth]{graph_2_grid}
\includegraphics<2>[width=.6\textwidth]{graph_2-3-5}\\
(3-6) \(y=-\cos\left(x-\frac\pi2\right)\)\\
\includegraphics<1>[width=.6\textwidth]{graph_2_grid}
\includegraphics<2>[width=.6\textwidth]{graph_2-3-6}\\
\end{frame}

%%
\begin{frame}{\secname}
(3-7) \(y=\sin\left(\frac12x-\frac\pi4\right)\)\\
\includegraphics<1>[width=.6\textwidth]{graph_2_grid}
\includegraphics<2>[width=.6\textwidth]{graph_2-3-7}\\
(3-8) \(y=\sin\left(2x+\frac\pi2\right)\)\\
\includegraphics<1>[width=.6\textwidth]{graph_2_grid}
\includegraphics<2>[width=.6\textwidth]{graph_2-3-8}\\
\end{frame}
\end{document}