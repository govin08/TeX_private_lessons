\documentclass{article}
\usepackage{amsmath,amssymb,amsthm,mdframed,kotex,paralist}
\usepackage{tabto}
%\TabPositions{0.5\textwidth}
\TabPositions{0.33\textwidth,0.66\textwidth}
\newcommand\bp[1]{\begin{mdframed}[frametitle={#1},skipabove=10pt,skipbelow=20pt,innertopmargin=5pt,innerbottommargin=40pt]}
\newcommand\ep{\end{mdframed}\par}

\begin{document}
\title{쎈 중등 수학 3(상) : 01--06}
\author{}
\date{\today}
\maketitle
\section{제곱근의 뜻과 성질}
%\begin{mdframed}[frametitle={0158}]
\bp{0158}
\(A\), \(B\)가 다음과 같을 때, \(A+B\)의 값을 구하여라.
\begin{align*}
A&=(-\sqrt{17})^2-\sqrt{3^4}\\
B&=\sqrt{169}+\sqrt{(-5)^2}-\sqrt{3^2\times(-2)^2}.
\end{align*}
\ep

\bp{0161}
\(A=\sqrt{(1+x)^2}-\sqrt{(1-x)^2}\)에 대한 설명으로 옳은 것만을 보기에서 있는 대로 고른 것은?
\begin{enumerate}[(a)]
\item
\(x>1\) 이면 \(A=2\)이다.
\item
\(-1<x<1\)이면 \(A=2x\)이다.
\item
\(x<-1\)이면 \(A=-2x\)이다.
\end{enumerate}
\noindent
\begin{inparaenum}
\item[①]
(a)
\tab\item[②]
(a), (b)
\tab\item[③]
(a), (c)
\tab\item[④]
(b), (c)
\tab\item[⑤]
(a), (b), (c)
\end{inparaenum}
\ep

\bp{0163}
두 개의 주사위를 던져서 나온 눈의 수를 각각 \(a\), \(b\)라 할 때, \(\sqrt{72ab}\)가 자연수가 될 확률은?\\
\begin{inparaenum}
\item[①]
\(\frac1{18}\)
\tab\item[②]
\(\frac1{12}\)
\tab\item[③]
\(\frac19\)
\tab\item[④]
\(\frac16\)
\tab\item[⑤]
\(\frac13\)
\end{inparaenum}
\ep

\bp{0166}
\(1<p<2\)에 대하여 \(n\)의 양의 제곱근이 \(5+p\)일 때, 자연수 \(n\)의 최댓값을 구하여라.
\ep

\bp{0167}
자연수 \(n\)에 대하여
\[A=\{x~|~2\le\sqrt{nx}<3,~nx\text{는 자연수}\}\]
이고 집합 \(A\)의 모든 원소의 합이 \(15\)이다.
이때 \(n\)의 값은?\\
\begin{inparaenum}
\item[①]
1
\tab\item[②]
2
\tab\item[③]
3
\tab\item[④]
4
\tab\item[⑤]
5
\end{inparaenum}
\ep

\bp{0173}
\(\sqrt{(a+2)^2}+\sqrt{(a-2)^2}=4\)를 만족시키는 \(a\)의 값의 범위를  구하여라.
\ep

\section{무리수와 실수}
\bp{0269}
\(500\) 이하의 자연수 \(n\)에 대하여 \(\sqrt n\), \(\sqrt{3n}\), \(\sqrt{5n}\)이 모두 무리수가 되도록 하는 \(n\)의 개수는?\\
\begin{inparaenum}
\item[①]
452개
\tab\item[②]
454개
\tab\item[③]
456개
\tab\item[④]
458개
\tab\item[⑤]
460개
\end{inparaenum}
\ep

\bp{0272}
다음 중 무리수에 해당하는 것을 모두 고른 것은?
\begin{enumerate}[(a)]
\item
한 원에서 지름의 길이에 대한 원의 둘레의 길이의 비
\item
넓이가 \(8\pi\)인 원의 반지름의 길이
\item
넓이가 \(1\)인 정사각형의 대각선의 길이
\item
 한 모서리의 길이가 \(\sqrt2\)인 정육면체의 겉넓이
\end{enumerate}
\noindent
\begin{inparaenum}
\item[①]
(a), (b)
\tab\item[②]
(b), (c)
\tab\item[③]
(a), (b), (c)
\tab\item[④]
(a), (c), (d)
\tab\item[⑤]
(b), (c), (d)
\end{inparaenum}
\ep

\bp{0275}
\(a=\sqrt7-1\), \(b=2\)일 때, 다음 중 가장 큰 수는?\\
\begin{inparaenum}
\item[①]
\(a+1\)
\tab\item[②]
\(\sqrt{(-a)^2}\)
\tab\item[③]
\(\sqrt{(-b)^2}\)
\tab\item[④]
\(\sqrt{(a-b)^2}\)
\tab\item[⑤]
\(\sqrt{(a+b)^2}\)
\end{inparaenum}
\ep

\section{근호를 포함한 식의 계산(1)}
\bp{0369}
\(\sqrt{0.08}\)은 \(\sqrt2\)의 \(a\)배이고, \(\sqrt{180}\)은 \(\sqrt5\)의 \(b\)배일 때, \(ab\)의 값은?
\\
\begin{inparaenum}
\item[①]
\(\frac5{36}\)
\tab\item[②]
\(\frac6{25}\)
\tab\item[③]
\(\frac65\)
\tab\item[④]
\(\frac{25}6\)
\tab\item[⑤]
\(\frac{36}5\)
\end{inparaenum}\ep

\bp{0370}
\(\sqrt5=a\), \(\sqrt{10}=b\)일 때, \(\sqrt{0.016}+\sqrt{2000}\)을 \(a\), \(b\)를 이용하여 나타내면?
\begin{inparaenum}
\item[①]
\(20a+\frac1{25}b\)
\tab\item[②]
\(20a+\frac1{24}b\)
\tab\item[③]
\(20a+\frac1{20}b\)
\tab\item[④]
\(25a+\frac1{25}b\)
\tab\item[⑤]
\(25a+\frac1{20}b\)
\end{inparaenum}
\ep

\section{근호를 포함한 식의 계산(2)}

\bp{0524}
\(\sqrt6\left(\frac6{\sqrt{32}}-\frac3{\sqrt2}\right)-\sqrt2\left(\frac2{\sqrt6}-\frac{10}{\sqrt{12}}\right)\)을 간단히 하면 \(a\sqrt3+b\sqrt6\)이다.
이때 유리수 \(a\), \(b\)에 대하여 \(\sqrt{b-6a}\)의 값을 구하여라.
\ep

\bp{0525}
다음 식을 계산하면?
\[
(1-\sqrt2+\sqrt3)^2+(2+\sqrt2-\sqrt3)(2-\sqrt2+\sqrt3)
\]
\begin{inparaenum}
\item[①]
\(5-2\sqrt2+2\sqrt3\)
\tab\item[②]
\(5+2\sqrt2-2\sqrt3\)
\tab\item[③]
\(5+2\sqrt2+2\sqrt3\)
\tab\item[④]
\(5+2\sqrt2-2\sqrt6\)
\tab\item[⑤]
\(5+2\sqrt2+2\sqrt6\)
\end{inparaenum}
\ep

\bp{0529}
\(x_1=\sqrt3+\langle\sqrt3\rangle\), \(x_{n+1}=x_n-\langle x_n\rangle\)이라 할 때, \(x_1x_2x_3\)의 값은?
(단, \(\langle x\rangle\)은 \(x\)보다 작지 않은 최소의 정수이다.)\\
\begin{inparaenum}
\item[①]
\(-\sqrt3-2\)
\tab\item[②]
\(-\sqrt3+2\)
\tab\item[③]
\(\sqrt3-2\)
\tab\item[④]
\(\sqrt3+2\)
\tab\item[⑤]
\(3\sqrt3-2\)
\end{inparaenum}
\ep

\bp{0532}
서로 다른 두 개의 주사위를 던져 나온 눈의 수가 각각 \(a\), \(b\)일 때, \(2<\sqrt a+\sqrt b<3\)일 확률은?\\
\begin{inparaenum}
\item[①]
\(\frac5{36}\)
\tab\item[②]
\(\frac16\)
\tab\item[③]
\(\frac29\)
\tab\item[④]
\(\frac14\)
\tab\item[⑤]
\(\frac5{12}\)
\end{inparaenum}
\ep

\bp{0543}
세 집합
\begin{align*}
A_k&=\{x\:|\:kx\text{는 자연수}\},\\
B_k&=\{x\:|\:\sqrt{kx}\text{의 정수 부분은 }2\},\\
C_k&=\{x\:|\:\sqrt{\frac{kx}2}\text{의 정수 부분은 }1\}
\end{align*}
에 대하여 \(n(A_k\cap B_k\cap C_k)\)를 구하여라. (단 k>0)
\ep

\section{인수분해 공식}
\bp{0665}
\(n\)이 자연수일 때, \(8n^3-2n\)은 어떤 수의 배수인가?\\
\begin{inparaenum}
\item[①]
4의 배수
\tab\item[②]
6의 배수
\tab\item[③]
8의 배수
\tab\item[④]
10의 배수
\tab\item[⑤]
12의 배수
\end{inparaenum}
\ep

\bp{0675}
두 집합
\begin{align*}
A&=\{X\:|\:X\text{는 }x^2+ax-4\text{의 인수}\},\\
B&=\{Y\:|\:Y\text{는 }4x^2+ax-b\text{의 인수}\}
\end{align*}
일 때, \((x-1)\in A\cap B\)이다.
이때 상수 \(a\), \(b\)의 합 \(a+b\)의 값을 구하여라.
\ep

\section{인수분해 공식의 활용}
\bp{0767}
\(2x+y=\frac1{\sqrt5+2}\), \(2x-y=\frac1{\sqrt5-2}\)일 때,
\(4x^2-8x-y^2+4y\)의 값을 구하여라.
\ep

\bp{0768}
\(x\), \(y\)가 양수이고 \(x^2+y^2=12\), \(xy=2\)일 때, \(x^3+x^2y+xy^2+y^3\)의 값은?\\
\begin{inparaenum}
\item[①]
36
\tab\item[②]
48
\tab\item[③]
\(48\sqrt3\)
\tab\item[④]
54
\tab\item[⑤]
\(54\sqrt3\)
\end{inparaenum}
\ep

\bp{0769}
\(a+b=3\), \(ab=1\)일 때, \((2a+b)^2-(a+2b)^2\)의 값을 구하여라. (단 \(a>b\))
\ep

\bp{0772}
부피가 \(x^3+x^2y-x-y\)인 직육면체의 밑면의 가로, 세로의 길이가 각각 \(x+y\), \(x+1\)일 때, 이 직육면체의 겉넓이를 구하여라.
\ep

\bp{0775}
다음 중 \((x^2-3x+2)(x^2+5x+6)-60\)의 인수가 아닌 것을 모두 고르면? (정답 2개)\\
\begin{inparaenum}
\item[①]
\(x-3\)
\tab\item[②]
\(x+3\)
\tab\item[③]
\(x+4\)
\tab\item[④]
\(x^2+x-4\)
\tab\item[⑤]
\(x^2+x+4\)
\end{inparaenum}
\ep

\bp{0776}
\(xy+x-4y-7=0\)을 만족시키는 정수 \(x\), \(y\)에 대하여 \(xy\)의 최솟값을 구하여라.
\ep

\bp{0778}
다음 식을 인수분해하면?
\[abc+ab+ac+a+bc+b+c+1\]
\begin{inparaenum}
\item[①]
\(a(b+1)(c+1)\)
\tab\item[②]
\(b(a+1)(c+1)\)
\tab\item[③]
\(c(a+1)(b+1)\)
\tab\item[④]
\((a+1)(b-1)(c-1)\)
\tab\item[⑤]
\((a+1)(b+1)(c+1)\)
\end{inparaenum}
\ep

\bp{0781}
\(7005\times7007+1\)이 어떤 자연수의 제곱일 때, 어떤 자연수를 구하여라.
\ep

\bp{0782}
\((\sqrt6+\sqrt3+\sqrt2-1)^2-(\sqrt6+\sqrt3-\sqrt2+1)^2\)을 계산하면?\\
\begin{inparaenum}
\item[①]
\(\sqrt3\)
\tab\item[②]
\(3\)
\tab\item[③]
\(3\sqrt3\)
\tab\item[④]
\(4\sqrt3\)
\tab\item[⑤]
\(6\sqrt3\)
\end{inparaenum}
\ep

\bp{0785}
\(x+y=14\)일 때, 다음 식의 값을 구하여라.
\[2x^2+4xy+2y^2-5x-5y-12\]
\ep

\bp{0786}
\(b-c=2\), \(c-a=4\)일 때, \(2a^2+2b^2+2c^2-2ab-2bc-2ca\)의 값은?\\
\begin{inparaenum}
\item[①]
52
\tab\item[②]
54
\tab\item[③]
56
\tab\item[④]
58
\tab\item[⑤]
60
\end{inparaenum}
\ep

\bp{0790}
인수분해 공식을 이용하여 다음을 계산하여라.
\[1^2-3^2+5^2-7^2+9^2-11^2+13^2-15^2+17^2-19^2\]
\ep

\bp{0791}
\(x=\frac1{\sqrt{10}-3}\), \(y=\frac1{\sqrt{10}+3}\)일 때, 다음에 답하여라.\\
(1) \(x\), \(y\)의 분모를 유리화하여라.\\
(2) \(x+y\), \(x-y\)의 값을 구하여라.\\
(3) \(x^2-y^2+6y-9\)의 값을 구하여라.
\ep

\bp{0793}
\(2x=-1+\sqrt5\)일 때, \(8x^3+8x^2+6x\)의 값을 구하여라.
\ep

\bp{0794}
\(x+y=8\)이고 \(x^2y+xy^2+3x+3y=40\)일 때, \(\frac{x^2y-xy^2}{x^2-y^2}\)의 값을 구하여라.
\ep

\bp{0795}
두 상수 \(a\), \(b\)에 대하여 \(a+b=7\), \(ab=10\)일 때, 다음 식의 값을 구하여라. (단, \(0<a<b\))
\[\frac1a\times\sqrt{a^4-2a^3b+a^2b^2}\]
\ep
\end{document}