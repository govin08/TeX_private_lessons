\documentclass{article}
\usepackage{amsmath,amssymb,amsthm,kotex,paralist,mathrsfs,centernot,marvosym}
\newcounter{num}[section]
\newcommand{\defi}[1]
{\bigskip\noindent\refstepcounter{num}\textbf{정의 \arabic{num}) #1}\par}
\newcommand{\theo}[1]
{\bigskip\noindent\refstepcounter{num}\textbf{정리 \arabic{num}) #1}\par}
%\newcommand{\axio}[1]
%{\bigskip\noindent\refstepcounter{num}\textbf{공리 \arabic{section}. \arabic{num}) #1}\par}

\newcommand{\notiff}{%
  \mathrel{{\ooalign{\hidewidth$\not\phantom{"}$\hidewidth\cr$\iff$}}}}
\newcommand{\LHS}{\text{LHS}}
\newcommand{\RHS}{\text{RHS}}
\newcommand{\irange}{\ensuremath{1\le i\le n}}
\newcommand{\jrange}{\ensuremath{1\le j\le n}}
\newcommand{\bb}[2]{\ensuremath{(^{#1}_{#2})}}
\newcommand{\cc}[2]{\ensuremath{_{#1}C_{#2}}}

\renewcommand{\figurename}{그림.}
\renewcommand{\proofname}{증명.}
\renewcommand{\contentsname}{목차}
\renewcommand\emph{\textbf}

%%%
\begin{document}

\title{준수-01:거듭제곱근식과 지수}
\author{}
\date{\today}
\maketitle
%\tableofcontents
\newpage

%%

%%
\newpage
%\section{거듭제곱근식과 지수}
%
\defi{자연수 지수\label{natural_exponent}}
실수 \(a\)와 자연수 \(n\)에 대해 \(a^n\)을
\begin{align*}
a^1&=a\\
a^2&=a\times a\\
a^3&=a\times a\times a\\
\vdots
\end{align*}
등으로 정의한다.

%
\defi{정수 지수\label{integer_exponent}}
실수 \(a\)와 정수 \(n\)에 대해 \(a^n\)을 다음과 같이 정의한다.
만약 \(n>0\)이면 정의 \ref{natural_exponent}에서처럼 정의한다.
만약 \(n=0\)이면 \(a^n=a^0=1\)로 정의한다.
만약 \(n<0\)이면 \(a^n=\frac1{a^{-n}}\)로 정의한다.
세 번째 경우 \(-n>0\)이므로 \(a^{-n}\)가 정의되는 데는 문제가 없다는 점에 주목하자.

%
\theo{지수법칙(정수)\label{law_of_exponents_1}}
실수 \(a\), \(b\), 정수 \(m\),  \(n\)에 대해
\begin{enumerate}[(1)]
\item
\(a^mb^n=a^{m+n}\)
\item
\((a^m)^n=a^{mn}\)
\item
\((ab)^m=a^mb^m\)
\item
\((\frac ab)^m=\frac{a^m}{b^m}\)(\(b\neq0\))
\end{enumerate}

\begin{proof}
증명은 생략한다.
\(m\), \(n\)이 각각 자연수일 때 네 법칙을 먼저 증명한 후 이를 토대로 정수 지수에 대해서 다시 증명하면 된다.
\end{proof}

%
\defi{거듭제곱근\label{n-th_root}}
\(a\)가 실수이고 \(n\)이 자연수일 때
\(x^n=a\)를 만족시키는 실수 \(x\)를 \emph{\(a\)의 \(n\)제곱근}이라고 부른다.

%
\theo{거듭제곱근의 존재\label{existence_of_n-th_root}}
정의 \ref{n-th_root}에서
\begin{enumerate}[(1)]
\item
\(a>0\)이면 \(x\)는 유일하게 하나 존재한다.
\item
\(a<0\)이고 \(n\)이 짝수이면 \(x\)는 존재하지 않는다.
\item
\(a<0\)이고 \(n\)이 홀수이면 \(x\)는 유일하게 하나 존재한다.
\end{enumerate}

\begin{proof}
(2)는 당연하다.
실수의 짝수승은 음수가 될 수 없기 때문이다.
(1)과 (3)의 증명은 생략한다.
\end{proof}

%
\defi{}
정리 \ref{existence_of_n-th_root}의 (1)번의 경우, \(a\)의 \(n\)제곱근을 \(\sqrt[n]a\)라고 표기한다.

%
\defi{유리수 지수\label{rational_exponent}}
양의 실수 \(x\)와 유리수 \(t=\frac mn\)에 대해(\(m\), \(n\)는 정수, \(n\neq0\)) \(x^t\)를 다음과 같이 정의한다.
만약 \(m=1\)이면 \(x^t=x^{\frac1n}=\sqrt[n]a\)로 정의한다.
일반적으로 \(m\)이 정수이면 \(x^t=x^{\frac mn}=(x^{\frac1n})^m\)으로 정의한다.

\(x\)가 음수이면 지수를 정의하기가 어려우므로 생각하지 않는다.

%
\theo{지수법칙(유리수)\label{law_of_exponents_2}}
양의 실수 \(a\), \(b\), 유리수 \(s\),  \(t\)에 대해
\begin{enumerate}[(1)]
\item
\(a^sb^t=a^{s+t}\)
\item
\((a^s)^t=a^{st}\)
\item
\((ab)^s=a^sb^s\)
\item
\((\frac ab)^s=\frac{a^s}{b^s}\)(\(b\neq0\))
\end{enumerate}
\begin{proof}
정리 \ref{law_of_exponents_1}를 사용하여 약간의 계산을 거치면 어렵지 않게 증명할 수 있다.
\end{proof}

%
\theo{실수 지수}
고등학교 과정의 극한 개념을 이용하면 실수 지수도 정의할 수 있다.
예를 들어, \(3^{\sqrt2}\)를 정의해보자.
\(\sqrt2=1.414213\cdots\)이다.
따라서 다음과 같은 수의 열을 생각해보자.
\begin{align*}
x_1&=3^1=3\\
x_2&=3^{1.4}=4.65553672175\cdots\\
x_3&=3^{1.41}=4.70696500172\cdots\\
x_4&=3^{1.414}=4.72769503527\cdots\\
x_5&=3^{1.4142}=4.72873393017\cdots\\
x_6&=3^{1.41421}=4.72878588091\cdots\\
x_7&=3^{1.414213}=4.72880146624\cdots\\
\vdots
\end{align*}
위 식들의 우변은 정의 \ref{rational_exponent}에 의하여 잘 정의되어 있다는 점에 주목하자.
이 수열(=수의 열)인 \(\{x_n\}\)은 \(n\)이 증가할수록 일정한 값에 가까워지는 것처럼 보인다.
이 값을 이 수열의 `극한'이라고 부른다.
그리고 이 극한값을 \(3^{\sqrt2}\)로 정의한다.


%
\theo{지수법칙(실수)}
양의 실수 \(a\), \(b\), 실수 \(s\),  \(t\)에 대해
\begin{enumerate}[(1)]
\item
\(a^sb^t=a^{s+t}\)
\item
\((a^s)^t=a^{st}\)
\item
\((ab)^s=a^sb^s\)
\item
\((\frac ab)^s=\frac{a^s}{b^s}\)(\(b\neq0\))
\end{enumerate}
\begin{proof}
정리 \ref{law_of_exponents_2}와 극한의 성질을 사용하여 증명할 수 있다.
\end{proof}
\end{document}