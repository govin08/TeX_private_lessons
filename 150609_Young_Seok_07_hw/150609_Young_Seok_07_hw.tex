\documentclass{article}
\usepackage{amsmath,amssymb,amsthm,kotex,mdframed,paralist,chngcntr}

\newcounter{num}
%\newcommand{\defi}[1]
%{\bigskip\noindent\refstepcounter{num}\textbf{정의 \arabic{num}) #1}\par}
%\newcommand{\theo}[1]
%{\bigskip\noindent\refstepcounter{num}\textbf{정리 \arabic{num}) #1}\par}
%\newcommand{\exam}[1]
%{\bigskip\noindent\refstepcounter{num}\textbf{예시 \arabic{num}) #1}\par}
%\newcommand{\prob}[1]
%{\bigskip\noindent\refstepcounter{num}\textbf{문제 \arabic{num}) #1}\par}
\newcommand{\howo}[1]
{\bigskip\noindent\refstepcounter{num}\textbf{숙제 \arabic{num}) #1}\par\bigskip}


\renewcommand{\proofname}{증명)}
\counterwithout{subsection}{section}


%%%
\begin{document}

\title{영석 : 07 숙제(\(\sim\)2015. 6. 13. 토요일)}
\author{}
\date{\today}
\maketitle
%\tableofcontents
\newpage


%
\howo{}
다음 방정식이 나타내는 원의 중심과 반지름의 길이를 구하여라.\\
(1) \(x^2+y^2=25\)\\
(2) \((x-2)^2+(y+1)^2=4\)\\
(3) \(x^2+y^2-4x=0\)\\
(4) \(x^2+y^2-6x+3=0\)

%
\howo{}
다음 원과 직선의 교점의 개수를 구하여라.
\[x^2+y^2=2,\quad y=x+1\]

%
\howo{}
다음 원과 직선의 교점의 개수를 구하여라.
\[x^2+y^2=2,\quad y=x+2\]

%
\howo{}
다음 원과 직선의 교점의 개수를 구하여라.
\[x^2+y^2=2,\quad y=x+3\]

%
\howo{}
기울기가 \(-1\)이고 원 \(x^2+y^2=1\)에 접하는 직선의 방정식을 구하여라.

%
\howo{}
원 \(x^2+y^2=5\) 위의 점 \((2,-1)\)에서의 접선의 방정식을 구하여라.

%
\howo{}
점 \((0,3)\)에서 \(x^2+y^2=8\)에 그은 접선의 방정식을 모두 구하여라.

\end{document}